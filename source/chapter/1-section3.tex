\chapter*{Fourieranalysis} \addcontentsline{toc}{chapter}{Fourieranalysis}

\setcounter{section}{2}

\section{Die Fouriertransformation auf $L^{1}$, $\Sz$ und $L^{2}$}



\begin{definition}
	Für $f \in L^{1}(\R^{d})$ setze $\hat{f}(\xi) = \ff f(\xi) = \int_{\R^{d}} e^{-2\pi i x \cdot \xi} f(x) dx$
\end{definition}


\begin{bemerkung*}
 	Für $d = 1$, $f \in L^{2}[0, \pi] \subseteq L^{2}(\R) \Rightarrow \hat{f}(n) = \int_{0}^{2\pi} e^{-2\pi n x} f(x) dx$ klassische Fourierkoeffizienten $f \in L^{2}[0, 2\pi]$ sind die Werte von $\hat{f}(\xi)$ für $\xi = n \in \Z$. $f = \sum_{n \in \Z} \hat{f}(n) e^{2\pi i n x}$ (Fourierreihen)
\end{bemerkung*}


\begin{prop}
	Für $f \in L^{1}(\R^{d})$ ist $\hat{f} \in L^{\infty}(\R^{d})$ und $\| \hat{f} \|_{L^{\infty}} \leq \| f\|_{L^{1}}$. Es gilt sogar $\hat{f} \in C_{0}(\R^{d})$.
\end{prop}

\begin{beweis}
	$|\hat{f}(\xi)| \leq \int |e^{-2\pi i x \xi}| |f(x)| dx = \| f \|_{L^{1}}$. Für $\hat{f} \in C_{0}(\R^{d})$ siehe Übung.
\end{beweis}


\begin{prop}
	\begin{enumerate}
		\item für die Dilation $f_{\delta}(x) \coloneqq f(\delta x), \delta > 0$ fest gilt $\hat{f}_{\delta}(x) = \delta^{-d} \hat{f}(\delta^{-1} \xi)$
		\item $f(x + h) \ft \hat{f}(\xi) e^{2\pi i \xi h}, h \in \R^{d}$ fest, $f(x)e^{-2\pi i \xi h} \ft \hat{f}(\xi + h)$
	\end{enumerate}
\end{prop}

\begin{beweis}
	\begin{enumerate}
		\setcounter{enumi}{1}
		\item $\ff[f(\cdot + g)](\xi) = \int e^{-2\pi i x \xi}f(x + h) dx \overset{x'=x+h}{=} \int e^{-2\pi i(x'-h) \xi} f(x') dx' = e^{2\pi ih\xi} \hat{f}(\xi)$
	\end{enumerate}
\end{beweis}


\begin{definition} \index{Faltung}
	Faltung von $f, g \in L^{1}(\R^{d})$:
	\[ (f \ast g)(x) \coloneqq \int_{\R^{d}} f(x-y)g(y) fy = \int_{\R^{d}} f(y)g(x-y) dy \]
\end{definition}


\begin{bemerkungnbr*}
	$\| f \ast g \|_{L^{1}} \leq \| f \|_{L^{1}} \cdot \| g \|_{L^{1}}$	
\end{bemerkungnbr*}


\begin{propnbr}
	$\widehat{(f \ast g)}(\xi) = \hat{f}(\xi) \cdot \hat{g}(\xi)$
\end{propnbr}

\begin{beweis}
	\begin{align*}
		\widehat{(f \ast g)}(\xi) & = \int e^{-2\pi i x \xi} \left( \int f(x-y)g(y) dy \right) dx \overset{Fubini}{=} \int e^{-2 \pi i y \xi} g(y) \left( \int e^{-2\pi i (x-y)\xi} f(x-y) dx \right) dy \\
			& \overset{x'=x-y}{=} \left( \int e^{-2\pi i y\xi} g(y) dy \right) \left( \int e^{-2\pi i x' \xi} f(x')dx' \right) = \hat{g}(\xi) \cdot \hat{f}(\xi)
	\end{align*}
\end{beweis}


\begin{definition} \index{schnell fallend} \index{Schwartzraum}
	$f \in C^{\infty}(\R^{d})$ hei{\ss}t schnell fallend, falls für alle Multiindizes $\alpha, \beta \in \N^{d}_{0}$ gilt
	\[ \| f \|_{\alpha, \beta} \coloneqq \sup_{\xi \in \R^{d}} | \xi^{\beta} (D^{\alpha}f)(\xi)| < \infty \]
	Notation $f \in \Sz(\R^{d})$ (schnell fallende Fkt, Schwartzraum). \\
	Dabei ist für $\alpha \in \N_{0}^{d}, x \in \R^{d}: D^{\alpha} = D_{x_{1}}^{\alpha_{1}} \cdots D_{x_{d}}^{\alpha_{d}}, x^{\alpha} = x_{1}^{\alpha_{1}} \cdot \dotsc \cdot x_{d}^{\alpha_{d}}, |a| = \alpha_{1} + \dotsc + \alpha_{d}$
\end{definition}


\begin{bemerkung*}
	Definition 3.6 hei{\ss}t, dass alle Ableitungen schneller gegen Null geht als jedes Polynom gegen $\infty$ für $\|x\| \rightarrow \infty$, d.h. $\forall \alpha, \beta \in \N_{0}^{d}$, $\forall m \in \N$ gilt
	\[ (1 + |x|)^{m} x^{\beta} D^{\alpha} f(x) \in L^{\infty}(\R^{d}) \]
	Insb $x^{\beta} D^{\alpha} f \in \Sz(\R^{d}) \subseteq \bigcap_{p \geq 1} L^{p}(\R^{d}) \Rightarrow$ Im Himmelreich für Fubini, Diff unter dem Integralzeichen, Leb. Konvergenzsatz usb.	
\end{bemerkung*}


\begin{beispiel}
	\begin{enumerate}
		\item $C_{c}^{\infty}(\R^{d}) \subseteq \Sz(\R^{d})$. Aber $\hat{f}(\xi) = \int_{K} e^{-2\pi i x \xi} f(x) dx \notin C_{c}^{\infty}(\R^{d})$ $(K = \supp f)$
		\item $h(x) = e^{-\pi |x|^{2}}$, $h \in \Sz(\R^{d})$. Später: $\hat{h} = h$
	\end{enumerate}
\end{beispiel}


\begin{satz} \label{satz:3.8}
	Mit $f \in \Sz(\R^{d})$ sind auch $x \in f(x), D^{\alpha}f, D^{\alpha}(\hat{f}), \widehat{D^{\alpha}f}$ in $\Sz(\R^{d})$ und
	\begin{enumerate}
		\item $D^{\alpha}\hat{f} = \ff[(-2\pi i x)^{\alpha} f(x)]$ $( (-2\pi i x)^{\alpha} = (-2\pi i)^{|\alpha|}\cdot x_{1}^{\alpha_{1}} \cdot \dotsc \cdot x_{d}^{\alpha_{d}})$
		\item $(2 \pi i \xi)^{\alpha} \hat{f}(\xi) = \widehat{D^{\alpha}f}(\xi)s$
	\end{enumerate}	
\end{satz}

\begin{beweis}
	\begin{enumerate}
		\item $D_{\xi_{1}}(\hat{f})(\xi) = \int D_{\xi_{1}}[e^{-2\pi i x \xi}]f(x) dx = \int (-2 \pi i x_{1})e^{-2\pi i x \xi} f(x) dx$
			\[ D_{\xi_{2}} D_{\xi_{1}} (\hat{f})(\xi) = \int D_{\xi_{2}} [e^{-2\pi i x \xi}] (-2\pi i x_{1})f(x) dx = \int e^{-2\pi i x \xi} (-2 \pi i x_{2})(-2 \pi i x_{1}) f(x) dx \]
		\item $~\hat{f}(x) = \lim_{R \rightarrow \infty} \int_{-R}^{R} \cdots \int_{-R}^{R} e^{-2 \pi i x y} f(y) dy$ 
			\begin{align*} 
				& = \lim_{R \rightarrow \infty} e^{-2 \pi i x_{1} y_{1}} \underbrace{\left( \int_{-R}^{R} e^{-2 \pi x_{2} y_{2}} \cdots \int_{-R}^{R} e^{-2 \pi i x_{d} y_{d}} f(y_{1} \dotsc y_{d}) dy_{2} \dotsc dy_{d} \right)}_{\eqqcolon I_{R}(y_{1}} dy_{d} ~ \\
				& \overset{P.I.}{=} \lim_{r \rightarrow \infty} - \int_{-R}^{R} \frac{1}{-2 \pi i x_{1}} e^{- 2 \pi i x_{1}y_{1}} D_{y_{1}} I_{R}(y_{1}) dy_{1} + \underbrace{\lim_{R \rightarrow \infty} \left[ \frac{1}{- 2 \pi i x_{1}} I_{R}(y_{1} \right]_{y_{1} = -R}^{y_{1} = R}}_{= 0} \\
				& = \frac{1}{2 \pi i x_{1}} \lim_{R \rightarrow \infty} \int_{-R}^{R} e^{-2 \pi i x_{1} y_{1}} \left( \int_{-R}^{R} \cdots \int_{-R}^{R} e^{-2 \pi i x_{2} y_{2}} \dotsc e^{-2 \pi i x_{d} y_{d}} D_{y_{1}} f(y_{1} \dotsc y_{d}) dy_{2} \dotsc dy_{d} \right) dy_{1} \\
				& = \frac{1}{2 \pi i x_{1}} \lim_{R \rightarrow \infty} \int_{[-R, R]^{d}} e^{-2 \pi i x \cdot y} D_{y_{1}} f(y) dy \\
				& = \frac{1}{2 \pi i x_{1}} \widehat{D_{y_{1}}(f)}(x)
			\end{align*}
	\end{enumerate}
\end{beweis}


\begin{beispielnbr}
	$\ff[e^{-\pi |x|^{2}}](\xi) = e^{-\pi|\xi|^{2}}$
\end{beispielnbr}

\begin{beweis}
	$d = 1$: Setze $h(\xi) = \int_{-\infty}^{\infty} e^{- \pi x^{2} - e \pi i x \xi} dx$, d.h. $h(\xi) = \ff[e^{-\pi|x|^{2}}](\xi)$
	\begin{align*}
		\xRightarrow[]{\hyperref[satz:3.8]{3.8}} h'(\xi) & = \int_{-\infty}^{\infty} (-2\pi i x)e^{-\pi x^{2} - 2 \pi i x \xi} dx = i \int_{-\infty}^{\infty} \frac{d}{dx} [e^{- \pi x^{2}}]d^{-2\pi x \xi} dx = i \ff [\frac{d}{dx}] e^{-\pi x^{2}}](\xi) \\
		& = i(2\pi i \xi) \ff[e^{-\pi x^{2}}](\xi) = - 2 \pi \xi h(\xi) \Rightarrow h'(\xi) = - 2 \pi \xi h(\xi)
	\end{align*}
	$\Rightarrow \frac{h'(\xi)}{h(\xi)} = - 2 \pi \xi \rightarrow ln h(\xi) = 2 \pi \int_{0}^{x}(-\xi) d\xi \Rightarrow h(\xi) = e^{- \pi \xi^{2}} = \ff[e^{-\pi x^{2}}](\xi) $ \\
	$d > 1$: $\ff[e^{-\pi|x|^{2}}](\xi) = \int e^{-2\pi i \xi x} e^{- \pi |x|^{2}} = \prod_{j = 1}^{d} \int_{\R} e^{-\pi x_{j}^{2}} e^{-2\pi i \xi_{j} x_{j} dx_{j}} dx_{j} = \prod_{j = 1}^{d}e^{-\pi \xi_{j}^{2}} = e^{- \pi |\xi|^{2}}$
\end{beweis}


% todo end missing
 
\newpage