\section{Don't know} %todo start missing


\setcounter{satz}{2}

\begin{satz} \label{satz:4.3}
	Sei $v : \Sz(\R^{d}) \rightarrow \C$ linear. Dann gilt:
	\[ v \text{ stetig (d.h. } v \in \Sz(\R^{d}) \text{)} \gdw \exists N \in \N: |v(f)| \leq C \| f \|_{N} \text{ für alle } f \in \Sz(\R^{d}) \]
\end{satz}

\begin{beweis}
	$"\Leftarrow"$ klar. \\
	$"\Rightarrow"$ Andernfalls gibt es zu jedem $n \in \N$ ein $f_{n} \in \Sz$ mit $\|f_{n}\|_{n} = 1$, aber $|v(f_{n}) \geq n$. Setze $g_{n} = \frac{f_{n}}{\sqrt{n}}$. Dann
		\[ \| g_{n} \|_{N} \leq n^{-\frac{1}{2}} \xrightarrow[n \rightarrow \infty]{} 0 \text{ für } n \geq N \]
		Aber: $|v(g_{n})| \geq \sqrt{n} \xrightarrow[n \rightarrow \infty]{} \infty$. Also ein Widerspruch zur Stetigkeit von $v$.
\end{beweis}


\begin{beispiele}
	\begin{enumerate}
		\item $"L^{p}(\R^{d}) \subset \Sz'(\R^{d})"$. Sei $u \colon \R^{d} \rightarrow \C$ lokal integrierbar und
			\[ (*) \quad \int_{|x|\leq R} |h(x)| dx \leq C R^{n} \text{ für } R > 1 \text{ und ein festes } C < \infty, n \in \N \text{ fest.} \label{eq:4.4-*} \]
			Dann wird durch
				\[ u_{n}(f) = \int_{\R^{d}} h(x) f(x)dx \]
			eine Distribution $u_{n} \in \Sz'(\R^{d})$ definiert, d.h. wir erhalten eine Einbettung 
				\[ h \in \{ \text{ Fkt. mit } \hyperref[eq:4.4-*]{(*)} \} \rightarrow v_{n} \in \Sz'(\R^{d}) \]
			\begin{beweis}
				Benutze \hyperref[satz:4.3]{4.3} mit $u$ auf $(*)$
				\begin{align*}
					|u_{n}(f)| & = |\int h(x) f(x) dx | = | \int h(x) (1 + |x|^{2})^{-2-n}(1+|x|^{2})^{2+n}f(x)dx| \\
					& \leq \underbrace{\int |h(x)(1+|x|^{2})^{-2-n}dx}_{\leq C \| f \|_{N} < \infty} \cdot \underbrace{\sup_{x \in \R^{d}} (1 + |x|^{2})^{2+R}|f(x)|}_{\underset{(N = 4+ 2R)}{\leq \| f \|_{4+2R}}}
				\end{align*}
				$h \in L^{p}(\R^{d}), \int_{|x|=R} |h(x) dx \overset{Hölder}{\leq} \left( \int_{|x| \leq R} |h(x)|^{p} dx \right)^{\frac{1}{p}} R^{\frac{d}{p'}}$ mit $1 = \frac{1}{p} + \frac{1}{p'}$. Hier ist $(*)$ erfüllt mit ... % todo check this whole line
			\end{beweis}
		\item Sei $\psi \in C^{\infty}(\R^{d})$ und langsam wachsend, d.h. für alle $\alpha \in \N_{0}^{d}$ gibt es $N_{\alpha}, C_{\alpha} < \infty$ mit
			\[ |D^{\alpha} \psi(x)| \leq C_{\alpha} (1 + |x|)^{N_{\alpha}}, x \in \R^{d} \]
			Dann kann man für jedes $v \in \Sz'(\R^{d})$ ein Produkt $\psi \cdot v \in S'(\R^{d})$ definieren, durch:
			\[ (\psi \cdot v)(f) \coloneqq u(\psi f) \text{ für alle } f \in S(\R^{d}) \]
			\begin{beweis}
				z.B.: $\psi f \in \R^{d}$
			\end{beweis}
		\item Dirac Distrubution: $\delta_{x} \in \Sz'(\R^{d})$
			\[ \delta_{x}(f) = f(x) \text{ für } f \in S(\R^{d}) \]
		\item $h(x) = e^{|x|^{2}}$. Dann $u_{h} \notin \Sz'(\R^{d})$, denn $f(x) = e^{-|x|^{2}} \in \Sz(\R^{d})$
			\[ u_{h} = \int h(x) g(x) dx = \int 1 dx = \infty \]	
	\end{enumerate}
\end{beispiele}


\begin{definition}[Prinzip der Dualität] \index{duale Abbildung} \index{Dualität}
	Sei $T \colon \Sz(\R^{d}) \rightarrow \Sz(\R^{d})$ linear, stetig. Dann definieren wir die \begriff{duale Abbildung} $T' \colon \Sz'(\R^{d}) \rightarrow \Sz'(\R^{d})$ durch
		\[  T' u(f) = u(Tf) \quad u \in \Sz'(\R^{d}), f \in \Sz(\R^{d}) \]
\end{definition}

\begin{beweis}
	$f_{n} \xrightarrow[]{\Sz} f \Rightarrow T f_{n} \xrightarrow[]{\Sz} T f$.  \\
	$u(tf_{n}) \rightarrow u(Tf)$, $T'u(f_{n}) \rightarrow T'u(f)$, da $T$ und $u$ stetig nach Definition.
\end{beweis}


\begin{definition} \label{def:4.6}
	$F \colon \Sz(\R^{d}) \rightarrow \Sz(\R^{d})$ \\
	$\Rightarrow F' \colon \Sz'(\R^{d}) \rightarrow \Sz'(\R^{d})$ mit $(F'u)(f) = u(\hat{f})$ und $f \in \Sz, u \in \Sz'$
\end{definition}


\begin{bemerkung}
	$h \in L^{1}(\R^{d}) \rightarrow u_{h} \in \Sz'(\R^{d})$. Für $f \in \Sz(\R^{d})$ gilt:
	\[ (F' u_{h})(f) = u_{h}(\hat{f}) = \int h(x) \hat{f}(x) dx \underset{\S 3}{=} \int \underbrace{\widehat{h(x)}}_{\underset{(4.4 a)}{hat (*)}} f(x) dx = u_{\widehat{h(x)}}(f), ~\forall f \in \Sz \]
\end{bemerkung} 


Notation: $F' = F$. Dann $\boxed{F(u_{h}) = u_{Fh} \text{ bzw. } \hat{u}_{h} = u_{\hat{h}}}$


\begin{prop} % todo diese prob überprüfen
	$F \colon \Sz'(\R^{d}) \rightarrow \Sz'(\R^{d})$ bijektiv, $(F^{-1}u)(f) = (Fu)(\tilde{f})$, wobei $\tilde{f}(x) = f(-x)$ \\
		\[ F \cdot F^{-1} = Id_{\Sz} \]
	Dualität: $(F^{-1})' F' = (F \cdot F^{-1})' = Id_{\Sz}' = Id_{\Sz}$ (und umgekehrt).
		\[ \Rightarrow (F')^{-1} = (F^{-1})' \text{ und daher bijektiv} \]
		Au{\ss}erdem $(F^{-1}u)(f) = u(F^{-1}f) \underset{\S 3}{=} u(Ff(-\cdot)) = F' u (F-\cdot) = F'u(\tilde{f})$
\end{prop}
 

\begin{definition}
	Sei $u \in \Sz'(\R^{d})$, $\alpha \in \N_{0}^{d}$. Definiere $D_{u}^{\alpha} \in \Sz'(\R^{d})$ durch:
	\[ (D^{\alpha}u)(f) = (-1)^{|\alpha|} u(D^{\alpha}f) \]
\end{definition} 
 

\begin{bemerkung}
	\begin{enumerate}
		\item $X^{\alpha} \colon \Sz(\R^{d}) \rightarrow \Sz(\R^{d}), \underset{\Sz'}{D^{\alpha}} u \coloneqq (-1)^{|\alpha|} (\underset{\Sz}{D^{\alpha}})'$ (im Sinne der Dualität in \hyperref[def:4.6]{4.6}), denn $(\underset{\Sz'}{D^{\alpha}}u)(f) = (-1)^{|\alpha|}\underbrace{u(D^{\alpha}f)}_{\underset{\Sz}{(D^{\alpha}}}$
		\item Sei $h \in \Sz(\R^{d})$ und $u_{h} \in \Sz(\R^{d})'$.
			\[ D^{\alpha}(u_{h})(f) = (-1)^{|\alpha|} u_{h}(D^{\alpha}f) = (-1)^{|\alpha|} \int h(x) D^{\alpha}f(x) dx \overset{P.I.}{=} \int (D^{\alpha}h)f(x)dx \]
			Also: $\boxed{D^{\alpha}(u_{h}) = u_{D^{\alpha}h}}$ 
	\end{enumerate}
\end{bemerkung}
 

\begin{prop}
	\begin{enumerate}
		\item Sei $h \in C^{\infty}(\R^{d})$ eine langsam wachsende Funktion und $u \in \Sz'(\R^{d})$. Dann:
			\[ D_{x_{i}}(h \cdot u) = (D_{x_{i}}h)\cdot u + h \cdot D_{x_{i}}u \]
		\item $u \in \Sz'(\R^{d}), \alpha, \beta \in \N_{0}^{d}$. Dann:
			\[ D^{\alpha + \beta} u = D^{\alpha} (D^{\beta} u) \]
	\end{enumerate}	
\end{prop}

\begin{beweis}
	Übung.
\end{beweis}
 

\begin{beispiel}
	\begin{enumerate}
		\item $d = 1$, $H(x) = \begin{cases}
			1 \text{ für } x \geq 0 \\ 0 \text{ für } x < 0 \end{cases}$. Beh.: $D(u_{H}) = \delta_{0}$ \\
			\begin{beweis}
				Für $\phi \in \Sz(\R)$ gilt:
					\[ (D u_{H})(\phi) = - u_{H}(D\phi) = - \int H(x) \phi'(x) dx = - \int_{0}^{\infty} \phi'(x) dx = \phi(0) \]
				Au{\ss}erdem: $(D^{\alpha}\delta_{a})(\phi) = (-1)^{|\alpha|}\delta_{a}(D^{\alpha}\phi) = (-1)^{|\alpha|} D^{\alpha} \phi(0)$
			\end{beweis} 
	\end{enumerate}
\end{beispiel}
 

Vorschau: Zu jedem $u \in \Sz'(\R^{d})$ gibt es eine langsam wachsende stetige Funktion $\psi \colon \R^{d} \rightarrow \C$ und $\alpha \in \N_{0}^{d}$, so dass $u = D^{\alpha} \psi$. (Keine Eindeutigkeit!)
 
 
 
\newpage