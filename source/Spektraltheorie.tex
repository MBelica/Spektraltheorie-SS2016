\documentclass[12pt]{extreport} % Schriftgröße: 8pt, 9pt, 10pt, 11pt, 12pt, 14pt, 17pt oder 20pt

% Language Setup (Deutsch)
\usepackage[utf8]{inputenc} 
\usepackage[ngerman]{babel}

%% Packages
\usepackage{scrextend}
\usepackage{amssymb}
\usepackage{amsthm}
\usepackage{amsmath}
\usepackage{changes}
\usepackage{cmap}
\usepackage{amsfonts}
\usepackage{apptools}
\usepackage{booktabs}
\usepackage{dsfont}
\usepackage{lmodern}
\usepackage{makeidx}
\usepackage{xcolor}
\usepackage{mathtools} 
\usepackage{graphicx}
\usepackage{geometry}
\usepackage{caption}
\usepackage{xpatch}
\usepackage{amssymb}
\usepackage{amsmath}
\usepackage{mathrsfs}
\usepackage{subfigure}
\usepackage{stmaryrd}
\usepackage{enumerate}
\usepackage{hyperref}
\usepackage{footnote}
\usepackage{mathrsfs}
%\usepackage{subcaption}
\usepackage[arrow, matrix, curve]{xy}
\geometry{a4paper,left=20mm,right=20mm,top=25mm,bottom=20mm}

% 1.1, 1.2 etc durchnummeriert, ebenso die Gleichungen mit (1.1), (1.2) ..
\newtheorem{Satz}{Satz}[subsection]
\newtheorem{Definition}[Satz]{Definition} 
\newtheorem{Klass}[Satz]{Klassifizierung} 
\newtheorem{Lemma}[Satz]{Lemma}	
\newtheorem{Bemerkung}[Satz]{Bemerkung}	
\newtheorem{Folgerung}[Satz]{Folgerung}  
\newtheorem{Beispiel}[Satz]{Beispiel} 
\newtheorem{DefBem}[Satz]{Definition + Bemerkung} 
\newtheorem{Prop}[Satz]{Proposition}
\newtheorem{PropDef}[Satz]{Proposition + Definition}
\newtheorem{Erinnerung}[Satz]{Erinnerung}
\newtheorem{Korollar}[Satz]{Korollar}
\newtheorem{Motivation}[Satz]{Motivation}
\newtheorem{Wiederholung}[Satz]{Wiederholung}
\newtheorem{Example}{Example}

\DeclareMathOperator{\trdeg}{trdeg}
\DeclareMathOperator{\Spec}{Spec}
\DeclareMathOperator{\Ann}{Ann}
\DeclareMathOperator{\Kern}{Kern}
\DeclareMathOperator{\Bild}{Bild}
\DeclareMathOperator{\Sh}{Sh}
\DeclareMathOperator{\Quot}{Quot}
\DeclareMathOperator{\Hom}{Hom}
\DeclareMathOperator{\Mor}{Mor}
\DeclareMathOperator{\id}{id}
\DeclareMathOperator{\ord}{ord}
\DeclareMathOperator{\Pic}{Pic}
\DeclareMathOperator{\GL}{GL}
\DeclareMathOperator{\Ob}{Ob}
\DeclareMathOperator{\Supp}{Supp}
\DeclareMathOperator{\Proj}{Proj}
\DeclareMathOperator{\Rang}{Rang}
\DeclareMathOperator{\res}{res}
\DeclareMathOperator{\Char}{char}
\DeclareMathOperator{\Res}{Res}
\DeclareMathOperator{\tr}{tr}
\DeclareMathOperator{\pic}{Pic}
\DeclareMathOperator{\ob}{ob}
\DeclareMathOperator{\SetOp}{Set}
\DeclareMathOperator{\KVecOp}{K-Vect}
\DeclareMathOperator{\RModOp}{R-Mod}
\DeclareMathOperator{\GroupOp}{Group}
\DeclareMathOperator{\AbelianOp}{Ab}
\DeclareMathOperator{\TopOp}{Top}
\DeclareMathOperator{\HoTopOp}{HoTop}
\DeclareMathOperator{\Graph}{Graph}
\DeclareMathOperator{\spann}{span}
\DeclareMathOperator{\supp}{supp}
\DeclareMathOperator{\cov}{cov}

\numberwithin{equation}{section} 

% einige Abkuerzungen
\newcommand{\C}{\mathbb{C}} % komplexe
\newcommand{\K}{\mathbb{K}} % komplexe
\newcommand{\R}{\mathbb{R}} % reelle
\newcommand{\Q}{\mathbb{Q}} % rationale
\newcommand{\Z}{\mathbb{Z}} % ganze
\newcommand{\N}{\mathbb{N}} % natuerliche

\newcommand{\f}{\hat{f}}
\newcommand{\g}{\hat{g}}
\newcommand{\F}{\mathcal{F}}
\newcommand{\G}{\mathcal{G}}
\newcommand{\om}{\omega}
\newcommand{\intR}{\int_{-\infty}^{\infty}}
\newcommand{\m}{\cdot}
\newcommand{\eF}{e^{-2i\pi \om t}}
\newcommand{\eIF}{e^{-2i\pi \om t}}
\newcommand{\Aff}{\mathbf{Aff}}
\newcommand{\A}{\mathbb{A}}
\newcommand{\PR}{\mathbb{P}}
\newcommand{\Bew}{\emph{Beweis: }}
\newcommand{\pw}{\wp}
\newcommand{\qscr}{\mathscr{Q}}
\newcommand{\Ocal}{\mathcal{O}}
\newcommand{\Hc}{\check{H}}
\newcommand{\Ccal}{\mathcal{C}}
\newcommand{\laplace}{\Delta}

\newcommand{\Set}{\underline{\SetOp}}
\newcommand{\KVect}{\underline{\KVecOp}}
\newcommand{\RMod}{\underline{\RModOp}}
\newcommand{\Grp}{\underline{\GroupOp}}
\newcommand{\Ab}{\underline{\AbelianOp}}
\newcommand{\Top}{\underline{\TopOp}}
\newcommand{\HoTop}{\underline{\HoTopOp}}



% Options
\makeatletter%
  % Linkfarbe, {0,0.35,0.35} für Türkis, {0,0,0} für Schwarz 
  \definecolor{linkcolor}{rgb}{0,0.35,0.35}
  % Zeilenabstand für bessere Leserlichkeit
  \def\mystretch{1.75} 
  % Publisher definieren
  \newcommand\publishers[1]{\newcommand\@publishers{#1}} 
  % Enumerate im 1. Level: \alph für a), b), \dotsc
  \renewcommand{\labelenumi}{\alph{enumi})} 
  % Enumerate im 2. Level: \roman für (i), (ii), \dotsc
  \renewcommand{\labelenumii}{(\roman{enumii})}
  % Zeileneinrückung am Anfang des Absatzes
  \setlength{\parindent}{0pt} 
  % Für das Proof-Environment: 'Beweis:' anstatt 'Beweis.'
  \xpatchcmd{\proof}{\@addpunct{.}}{\@addpunct{:}}{}{} 
  % Nummerierung der Bilder, z.B.: Abbildung 4.1
  \@ifundefined{thechapter}{}{\def\thefigure{\thechapter.\arabic{figure}}} 
\makeatother%

% Meta Setup (Für Titelblatt und Metadaten im PDF)
\title{Spektraltheorie}
\author{Prof. Dr. Lutz Weiss}
\date{Vorlesungsmitschrieb ~\vspace{0.2cm} \\ Sommersemester 2015/16}
\publishers{Karlsruher Institut für Technologie}


%% Template
\makeatletter%
\hypersetup{colorlinks,breaklinks, urlcolor=linkcolor, linkcolor=linkcolor, pdftitle=\@title, pdfauthor=\@author, pdfsubject=\@title, pdfcreator=\@publishers}\DeclareOption*{\PassOptionsToClass{\CurrentOption}{report}} \ProcessOptions \def\baselinestretch{\mystretch} \setlength{\oddsidemargin}{0.125in} \setlength{\evensidemargin}{0.125in} \setlength{\topmargin}{0.5in} \setlength{\textwidth}{6.25in} \setlength{\textheight}{8in} \addtolength{\topmargin}{-\headheight} \addtolength{\topmargin}{-\headsep} \def\pulldownheader{ \addtolength{\topmargin}{\headheight} \addtolength{\topmargin}{\headsep} \addtolength{\textheight}{-\headheight} \addtolength{\textheight}{-\headsep} } \def\pullupfooter{ \addtolength{\textheight}{-\footskip} } \def\ps@headings{\let\@mkboth\markboth \def\@oddfoot{} \def\@evenfoot{} \def\@oddhead{\hbox {}\sl \rightmark \hfil \rm\thepage} \def\chaptermark##1{\markright {\uppercase{\ifnum \c@secnumdepth >\m@ne \@chapapp\ \thechapter. \ \fi ##1}}} \pulldownheader } \def\ps@myheadings{\let\@mkboth\@gobbletwo \def\@oddfoot{} \def\@evenfoot{} \def\sectionmark##1{} \def\subsectionmark##1{}  \def\@evenhead{\rm \thepage\hfil\sl\leftmark\hbox {}} \def\@oddhead{\hbox{}\sl\rightmark \hfil \rm\thepage} \pulldownheader }	\def\chapter{\cleardoublepage  \thispagestyle{plain} \global\@topnum\z@ \@afterindentfalse \secdef\@chapter\@schapter} \def\@makeschapterhead#1{ {\parindent \z@ \raggedright \normalfont \interlinepenalty\@M \Huge \bfseries  #1\par\nobreak \vskip 40\p@ }} \newcommand{\indexsection}{chapter} \patchcmd{\@makechapterhead}{\vspace*{50\p@}}{}{}{}
	% Titlepage
	\def\maketitle{ \begin{titlepage} 
			~\vspace{3cm} 
		\begin{center} {\Huge \@title} \end{center} 
	 		\vspace*{1cm} 
	 	\begin{center} {\large \@author} \end{center} 
	 	\begin{center} \@date \end{center} 
	 		\vspace*{7cm} 
	 	\begin{center} \@publishers \end{center} 
	 		\vfill 
	\end{titlepage} }
\makeatother%

% Indexdatei erstellen
\makeindex 

\begin{document}
	\maketitle
	
	\tableofcontents
	\newpage
	\chapter{Einführung}
	
	\section{Wiederholung aus der Funktionalanalysis}
	
	\subsubsection{Abgeschlossene Operatoren}
	
	Sei $X$ ein Banachraum, $X\supset D(A)\overset{A}{\rightarrow} X$ linear auf einem linearen Teilraum $D(A)$ von $X$. $D(A)$ heißt \textbf{Definitionsbereich}. 
	
	\begin{Definition}[Abgeschlossener Operator]
		$A$ heißt abgeschlossen, falls für alle $x_n\in D(A)$, $x_n\rightarrow x$ in $||.||_X$, $Ax_n\rightarrow y$ in $||.||_y$ $\Rightarrow$ $x\in D(A)$, $Ax= y$.
	\end{Definition}
	
	Notation: $||x||_A = ||x||+||Ax||$, $x\in D(A)$ heißt Graphennorm von $A$. Also: 
	$$A:(D(A),||.||_A)\rightarrow (X,||.||_X)$$
	stetig.
	
	\begin{Satz}
		$A:X\supset D(A)\rightarrow X$ ist abgeschlossen $\Leftrightarrow$ $\Graph(A) = \{(x,Ax)\in X\times X,~ x\in D(A) \}$ ist abgeschlossen in $X\times X$ $\Leftrightarrow$ $(D(A),||.||_A)$ vollständig normierter Raum.
	\end{Satz}
	
	\begin{Korollar}
		$X = D(A)$, $A$ abgeschlossen $\Leftrightarrow$ $A:X\rightarrow X$ stetig $\Leftrightarrow$ $A(U_X)$ beschränkt in $X$, $U_X = $ offene Einheitskugel in $X$.
	\end{Korollar}
	
	Notation: $B(X)$ ist der Banachraum aller beschränkten/stetigen linearen Operatoren $A:X\rightarrow X$ mit der Norm $||A|| = \sup_{x\in U_X}||Ax|| <\infty$.
	
	\begin{Bemerkung}
		Sei $D\subseteq X$ ein dichter, linearer Teilraum von $X$, d.h. $\bar{D} = X$. Sei $A:D\rightarrow X$ ein linearer Operator mit $||Ax||\leq C||x||$ für alle $x\in D$. Dann gibt es \textbf{genau eine} stetige Fortsetzung $\tilde{A}: X\rightarrow X$, d.h. $\tilde{A}\in B(X)$, $\tilde{A}|D = A$, $||\tilde{A}||\leq C$ (Sei $x\in X$ mit $x_n=\lim x_n$, $x_n\in D$. Dann $\tilde{A}x=\lim_{n\rightarrow\infty} Ax_n$).
	\end{Bemerkung}
	
	\subsubsection{Spektrum und Resolvente}
	
	Sei $X$ ein Banachraum, $A:X\supset D(A)\rightarrow X$ linearer, abgeschlossener Operator.
	
	Sei gegeben: $\lambda x-Ax=y$,$\lambda\in \C$, $y\in X$, $x\in D(A)$ ist gesucht. Formal $x=(\lambda-A)^{-1}y$ ist Lösung, falls $(\lambda-A)^{-1}$ existiert.
	
	\begin{Definition}
		$\lambda\in \rho(A)$ falls $\lambda -A: D(A)\rightarrow X$ bijektiv oder äquivalent: 
		$$\lambda-A:(D(A),||.||_A)\rightarrow (X,||.||_X)$$
		ist ein Isomorphismus. $\rho(A)$ heißt die \textbf{Resolventenmenge} von $A$. $\sigma(A)=\C\backslash\rho(A)$ heißt das \textbf{Spektrum} von $A$.
		
		$R(\lambda, A):= (\lambda-A)^{-1}: X\rightarrow D(A)\subset X$, für ein $\lambda\in \rho(A)$. $R(\lambda,A)\in B(X)$, aber $\Bild(R(\lambda,A))\subset D(A)$.
	\end{Definition}
	
	\begin{Bemerkung}
		Fall $A\in B(X)$, $D(A) = X$, dann ist $R(\lambda, A): X\rightarrow X$ ein 
		Isomorphismus.
	\end{Bemerkung}
	
	\begin{Satz}~
		\begin{enumerate}
			\item[a)] $\rho(A)$ ist offen, $\sigma(A)$ abgeschlossen.
			\item[b)] Falls $A\in B(X)$, dann: $|\lambda|\leq ||A||$ für alle $\lambda\in \sigma(A)$. Insbesondere: $\sigma(A)$ Kompakt. $\sigma(A)\neq\emptyset$.
		\end{enumerate}
	\end{Satz}
	
	\begin{Satz}[Resolventendarstellung]~
		\begin{enumerate}
			\item[a)] Sei $\lambda_0\in \rho(A)$, $|\lambda-\lambda_0|\leq\frac{1}{||R(\lambda_0,A)||}$. Dann ist
			$$R(\lambda,A) =\sum_{n = 0}^{\infty} (\lambda_0-\lambda)^n R(\lambda_0, A)^{n+1}$$
			analytisch.
			\item[b)] Sei $A\in B(X)$ und $|\lambda|>||A||$, dann ist 
			$$R(\lambda, A) = \sum_{n = 0}^{\infty}\lambda^{-(n+1)}A^n.$$ 
		\end{enumerate}
	\end{Satz}
	
	
	\begin{Satz}[Resolventenregel]
		Für $\lambda,\mu\in \rho(A)$ gilt:
		$$R(\lambda,A)-R(\mu,A) = (\mu-\lambda)R(\lambda,A)R(\mu,A).$$
		Für $\mu\rightarrow \lambda$:
		$$\frac{d}{d\lambda}R(\lambda,A) = -R(\lambda,A)^2.$$
	\end{Satz}
	
	\begin{Beispiel}~
		\begin{enumerate}
			\item[a)] Sei $X=l^p$, $(x_n)\subset\C$.
			$$D(A) = \{(x_n)\in l^p: (\sum_n |\lambda_n x_n|^p)^{1/p} \}$$
			$A(x_n) = (\lambda_n x_n)\in l^p$ für $(x_n)\in D(A)$.
			
			Diagonaloperator: $\sigma(A) = \{\lambda_n\}$, $\lambda\notin \overline{\{\lambda_n\}}$ ist 
			$$R(\lambda,A)(x_n) = (\lambda-A)^{-1}(x_n) = (\frac{1}{\lambda-\lambda_n}x_n).$$
			\item[b)] $X=l^p$, $A(x_n) = (0,x_1,x_2,...)$ (Rechts-Verschiebeoperator). $\sigma(A) = \{\lambda:|\lambda|\leq 1 \}\subset \C$. 
			\item[c)] $X = C[0,1]$, $Tf(t) = \int_0^t f(s)ds$, $t\in [0,1]$ (Volterraoperator). $\sigma(T) = \{0\}$, $||T|| \neq 0$.
		\end{enumerate}
	\end{Beispiel}
	
	\subsubsection{Spektrum und Kompaktheit}
	
	\begin{Satz}
		Sei $X$ ein Banachraum, $A\in B(A)$ kompakt, d.h. $A(U_X)$ ist relativ kompakt in $(X,||.||)$ (z.B. $A\in \{T: X\rightarrow X |\dim\Bild(T)<\infty \}$). Falls $\dim X = \infty$, dann gilt $0\in \sigma(A)$. $\sigma(A)\backslash \{0\}$ besteht aus einer Folge $(\lambda_n)\subset \C$, die aus Eigenwerten von $A$ besteht, mit endlich dimensionalem Eigenraum $\Kern(\lambda_n-A).$ $(\lambda_n)$ ist endlich oder $|\lambda_n|\rightarrow 0$.
	\end{Satz}
	
	Sei ab jetzt $X$ ein Hilbertraum mit $<.,.>$. 
	
	\begin{Definition}
		$(h_n)\subset X$ heißt Orthonormalbasis von $X$, falls $<h_n, h_m> = \delta_{n,m}$, $\overline{\spann(h_n)} = X$.
	\end{Definition}
	
	\begin{Satz}
		Jeder seperable Hilbertraum besitzt eine Orthonormalbasis $(h_n)$ und für alle $x\in X$ gilt
		\begin{eqnarray}
			x &=& \sum_n <x,h_n>h_n,\nonumber\\
			||x||^2 &=& \sum_n|<x,h_n>|^2 \nonumber
		\end{eqnarray}	
	\end{Satz}
	
	\begin{Satz}[Spektralsatz]
		Für kompakte, selbstadjungierte Operatoren auf einem seperablen Hilbertraum.
		Sei $A$ kompakt, selbstadjungiert, d.h. $<Ax, y> = <x, Ay>$. Dann gibt es eine Orthonormalbasis $(h_n)$ von $H$ und eine Folge $(\lambda_n)\subset\R$ mit $|\lambda_1|\geq |\lambda_2|\geq...\geq 0$, $\lim\limits_{n\rightarrow\infty} \lambda_n = 0$, sodass
		$$ Tx= \sum_n \lambda_n<x,h_n>h_n.$$
	\end{Satz}
	
	\begin{Bemerkung}~
		\begin{enumerate}
			\item[a)] $\lambda_n$ sind Eigenwerte von $A$, denn $Th_n = \lambda_nh_n$, $|\lambda_n|\rightarrow 0$.
			\item[b)] $||A|| = \sup_{n\in \N}|\lambda_n| = \lambda_1$.
			\item[c)] $\Kern A = \overline{\spann}\{h_n:\lambda_n = 0 \}$. $\overline{\Bild(A)} = \overline{\spann}\{h_n: \lambda_n\neq 0\}$. $X = (\Kern(A))\oplus\overline{\Bild(A)}$ (orthogonal). $A$ injektiv $\Leftrightarrow$ $\overline{\Bild(A)} = X$ $\Leftrightarrow$ $\lambda_n \neq 0$ $\forall n\in \N$.
			\item[d)] $J:l^2\rightarrow X$, $J(e_n) = h_n$ ($e_n$ Einheitsvektor). $J$ ist Isometrie, denn
			$$||\sum \alpha_n e_n||_{l^2} = (\sum |\alpha_n|^2)^{1/2} = ||\sum_n \alpha_n h_n||_X$$
			$$\xymatrix{
				l^2\ar[r]^D \ar[d]_J & l^2\\
				X\ar[r]_A & X\ar[u]_{J^{-1}}
				}$$
			$D = JJ^{-1}A$, $D(x_n) = (\lambda_n x_n)$ für $x= (x_n) \in l^2$. $A$ ist ähnlich zu einem Diagonaloperator $D$ mit $|\lambda_n|\rightarrow 0$.
		\end{enumerate}
	\end{Bemerkung}
	
	\section{Verpasst}
	
	\newpage
	\section{Fourieranalysis}
	
	\subsubsection{Die Fouriertransformation auf $L^1$, $S$ und $L^2$}
	
	\begin{Definition}
		Für $f\in L^1(\R^d)$ setze 
		$$\f(\xi) = \F(f(\xi))) = \int_{\R^d} e^{-2\pi ix\xi}f(x)dx$$
		mit $x\m \xi = \langle x,\xi\rangle$.
	\end{Definition}
	
	\paragraph{Bemerkung:} $d = 1$, $f\in L^2[0,\pi]\subset L^2(\R)$.
	$$\f(n) = \int_{0}^{2\pi} e^{-2\pi i n x}f(x) dx = \F(f(n))$$
	Klassische Fourier Koeffizienten $f\in L^[0,2\pi]$ sind die Werte von $\F f(\xi)$ für $\xi = n\in \Z$.
	$$f= \sum_{n \in \Z} \f(n) e^{2\pi i n x} \text{ Fourierreihen.}$$
	Ziel: $\sum\rightarrow \int$. Damit folgt $f(x) = \int_{\R}e^{2\pi i x \xi}\f(\xi)d\xi$.
	
	\begin{Prop}
		Für $f\in L^1(\R^d)$ ist $\F f\in L^{\infty}(\R^d)$ und $||\F f||_{L^\infty}\leq ||f||_{L^1}$. Es gilt sogar $\F f\in C_0(\R^d)$, d.h. 
		$$\lim\limits_{|x|\rightarrow 0}\F f(x) = 0.$$
	\end{Prop}
	
	\Bew $|\F f(\xi)|\leq \int |e^{-2\pi i x\xi}| |f(x)| dx$. FÜr $\F f\in C_0(\R^d)$ siehe Übung.
	\qed
	
	\begin{Prop}
		~
		\begin{enumerate}
			\item[a)] Für die Dilation $f_\delta(x) = f(\delta x)$, $\delta>0$ fest:
			$$f(\delta x) \overset{\F}{\rightarrow} \delta^{-d}\f(\delta^{-1}\xi).$$
			Beweis durch Substitution $x' = x\m \delta$ in der Definition von $\F$.
			\item[b)] $f(x+h) \overset{\F}{\rightarrow} \f(\xi)e^{2i\pi \xi h}$, $h\in \R^d$ fest. 
			
			$f(x)e^{-2i\pi x h}\overset{\F}{\rightarrow} \f(\xi+h)$. 
		\end{enumerate}
	\end{Prop}
	\Bew b) $\F[f(\m+h)](\xi) = \int e^{-2i\pi x\xi}f(x+h) dx = (*)$. Substituiere $x' = x+h$
	\begin{eqnarray}
		(*) &=& \int e^{-2i\pi (x'-h)\xi}f(x')dx'\nonumber\\
		&=& e^{2\pi i h\xi}\f(\xi).\nonumber
	\end{eqnarray}
	\qed
	
	\begin{Definition}[Faltung von $f,g\in L^1(\R^d)$]
		$$(f*g)(x):= \int_{\R^d}f(x-y)g(y)dy = \int_{\R^d}f(y) g(x-y) dy$$
		\textbf{Bemerkung:} $||f*g||_{L^1}\leq ||f||\m ||g||$.
	\end{Definition}
	
	\begin{Prop}
		$$\F(f*g)(\xi) = \f(\xi)\m \g(\xi).$$
	\end{Prop}
	
	\Bew $\F(f*g)(x) = \int e^{-2i\pi x\xi}(\int f(x-y)g(y)dy)dx$. Fubini liefert:
	\begin{eqnarray}
		\F(f*g)(x) &=& \int e^{-2i\pi y\xi}g(y) (\int e^{-2i\pi (x-y)\xi}f(x-y)dx)dy\nonumber\\
		&=&(\int e^{-2i\pi y\xi}g(y)dy)\m(\int e^{-2i\pi x\xi}f(x)d(x)) = \g(\xi)\m \f(\xi)\nonumber
	\end{eqnarray}
	\qed
	
	\paragraph{Zur Erinnerung:} $f\rightarrow f*g$ ist eine Glättung.
	
	\begin{Definition}
		$f\in C^{\infty}(\R^d)$ heißt schnell fallend, falls für alle Multiindizes $\alpha,\beta \in \N_0^d$ gilt: 
		$$||f||_{\alpha,\beta} := \sup_{\xi\in \R^d}|\xi^\beta(D^\alpha f)(\xi)| <\infty.$$
		Notation: $f\in S(\R^d)$ (Raum der schnell fallenden Funktionen/Schwarzraum). $\alpha\in \N_0^d$, $\alpha = (\alpha_1,...,\alpha_d)$, $D^\alpha = D_{x_1}^{\alpha_1}...D_{x_n}^{\alpha_n}$ und $x^\alpha$ beschreibt das komponentenweise Potenzieren mit $\alpha_i$.
	\end{Definition}
	
	\paragraph{Bemerkung:} Alle Ableitungen gehen schneller gegen Null als jedes Polynom gegen $\infty$ geht für $|x|\rightarrow \infty$, d.h. $\forall\alpha,\beta\in \N_0^d$, $\forall m\in \N$ gilt:
	$$(I+|x|)^m x^\beta D^\alpha f(x)\in L^\infty(\R^d).$$
	Insbesondere: $f\in S(\R^d)$ $\Rightarrow$ $f\in \bigcap_{p\geq 1} L^p(\R^d)$ (Himmelreich für Fubini, Differentiation unter dem Integralzeichen, Lebesgue Konvergenz...).
	
	\begin{Beispiel}
		~
		\begin{enumerate}
			\item[a)] $C_c^{\infty}(\R^d)\subset S(\R^d)$. Aber $\f(\xi) = \int_K e^{-2i\pi x\xi}f(x) dx\notin C_c^{\infty}(\R^d)$, $K$ kompakt = $\supp f$.
			\item[b)] $h(x) = e^{-\pi|x|^2}$, $h\in S(\R^d)$. Später: $\F(h) = h$.
		\end{enumerate}
	\end{Beispiel}
	
	\begin{Satz}
		Mit $f\in S(\R^d)$ sind auch $x\m f(x)$, $D^\alpha f$, $D^\alpha\F f$, $\F(D^\alpha f)$ in $S(\R^d)$ und 
		\begin{enumerate}
			\item[a)] $D^{\alpha}\F f= \F[(-2i\pi x)^\alpha f(x)]$.
			\item[b)] $(2i\pi \xi)^\alpha\F(f)(\xi) = \F(D^\alpha f)(\xi)$.
		\end{enumerate}
	\end{Satz}
	
	\Bew a) $D_{\xi_1}(\F f)(\xi) = \int D_{\xi_1}[e^{-2i\pi x\xi}]f(x) dx$
	\begin{eqnarray}
		D_{\xi_1}(\F f)(\xi) = \int (2i\pi x_1)e^{-2i\pi x\xi}f(x)dx\nonumber
	\end{eqnarray}
	Analog 
	$$D_{\xi_1}D_{\xi_2}\F f(\xi) = \int e^{-2i\pi x\xi}(-2i\pi x_2)(-2i\pi x_1)f(x) dx...$$
	b) \begin{eqnarray}
		\F f(x) &=& \lim\limits_{R\rightarrow \infty} \int_{-R}^R...\int_{-R}^{R} e^{-2i\pi x y}f(y)dy\nonumber\\
		&=& \lim\limits_{R\rightarrow\infty}\int_{-R}^R e^{-2i\pi x_1y_1}\left(\int_{-R}^{R} e^{-2i\pi x_2y_2}...e^{-2i\pi x_dy_d} f(y_1,...,y_d) dy_2...dy_d\right)dy_1 \nonumber\\
		&=& \lim\limits_{R\rightarrow\infty} \int_{-R}^{R}\frac{1}{-2i\pi x_2}e^{-2i\pi x_1 y_1}D_{y_1}I_R(y_1)dy_1 + \lim\limits_{R\rightarrow\infty}\left[\frac{1}{-2i\pi x_2} I_R(y_1)\right]_{y_1 = -R}^R \nonumber\\
		&=& \frac{1}{-2i\pi x_1}\F(Dy_1 f)(x)\nonumber
	\end{eqnarray}
	mit $I_R = \int_{-R}^{R} e^{-2i\pi x_2y_2}...e^{-2i\pi x_dy_d} f(y_1,...,y_d) dy_2...dy_d$.
	
	\begin{Beispiel}
		$\F[e^{-\pi |x|^2}](\xi) = e^{-\pi|\xi|^2}$, denn (für $d= 1$):
		
		Setze $h(\xi) = \intR e^{-\pi x^2-2i\pi x\xi}dx$, d.h. $h(\xi) = \F[e^{-\pi|x|^2}](\xi)$.
		\begin{eqnarray}
			h'(\xi) &=& \intR (-2i\pi x)e^{-\pi x^2- 2i\pi x\xi}dx \nonumber \\
			&=& i\intR \frac{d}{dx}[e^{-\pi x^2}]e^{-2i\pi x\xi}dx \nonumber\\
			&=& i \F\left[\frac{d}{dx}e^{-\pi x^2} \right](\xi)\nonumber\\
			&=& i(2i\pi\xi)\F\left[e^{-\pi x^2} \right](\xi) = -2\pi\xi h(\xi)\nonumber
		\end{eqnarray}
		Differentialgleichung: $h'(\xi) = -2\pi\xi h(\xi)$.
		\begin{eqnarray}
			\frac{h'(\xi)}{h(\xi)}=-2\pi \xi ~~\Rightarrow~~ \ln(h(\xi)) = 2\pi\int_0^x(-\xi)d\xi\nonumber\\
			\Rightarrow h(\xi) = e^{-\pi\xi^2} = \F[e^{\pi x^2}](\xi)\nonumber
		\end{eqnarray}
		Für $d>1$: 
		\begin{eqnarray}
			\F[e^{-\pi|x|^2}](\xi) &=& \int e^{-2i\pi \xi x}e^{-\pi|x|^2}dx \nonumber\\
			&=& \prod_{j = 1}^{d}\int_{\R}e^{-\pi x_j^2}e^{-2i\pi \xi_j x_j}dx_j = \prod_{j = 1}^{d} e^{-\pi\xi_j^2}\nonumber\\
			&=& e^{-\pi|\xi|^2}\nonumber
		\end{eqnarray}
		Damit folgt $\F(\exp(-\pi\epsilon^2|x|^2)) = \epsilon^{-d}\exp(-\pi|\xi|^2/\epsilon^2)$.
	\end{Beispiel}
	
	\begin{Satz}
		$\F:S(\R^d)\rightarrow S(\R^d)$ ist bijektiv und 
		$$\F^{-1}\phi(x) = \int_{R^d} e^{2i\pi x\xi}\phi(\xi)d\xi.$$
	\end{Satz}
	
	\Bew Seien $\phi,\psi\in S(\R^d)$, $\F(\phi)\in S(\R^d)$, $x\in \R^d$ fest:
	\begin{eqnarray}
		\int e^{2i\pi x\xi}(\F\phi)(\xi)d\xi &=& \int \left(\int e^{-2i\pi \xi y}\phi(y) dy\right) e^{2i\pi x\xi} \psi(\xi)d\xi\nonumber\\
		&\overset{\text{Fubini}}{=}& \int\left(\int e^{2i\pi (x-y)\xi}\psi(\xi)d\xi\right)\phi(y)dy\nonumber\\
		&=& \int\F\psi(x-y)\phi(y)dy \nonumber\\
		&=& \int \F\psi(y)\phi(x+y)dy\nonumber
	\end{eqnarray}
	Wähle $\psi(x) = e^{-\pi\epsilon^2|x|^2}$.
	\begin{eqnarray}
		\int e^{2i\pi x\xi}(\F\phi)(\xi)e^{-\pi\epsilon^2|\xi|^2}d\xi &=& \int \epsilon^{-d} e^{-pi|y|^2}\phi(y+x)dy\nonumber\\
		&\overset{y' = y/\epsilon}{=}& \int e^{2i\pi x\xi}\F\phi (\xi) e^{-\pi\epsilon^2|\xi|^2}d\xi \nonumber \\
		&=& \int e^{-\pi |y|^2}\phi (x+\epsilon y)dy\nonumber
	\end{eqnarray}
	Für $\epsilon>> 0$ folgt mit Lebesgue Konvergenz und
	\begin{eqnarray}
		\phi(x+\epsilon y)&\overset{\epsilon\rightarrow 0}{\longrightarrow}&\phi(x),~|\phi(x)|\leq C\nonumber\\
		e^{-\pi\epsilon^2|\xi|^2}&\overset{\epsilon\rightarrow 0}{\longrightarrow}& 1, ~|e^{-\pi\epsilon^2|\xi|^2}\leq 1\nonumber
	\end{eqnarray}
	Für $\epsilon\rightarrow 0$:
	\begin{eqnarray}
		\int e^{2i\pi x\xi}\F\phi(\xi)d\xi &=& \left(\int e^{-\pi|y|^2}dy \right)\phi(x)\nonumber\\
		\G f(x)&:=& \int e^{2\i\pi x\xi}f(\xi)d\xi\nonumber
	\end{eqnarray}
	Also $\G(\F) = \text{Id}$ und $\F(\G) = \text{Id}$ $\Rightarrow$ $\F^{-1} = \G$.\qed
	
	\paragraph{Folgerung:} $\F^{-1}\phi(x) = \F\phi(-x)$.
	
	\paragraph{Bemerkung} Vergleich zu Fourierreihen: 
	$$f(x)=\sum_{n\in \Z} e^{2i\pi n x}\f(n),~~ \f(n) = \int_{0}^{2\pi}e^{-2i\pi n x}f(x) dx.$$
	
	\begin{Prop}
		Seien $f, g\in S$. Dann sind $f\m g$ und $f*g$ wieder $\in S$.
	\end{Prop}	
	\Bew Zeige $f(x)g(x) \in S$, benutze Produktformel.\\
	$D^\alpha(f*g)(x) = \int (D_\alpha f)(x-y)g(y) dy$.
	\begin{eqnarray}
		(1+|x|)^n D^{\alpha}(f*g)(x) &=& \int(1+|x-y|)^n D^\alpha f(x-y) (1+|y|)^ g(y) \frac{(1+|x|)^n}{(1+|x-y|)^n (1+|y|)^n}dx \nonumber\\
		&\leq& C\int(1+|y|)^{-m}(1+|y|)^{n+m}g(y) dy\nonumber
	\end{eqnarray}
		
	\paragraph{Zusammenfassung:} 
	$$\xymatrix{
		S\ar[d]_\F \ar[r]^{D^\alpha} & S\ar[d]^\F\\
		S\ar[r]_{M_\alpha} & S
		}$$
	Mit $M_\alpha f(x) = (2i\pi x)^{\alpha} f(x)$.
	$$\xymatrix{
		S\ar[r]^{T_g}\ar[d]_\F & S\ar[d]^\F\\
		S\ar[r]_{M_g} & S
		}$$
	Mit $T_g f = f*g$, $M_g f= \g f$.
	
	\begin{Bemerkung}[Zur Lösung von Differentialgleichungen (formale Rechnung)]~
		\\
		$(*)$ $(I-D^{\alpha})f = g$, $g$ gegeben, $f$ gesucht.
		\begin{eqnarray}
			[1-(2i\pi x)^{\alpha}]\f(x) = \g(x) &\Rightarrow& \f(x) = [1 - (2i\pi x)^{\alpha}]^{-1}\g(x)\nonumber 
		\end{eqnarray}
		$m(x) = 1-(2i\pi x)^{\alpha}$. Falls $m(x)\neq 0$.
		\begin{eqnarray}
			f = [\F^{-1}[(1-2i\pi x)^{\alpha}]\g(x)]\nonumber
		\end{eqnarray}
		Angenommen $\exists k\in L^1$ mit $\hat{k}(x) = \frac{1}{m(x)}$.
		\begin{eqnarray}
		\Rightarrow f &=& [\F^{-1}[(1-2i\pi x)^{\alpha}]\g(x)]\nonumber\\
		 &=& \F^{-1}[\hat{k}\m \g]\nonumber\\
		&=& \F^{-1}\hat{k}*\F^{-1} \g = k*g\nonumber
		\end{eqnarray}
		Die Lösung von $(*)$ ist oft durch einen Faltungsoperator gegeben.
	\end{Bemerkung}
	
	\subsubsection{Fouriertransformation auf $L^2(\R^d)$}
	
	\begin{Lemma}
		Für $f, g\in S(\R^d)$ gilt mit $<f,g> = \int_{\R^{d}} f(x) \overline{g(x)}dx$.
		\begin{enumerate}
			\item[a)] $<\F f,\F g> = <f,g>$, $\F$ unitär.
			\item[b)] $||\F f||_{L^2} = ||f||_{L^2}$, $\F$ Isometrie.
		\end{enumerate}
	\end{Lemma}
	\Bew a) $\F^{-1}\F f = f$. Daraus folgt:
	\begin{eqnarray}
		<f,g> &=&\int f(x)\overline{g(x)}dx\nonumber\\
		&=& \int\left(\int e^{2i\pi x y}\F f(y)dy \right)\overline{g(x)}dx \nonumber \\
		&\overset{\text{Fubini}}{=}& \int \F(y)\left(\int e^{2i\pi xy} \overline{g(x)}dx\right)dy\nonumber\\
		&=& \int \F f(y) \int e^{-2i\pi xy}g(x)dx dy\nonumber\\
		&=& <\F f, \F g>\nonumber
	\end{eqnarray}
	b) $<f,f>$ liefert Behauptung.
	\qed
	
	\begin{Definition}[der Fouriertransformation auf $L^2(\R^d)$]
		Da $S(\R^d)$ dicht in $L^{2}(\R^d)$ liegt, gibt es zu jedem $f\in L^{2}(\R^d)$ eine Folge $(f_n)\subset S(\R^d)$ mit $||f-f_n||_{L^2}\rightarrow 0$.
		
		Setze $\F f =\lim\limits_{n\rightarrow\infty}\F(f_n)$ (in $L^2$), d.h. $\F$ ist die stetige Fortsetzung von 
		$$\F:S(\R^d)\rightarrow L^2(\R^d) \text{ auf } L^2(\R^d),$$	
		denn $||\F f||_{L^2} = ||f||^{L^2}$ für alle $f\in S(\R^d)$.
	\end{Definition}
	Achtung: Für $f\in L^2(\R^d)$ ist $\int e^{-2i\pi xy}f(y)dy$ nicht immer für fast alle $x$ als Lebesgue-Integral definiert.
	
	\begin{Korollar}
		Für $f, g\in L^2(\R^d)$ gilt:
		\begin{enumerate}
			\item[a)] $<\F f,\F g> = <f,g>$.
			\item[b)] $||\F f||_{L^2} = ||f||_{L^2}$.
		\end{enumerate}
	\end{Korollar}
	\Bew $f_n,g_n\in S^n$ mit $f_n\rightarrow f$, $g_n\rightarrow g$, $f,g\in L^2$. 3.13 und 3.14 liefern Behauptung.
	\qed
	
	\begin{Korollar}
		Sei $g\in L^{1}(\R^d)$, $T_g f= g*f$. Dann ist 
		$$||T_g||_{L^2\rightarrow L^2} = \sup_{x\in \R^d}|\g(x)|.$$
	\end{Korollar}
	\Bew $||f*g||_{L^2} = ||\F[f*g]||_{L^2} = ||\f\m\g||_{L^2}$.  Damit folgt
	$$||\f\m\g||_{L^2} = \left(\int |f(x)\bar{\g(x)}|^2 dx \right)^{1/2}\leq \sup_{x\in \R^d}|\g(x)|\m||\f||_{L^2} = \sup|\g(x)|\m ||f||_{L^2}.$$
	Somit $||T_g||\leq \sup|\g(x)|$.
	\qed 
	
	\paragraph{Bemerkung:} $||T_g||_{L^2\rightarrow L^2}\leq||g||_{L^1}$ (Youngsche Ungleichung), denn $\sup|\g(x)| \leq ||g||_{L^1}$.
	
	\section{Temperierte Distributionen und die Fouriertransformation}
	
	Idee $S'(\R^d)$ soll der Dualraum von $S(\R^d)$ sein. 
	\paragraph{Wiederholung:} Für $N\in \N$ setze $||f||_N = \sup\{|x^\alpha D^\beta f(x)|: x\in \R^d, \alpha,\beta\in \N_0^d, |\alpha|,|\beta|\leq N \}$.
	
	\begin{Definition}
		Für $f_n,f\in S(R^d)$ gilt $f_n\overset{S}{\rightarrow}f$ $\Leftrightarrow$ $||f-f_n||_N\rightarrow 0$ $\forall N\in \N$.
	\end{Definition}
	
	\paragraph{Bemerkung:} $f_n\overset{S}{\rightarrow} f$ ist äquivalent zu $d(f_n,f)\rightarrow 0$ für die Metrik
	$$d(f,g)=\sum_{N\in \N}2^{-N}\frac{||f-g||_N}{1+||f-g||_N}$$
	und $d(f_n,f)\rightarrow 0$ $\Leftrightarrow$ $||f-f_n|_N\rightarrow 0$ für alle $N$.
	
	\begin{Definition}
		$S'(\R^d) = \{u:S(\R^d)\rightarrow K, \text{ linear, stetig bezüglich } f_n\overset{S}{\rightarrow} f \}$. $S'(\R^d)$ ist der Dualraum von $S(\R^d)$. Mann nennt ihn auch den Raum der temperierten Distributionen.
	\end{Definition}
	
	\begin{Beispiel}
		~
		\begin{enumerate}
			\item[a)] Sei $h:\R^d\rightarrow \C$ lokal integrierbar und es gebe $n,C$, sodass $\int_{|x|\leq R}|h(x)|dx\leq C R^n$ für $R\rightarrow \infty$. Dann setze $$u_h(f) = \int f(x)h(x) dx.$$
			Zu zeigen: $u_h\in S'$. Insbesondere: $L^p(\R^d)\subset S'(\R^d)$ für alle $p$, denn 
			$$\int_{|x|\leq R}|f(x)|d<\leq R^{1/p}||f\chi_{|x|\leq R}||_{L^p}.$$
			\item[b)] Dirac Maß: $\delta_x\in S'(\R^d)$, $\delta_x(u) = u(x)$.
		\end{enumerate}
	\end{Beispiel}
	
	\begin{Satz}
		Sei $u:S(\R^d)\rightarrow \C$ linear, $u$ stetig, d.h. $u\in S'(\R^d)$ genau dann, wenn es ein $N$ gibt, sodass 
		$$|u(f)|\leq C||f||_N~~\forall f\in S(\R^d).$$
	\end{Satz}
	
	\Bew \glqq$\Leftarrow$\grqq\ klar. \glqq$\Rightarrow$\grqq: Andernfalls gibt es zu jedem $n\in \N$ ein $f_n\in S$ mit $||f_n||_n = 1$ aber $|u(f_n)|\geq n$. Setze $g_n = f_n/\sqrt{n}$. Dann gilt $||g_n||_N \leq n^{1/2}\overset{n\rightarrow\infty}{\rightarrow} 0$ für $n\geq N$.
	
	Aber $|u(g_n)|\geq \sqrt{n}\overset{n\rightarrow\infty}{\rightarrow}\infty$. Also Widerspruch zur Stetigkeit von $u$.
	
	
	\begin{Beispiel}
		~
		\begin{enumerate}
			\item[a)] \glqq$L^p(\R^d)\subset S'(\R^d)$\grqq. Sei $h:\R^d\rightarrow \C$ lokal integrierbar und
			$$(*)~~~\int_{|x|\leq R}|h(x)|dx\leq C R^n\text{ für } R> 1$$
			für ein festes $C<\infty$, $n\in \N$ fest. Dann wird durch 
			$$u_h(f)= \int_{\R^d} h(x)f(x)dx$$
			eine Distribution $u_h\in S'(\R^d)$ definiert, d.h. man erhält eine Einbettung von Funktionen $h$ mit $(*)$ nach $u_h\in S'(\R^d)$.
			\item[b)] Sei $\psi\in C^\infty(\R^d)$ und langsam wachsend, d.h. für alle $\alpha\in \N$ gibt es $N_\alpha, C_\alpha\leq \infty$ mit 
			$$|D^\alpha\psi(x)|\leq C_\alpha(1+|x|)^{N_\alpha},~~ x\in \R^d.$$
			Dann kann man für jedes $u\in S'(\R^d)$ ein Produkt $\psi\m u\in S'(\R^d)$ definieren durch $(\psi\m u)(f) = u(\psi f)$ für $f\in S(\R^d)$.
			\item[c)] Dirac Distribution: $\delta_x\in S'(\R^d)$. $\delta_x(f) = f(x)$ für $f\in S(\R^n)$.
			\item[d)] $h(x) = e^{|x|^2}$. Dann $u_h\notin S'(\R^d)$, den $f(x) = e^{-|x|^2}\in S(\R^d)$
			$$u_h(f)=\int h(x) g(x) dx = \int 1 dx = \infty.$$
		\end{enumerate}
	\end{Beispiel}
	\Bew a) Benutze 4.4. 
	\begin{eqnarray}
		|u_h(f)| &=& \int h(x)f(x)dx\nonumber\\
		&=& \int h(x)(1+|x|^2)^{-2-n}(1+|x|^2)^{2+n}f(x) dx \nonumber\\
		&\leq& \int |h(x)|(1+|x|^2)^{-2-n}dx\m \sup_{x\in \R^d}(1+|x|^2)^{2+n}|f(x)|\nonumber\\
		&\leq& C||f||_N\nonumber
	\end{eqnarray}
	Für $h\in L^p(\R^d)$ gilt mit Hölder $\int_{|x|\leq R}|h(x)|dx\leq \left(\int_{|x|\leq R}|h(x)|^p dx \right)^{1/p} R^{d/p'}$. Hier ist $(*)$ erfüllt mit $n>d/p'$.\\
	b) z.B. $\psi f\in S(\R^d)$.
	\qed
	
	\begin{Definition}[Prinzip der Dualität]
		Sei $T:S(\R^d)\rightarrow S(\R^d)$ linear und stetig. Dann definiere die \textbf{duale Abbildung} $T':S'(\R^d)\rightarrow S'(\R^d)$ durch $(U\in S'(\R^d), f\in S(\R^d))$
		$$(T'u)(f) = u(T(f)).$$
	\end{Definition}
	
	
	\paragraph{Bemerkung.} $f_n\overset{S}{\rightarrow} f$ $\Rightarrow$ $Tf_n\overset{S}{\rightarrow} Tf$. $u(Tf_n) \rightarrow u(Tf)$, $T'u(f_n)\rightarrow T'u(f)$, da $T$ und $u$ stetig und linear nach Definition.
	
	\begin{Definition}
		$\F:S(\R^d)\rightarrow S(\R^d)$ $\Rightarrow$ $\F':S'(\R^d)\rightarrow S'(\R^d)$ mit 
		$$(\F'u)(f) = u(\f),~~ f\in S,~ u\in S'.$$
	\end{Definition}
	
	\paragraph{Bemerkung.} $h\in L^1(\R^d)\rightarrow u_h\in S'(\R^d)$. Für $f\in S(\R^d)$ gilt 
	$$(\F'u_h)(f) = u_h(\f) = \int h(x) \f(x) dx \overset{\text{Sec. 3}}{=}\int \hat{h}(x)f(x) dx = u_{\hat h (x)}(f),~\forall f\in S.$$
	Also $\F'u_h = u_{\F h}$.
	
	Notation: $\F'\hat{=}\F$. Dann $\boxed{\F(u_h)=u_{\F h}$, $\hat{u}_h = u_{\hat h}.}$
	
	\begin{Prop}
		$\F: S'(\R^d)\rightarrow S'(\R^d)$ ist bijektiv.
		$$(\F^{-1}u)(f) = (\F u)(\tilde{f}), \text{ wobei } \tilde{f}(x) = f(-x).$$
	\end{Prop}
	
	\Bew $\F\m\F^{-1} = \id_S$. Dualität: 
	$$(\F^{-1})'\F' = (\F\F^{-1})' = \id_S'= \id_{S'}.$$
	$\Rightarrow$ $(\F')^{-1} = (\F^{-1})'$ und damit bijektiv. Außerdem 
	$$(F^{-1}u)(f) = u(\F^{-1}f) = u(\F f(-\bullet)) =  \F'u(f(-\bullet)) = \F'u(\tilde{f}).$$
	\qed
	
	
	\begin{Definition}
		Sei $u\in S'(\R^d)$, $\alpha\in \N_0^d$. Definiere $D^\alpha u$ durch
		$$(D^\alpha u)(f) = (-1)^{|\alpha|}u(D^\alpha f)\text{ und } D^\alpha u\in S'(\R^d).$$
	\end{Definition}
	
	\paragraph{Bemerkung:} ~
	\begin{enumerate}
		\item[a)] $D^\alpha: S(\R^d)\rightarrow S(\R^d)$. 
		$$D^\alpha u := (-1)^{|\alpha|}(D_S^\alpha)'u \text{ (im Sinne der Dualität)}$$
		denn $(D_{S'}^\alpha u)(f) = (-1)^{|\alpha|} u(D^\alpha f)$.
		\item[b)] Sei $h\in S(\R^d)$ und $u_h\in S'(\R^d)$. 
		$$D^\alpha(u_h)(f) = (-1)^{|\alpha|} u_h(D^\alpha f) = (-1)^{|\alpha|}\int h(x) D^\alpha f(x) dx \overset{\text{p.I.}}{=} \int(D^\alpha h)(x) f(x) dx.$$
		Also: $\boxed{D^\alpha (u_h) = u_{D^\alpha h}.}$
	\end{enumerate}
	
	\begin{Prop}
		Sei $h\in C^{\infty}(\R^d)$ eine langsam wachsende Funktion, $u\in S'^{\R^d}$. Dann gilt
		$$D_{x_i}(h\m u) = (D_{x_i}h) \m u + h D_{x_i}u$$
		sowie für $\alpha,\beta\in \N_0^d$
		$$D^{\alpha+\beta}u = D^\alpha(D^\beta u).$$
	\end{Prop}
	
	\Bew Übung.
	\qed
	
	\begin{Beispiel}
		~
		\begin{enumerate}
			\item[a)] Sei $d= 1$, $H(x) =\left\{\begin{array}{l}
			1, ~x\geq 0\\
			0, ~x< 0
			\end{array}
			\right.$. Behauptung: $D(u_H) = \delta_0$, denn für $\phi\in S(\R)$ gilt
			\begin{eqnarray}
				(Du_h)(\phi) = -u_h(D\phi) = -\int H(x)\phi'(x)dx = \int_0^\infty \phi'(x) dx = \phi(0)\nonumber
			\end{eqnarray}
			Außerdem: für $\delta_a$
			$$(D^\alpha\delta_a)(\phi) = (-1)^{|\alpha|}(D^\alpha\phi) = (-1)^{|\alpha|}D^\alpha\phi(0).$$
		\end{enumerate}
	\end{Beispiel}
	
	\paragraph{Vorschau:} Zu $u\in S'(\R^d)$ gibt es eine langsam wachsende, stetige Funktion $\psi:\R^d\rightarrow\C$ und $\alpha\in \N_0^d$, sodass $u= D^\alpha\psi$.
	
	\begin{Prop}
		Für $u\in S'(\R^d)$ und $\alpha\in \N_0^d$ gilt:
		\begin{enumerate}
			\item[a)] $\F(D^\alpha u) = (2i\pi \m)^\alpha \F u$.
			\item[b)] $D^\alpha (\F u) = \F((-2i\pi \m)^\alpha u)$.
		\end{enumerate}
	\end{Prop}
	\Bew a) Für $f\in S(\R^d)$ erhält man nach Kapitel 3
	\begin{eqnarray}
		[\F(D^\alpha u)](f)&\overset{\text{Def.}}{=}& (D^\alpha u)(\F f)\nonumber\\
		&\overset{\text{Def.}}{=} & (-1)^{|\alpha|} u(D^\alpha \F f)\nonumber\\
		&=& (-1)^{|\alpha|} u (\F((-2i\pi\m)^\alpha f)) \nonumber\\
		&\overset{\text{Def.}}{=}& (-1)^{|\alpha|}(\F u)((-2i\pi\m)^\alpha f)\nonumber\\
		&=& [(2i\pi)^\alpha \F u](f)\nonumber
	\end{eqnarray}
	b) analog.
	\qed
	
	\paragraph{Erinnerung:} Für $f\in S(\R^d)$, $y\in \R^d$, $a>0$ gilt
	\begin{eqnarray}
		(\tau^Y f)(x) &:=& f(x-y)\nonumber\\
		(\delta^a f)(x)&:=& f(ax)\nonumber\\
		\tilde{f}(x) &:= & f(-x)\nonumber
	\end{eqnarray}
	Dann folgt mit einfacher Substitution
	\begin{eqnarray}
		u_{\tau^y g}(f) &=& \int_{\R^d} \tau^Y(g)f dx \nonumber\\
		&=& \int_{\R^d}g(x-y) f(x) dx \nonumber\\
		&\overset{\text{Subst.}}{=}& \int_{\R^d} g(x)f(x+y)dx = u_g(\tau^{-y}f)\nonumber\\\nonumber \\
		u_{\delta^a g}(f) &=& \int_{\R^d}g(ax)f(x)dx \nonumber\\
		&\overset{\text{Subst.}}{=}& a^{-d}\int_{\R^d} g(x)f(\frac{1}{a}x)dx \nonumber\\
		&=& a^{-d}u_g(\delta^{1/a}f)\nonumber\\\nonumber\\
		u_{\tilde{g}}(f) &=& \int_{\R^d}g(x)f(x)dx \nonumber\\
		&=& \int_{\R^d}g(x)f(-x) dx\nonumber\\
		&=& u_g(\tilde{f})\nonumber
	\end{eqnarray}
	
	\begin{Prop}
		Es gelten für $u\in S'(\R^d)$, $y\in \R^d$, $a>0$ die Regeln:
		\begin{enumerate}
			\item[a)] $\F(\tau^y u) = e^{-2i\pi y x} \hat{u}$.
			\item[b)] $\tau^y \hat u =\F(e^{2i\pi x y} u)$.
			\item[c)] $\F(\delta^a u) = a^{-d}\delta^{1/a} \hat u$.
			\item[d)] $\F(\tilde{u}) = \widetilde{\F(u)}$.
		\end{enumerate}
	\end{Prop}
	
	\Bew a) Für $f\in S(\R^d)$ gilt
	\begin{eqnarray}
		[F(\tau^y u)](f) &\overset{\text{Def}}{=}& (\tau^y u)(\f)\nonumber\\
		&\overset{\text{Def}}{=}& u(\tau^{-y}\f)\nonumber\\
		&\overset{\text{Def.}}{=}& u(\F(e^{-2i\pi xy}f))\nonumber\\
		&\overset{\text{Def.}}{=}& \hat u (e^{-2i\pi xy}f)\nonumber\\
		&\overset{\text{Def.}}{=}& [e^{-2i\pi xy}\hat u](f)\nonumber
	\end{eqnarray}
	b), c), d) analog
	\qed
	
	\begin{Satz}
		Zu $u\in S'(\R^d)$ existiert eine langsam wachsende Funktion $\varphi:\R^d\rightarrow \C$ und $\gamma\in \N_0^d$, sodass
		\begin{eqnarray}
			u &=& D^\gamma\varphi (= D^\gamma u_\varphi)\nonumber\\
			( u(f) &=& (-1)^{|\gamma|}\int_{\R^d}\varphi(x) D^{\gamma} f(x)dx ) \nonumber
		\end{eqnarray}
	\end{Satz}
	
	\Bew Idee: Verwende Darstellungssatz von Riesz für $L^1$.
	\begin{enumerate}
		\item Für $N\in \N$ definiere $S_N:=(S(\R^d),||.||_N)$, wobei
		$$||f||_N = \sup_{\lambda\in \R^d}\sup_{|\alpha|,|\beta|\leq N}|x^\beta D^\alpha f(x)|$$
		und sei $S'_N$ der Dualraum von $S_N$. Definiert man 
		$$I_N:= \{\alpha\in \N_0^d: |\alpha|\leq N \}$$ 
		und $M:= \# I_N$ zu $f\in S(\R^d)$ und außerdem 
		$$\psi_f:\R^d\rightarrow \R^d,~~\psi_{f,\alpha}(x)= (1+|x|)^N D^{\alpha+1}f(x),~~\alpha \in I_N$$
		mit $1=(1,...,1)\in \N_0^d$. Dann ist $\psi_{f,\alpha}\in L^1(\R^d)$ für alle $\alpha\in I_N$, d.h. $\psi_f\in L^1(\R^d)^M$. 
		Außerdem ist $\psi_f$ durch $f$ eindeutig bestimmt.
		\item Sei nun $u\in S'(\R^d)$. Nach Satz 4.3 ist dann $u\in S'_N$ für ein $N\in \N$. Definiere außerdem $L(\psi_f):= u(f)$, $f\in S(\R^d)$. Nach 1. ist dann $L$ wohldefiniert.
		
		Weiter $|L(\psi_f)| = |u(f)|\leq C||f||_N = C\sup_x \sup_{|\alpha|,|\beta|\leq N}|x^\beta D^\alpha f(x)$
		\begin{eqnarray}
			|L(\psi_f)| &=& |u(f)|\leq C||f||_N\nonumber\\
			&=& C\sup_x \sup_{|\alpha|,|\beta|\leq N}|x^\beta D^\alpha f(x)\nonumber\\
			&\leq & C\sum_{|\alpha|\leq N} \sup_x(1+|x|)^N|D^\alpha f(x)|\nonumber\\
			&\overset{x_0 = x_{0,\alpha}}{=}& C\sum_{|\alpha|\leq N} (1+|x_{0,\alpha})^N|D^\alpha f(x_{0,\alpha}|)\nonumber\\
			&=& C \sum_{|\alpha|\leq N} (1+|x_0|)^N \left( \int_{-\infty}^{x_{0,d}}...\int_{-\infty}^{x_{0,k+1}}...\int_{x_{0,k}}^{\infty}... D^{\alpha+1}(f(y_1,...,y_d))dx  \right)\nonumber\\
			&\leq& C||\psi_f||_{L^1(\R^d)^M}\nonumber
		\end{eqnarray}
		Definiere nun $\mathcal{L}^M:=\{\psi_f:\R^d\rightarrow\R^M: f\in S(\R^d) \}\subseteq L^1(\R^d)^M$. Wegen $\alpha\psi_f+\psi_g = \psi_{\alpha f+g}$ und der Linearität von $U$ ist $L$ linear auf $\mathcal{L}^M$ und nach obigem beschränkt auf $(\mathcal{L}^M,||.||_{L^1(\R^d)^M})$, d.h. $L\in (\mathcal{L}^{M})'$. Nach Hahn-Banach existiert nun eine Erweiterung $\tilde{L}\in (L^1(\R^d)^M)'$ und nach Darstellungssatz von Riesz gilt
		$$(L^1(\R^d)^M)'\cong (L^1(\R^d)')^M\cong L^\infty(\R^d)^M.$$
		Damit gilt 
		\begin{eqnarray}
			u(f) = \tilde{L}(\psi_f) &=& \sum_{|\alpha|\leq N}\int_{\R^d}\psi_{f,\alpha}g_\alpha dx\nonumber\\
			&=&\sum_{|\alpha|\leq N}\int_{\R^d} (1+|x|)^N D^{\alpha + 1}f(x) g_\alpha(x) dx\nonumber
		\end{eqnarray}
		mit $g_\alpha\in L^\infty(\R^d)$ für $\alpha\in I_N$.
		Mit mehrfacher partieller Integration kann man dies weiter vereinfachen zu
		$$u(f) = \int_{\R^d} \tilde{\varphi}D^{\tilde{\gamma}}f dx$$
		mit $\tilde{\gamma} = (N+1,...,N+1)$ und $\tilde{\varphi}=$ Linearkombinationen aus $(1+|x|)^Ng_\alpha$ und deren Stammfunktionen. Insbesondere ist $\tilde{\varphi}$ langsam wachsende Funktion. Nochmals partielles Integrieren liefert dann ein stetiges $\varphi$ und $\gamma\in \N_0^d$ mit gewünschter Eigenschaft.
	\end{enumerate}
	\qed
	
	\section{Faltung und Distributionen}
	
	Wir begnügen uns damit Distributionen $u$ mit Schwarzfunktionen $f$ zu falten. Die Definition wird motiviert durch
	$$(f*g)(x)=\int_{\R^d}f(x-y)g(y) dy = u_g(f(x-\m)).$$
	
	\begin{Definition}
		Sei $u\in S'(\R^d)$ und $f\in S(\R^d)$. Dann definiere
		$$(f*u)(x):= u(f(x-\m)),~ x\in \R^d.$$
	\end{Definition}
	
	\begin{Satz}
		Sei $u\in S'(\R^d)$, $f\in S(\R^d)$. Dann ist die Funktion $x\mapsto (f*u)(x)$ eine langsam wachsende $C^\infty-$Funktion, mit 
		$$D(f*u) = (D^\alpha f)*u = f*(D^\alpha u).$$
	\end{Satz}
	
	\Bew Nach Kapitel 4 existiert zu $u\in S'(\R^d)$ ein $N\in \N$, $C<\infty$ mit 
	$$|u(f)|\leq C||f||_N.$$
	Damit folgt
	\begin{eqnarray}
		|(f*u)(x)| &=& |u(f(x-\m))|\nonumber\\
		&\leq& C\sup_y \sup_{|\alpha|,|\beta|\leq N}|y^\alpha D^\beta f(x-y)|\nonumber\\
		&=& C\sup_y\sup_{|\alpha|,|\beta|\leq N} |(x-y)^\alpha D^\beta f(y)|\nonumber\\
		&\leq& C\m \tilde{C}(1+|x|)^N, \nonumber
	\end{eqnarray}
	wobei $\tilde{C}\geq \sup_{|\alpha|,|\beta|\leq N}(1+|y|)^N|D^\beta f(u)|$.
	
	Zeige nun
	$$D_{x_i}(f*u) = (D_{x_i}f*u).$$
	Betrachte dazu den Differenzenquotienten
	\begin{eqnarray}
		\frac{1}{n}[(f*u)(x+h e_i)-(f*u)(x)] &\overset{\text{Def.}}{=}& u(\frac{1}{n}[f(x+he_i -\m)-f(x-\m)])\nonumber
	\end{eqnarray}
	Da jedes $f\in S(\R^d)$ gleichmäßig stetig (und alle Ableitungen ebenso) ist, folgt für festes $x\in \R^d$ und $N\in \N$.
	$$||[\frac{1}{n}f(x+he_i-\m)-f(x-\m)]-D_{x_i}f(x-\m)||_N\rightarrow 0\text{ für } h\rightarrow 0.$$
	Mit der Stetigkeit von $u$ folgt nun
	\begin{eqnarray}
		D_{x_i}(f*u)(x) &=& u(D_{x_i}f(x-\m))\nonumber\\
		&=& ((D_{x_i}f)*u)(x)\nonumber
	\end{eqnarray}
	Wiederholung des Arguments liefert
	$$D^\alpha (f*u) = (D^\alpha f)*u).$$
	Die letzte Behauptung folgt aus 
	\begin{eqnarray}
		(D^\alpha f*u)(x) &=& u((D^\alpha f)(x-\m))\nonumber\\
		&=& u((-1)^{|\alpha|}D^\alpha (f(x-\m))\nonumber\\
		&=& (D^\alpha u)(f(x-\m)) = (f*D^\alpha u)(x)\nonumber
	\end{eqnarray}
	\qed
	
	Für alternative Darstellungen benötigen wir das folgende
	
	\begin{Lemma}
		Seien $u\in S'(\R^d),~f,g\in S(\R^d)$. Dann gilt 
		$$u\left(\int_{\R^d}f(u-\m)g(y) dy \right) = \int_{\R^d}u(f(y-\m))g(y) dy.$$
	\end{Lemma}
	
	\Bew 
	\begin{enumerate}
		\item Für jedes $N\in \N$ sei $(Q_m)_{m= 1}^{(2N^2)^d}$ die Zerlegung von $[-N,N]^d$ in Würfel der Seitenlänge $1/N$ mit Mittelpunkt $y_m$. Zeige nun, dass die Riemannsumme $R_N(x) := \sum_{m=1}^{(2N^2)^d}f(y_m-x)g(y_m)|Q_m|$ in $S(\R^d)$ gegen $\int_{\R^d}f(y-x)g(y)dy$ konvergiert, d.h.
		$$||R_N-\int_{\R^d}f(y-\m)g(y) dy||_{x,\beta} = \sup_{\lambda\in \R^d}|\sum_{m= 1}^{(2N^2)^d}x^\alpha D^\beta(f(y_m-x))g(y_m)|Q_m| - \int_{\R^d}x^\alpha D_x^\beta (f(y-x))g(y)dy|\overset{N\rightarrow\infty}{\rightarrow} 0.$$
		Es gilt zum einen: 
		\begin{eqnarray}
			x^\alpha (-1)^{|\beta|}(D^\beta f)(y_m-x)g(y_m)|Q_m|&-&(-1)^{|\beta|}\int_{\Q_m} x^\alpha(D^\beta f)(y-x)g(y) dy\nonumber\\
			&=& (-1)^{|\beta|}\int_{Q_m}x^\alpha ((D^\beta f)(u_m-x)g(y_m)-(D^\beta f) (y-x)g(y)) dy \nonumber\\
			&=& (-1)^{|\beta|}\int_{Q_m} x^\alpha(y_m-y)[\nabla_y (D^\beta f(\m-x)g)](\xi) dy = (*)\nonumber
		\end{eqnarray}
		 für $\xi = y+\theta(y_m-y)$, $\theta \in [0,1]$. Wegen $|y|\leq |\xi|+\theta|y_m-y|\leq |\xi|+\frac{\sqrt{d}}{N}\leq |\xi|+1$ für $N>\sqrt{d}$, folgt
		 \begin{eqnarray}
		 	|(*)| &\leq& c_1\frac{|x|^{|\alpha|}}{(1+|x|)^M}\frac{\sqrt{d}}{N}\int_{Q_m} \frac{1}{(1+|y|)^M}dy\nonumber
		 \end{eqnarray}
		 Damit erhält man 
		 \begin{eqnarray}
		 	|D_N(x)|&\leq& c_1\frac{|x|^{|\alpha|}}{(1+|x|)^M}\frac{\sqrt{d}}{N} \int_{|y|\leq N}\frac{1}{(1+|y|)^M}dy+ \int_{|y|> N} |x^{|\alpha|}D^\beta f(y-x)g(y)dy \nonumber\\
		 	&\leq& c_1 \frac{|x|^{|\alpha|}}{(1+|x|)^M}\frac{\sqrt{d}}{N}\int_{|y|_\infty\leq N} \frac{1}{(1+|y|)^M}dy + c_2 \frac{|x|^{|\alpha|}}{(1+|x|)^M} \int_{|y|_\infty> N} \frac{1}{(1+|y|)^M} dy\nonumber\\
		 	&\rightarrow& 0 \text{ für } N\rightarrow 0 \text{ (unabhängig von x)}\nonumber
		 \end{eqnarray}
		 \item Mit 1. erhält man schließlich 
		 \begin{eqnarray}
		 	u\left(\int_{\R^d} f(y-\m)g(y) dy \right) &=& \lim\limits_{N\rightarrow \infty} u(R_N)\nonumber\\
		 	&=& \lim\limits_{N\rightarrow \infty}\sum_{m = 1}^{(2N^2)^d}u(f(y_m-\m))g(y_m)|Q_M|\nonumber\\
		 	&=& \int_{\R^d} u(f(y-\m))g(y)dy\nonumber
		 \end{eqnarray}
		 da $y\mapsto u(f(y-\m))g(y)\in S(\R^d)$ und damit Riemann-integrierbar
	\end{enumerate}
	\qed
	
	Damit erhält man
	\begin{Prop}
		Für $u\in S'(\R^d)$, $f\in S(\R^d)$ gilt $$(f*u)(g) = u(\tilde{f}*g) ~\forall g\in S(\R^d)$$
		($f*u$ als Distribution aufgefasst).
	\end{Prop}
	
	\Bew Nach Kapitel 3 gilt $\tilde{f}*g\in S(\R^d)$, d.h. die rechte Seite ist wohldefiniert. Außerdem gilt
	\begin{eqnarray}
		u(\tilde{f}*g) &=& u\left(\int_{\R^d}f(y-\m)g(y) dy \right)\nonumber\\
		&\overset{\text{5.3}}{=} & \int_{\R^d} u(f(y-\m))g(y)dy\nonumber\\
		&=& \int_{\R^d}(f*u)(y) g(y) dy\nonumber\\
		&=& (f*u)(g)\nonumber.
	\end{eqnarray}
	\qed
	
	\begin{Prop}
		Für $u\in S'(\R^d)$, $f,g\in S(\R^d)$ gilt 
		$$f*(g*u)= (f*g)*u.$$
	\end{Prop}
	
	\Bew 
	\begin{eqnarray}
		\left[(f*g)*u \right](x) &=& u((f*g)(x-\m)\nonumber\\
		&=& u\left(\int_{\R^d} g(x-y-\m)f(y) dy \right) \nonumber\\
		&\overset{\text{5.3}}{=}& \int_{\R^d} u(g(x-y-\m))f(y) dy\nonumber\\
		&=& \int_{\R^d} (g*u)(x-y)f(y)dy\nonumber\\
		&=& [f*(g*u)](x)\nonumber.
	\end{eqnarray}
	\qed
	
	\begin{Prop}
		Für $u\in S'(\R^d)$, $f\in S(\R^d)$ gilt
		\begin{enumerate}
			\item[a)] $\F(f*u) = \f\m \hat{u}$.
			\item[b)] $\F(f\m u) = \f *\hat{u}$.
		\end{enumerate}
	\end{Prop}
	
	\Bew a) Für $g\in S(\R^d)$ gilt
	\begin{eqnarray}
		\F(f*u)(g) &=& (f*u)(\g) \overset{\text{5.4}}{=} u(\tilde{f}*\tilde{g})\nonumber\\
		(\f\m \hat{u})(g) &=& \hat{u}(\f\m g) = u(\F(\f\m g)) = u(\F(\f)*\g) = u(\tilde{f}* g). \nonumber
	\end{eqnarray}
	b) analog.
	\qed
	
	So kommt man schließlich zu einem Dichtheitsresultat:
	
	\begin{Satz}
		Zu jedem $u\in S'(\R^d)$ existiert eine Folge $(f_k)_{k\in \N}\subseteq C_c^\infty(\R^d)$ mit $f_k\rightarrow u$ \glqq im distributiven Sinne\grqq, d.h.
		$$u_{f_k}(g)\overset{k\rightarrow \infty}{\rightarrow}u(g) ~\forall g\in S(\R^d).$$
	\end{Satz}
	
	\Bew Sei $\varphi\in C_c^\infty(\R^d)$ mit $\varphi(x) = 1$ auf $B(0,R)$ für ein $R>0$ und setze $\varphi_k(x) = \varphi(\frac{1}{k}x)$. Zeige zunächst zwei Hilfsbehauptungen. Für $f\in S(\R^d)$ gilt
	\begin{enumerate}
		\item $\lim\limits_{k\rightarrow \infty}\varphi_k f= f$ in $S(\R^d)$.
		\item $\lim\limits_{k\rightarrow\infty}\varphi_k\F(\varphi_k f) = \f$ in $S(\R^d)$.
	\end{enumerate}
	
	\Bew 1. Seien $\alpha,\beta\in \N_0^d$ beliebig. Dann gilt
	\begin{eqnarray}
		||\varphi_k f- f||_{\alpha,\beta} &=& \sup_{x\in \R^d}|x^\alpha D^\beta((\varphi_k(x)-1)f(x))| \nonumber\\
		&\overset{\text{Leibnitz}}{=}& \sup_{x\in \R^d} |x^\alpha \sum_{\gamma_1 = 0}^{\beta_1}...\sum_{\gamma_d = 0}^{\beta_d}\left(\begin{matrix}
		\beta_1\\
		\gamma_1
		\end{matrix}\right)\m...\m \left(\begin{matrix}
		\beta_d\\
		\gamma_d
		\end{matrix}\right) (D^\gamma(\varphi_k-1)) (D^{\beta\gamma}f)|\nonumber\\
		&\leq& \sup_{x\in \R^d}\left[|x^\alpha (\varphi_k (x)-1) D^\beta f(x)  + \sum_{\gamma\neq 0} \left(\begin{matrix}
		\beta\\
		\gamma
		\end{matrix}\right) \frac{1}{k^{|\gamma|}}|x^\alpha (D^\gamma\varphi)(\frac{x}{k}) D^{\beta-\gamma} f(x)|\right]\nonumber\\
		&\leq& \sup_{x\in \R^d}|x^\alpha(\varphi_k(x)- 1)D^\beta f(x)| +\frac{1}{k} c_\beta ||\varphi||_N ||f||_N
	\end{eqnarray} Nun ist $\varphi_k(x)-1 = 0$  für $|x|<kR$ und $|x^\alpha D^\beta f(x)| \leq c\frac{1}{|x|}\leq c\frac{1}{kR}$ für $|x|\leq kR$.
	Damit folgt
	$$||\varphi_k f-f||_{\alpha,\beta}\leq \frac{c}{kR}+\frac{1}{k}c_\beta||\varphi||_N ||f||_N\rightarrow 0 ~~(k\rightarrow 0).$$
	
	2. Seien $\alpha,\beta\in \N_0^d$ beliebig. Nach 1. und wegen der Stetigkeit von $\F: S\rightarrow S$ gilt
	$$\lim\limits_{k\rightarrow\infty}\F(\varphi_k f) = \f$$ 
	in $S(\R^d)$ für alle $f\in S(\R^d)$. Nach 1. und da $\F(\varphi_j f)\in S(\R^d)$ folgt dann auch 
	$$\lim\limits_{k\rightarrow\infty}\varphi_k \F(\varphi_j f) = \F(\varphi_j f)$$
	in $S(\R^d)$ für alle $j\in \N$. Wähle nun zu $\epsilon>0$ ein $j_\epsilon\in \N$ mit $||\F(\varphi_j f)-\f||_{\alpha,\beta} <\epsilon$ und $||\F(\varphi_k f)-\F(\varphi_j f)||_{\alpha,\beta}<\epsilon$ für alle $k,j\geq j_\epsilon$. Dann gilt für $j\geq j\epsilon$
	\begin{eqnarray}
		||\varphi_k\F(\varphi_k f)-\f||_{\alpha,\beta}&\leq & ||\varphi_k\F(\varphi_k f) - \varphi_k\F(\varphi_j f)||_{\alpha,\beta}\nonumber\\
		&+& ||\varphi_k\F(\varphi_j f) - \F(\varphi_j f)||_{\alpha,\beta}\nonumber\\
		&+& ||\F(\varphi_j f) - \f||_{\alpha,\beta}\nonumber\\
		&\rightarrow& 2\epsilon \text{ für } k\rightarrow\infty\nonumber
	\end{eqnarray} % TODO fehlt ein Teil
	\qed
	
	Sei nun $u\in S'(\R^d)$. Dann definiere 
	$$f_k: = \varphi_k(\hat{\varphi}* u)\in C_c^\infty(\R^d) \text{ nach Satz 5.2}.$$
	Damit folgt 
	\begin{eqnarray}
		u_{f_k}(g) &\overset{\text{5.6}}{=}& \varphi_k\m\F(\varphi_k\m\F^{-1}(u))(g) \nonumber\\
		&=& \F^{-1}(\varphi_k\F(\varphi_k g))\nonumber\\
		&\rightarrow& \F^{-1}u(\g) = u(g)~(k\rightarrow\infty)\nonumber
	\end{eqnarray}
	nach Behauptung 2.
	\qed

	\newpage
	
	\section{Sobolevräume}
	
	Sei $f\in L^2(\R^d)\rightarrow u_f\in S'$, $u_f(h) =\int f(x)g(x) dx$. Wann gilt $D^\alpha u_f\in S'$ $\Rightarrow$ $D^\alpha u_f \in L^2$?.
	
	\begin{Definition}
		Für $K\in \N$: 
		\begin{eqnarray}
			H^K(\R^d) =\{f\in L^2(\R^d): D^\alpha f\in L^2, \forall |\alpha|\leq K \}\nonumber\\
			\text{genauer } f\in H^\alpha\Leftrightarrow \forall|\alpha|\leq K\exists f_\alpha: U-f=D^\alpha u_f\nonumber
		\end{eqnarray}
		mit der Norm
		$$||f||_K:= \left(\sum_{|\alpha|\leq K}||D^\alpha f||_{L^2(\R^d)}^2 \right)^{1/2}.$$
	\end{Definition}
	
	\begin{Bemerkung}
		~\begin{enumerate}
			\item[a)] (schwache Ableitung) Zu $D^\alpha f\in L^2(\R^d)$ gibt es ein $f_\alpha\in L^2$, sodass für $h\in S(\R^d)$ gilt:
			\begin{eqnarray}
				\int f_\alpha(x) h(x) dx &=& D^\alpha(u_f)(h) \nonumber\\
				&=& (-1)^{|\alpha|}u_f(D^\alpha h)\nonumber\\
				&=& (-1)^{|\alpha|}\int f(x) (D^\alpha h)(x) dx.\nonumber
			\end{eqnarray}
			\item[b)] Die Räume $H^K$ sind vollständig.
			\item[c)] $H^K$ sind Hilberträume bezüglich 
			$$<f,g>_K = \sum_{|\alpha|\leq K} \int_{\R^d} (D^\alpha f)(x) \overline{D^\alpha g(x)}dx,~~ <f,f>_K = ||f||_K^2.$$
			\item[d)] Berechne $||f||_K$ mit Hilfe von $\F$.
			\begin{eqnarray}
				||D^\alpha f||_{L^2}^2 &=&||\F(D^\alpha f)||_{L^2}^2\nonumber\\
				&=& ||(2i\pi\xi)^\alpha\f(\xi)||_{L^2}^2 \nonumber\\ 
				&=& (2\pi)^{|\alpha|}\int_{\R^d}|\xi^\alpha||\f(\xi)^2 d\xi\nonumber\\
				\Rightarrow \sum_{|\alpha|\leq K}(2\pi)^{|\alpha|}\int |\xi^\alpha|^2 |\f(\xi)|^2d\xi &=& \int_{\R^d}\left(\sum_{|\alpha|\leq K} (2\pi)^{|\alpha|} |\xi^\alpha| \right)|\f(\xi)|^2d\xi\approx (1+|\xi|^2)^K\nonumber
			\end{eqnarray}
			denn:
			\begin{eqnarray}
				|x^\alpha|&\leq& 1+|x|^{2k}\nonumber\\
				(1+|x|^K)^2 &=& (1+(\sum |x_i|^2)^{K/2})^2\nonumber\\
				&\leq& (1+(\sum_{i = 1}^{d}|x_i|^K))^2\nonumber\\
				&\leq& C(\sum_{|\alpha|\leq K}|x^\alpha|^2)\nonumber
			\end{eqnarray}
			Also 
			$$||f||_K^2\cong \int_{\R^d}(1+|\xi|^2)^{K/2}|\f(\xi)|^2 d\xi,K\in \N. $$
		\end{enumerate}
	\end{Bemerkung}
	
	\begin{Definition}
		Für alle $s\in \R$ definiere den Sobolevraum
		\begin{eqnarray}
			H^s(\R^d) &=& \{f\in S'(\R^d): (1+|\xi|^2)^{s/2}\f(\xi)\in L^2(\R^d) \}\nonumber\\
			||f||_{H^s} &=& \left(\int_{\R^d}|\f(\xi)|^2(1+|\xi|^2)^sd\xi \right)^{1/2}\nonumber\\
			<f,g>_{H^s} &=& \int_{\R^d} \f(\xi)\overline{\g(\xi)}(1+|\xi|^2)^s d\xi\nonumber
		\end{eqnarray}
		für $s=K\in\N: ||f||_{H^s}\cong ||f||_{H^\alpha}$.
	\end{Definition}
	
	\begin{Prop}
		Die Räume $H^s(\R^d)$, $s\in \R$ sind Hilberträume, also insbesondere vollständig. $C_c^\infty(\R^d)$ ist dich in $H^s(\R^d)$.
	\end{Prop}
	
	\Bew $\F:S'(\R^d)\rightarrow S'(\R^d)$, $H^s\overset{\F|_{H^s}}{\leftrightarrow} L^2(\R^d, (1+|\xi|^2)^s)$, Isometrie von $H^s$ auf $L^2(\R^d, (1+|\xi|^2)^s)$, wobei $L^2(\R^d, (1+|\xi|^2)^s)$ vollständig ist. Somit ist auch $H^s$ vollständig und damit ein Hilbertraum.
	\begin{eqnarray}
		\F^{-1}: L^2(\R^d,(1+|\xi|^2)^s d\xi) &\overset{\F^{-1}}{\rightarrow}& H^s\nonumber\\
		S(\R^d)&\overset{\F^{-1}}{\rightarrow}& S(\R^d)\nonumber
	\end{eqnarray}
	Also liegt $S(\R^d)$ dicht in $H^s(\R^d)$. Elementar: $C_0^\infty(\R^d)$ in $S(\R^d)$.
	\qed
	
	\begin{Prop}
		$J:H^{-s}\rightarrow (H^s)'$, $J(u)(h) = \int \hat u(\xi) \hat h(\xi)d\xi$ für $u\in H^{-s}$, $h\in H^s$ definiert eine surjektive Isometrie von $H^{-s}$ auf $(H^s)'$ - den Banachraum dual von $H^s$ bezüglich der Dualität 
		$$(f,g) = \int \f(\xi)\g(\xi)d\xi \text{ (bilineare Abbildung)}.$$
	\end{Prop}
	
	\Bew Sei $u\in H^{-s}$, $h\in H^s$. 
	\begin{eqnarray}
		J(u)(h) &=& \int\hat u(\xi)\hat h(\xi)d\xi\nonumber\\
		&=& \int\hat u(\xi) (1+|\xi|^2)^{-s/2}\hat h(\xi)(1+|\xi|^2)^{s/2}d\xi\nonumber\\
		&\leq& \left(\int |\hat u(\xi)|^2 (1+|\xi|^2)^{-s} d\xi \right)^{1/2}\m \left(\int |\hat h(\xi)|^2 (1+|\xi|^2)^{s} d\xi \right)^{1/2}\nonumber\\
		&=& ||u||_{H^{-s}} ||h||_{H^s} \nonumber
	\end{eqnarray}
	Da $J$ stetig gilt $||J u||_{(H^s)'}\leq ||u||_{H^{-s}}$. Sei nun $u\in H^{-s}$, $||u||_{H^{-s}}= 1$ gegeben. Wähle $g = \F^{-1}[\hat u(\xi)(1+|\xi|^2)^{-s}]$. Dann gilt
	\begin{eqnarray}
		||g||_s &=& \left(\int|\hat u(1+|\xi|^2)^{-s/2}|^s d\xi \right)^{1/2}\nonumber\\
		&=& ||u||_{H^{-s}} = 1\nonumber\\
		(Ju)(g) &=& \int |\hat u(\xi)|^2(1+|\xi|^2)^{-s}d\xi\nonumber\\
		&=& ||u||_{H^{-s}} = 1\nonumber\\
		\Rightarrow ||Ju||_{(H^s)'} &=& 1\text{ für } ||u||_{H^{-s}} = 1\nonumber
	\end{eqnarray}
	Somit ist $J:H^{-s}\rightarrow (H^s)'$ eine isometrische Einbettung. 
	
	Zu zeigen bleibt, $J$ ist surjektiv. Zu $v\in(H^s)'$ gibt es ein $g\in H^s$ mit $v(h) = <h,g>_{H^s}$ nach Riesz.
	\begin{eqnarray}
		v(h) &=& \int \hat h(\xi)\overline{\g(\xi)} (1+|\xi|^2)^s d\xi\nonumber
	\end{eqnarray}
	für alle $h\in H^s$. Wähle $u=\F^{-1}[\overline{\g(\xi)}(1+|\xi|^2)^{s}]$. Dann:
	\begin{eqnarray}
		||u||_{H^{-s}} &=& ||\bar g||_{H^s} ~=~ ||g||_{H^s}\nonumber\\
		J(u)(h) &=& <h,g>_{s}~=~ v(h),~\forall h\in H\nonumber
	\end{eqnarray}
	\qed
	
	\begin{Prop}
		Ist $s<t$, so ist $H^t\subset H^s$.
	\end{Prop}
	
	\Bew Da $(1+|\xi|^2)^s\leq (1+|\xi|^2)^t$.
	\qed
	
	\begin{Satz}[Sobolevscher Einbettungssatz]
		Sei $s>d/2$. Dann gilt
		\begin{enumerate}
			\item[a)] $H^s\subset C_b(\R^d)$.
			\item[b)] $H^{s+K}\subset C_b^K(\R^d)$.
		\end{enumerate}
	\end{Satz}
	
	\Bew a) Für $u\in S(\R^d)$. 
	\begin{eqnarray}
		u(x) &=& \int_{\R^d} e^{2i\pi x\xi}(1+|\xi|^2)^{-s/2} \hat u(1+|\xi|^2)^{s/2} d\xi\nonumber\\
		&\leq& \left(\int (1+|\xi|^2)^{-s}d\xi \right)^{1/2} \left(\int |\hat u|^2 (1+|\xi|^2)^sd\xi \right)^{1/2}\nonumber 
	\end{eqnarray}
	mit $\left(\int (1+|\xi|^2)^{-s}d\xi \right)^{1/2}\leq \infty$ und $\left(\int |\hat u|^2 (1+|\xi|^2)^sd\xi \right)^{1/2} = ||u||_{H^s}$. 
	
	b) $|\alpha|\leq K$, $u\in H^{s+K}$ $\Rightarrow$ $D^\alpha\in H^s$, wende nun a) an $\Rightarrow$ Beh.
	\qed
	
	\begin{Korollar}
		Sei $s>d/2$, dann gilt: Der Raum der endlichen Maße $M(\R^d)\subset H^{-s}(\R^d)$.
	\end{Korollar}
	
	\paragraph{Bemerkung:} $\mu\in M(\R^d)$, $u_\mu\in S'(\R^d)$.
	\begin{eqnarray}
		u_\mu(h) &=& \int_{\R^d}h(x)d\mu(x)\nonumber\\
		\mu\in M(\R^d) &\Rightarrow& u_\mu \in H^{-s}(\R^d)\nonumber
	\end{eqnarray}
	
	\Bew 
	\begin{eqnarray}
		|u_\mu(h)| &=& |\int h(x)d\mu(x)|\nonumber\\
		&\leq& ||h||_{C_b(\R^d)}||\mu||_{M(\R^d)}\nonumber\\
		&\leq& ||h||_{H^s}\m ||\mu||_{M(\R^d)}\nonumber
	\end{eqnarray}
	Also: $u_\mu\in (H^s)'$, $||u_\mu||_{(H^s)'}\leq ||\mu||_{M(\R^d)}$, d.h. $u_\mu\in H^{-s}$. 
	
	z.B. $\delta_x\in H^{-s}$, $s>d/2$, $x\in \R^d$, denn $h\in H^s$, $h\in C_b(\R^d)$, $\delta_x(f) = f(x)$.
	
	\begin{Satz}
		Sei $s<t$ (d.h. $H^t\subseteq H^s$), $(u_n)_{n\in\N}\subset H^t$ und $||u_n||_{H^t}\leq 1$ und $\supp(u_n)\subset U$, $U\subset \R^d$ beschränkt. Dann hat $(u_n)$ eine in $H^s$ konvergente Teilfolge.
	\end{Satz}
	
	\paragraph{Ergänzung zur Einbettung.} $H^t\subset H^s(\R^d)$ für $t>s$.
	
	\begin{Satz}[Kompakte Einbettung auf beschränkten Mengen]
		Sei $u_n\subset H^t$, $||u_n||_{H^t}\leq C$,  und für ein $M<\infty$ $\supp u_n\subset B(0,M)$ für alle $n\in \N$. Dann hat $u_n$ eine Teilfolge, die in $H^s(\R^d)$ konvergiert für $s<t$.
	\end{Satz}
	
	\Bew 1. Schritt: $\hat u_n|_K, n\in \N$ ist relativ kompakt in $C(K)$ für alle kompakten Teilmengen $K\subset \R^d$. Denn: wähle $\psi\in C_c^\infty(\R^d)$ und $\psi\equiv 1$ auf $B(0, m)$.
	\begin{eqnarray}
		D^\alpha \hat u_n &=& D^\alpha \F(\psi u_n) \nonumber\\
		&=& D^\alpha[\hat{\psi}*\hat u_n]\nonumber\\
		&=& (D^\alpha\hat{\psi})*\hat u_n,\nonumber~~~\text{ also}\\
		|D^\alpha \hat u_n(\xi) |&\leq& \int D^\alpha \hat{\psi}(\xi-\eta)\hat u_n(\eta) d\eta\nonumber\\
		&\leq&\left(\int (1+|\eta|^2)^{-t}|D^\alpha\hat{\psi}(\xi-\eta)|^2d\eta\right)^{1/2}\left(\int |\hat u_n(\eta)|^2(1+|\eta|^2)^ d\eta \right)^{1/2}\nonumber\\
		&=:& f_\alpha(\xi)\m||u_n||_{H^t}\nonumber\\
		&\leq& Cf_\alpha(\xi)\in L^\infty(\R^d),\nonumber
	\end{eqnarray}
	denn $(1+|\eta|^2)^{-1}\in L^\infty$, $\xi\rightarrow |D^\alpha\hat{\psi}(\xi)|^2\in L^1(\R^d)$ (Youngsche Ungleichung).
	
	Da $\hat u_n|_K$ beschränkt und gleichgradig stetig (denn $D^\alpha f_n$ gleichmäßig beschränkt), folgt aus Azela-Ascoli, dass $(u_n)_{n\in \N}$ gleichmäßig konvergente Teilfolgen hat.
	
	2. Schritt: $u_n$ hat eine konvergente Teilfolge in $H^s$. Denn: Zu $\epsilon>0$ wähle ein $0<R<\infty$, sodass 
	\begin{eqnarray}
		(1+R^2)^{s-t}\leq\frac{\epsilon}{4c^2}\nonumber
	\end{eqnarray}
	Für $K=\{x\in \R^d: |x|\leq R \}$ wähle nach Schritt 1 eine Teilfolge von $(u_n)$ (wieder $u_n$ genannt), sodass $\hat u_n|_K$ gleichmäßig in $C(K)$ konvergiert. Dann
	\begin{eqnarray}
		||u_n-u_m||^2 &=& \int_{\R^d} |\hat u_n(\xi)-\hat u_m(\xi)|^2(1+|\xi|^2)^s d\xi \nonumber\\
		&=& \int_{|\xi|\leq R}...+\int_{|\xi|> R}...\nonumber
	\end{eqnarray}
	mit
	\begin{eqnarray}
		\int_{|\xi|>R}|\hat u_n(\xi)-\hat u_m(\xi)|^2(1+|\xi|^2)^sd\xi &\leq& \sup_{|\xi|>R} (1+|\xi|^2)^{s-t}||u_n-u_m||^2_{H^t}\nonumber\\
		&\leq& (1+R^2)^{s-t}(||u_n||_{H^t}+||u_m||_{H^t})\nonumber
		\leq \epsilon\nonumber
	\end{eqnarray}
	nach Wahl von $R$ und
	\begin{eqnarray}
		\int_{|\xi|\leq R} |\hat u_n(\xi)-\hat u_m(\xi)^2(1+|\xi|)^sd\xi&\leq& \int_{|\xi|\leq R}(1+|\xi|^2)^s d\xi \sup_{\xi\in K}|\hat{u_n}(\xi)-\hat u_m(\xi)|^2\nonumber\\
		&=& C_1\m\text{ Konst }\m ||\hat u_n-\hat u_m||_{C(K)}\overset{n,m\rightarrow\infty}{\longrightarrow} 0\nonumber
	\end{eqnarray}
	
	
	\begin{Prop}[Bernsteinsche Ungleichung]
		Für $f\in L^2(\R^d)$ mit $\supp\f\subseteq\{\xi: |\xi|>R \}$ gilt:
		\begin{enumerate}
			\item[a)] $f$ hat eine analytische Fortsetzung auf $\C^d$.
			\item[b)] Für alle $s<r$ gilt $f\in H^r$ und 
			$$||f||_{H^r}\leq (1+R)^{r-s}||f||_{H^s}.$$
		\end{enumerate}
	\end{Prop}
	
	\Bew a) Für $(z_1,..., z_d)\in \C^d$ setze 
	$$F(z_1,...,z_d) = \int_{|\xi|\leq R} \exp\left[2i\pi\left(\sum_{j = 1}^{d}z_j \xi_j\right)\right]\f(\xi_1,...,\xi_d)d\xi_1...\xi_d.$$
	Nach der Umkehrformel: $F(x_1,...,x_d) = f(x_1,...,x_d)$ für $x_1,...,x_d\in \R$. Also ist $F$ eine Fortsetzung von $f$ von $\R^d$ auf $\C^d$. Zu zeigen: $F$ ist partiell komplex differenzierbar nach $z_1,...,z_d$. Sei $w=w_1,...,w_d\in\C^d$ fest. Auf der Kugel $K=\{z\in \C^d: |z-w|\leq 1 \}$ ist 
	$\exp\left[2i\pi \left(\sum_{j = 1}^{d}z_j \xi_j\right) \right]$, $|\xi|\leq R$, $z\in K$ gleichmäßig beschränkt. Also ist Differentiation nach $w_1,...,w_d$ unter dem Integral erlaubt und die Behauptung folgt.
	
	b) 
	\begin{eqnarray}
		||f||_{H^r}^2 &=& \int |\f(\xi)|^2(1+|\xi|^2)^rd\xi\nonumber\\
		&\leq& \sup_{|\xi|\leq R}(1+|\xi|^2)^{r-s}\int_{|\xi|\leq R}|\f(\xi)|^2(1+|\xi|^2)d\xi\nonumber\\
		&\leq& (1+R^2)^{r-s}||f||_{H^s}^2\nonumber\\
		&\leq& (1+R)^{2(r-s)}||f||_{H^s}^2\nonumber
	\end{eqnarray}
	\qed
	
	
	\section{Der Funktionalkalkül des Laplace Operators}
	
	\begin{Definition}[des Laplace Operators]
		Für $f\in S(\R^d)$: $\Delta f(\m)=\sum_{j = 1}^{d}\frac{\partial^2}{\partial x_j^2} f(\m)$. 
		$$\xymatrix{
			S(\R^d)\ar[r]^{\Delta}\ar[d]_{\F} & S(\R^d)\ar[d]^{\F}\\
			S(\R^d)\ar[r]_M & S(\R^d)
			}$$
		\begin{eqnarray}
			\F(\Delta f)(\xi) &=& \sum_{j = 1}^{d}\F(\frac{\partial^2}{\partial x_j^2}f)(\xi) \nonumber\\
			&=& \sum_{j = 1}^d (2i\pi)^2\xi_j^2 (\xi)\nonumber\\
			&=& -(2\pi)^2 |\xi|^2 \f(\xi)\nonumber
		\end{eqnarray}
		$$Mg(\xi) = -(2\pi)^2|\xi|^2 g(\xi)$$
		$$\boxed{\Delta= \F^{-1}\circ M\circ \F.}$$
		$$D(A) = H^2(\R^d),~ f\in D(A):~ \F(\Delta f)(\xi) = -(2\pi)^2|\xi|^2\f(\xi).$$
	\end{Definition}
	
	\begin{Bemerkung}
		~
		\begin{enumerate}
			\item[a)] Der ungewohnte Faktor $(2\pi)^2$ kommt von der Definition der Fouriertransformation durch $e^{\underline{2\pi}i\xi x}$.
			\item[b)] $H^2 = \{f\in S': \frac{\partial^i}{\partial x_j^i}\in L^2(\R^d), i=1,2 \}$
			\begin{eqnarray}
				\Delta f = \sum_{j = 1}^d \frac{\partial^2}{\partial x_j^2} f\nonumber
			\end{eqnarray}
			ist in Ordnung, falls man $\frac{\partial^i}{\partial x_j^i}$ als distributionelle oder \textbf{schwache} Ableitungen versteht. 
			$$\int \varphi(x)\left[\frac{\partial^2}{\partial x_j^2} f(x)\right]dx = (-1)^{2}\int\left(\frac{\partial^2}{\partial x_j^2} \varphi(x) \right)dx$$
			für alle $\varphi\in C_c^\infty(\R^d)$.
			\item[c)] $\Delta: H^2\rightarrow L^2$ ist stetig und mit der Graphennorm
			$$||f||_{H^2}\cong ||f||_{L^2} + ||\Delta f||_{L^2}$$
			ist $\Delta$ insbesondere auf $L^2$ ein abgeschlossener Operator, d.h. $f_n\in D(A)$, $f_n\rightarrow f$ in $L^2$, $\Delta f_n\rightarrow g$ in $L^2$ $\Rightarrow$ $f\in D(A)$, $\Delta f = g$.
		\end{enumerate}
	\end{Bemerkung}
	
	\Bew 
	\begin{eqnarray}
		||f||_{H^2}^2 &=& \int |\f(\xi)|^2(1+|\xi|^2)^2 d\xi\nonumber\\
		&\approx& \int |\f(\xi)|^2d\xi + \int\left[|\f(\xi)||\xi|^2 \right]^2d\xi\nonumber\\
		&\approx& ||f||_{L^2}^2 + ||\Delta f||_{L^2}^2\nonumber
	\end{eqnarray}
	Also: $||\Delta u||_{L^2} \leq ||u||_\Delta\approx ||u||_{H^2}$ - Stetigkeit.
	\qed
	
	\begin{Bemerkung}[Ziele des Funktionalkalküls]
		Definition neuer Operatoren, z.B.
		\begin{itemize}
			\item $e^{-t\Delta}$ $\Rightarrow$ Lösung der Wärmeleitungsgleichung: $y'(t)=\Delta y(t)$, $y(0) = y_0$.
			\item $e^{-i t\Delta}$ $\Rightarrow$ Lösung der Schrödingergleichung $y'(t) = (i\Delta) y(t)$.
			\item $\sin(t\Delta^{1/2})$, $\cos(t\Delta^{1/2})$ $\Rightarrow$ Lösung der Wellengleichung $y''(t) = \Delta y(t)$.
		\end{itemize}
		Berechnen der Operatorennormen, z.B.
		$$||\Delta^n e^{-z\Delta}|| = ?$$
		Übertragung von Funktionalgleichungen in Operatorengleichungen, z.B. 
		$$e^{-t\lambda}e^{-s\lambda} = e^{-(s+t)\lambda} \Rightarrow e^{-t\Delta}e^{-s\Delta} = e^{-(s+t)\Delta} ?$$
	\end{Bemerkung}
	
	\begin{Definition}
		Der Funktionalkalkül des Laplaceoperators ist eine Abbildung
		$$\Phi: B_b(\R_+)\rightarrow B(L^2(\R^d))$$
		mit
		\begin{enumerate}
			\item[(i)] $\Phi(\varphi +\phi) = \Phi(\varphi)+ \Phi(\phi)$. $\Phi(\varphi\m \phi) = \Phi(\varphi)\circ\Phi(\phi)$. \textbf{Algebrahomomorphismus} von $B_b(\R_+)$ - punktweise Operationen - nach $B(L^2)$ - Operatorverknüpfungen.
			\item[(ii)] $||\Phi(\varphi)|| = ||\varphi||_{L^{\infty}(\R^d)}$. \textbf{Beschränktheit} des Kalküls.
			\item[(iii)] Sei $\varphi_n(t)|\leq 1$, $\varphi_n(t)\rightarrow \varphi(t)$, $n\rightarrow\infty$ für alle $t>0$. Dann gelte
			$$\Phi(\varphi_n)f\overset{n\rightarrow\infty}{\longrightarrow} \Phi(\varphi)f, ~\forall f\in L^2(\R^d)$$
			\textbf{Konvergenzeigenschaft}.
			\item[(iv)] Für $r_\mu(t) =\frac{1}{\mu-t}$ folgt $\Phi r_\mu = R(\mu, \Delta)$.
		\end{enumerate}
	\end{Definition}
	
	\paragraph{Notation:} Schreibe $\Phi(\varphi):= \varphi(\Delta)$.
	
	Idee zur Konstruktion von $I$:
	$$-\laplace = \F^{-1}M\F$$
	mit $Mg(\xi) = (4\pi^2 \m |\xi|^2)g(\xi)$, $m(\xi) = 4\pi^2|\xi|^2$, $$D(M) = \{g\in L^2(\R^d): ~ m\m g\in L^2, \text{ d.h. } |\xi|^2g(\xi) \in L^2 \}.$$ Konstruiere zuerst
	
	\begin{Definition}[Funktionalkalkül für $M$]
		$$M^2 g = M(Mg) = M(m\m g) = m^2 g, M^g = (m^n)g.$$
		\begin{eqnarray}
			p(\lambda)&=& \sum a_n\lambda^n,\nonumber\\
			p(M)g &=& \left(\sum a_nM^n \right)g = \left(\sum a_n m^n \right)g = \left[p(m) \right]g\nonumber
		\end{eqnarray}
		wobei $p(m)(u) = p(m(u))$ (Komposition von Funktionen).
		
		Sei $\varphi \in B_b(\R_+)$: 
		\begin{eqnarray}
			\Phi_M(\varphi)g &=& \varphi(M)g~=~\boxed{\varphi(m)}g,~ \varphi(m) =\varphi\circ m\nonumber
		\end{eqnarray}
	\end{Definition}
	
	\emph{Nachweis der Eigenschaften (i)-(iv) für $M$:} (i)
	\begin{eqnarray}
		\varphi(M)g +\psi(M)g &=& (\varphi\circ m)g+(\psi\circ m)g\nonumber\\
		&=& \left[(\varphi+\psi)\circ m \right]g \nonumber\\
		&=& (\varphi+\psi)(M)g\nonumber\\ \nonumber\\
		\left[\varphi(M)\m \psi(M) \right](g) &=& \varphi(M)\left[\psi(M)g \right]\nonumber\\
		&=& \varphi(M)\left[(\psi\circ m)g \right]\nonumber\\
		&=& (\varphi\circ m)(\psi\circ m)g\nonumber\\
		&=& \left[(\varphi\m \psi)\circ m\right] g\nonumber\\
		&=& (\varphi\m \psi)(M)g\nonumber
	\end{eqnarray}
	(ii)
	\begin{eqnarray}
		||\varphi(M)||_{B(L^2)} &=& \int |\varphi(M)g(x)|^2 dx\nonumber\\
		&=& \int|\varphi(m(x))g(x)|^2dx\nonumber\\
		&=& ||m\circ\varphi||_{L^\infty(\R^d)}\nonumber\\
		&=& \text{ess}\sup_{x\in \R^d}|\varphi(m(x))|\nonumber\\
		&=& \text{ess}\sup_{\lambda>0}|\varphi(\lambda)|\nonumber
	\end{eqnarray}
	(iii) Zu zeigen:
	\begin{eqnarray}
		\varphi_n(M)f &\rightarrow& \varphi(M)f\text{ in } L^2(\R^d).\nonumber\\
		\nonumber\\
		|\varphi(M)f-\varphi_n(M)f||_{L^2}^2 &=&\int |\varphi(m(x)) - \varphi_n(m(x))|^2 |f(x)|^2 dx\overset{n\rightarrow \infty}{\longrightarrow} 0\nonumber
	\end{eqnarray}
	nach Satz von Lebesgue, da $|\varphi(m(x)) - \varphi_n(m(x))|^2\rightarrow 0$ für $n\rightarrow\infty$ für alle $x\in \R^d$.\\
	(iv) Zu zeigen:
	\begin{eqnarray}
		r_\mu(M) &=& R(\mu, M)\nonumber\\
		\nonumber\\
		R(\mu,M)&=& (\mu-M)^{-1},\nonumber\\
		(\mu-M)g&=& (\mu-m(x))g(\m)\nonumber\\
		\Rightarrow (\mu-M)^{-1}g &=& (M-m(\m))^{-1}g(\m) = r_\mu(M)\nonumber
	\end{eqnarray}
	\qed
	
	\begin{Definition}[Konstruktion des Funktionalkalküls für $\laplace$]
		$$\varphi(-\laplace) = \F^{-1}\varphi(M)\F.$$
	\end{Definition}
	
	Nachprüfen der Eigenschaften von $\Phi$: (i)
	\begin{eqnarray}
		\varphi(-\laplace)\psi(-\laplace) &=& \left(\F^{-1}\varphi(M)\F \right)\left(\F^{-1}\varphi(M)\F \right)\nonumber\\
		&=&\F^{-1}[\varphi(M)\psi(M)]\F \nonumber\\
		&=& (\varphi\m \psi)(-\laplace)\nonumber
	\end{eqnarray}
	(ii)
	\begin{eqnarray}
		||\varphi(-\laplace)|| \overset{\text{Isometrien}}{=}||\varphi(M)|| =||\varphi||_{L^{\infty}(\R^d)}.\nonumber
	\end{eqnarray}
	(iii) 
	\begin{eqnarray}
		\varphi_n(M)f&\overset{n\rightarrow\infty}{\longrightarrow}&\varphi(M)f,~f\in L^2\nonumber\\
		\Rightarrow \varphi_n(-\laplace)g = \F^{-1}[\varphi_n(M)(\F g)]&\rightarrow &\F^{-1}[\varphi(M)(\F g)] \text{ da }\F\text{ stetig.}\nonumber
	\end{eqnarray}
	(iv)
	\begin{eqnarray}
		(\lambda-M)r_\mu(M)&=& \id\nonumber\\
		\Rightarrow \F^{-1}[(\lambda-M)r_\mu(M)]\F &=&\id\nonumber\\
		\Rightarrow R(\lambda,-\laplace) &=& r_\mu(-\laplace)\nonumber
	\end{eqnarray}
	\qed
	
	\paragraph{Beispiel.} Definition von $-(-\laplace)^{1/2}$.
	\begin{eqnarray}
		\F(-(-\laplace)^{1/2} f)(\xi) &=& 2\pi|\xi|\f(\xi):\nonumber\\
		(-(-\laplace))^2 &=& -\laplace\nonumber
	\end{eqnarray}
	
	\newpage
	
	\section{Das Cauchyproblem für die Schrödinger- und die Wellengleichung}
	
	\begin{Motivation}
		$$y'(t) = a(y(t))+ f(t), ~y(0) = y_0,$$
		wobei $a\in \R$, $f\in L^1(\R_+)$.
		$$y(t) = e^{at}y_0+\int_{0}^{t}e^{a(t-s)}f(s)ds.$$
	\end{Motivation}
	
	\begin{Bemerkung}[Cauchyproblem für die Schrödingergleichung]
		$$(+)~u'(t) = i\laplace u(t) +f(t),~ u(0) = y_0,$$
		wobei $y_0\in L^2(\R^d)$, $f(t)\in L^2(\R^d)$, $\int_{0}^{\infty}||f(t)||_{L^2}dt<\infty$, $A = i\laplace$.
	\end{Bemerkung}
	
	\paragraph{Notation:} $\hat{u}(t,\xi)  =\F(u(t,\m))(\xi)$ für festes $t\in \R$ heißt partielle Fouriertransformation bezüglich $x$.
	
	Anwendung der partiellen Fouriertransformation auf $(+)$.
	\begin{eqnarray}
		\partial_t \hat u(t,\xi) &=& i(-4\pi^2|\xi|^2)\hat u(t,\xi) +\f(t,\xi) 
	\end{eqnarray}
	Für festes $\xi\in \R^d$ ist das eine gewöhnliche Differentialgleichung wie in 8.1. Die Lösung von (1) für festes $\xi$ ist
	\begin{eqnarray}
		\hat u(t,\xi) &=& e^{-4\pi^2i|\xi|^2}\hat{y}_0(\xi)+\int_{0}^{\infty} e^{-4i\pi^2(t-s)|\xi|^2}\f(s,\xi)ds
	\end{eqnarray}
	Mit der inversen partiellen Fouriertransformation bezüglich $\xi$ ($t$ fest) erhält man
	\begin{eqnarray}
		u(t,\xi) &=& \F^{-1}\left[e^{-4i\pi^2|\xi|^2}\hat{y}_0(\xi) \right](x) +\int_{0}^{t}\F^{-1}\left[e^{-4i\pi^2|\xi^2(t-s)}\f(x,\xi)\right]ds
	\end{eqnarray}
	wobei im letzten Ausdruck mit Hilfe von Fubini die Integration und $\F^{-1}$ vertauscht wurden. Mit Hilfe des Funktionalkalküls kann man (3) interpretieren als
	\begin{eqnarray}
		\boxed{u(t) = e^{it\laplace}y_0 +\int_0^t e^{i(t-s)\laplace}f(s) ds,}
	\end{eqnarray}
	denn
	\begin{eqnarray}
		\F(e^{it\laplace}y_0)(\xi) &=& e^{itm(\xi)}\hat{y}_0(\xi),~~ m(\xi) = 4\pi|\xi|^2\nonumber\\
		&=& e^{-4\pi^2 it|\xi|^2}\hat{y}_0(\xi)\nonumber
	\end{eqnarray}
	
	\begin{Prop}
		Die Operatoren $T(t) = e^{it\laplace}$ erfüllen:
		\begin{enumerate}
			\item[(i)] $T(t+s) = T(s)T(t)$, $t,s\in \R$.
			\item[(ii)] $T^{-1}(t) = T(-t) = T(t)^*$, d.h. die $T(t)$ sind eine unitäre Gruppe in $B(L^{2})$. $||T(t) ||= 1$.
			\item[(iii)] $T_t f\overset{t\rightarrow 0}{\longrightarrow}f$ in $L^2$, $f\in L^2$.
			\item[(iv)] Für $f\in D(\laplace)$ gilt:
			$$\lim\limits_{t\rightarrow 0}\frac{T(t)-I}{t}f = i\laplace f, ~~ \frac{d}{dt}e^{it\laplace}|_{t=0}f = i\laplace f.$$
		\end{enumerate}
	\end{Prop}
	
	\Bew (i) 
	\begin{eqnarray}
		T(t)&=&\varphi_t(-\laplace),~ \varphi_t(x) = e^{-i\lambda t}\nonumber\\
		\varphi_t(\lambda)\varphi_s(\lambda) = \varphi_{s+t}(\lambda)&\Rightarrow& T(t) T(s) = T(t+s)\nonumber\\
		\varpi_t(\lambda)\m\varphi_{-t}(\lambda)= 1&\Rightarrow& T(t)\m T(-t) = \id\nonumber
	\end{eqnarray}
	(ii) 
	$$||T(t)|| = \sup_{\lambda>0}|e^{i\lambda t}| =1.$$
	(iii)
	$$|\rho_t(\lambda)|\leq 1, \varphi_t(\lambda)\rightarrow 1\Rightarrow T(t)f\rightarrow T(0) =\id.$$
	(iv) Da $f\in D(\laplace)$, ist $(i\laplace)f=g\in L^2$.
	\begin{eqnarray}
		\frac{1}{-i\lambda t}(e^{-i\lambda t}-1)&\rightarrow& 1\nonumber\\
		\Rightarrow \frac{1}{t}(e^{i\lambda\laplace}- I)(i\laplace)^{-1}g&\rightarrow& g\nonumber\\
		\Rightarrow \lim\limits_{t\rightarrow 0}\frac{(T(t)-I)}{t}f &=& i\laplace f.\nonumber
	\end{eqnarray}
	
	\newpage
	\chapter{Spekraltheorie selbstadjungierter Operatoren}
	
	\section{Beschränkte normale Operatoren}
	
	\begin{Definition}
		Seien $H_j$ Hilberträume mit $<.,.>_j$, $j = 1,2$. Zu jedem $T\in B(H_1,H_2)$ gibt es einen Hilbertraum adjungierten Operator $T^*\in B(H_2,H_1)$ mit 
		$$<Tx,y> = <x,T^*y>.$$
	\end{Definition}
	
	\begin{Bemerkung}
		~
		\begin{enumerate}
			\item[a)] $A^*$ ist eindeutig, $A^* \in B(H)$ mit $||A^*|| = ||A||$.
			\item[b)] Beziehung zwischen der Hilbertraum-Adjungierten $T^*$ und der Banachraum-Adjungierten $T'$: 
			\begin{eqnarray}
				\Phi_1:H_1&\rightarrow& H_1', \Phi_1(x)(y) = <y,x>_1\nonumber\\
				\Phi_2: H_2&\rightarrow& H_2', \Phi_2(x)(y) = <y,x>_2,~x,y\in Hz.\nonumber
			\end{eqnarray}
			(Riesz-Homomorphismen).
		\end{enumerate}
	\end{Bemerkung}
	
	Behauptung:
	$$\xymatrix{
		H_1\ar[r]^{\Phi_1} & H_1'\\
		H_2\ar[u]^{T^*}\ar[r]_{\Phi_2} & H_2' \ar[u]_{T'}
		}, ~~ T^* = \Phi_1^{-1}(T')\Phi_2.$$
		
	\begin{eqnarray}
		\Phi_1(T^*x)(y) &=& <y,T^*x>_1\nonumber\\
		&=& <Ty,x>_2\nonumber\\
		&=& <\Phi_2(x),Ty>_1
	\end{eqnarray}
	
	\begin{Prop}
		Seien $S,T\in B(H_1,H_2)$, $R\in B(H_2,H_3)$, $\lambda\in \C$. Dann
		\begin{enumerate}
			\item[a)] $(S+T)^* = S^*+T^*$.
			\item[b)] $(\lambda S)^* = (\overline{\lambda})S^*$, $(\lambda S)^* = \lambda S'$.
			\item[c)] $(RS)^* = S^* R^*$.
			\item[d)] $S^{**} = S$.
			\item[e] $||S S^*|| = ||S^* S|| = ||S||^2$.
			\item[f)] $\Kern(S) = \Bild(S^*)^{\perp}$, $\Kern(S^*) = \Bild(S)^{\perp}$.
		\end{enumerate}
	\end{Prop}
	
	\Bew e) 
	\begin{eqnarray}
		||Sx||^2 = <Sx,Sx> = <x,S^*Sx> \leq ||x||^2||S^*S|| = ||x||^2||S^*||\m||S^*|| = ||x||^2||S||^2\nonumber
	\end{eqnarray}
	Supremum über $x$ mit $||x|| = 1$:
	$$||S||^2\leq ||S^* S||\leq ||S||^2.$$
	
	
	\begin{Definition}
		~
		\begin{enumerate}
			\item[a)] $T: H_1\rightarrow H_2$ ist \textbf{unitär}, falls $T$ invertierbar ist und $T^{-1} = T^*$.
			\item[b)] $T\in B(H)$ ist \textbf{selbstadjungiert}, falls $T^* = T$.
			\item[c)] $T\in B(H)$ ist \textbf{normal} fallst $TT^* = T^* T$.
		\end{enumerate}
	\end{Definition}
	
	\begin{Bemerkung}
		~
		\begin{enumerate}
			\item[a)] Selbtstadjungierte und unitäre Operatoren sind normal.
			\item[b)] $T$ unitär $<Tx, Ty> = <x,T^*Ty> = <x,T T^*y> = <x, TT^{-1}y> ? <x,y>$ $\Rightarrow$ eine unitäre Abbildung erhält das Skalarprodukt und $||Tx|| = ||x||$.
		\end{enumerate}
	\end{Bemerkung}
	
	\begin{Beispiel}
		$H = L^2(\Omega)$, $\Omega\subset \R^d$, $m\in L^\infty(\Omega)$. 
		$$T:H\rightarrow H,~~ Tf(\om) = m(\om)f(\om).$$
		Dann gilt $||T|| = ||m||_{L^\infty}$.
		\begin{eqnarray}
			<Tf,g> &=& \int_{\Omega} m(x)f(x) \overline{g(x)} dx \nonumber\\
			&=& \int f(x)\overline{\overline{m(x)}g(x)}dx\nonumber\\
			&=& <f,T^* g>\nonumber
		\end{eqnarray}
		$\Rightarrow$ $T^*g(x) = \overline{m(x)}g(x)$. Also 
		\begin{itemize}
			\item $T$ selbstadjungiert $\Leftrightarrow$ $m(x)\in \R$.
			\item $T$ unitär $\Leftrightarrow$ $|m(x)| = 1$.
		\end{itemize}
		$$T^{-1}f(x) = m(x)^{-1} f(x),~m(x)^{-1} = \overline{m(x)}$$.
	\end{Beispiel}
	
	\begin{Satz}
		Sei $T\in B(H)$. Dann
		\begin{enumerate}
			\item[a)] $T$ normal: $r(T) = ||T||$.
			\item[b)] $T$ selbstadjungiert $\Rightarrow$ $\sigma(T)\subset \R$.
			\item[c)] $T$ unitär $\Rightarrow$ $\sigma(T)\subset \{\lambda:|\lambda| = 1 \}$.
		\end{enumerate}
	\end{Satz}
	
	\Bew a) $$||T^2||^2 = ||(T^2)(T^2)^*|| = ...= ||T T^*||^2 = ||T||^4.$$
	$\Rightarrow$ $||T^2|| = ||T||^2$ $\Rightarrow$ $ ||T^{2^n}|| = ||T||^{2n}$.

	b),c) Übung.
	\qed
	
	\begin{Satz}
		$T\in B(H)$ ist selbstadjungiert genau dann, wenn $<Tx,x>\in \R$ $\forall x\in H$ und $$||T|| = \sup_{||x||= 1}<Tx,x>.$$
	\end{Satz}
	
	\Bew Übung.
	
	\begin{Satz}
		$P\in B(H)\backslash\{0\}$ sei eine Projektion. Dann sind äquivalent:
		\begin{enumerate}
			\item[a)] $P$ ist eine Orthogonalprojektion, d.h. $\Kern P\perp \Bild P$.
			\item[b)] $||P|| = 1$.
			\item[c)] $P$ selbstadjungiert.
			\item[d)] $P$ normal.
			\item[e)] $<Px,x>\geq 0$.
		\end{enumerate}
	\end{Satz}
	
	\Bew Funkana Übung, Buch Werner.
	
	\begin{Satz}[Lax-Milgram]
		Sei $H$ ein komplexer Hilbertraum mit $<.,.>$ und $b:H\times H\rightarrow\C$ sei eine sesquilineare Form ($b(x,\lambda y)= \overline{\lambda}b(x,y)$).
		\begin{enumerate}
			\item[a)] Falls $|b(x,y)|\leq C||x||\m||y||$ $(+)$, dann existiert ein eindeutig bestimmter Operator $T\in B(H)$ mit 
			$$b(x,y) = <x,Tx>\text{ und } ||T||\leq C.$$
			\item[b)] Falls zusätzlich $b(x,x)\geq \delta||x||^2$, $\delta\geq 0$, dann ist $T$ invertierbar und $||T^{-1} \leq \delta^{-1}$.
		\end{enumerate}
	\end{Satz}
	
	\Bew a) Sei $y\in H$ fest. Dann ist $x\rightarrow b_y(x) := b(x,y)$ ein lineares Funktional auf $H$ mit $||b_y||\leq C||y||$ (nach $(+)$). Da $b_y\in H'$, gibt es nach dem Satz von Riesz ein $z\in H$ mit $<x,z> = b_y(x) = b(x,y)$ und $||z|| = ||b_y||_{H'}\leq C||y||$.
	
	Definiere $Ty:= z$. Dann ist $||Ty||\leq ||z||\leq C||y||$.
	$$<x,Ty> = <x,z>= b(x,y)$$
	$||T||\leq C$, $T$ linear.
	
	b) Es gilt $b(x,x) \geq \delta ||x||^2$. Für $T$ aus a) und $y\in H$ gilt 
	$$<y, Ty> = b(y,y)\geq \delta ||y||^2.$$
	\begin{itemize}
		\item Aus $Ty = 0$ folgt $||y|| = 0$, d.h. $T$ ist injektiv.
		\item $\Bild T$ ist abgeschlossen, denn für $y_n\in H$ mit $T(y_n)\rightarrow z\in H$ folgt 
		$$||y_n-y_m||^2\leq \delta^{-1} b(y_n-y_m,y_n-y_m) = \delta^{-1}<y_n-y_m,T(y_n-y_m)> \leq \delta^{-1}2\sup ||y_n||\m||T(y_n)-T(y_m)||$$
		Da $\sup ||y_n||<0$, $T(y_n)\rightarrow z$, gilt $||y_n-y_m||\rightarrow 0$ für $n,m\rightarrow \infty$. Also $(y_n)$ sind Cauchy Folge und $y=\lim y_n$ existiert. Dann $Ty=\lim Ty_n = z$ und $z\in \Bild T$, d.h. $\Bild T$ ist abgeschlossen.
		\item $\Bild T = H$. Wähle $z\in (\Bild T)^{\perp}$. Dann 
		$$||z||^2\leq \delta^{-1}<z,Tz> = 0\Rightarrow z= 0$$
		Somit gilt $\Bild T = H$.
	\end{itemize}
	Also $T\in B(H)$ ist surjektiv und $T^{-1}\in B(X)$ (open-map).
	\begin{eqnarray}
		||T^{-1}x||^2\leq \delta^{-1}<T^{-1}, T^{-1}Tx> \leq \delta^{-1}||T^{-1}x||\m ||x||\nonumber
	\end{eqnarray}
	Also $||T^{-1}x||\leq \delta^{-1}||x||$.
	
	\section{Funktionalkalkül für beschränkte selbstadjungierte Operatoren}
	
	\paragraph{Notation} $\mathcal{P}$: Menge der Polynome auf $\R$ mit komplexen Koeffizienten. 
	
	\begin{Definition}[Funktionalkalkül für Polynome]
		Sei $A\in B(X)$, $X$ Hilbertraum, $<.,.>$, $A$ selbstadjungiert. Für 
		$$p(\lambda) = \sum_{j = 0}^{n} a_n\lambda^n, ~ a_n\in \mathbf{K}$$
		setze 
		$$p(A) = \sum_{j = 0}^{^n}a_n A^n\in B(X).$$
	\end{Definition}
	
	\begin{Definition}
		Die Abbildung $p\in\mathcal{P}\rightarrow p(A) \in B(X)$ hat die Eigenschaften ($f,g\in \mathcal{P}, \lambda\in \C$)
		\begin{enumerate}[(i)]
			\item $(\alpha f +g)(A) = \alpha f(A)+g(A)$, linear. \\
			$(f\m g)(A) = f(A)\m g(A)$, multiplikativ.
			\item $f_0(\lambda)\equiv 1$, $f_1(\lambda) =\lambda$ $\Rightarrow$ $f_0(A) = \id_X$, $f_1(A) = A$.
			\item $f(A)^* =\overline{f}(A)$.
			\item $\boxed{\|p(A)\|\leq \sum_{j = 0}^{n}|a_j|\m\|A\|^j}$ für $p(\lambda) = \sum_{j = 0}^{n} a_j\lambda^j$
		\end{enumerate}
	\end{Definition}
	
	\Bew (i) Übung, (ii) folgt aus Definition, (iii) 
	\begin{align*}
		p(A)^* = (\sum_{j = 0}^{n}a_j A^n)* = \sum_{j = 0}^{n} (a_n A^n)^* = \sum_{j = 0}^{n} \overline{a}_j (A^*)^n = \overline{p}(A^*) = \overline{p}(A)
	\end{align*}
	(iv) $\|p(A)\| \leq\sum_{j = 0}^{n}|a_j|\m\|A^j\|\leq \sum_{j = 0}^{n}|a_j|\m\|A\|^j$.
	\qed
	
	\begin{Satz}[Spektralabbildungssatz]
		Für einen selbstadjungierten Operator $A\in B(X)$ und $p\in \mathcal{P}$ gilt 
		\begin{align*}
			\sigma(p(A)) = p(\sigma(A)) = \{p(\lambda):\lambda\in \sigma(A) \}.
		\end{align*}
	\end{Satz}
	
	\Bew \glqq$\supseteq$\grqq\ Sei $\mu\in \sigma(A)$. Zu zeigen: $p(\mu)\in \sigma(p(A))$. Dazu wähle ein Polynom $q$, sodass 
	\begin{align*}
		&~ p(\mu) - p(\lambda) = (\mu-\lambda)q(\lambda), ~\lambda>0\\
		\Rightarrow & ~p(\mu)-p(A) = (\mu-A)q(A) = q(A)(\mu-A)
	\end{align*}
	Da $\mu-A$ keine Inverse hat ($\mu\in \sigma(A)$), hat auch $p(\mu)-p(A)$ keine Inverse, d.h. $p(\mu)\in \sigma(p(A))$.
	
	\glqq$\subseteq$\grqq\ Sei $\mu\in \sigma(p(A))$. Zeige: $\mu\in p(\sigma(A))$. Seien $\lambda_1,...,\lambda_n$ die Wurzeln von $\lambda\rightarrow \mu-p(\lambda)$, d.h. $\mu-p(\lambda) = a(\lambda-\lambda_1)\m...\m (\lambda-\lambda_n)$, $(+)$. Damit folgt $\mu-p(A) = a(A-\lambda_1)\m...\m (A-\lambda_n)$. Wäre $\lambda_1,...,\lambda_n\notin \sigma(A)$, dann folgt 
	\begin{align*}
		(\mu-p(A))^{-1} = a^{-1}(A-\lambda_n)^{-1}\m...\m (A-\lambda_1)^{-1}
	\end{align*}
	also wäre $\mu\notin \sigma(p(A))$. Da $\mu \in \sigma(p(A))$, ist mindestens eines der $\lambda_i$ in $\sigma(A)$. Dann folgt aus $(+)$ $\mu=p(\lambda_j)$, $\mu \in p(\sigma(A))$.
	\qed
	
	
	\begin{Lemma}
		Für $A\in B(X)$ selbstadjungiert und $p\in \mathcal{P}$ gilt 
		$$\|p(A)\|_{B(X)} = \sup_{\lambda\in \sigma(A)}|p(x)|.$$	
	\end{Lemma}
	
	\Bew 
	\begin{align*}
		\|p(A)\|^2 = \|p(A)p(A)^*\| = \|p(A)\overline{p}(A)\| = \| (p\m \overline{p})(A)\| = \||p|^2(A)\| = r(|p|^2(A))
	\end{align*}
	da $|p(\m)|^2$ reell und $|p|^2(A)$ selbstadjungiert ist. Weiter ist
	\begin{align*}
		r(|p|^2(A)) = \sup\{|\lambda| : \lambda\in \sigma(|p|^2(A)) \} \overset{\text{2.2.3}}{=} \sup\{|p|^2(\lambda): \lambda\in \sigma(A) \}
	\end{align*}
	Durch Wurzelziehen erhält man 
	$$\|p(A)\| = \sup|\{p(\lambda)\in \sigma(A) \}|.$$
	\qed
	
	Idee: $f(A) = \lim p_n(A)$ für $p_n\in \mathcal{P}$ mit $\|p_n-f\|_{C(\sigma(A))}\rightarrow 0$.
	
	
	\begin{Satz}[Funktionalkalkül für stetige Funktionen]
		Es gibt eine lineare, multiplikative Abbildung 
		\begin{align*}
			\Phi: C(\sigma(A)) \rightarrow B(X) ~(\text{schreibe} \Phi(f) = f(A))
		\end{align*}
		mit $\Phi(p) = p(A)$ für $p\in \mathcal{P}$ und
		\begin{enumerate}[(i)]
			\item $\|f(A)\| = \sup\{|f(\lambda)|: \lambda\in \sigma(A) \}$.
			\item $f(A)^* = \bar{f}(A)$, $f(A)$ normal, $f(A)$ selbstadjungiert $\Leftrightarrow$ $f$ reellwertig. $f(A)\geq 0$ $\Leftrightarrow$ $f(\lambda)\geq 0$ für $\lambda\in \sigma(A)$.
			\item $Ax = \lambda x$ $\Rightarrow$ $f(A)x = f(\lambda)x$.
			\item $\sigma(f(A)) = f(\sigma(A))$.
		\end{enumerate}
	\end{Satz}
	
	\Bew Nach dem Satz von Weierstraß sind die Polynome dicht in $(C(\sigma A),\|\m\|_\infty)$, da $\sigma(A)\subset \R$. $\Phi$ ist die stetige Fortsetzung der Abbildung $p\in \mathcal{P}\rightarrow p(A)\in B(X)$, d.h. für $p\in \mathcal{P}$ mit $\|p_n-f\|_\infty\rightarrow 0$ setze $\Phi(f) = \lim\lim\limits_{n\rightarrow \infty} p_n(A)$ in $B(X)$. Das ist möglich, da $p\in \mathcal{P}\rightarrow p(A)\in B(X)$ eine Isometrie ist, wegen 2.2.3 
	$$\|p(A)\| = \sup\{|p(\lambda)|:\lambda\in \sigma(A) \}.$$
	Da $$\|f(A)\| = \lim\limits_{n\rightarrow\infty} \|p_n(A)\| = \lim\lim\limits_{n\rightarrow\infty}sup_{\lambda\in \sigma(A)}|p_n(x)| = \sup_{\lambda\in \sigma(A)}|f(x)$$
	gilt also 
	$$\|\Phi(f)\| = \sup_{\lambda\in\sigma(A)}|f(\lambda)| \text{ nach (i)},$$
	d.h. $\Phi$ ist linear und multiplikativ.
	$$f(A)^* = \lim p_n(A)^* = \lim \overline{p}_n(A) = \overline{f}(A)$$
	Zeige nun $f(A)$ ist normal: 
	$$f(A)\m f(A)^* = f(A) \m \overline{f}(A) = (f\m \overline{f})(A) = (\overline{f}\m f)(A) = \overline{f}(A)\m f(A) = f(A)^*\m f(A).$$
	$f(A)$ selbstadjungiert $\Leftrightarrow$ $f$ ist reelwertig:
	$$f(A) = f(A)^* = \overline{f}(A) \Leftrightarrow f = \overline{f}	\Leftrightarrow f\text{ ist reelwertig.}$$
	$f\geq 0$ $\Rightarrow$ $f(A)\geq 0$, d.h. $<f(A)x,x> \geq 0$ für alle $x$. Wähle $g\geq 0$ mit $f = g^2$. Dann 
	$$<f(A)x,x> = <g^2(A)x,x> = <g(A)g(A)x,x> = <g(A)x,g(A)^*x> = <g(A)x,g(A)x> \geq 0.$$
	(iii) $p\in \mathcal{P}$, $Ax = \lambda x$
	\begin{align*}
		\Rightarrow p(A) = \left(\sum_{j = 0}^{n}a_j A^j \right)x = \left(\sum_{j = 0}^{n} a_j \lambda^j \right)x
	\end{align*}
	$\Rightarrow$ $f(A)x = f(\lambda)x$ mit Hilfe von Approximationen von $f$ durch Polynome $p_n$.
	(iv) \glqq$\subseteq$\grqq\ Sei $\mu\in f(\sigma(A))$. Setze $g(\lambda) = (f(\lambda)-\mu)^{-1}\in C(\sigma(A))$. Dann gilt
	\begin{align*}
		g(f-\mu) = (f-\mu) g\equiv 1\\
		\Rightarrow g(A)(f(A)-\mu) = (f(A)-\mu)g(A) = \id_X\Rightarrow \mu\notin\sigma(f(A)).
	\end{align*}
	\glqq$\supseteq$\grqq\ Sei $\lambda\in\sigma(A)$. Zeige: $f(\mu)\in \sigma(f(A))$. Wähle $p_n\in \mathcal{P}$ mit $\|f-p_n\|_{C(\sigma(A))}\rightarrow 0$. 
	\begin{align*}
		\|f(\mu)-f(A)-(p_n(\mu)-p_n(A))\|_{B(X)} \leq \sup_{\lambda\in \sigma(A)}|f(\mu)-f(\lambda)-p_n(\mu)+p_n(\lambda)| \overset{n\rightarrow\infty}{\rightarrow}0\\
		f(A)^* = \lim p_n(A)^* = \lim \overline{p}_n(A) = \overline{f}(A)
	\end{align*}
	Nach 2.2.3 gilt $p_n(\mu)\in \sigma(p_n(A))$. Wäre $f(\mu)-f(A)$ invertierbar, so wäre wegen $p_n(\mu)-p_n(A)\rightarrow f(\mu)-f(A)$ in $B(X)$ auch $p_n(\mu)-p_n(A)$ für große $n$ invertierbar, da die Menge der invertierbaren Operatoren in $B(X)$ offen ist. Also $f(\mu)-f(A)$ nicht invertierbar, d.h. $f(\mu)\in\sigma(f(A))$.
	
	\paragraph{Ziele.} 
	\begin{enumerate}[1)]
		\item $\Phi: B_b(\sigma(A)) \rightarrow B(X)$, $B_b$ beschränkte Borelfunktionen.
		\item Finde ein Maß $\mu$ auf $\C$ und eine Isometrie $U: L^2(\C,\mu)\rightarrow X$, sodass 
		$$A=UMU^{-1},\text{ wobei } Mg(\lambda = m(\lambda)g(\lambda).$$
	\end{enumerate}
	
	Nach Riesz gilt $C(\sigma(A))' = M(\sigma(A))$. Zu $|l\in C(\sigma(A))$ gibt es ein Maß $\mu$:
	$$l(f)\int_{\sigma(A)}f(\lambda d\mu(\lambda)).$$
	
	\begin{Satz}
		$f\in C(\sigma(A))\rightarrow f(A)\in B(X)$ ist ein isometrischer Algebrahomomorphismus.
	\end{Satz}
	
	\begin{Satz}
		Sei $A\in B(X)$, $A$ selbstadjungiert. Zu jedem $x\in X$, $\|x\| = 1$ gibt es ein Maß $\mu_x$ auf $(\sigma(A),\text{ Borelmengen})$ mit $\mu_x(\sigma(A)) = 1$, sodass
		$$\boxed{<f(A) x,x> = \int_{\sigma(A)}f(x) d\mu_x(x) }$$
		für alle $f\in C(\sigma(A))$.
	\end{Satz}
	
	\begin{Definition}
		$\mu_x$ heißt \textbf{Spektralmaß} von $A$ bezüglich $x\in X$.
	\end{Definition}
	
	\Bew Sei $x\in X$, $\|x\| = 1$ fest. Definiere $l_x:C(\sigma(X))\rightarrow\C$. Für $f\in C(\sigma(A))$ setze $l_x(f) = <f(A) x,x>$. Dann gilt 
	\begin{itemize}
		\item $l_x$ ist linear, denn 
		$$l_x(f+g) = <(f+g)(A)x,x> = <f(A)x,x> + <g(A)x,x> = l_x(f)+l_x(g).$$
		\item $|l_x(f)|\leq |<f(A)x,x>|\leq \|f(A)\|\m\|x\|^2 = \sup_{\lambda\in \sigma(A)}f(x)$. \\
		$l(1_{\sigma(A)}) = <1(A)x,x> = \|x\|^2 = 1$ $\Rightarrow$ $\|l\|_{C(\sigma(A))} = 1$.
		\item $f$ ist positiv, d.h. $f\geq 0$, $\overset{2.6}{\Rightarrow}$ $f(A)\geq 0$ $\Rightarrow$ $l_x(f) = <f(A)x,x> \geq 0$.
	\end{itemize}
	Nach Riesz: $\exists \mu_x$ Maß mit $\mu_x(\sigma(A)) = 1$, $\int f(\lambda)d\mu_x(\lambda) = l_x(f) = <f(A)x,x>$.
	\qed
	
	\section{Spektraldarstellung selbstadjungierter beschränkter Operatoren}
	
	\begin{Definition}
		Sei $A\in B(X)$. Dann heißt $x\in X$ ein zyklischer Vektor von $A$, falls
		$$X=\overline{\spann\{A^nx: n\in \N_0 \}} =\overline{\{p(A)x:p\in \mathcal{P} \}}$$
	\end{Definition}
	
	\begin{Satz}
		Sei $A$ selbstadjungiert, $A\in B(X)$ und habe einen zyklischen Vektor $x\in X$. Dann gibt es eine unitäre Abbildung 
		$$U: L^2(\sigma(A),\mu_x)\rightarrow  X ~~(\mu\text{ Spektralmaß}),$$
		sodass $A = UMU^{-1}$, wobei $Mg(\lambda) = \lambda g(\lambda)$.
	\end{Satz}
	
	\Bew Sei $x$ ein fester, zyklischer Vektor mit $\|x\| = 1$ und $\mu_x$ das zugehörige Spektralmaß.
	\begin{align*}
		U : C(\sigma(A))\rightarrow X
	\end{align*}
	Für $f\in C(\sigma(A))$ setze $U f:= f(A)x\in X$
	\begin{itemize}
		\item $U$ ist linear, denn der Funktionalkalkül ist linear.
		\item $\|Uf\|_x^2 = <Uf, Uf>_X = <f(A)x,f(A)x> = <f(A)^*f(A)x,x> = <\bar f(A)f(A)x,x> $ \\$=<(\bar f\m f)(A)x,x> = \int_{\sigma(A)} |f(\lambda)|^2d\mu_x(\lambda) = \|f\|_{L^2(\sigma(A),\mu_x)}^2$. Also: $\|Uf\|_x = \|f\|_{L^2(\sigma(A),\mu_x)}$.
	\end{itemize}
	Da $C(\sigma(A))$ dicht in $L^2(\sigma(A),\mu_x)$, kann man $U$ stetig fortsetzen.
	\begin{align*}
		L^2(\sigma(A),\mu_x) \rightarrow X \text{ linear, isometrisch.}
	\end{align*}
	Zu zeigen: $U:L^2\rightarrow X$ ist surjektiv.
	
	$\Bild U$ ist abgeschlossen in $X$, da $U$ isometrisch. Ferner gilt $\{Up: p\in \mathcal{P}\} = \{p(A)x: p\in \mathcal{P} \}$ ist dicht in $X$, da $x$ ein zyklischer Vektor ist. $\Rightarrow$ $\Bild U = X$. 
	
	Bleibt zu zeigen: $A = UMU^{-1}$. $(AU)(p) = A(p(A)x) = (A\m p(A))x = q(A)x = M(p)(A)x$ mit $q(\lambda) = \lambda p(\lambda) = M p(\lambda) = UM(p)$. Damit gilt $AU = UM$ $\Rightarrow$ Behauptung.
	\qed
	
	\begin{Lemma}
		Sei $X$ separabel, $A\in B(X)$ selbstadjungiert. Dann gibt es eine (möglicherweise endliche) Folge $H_j$, $j\in J$, von Teilräumen von $X$ mit 
		\begin{enumerate}[(i)]
			\item $A(H_j)\subset H_j$, $j\in J$.
			\item $A_j = A|_{H_j}$ hat $A_j$ einen zyklischen Vektor in $H_j$, d.h.
			$$\overline{\{p(A_j)x_j: p\in \mathcal{P} \}} = H_j,~ j\in J.$$
			\item $H_j\perp H_k$ für $j\neq k$.
			\item Für $x\in X$ gilt mit $x_j = P_{H_j}x$ ($P_{H_j}: X\rightarrow H_j$ orthogonale Projektion).
			$$x = \sum_{j\in J} x_j, ~ \|x\|^2 =\sum_{j\in J} \|x_j\|^2.$$
		\end{enumerate}
	\end{Lemma}
	
	\Bew (i) Wähle $x_1\in H_1$, $x_1\neq 0$ und setze
	$$H_1 = \overline{\spann}\{A_jx_1, j = 0,1,2,... \} = \overline{\{p(A)x_1:p\in \mathcal{P} \}}$$
	Dann $A(H_1) \subset H_1$, denn 
	\begin{align*}
		x = p(A)x_1\Rightarrow Ax = Ap(A)x_1 = q(A)x_1 \in H_1
	\end{align*}
	mit  $q(\lambda) = \lambda p(\lambda)$. $x_1$ ist ein zylischer Vektor von $A_1 = A|_{H_1}$ in $H_1$.
	
	(ii) Falls $X\neq H_1$, wähle $x_2\in H_1^{\perp}$, $x_2\neq 0$. Setze
	$$H_2=\overline{\{p(A)x_2: p\in \mathcal{P} \}}.$$
	Genauso folgt: $A(H_2)\subset H_2$, $x_2$ ist ein zyklischer Vektor für $A_2 = A|_{H_2}$ in $H_2$. 
	
	Zeige nun $H_1\perp H_2$. $p(A)x_1\in H_1$, $q(A)x_2 \in H_2$, $p,q\in \mathcal{P}$. 
	\begin{align*}
		<p(A)x_1, q(A)x_2> = <q(A)^* p(A)x_1,x_2> = <(\bar q\m p)(A)x_1,x_2> = 0
	\end{align*}
	denn $(\bar{q}\m p)(A)x_1 \in H_1$ und $x_2\perp H_1$.
	
	(iii) Falls $x\neq \overline{H_1+H_2}$, wähle $x_3\perp H_1,H_2$ und setze 
	$$H_3 = \overline{\{p(A)x_3:p\in \mathcal{P} \}}.$$
	Frage 
	$$\overline{\bigcup_{j\in J}H_j} = X.$$
	(iv) In dieser Konstruktion kann man $x_1,x_2,...$ so wählen, dass
	$$\overline{\bigcup_{j\in J}H_j} = X.$$
	Wähle eine Folge $(y_k)$ mit $\overline{\spann} (y_k) = X$. Wähle $x_1 = y_1$. Falls $H_1\neq X$, wähle zuerst den ersten Index $j_1$, sodass $y_{j_1}\notin H_1$ und setze $x_2 = y_{j_1} - P_{H_1}y_{j_1}\perp H_1$. Iteriere nun. Dann
	\begin{align*}
		\{y_j:j\in \N \}\subset \overline{\sum_{j\in J}H_j } = \overline{\{x_1+\cdots+x_{j_m}, x_k\in H_k \}}.
	\end{align*}
	(v) $x_j\perp x_k$ für $k\neq j$. Wie bei ONB:
	\begin{align*}
		\|\sum_{j = 1}^{m}x_j\|^2 = <\sum_{j = 1}^{m}x_j,\sum_{j = 1}^{m}x_k> = 	\|\sum_{j = 1}^{m}x_k\|^2 \Rightarrow \text{ Behauptung.}
	\end{align*}
	\qed
	
	\begin{Satz}
		Sei $A\in B(X)$ selbstadjungiert und $X$ seperabel. Dann existieren
		\begin{itemize}
			\item eine abgeschlossene Menge $S\subset \R^2$,
			\item ein Borelmaß $\mu$ auf $S$
			\item und ein unitärer Operator $U:L^2(S,\mu)\rightarrow X$.
			\item sowie $m:S\rightarrow \C$ stetig und beschränkt,
		\end{itemize}
		sodass $A = UMU^{-1}$, wobei $Mg(\lambda) = m(\lambda)g(\lambda)$.
	\end{Satz}
	
	\Bew Nach Lemma 3.3 gibt es eine Folge $H_j$, $j\in J$ von Teilrämen von $X$ mit 
	\begin{itemize}
		\item $A(H_j) \subset H_j$, $A_j = A|_{H_j}$ hat einen zyklischen Vektor $x_j$.
		\item $x = \sum x_j$, $x_j= P_{H_j}x$, $\|x\|^2 = \sum_{j\in J}\|x_j\|^2$ (orthogonale Zerlegung von $X$).
	\end{itemize}
	Auf $A_j\in H(H_j)$ wende Satz 3.2 an: Zum zyklischen Vektor $x_j$ und seinem Spektralmaß $\mu_j$ gibt es eine unitäre Abbildung: $L^2(\sigma(A_j), \mu_j)\rightarrow H_j$, sodass $A_j = U_jM_j U_j^{-1}$, wobei $M_jg(\lambda) = \lambda g(\lambda)$. Sei $S$ die "disjunkte Vereinigung" der $\sigma(A_j)\subset \R$
	$$S = \bigcup_{j\in J} S_j,~ S_j = \{j\}\times \sigma(A_j)\subset \R^2.$$
	$\tilde{\mu}_j(\{j\}\times B_j) = \mu_j(B_j)$ für $B_j\subset \R$ Borel, $j\in J$.
	
	Definiere Maß $\mu$ auf $S$:
	\begin{align*}
		\mu(B) = \sum_{j \in J} \mu_j(B_j)\text{ für } B = \bigcup_{j\in J} \{j\}\times B_j.
	\end{align*}
	$f\in L^{2}(S,\mu)$ hat die Form
	$$f(j,\lambda) = f_j(\lambda),~ \lambda\in \R, j\in J.$$
	\begin{align*}
		\|f\|_{L^2(S,\mu)}^2 = \int_{\bigcup S_j} |f(\lambda)|^2d\mu(\lambda) = \sum_{j\in J} \int_{S_j}|f_j (t)|^2 d\tilde{\mu}_j(t) = \sum \int_{\sigma(A)}|f_j(t)|^2 d\mu(t) = \sum_{j\in J} \|f_j\|_{L^2(\sigma(A_j),\mu_j)}
	\end{align*}
	$U:L^2(S,\mu)\rightarrow X$ $Uf=\sum_{j} U_j f_j$.
	$$\|U(f)\|_X^2 = \sum \|U_j(f_j)\|_{H_j}^2 = \sum \|f_j\|_{L^2(\sigma(A_j), \mu_j)} = \|f\|_{L^2(S,\mu)}$$
	Also ist $U$ unitär.
	
	Setze $m(j,\lambda) = m_j(\lambda)$ für $(j,\lambda)\in S$. Damit folgt 
	$$[UAU^{-1}] (j,d) = U_jA_jU_j^{-1}f_j(\lambda) = m_j(\lambda)f_j(\lambda) = (m\m f)(j,\lambda), (j,\lambda)\in s.$$
	\qed
	
	Erweiterung des Funktionskalküls auf $B_b(\sigma(A))$, z.B. $f = \chi_{[a,b]}$, $f = \chi_B$. Problem: $\|f(A)\| = \sup_{\lambda \in \sigma A}|f(\lambda)|$.
	
	\begin{Beispiel}
		Sei $M_m:L^2(S,\mu)\rightarrow L^2(S,\mu)$, $M(f) = m(\lambda)f(\lambda).$
		Wiederhole: 
		$\|M_m\|_{B(L^2)} = \|m\|_{L^\infty(S,\mu)}$.
		\begin{align*}
			\sigma(M_m) &= \overline{\{m(s): s\in S \}} = \{\mu\text{-wesentlicher Wertebereich von }m\text{ auf } S\} \\ 
			&= \{t\in \R:\mu(\{s\in S: |m(s)-t|<\epsilon \})>0\forall\epsilon > 0 \}.
		\end{align*}
		ist abgeschlossen, $t_n\in R_\mu(m)$, $t_n\rightarrow t$, $\mu(\{s\in S:|m(s)-t_n|<\epsilon/2 \})>0$.
		\begin{align*}
			|m(s)-t|\leq |m(s)-t_n|+|t_n-t|\\
			\{|m(s)-t|<\epsilon \} \supset \{s\in S:|t_n-t|<\epsilon/2 \}
		\end{align*}
		Damit folgt $\mu(\{s\in S:|m(s)-t_n|<\epsilon/2 \})> 0$, d.h. $t\in R_\mu(m)$.
		
		Zu Zeigen: $t\in R_\mu(m)$ $\Rightarrow$ $t\in \sigma(M_m)$. $(t\id - M_m)f$.
		\begin{align*}
			B_\epsilon &= \{s\in S: |m(s)-t|<\epsilon \}, \mu(B_\epsilon)>0\\
			f &= \chi_{B_\epsilon}\m \mu(B_\epsilon)^{-1/2}, \|f\|_{L^2(\mu)} = 1\\
			\|(t\id - M)f\|_{L^2}^2 &= \int_{B_\epsilon}|(t-m(s))f(s)|^2ds\\
			&\leq \epsilon^2 \|f\|^2
		\end{align*}
		Wähle eine Folge $\epsilon_n\rightarrow 0$. Dann gilt für die Konstruktion $f_{\epsilon_n}$:
		\begin{align*}
			\|(t\id - M) f_{\epsilon_n}\| \rightarrow 0, \|f_{\epsilon_n}\| = 1
		\end{align*}
		d.h. $t\in \sigma(M_m)$.
		Bleibt zu zeigen: $t\in \sigma(M_m)$ $\Rightarrow$ $t\in R_\mu(m)$, d.h. $t\notin R_\mu(m)$ $\Rightarrow$ $t\in \rho(M_m)$
		
		$t\notin R_\mu(m)$ heißt: $\exists \epsilon_0>0:\mu(s\in S_|t-m(s)|<\epsilon_0) = 0$, d.h. $|t-m(s)|^{-1}\leq \epsilon_0 >0$ für $\mu$-fas alle $t\in S$. $R(f(t)) = (t-m(s))^{-1}f(s)$ gilt
		\begin{align*}
			\|R\|_{B(L^2)} &= \|(t-m(s))^{-1}\|_{L^\infty(S,\mu)}\leq \infty\\
			R(t\id-M_m)f &= (t-m(s))^{-1}(t-m(s))f(s) = f(s)\\
			R &= (t\id - M_m)^{-1},
		\end{align*}
		d.h. $t\in \rho(M)$.
		%TODO Tafel wurde einfach gewischt, nachdem es gerade erst dort stand.
	\end{Beispiel}
	
	\subsection*{Funktionalkalkül von $\mathbf{M_m}$} $f\in B_b(\sigma(M_m)) \rightarrow f(M_m)$. $M_m^2 = M_{m^2}$, $M_m^n = M_{m^n}$. Somit gilt für ein Polynom
	$$p(M) = \sum_{n = 1}^{m} a_n M_m^n = M_{p(m)}$$
	Definiere nun für $f\in B_b(\sigma(M_m))$: $$f(M_m):= M_{f\circ m}.$$
	Die Abbildung $f\in B_b(\sigma(M_m)) \rightarrow f(M_m)\in B(H)$ ist linear und multiplikativ.
	
	\Bew Folgt direkt aus der Multiplikativität von Funktionen.
	
	Frage: Was ist $\|f(M)\|_{B(L^2)}$?
	\begin{align*}
		\|f(M)\|_{B(L^2)} &= \|M_{f\circ m}\|_{B(L^2)} = \text{ess}\sup |(f\circ m)(s)| = \|f\circ m\|_{L^\infty}
	\end{align*}
	$(S,\mu)\overset{m}{\rightarrow} (\sigma(M_m),\rho)\overset{f}{\rightarrow}\R$. Ziel:
	$$\|f\circ m\|_{L^{\infty}(S,\mu)} = \|f\|_{L^\infty(\sigma(M_m),\rho)}$$
	z.B. $m(\lambda) = \lambda$, $\sigma(M_m) = S$, $\mu= \mu_X$ $\|f\circ m\|_{L^\infty(\mu) = \|f\|_{L^\infty}}$.
	
	\begin{Definition}[eines Maßes $\nu$ auf $\sigma(M_m)$]
		Für $B\subset \sigma(M_m)$ setze $\nu(B) = \mu(m^{-1}(B))$ (da $m^{-1}(B)$ Borelmenge in S).
	\end{Definition}
	
	 $\nu$ ist die Verteilung von $m$ auf $\sigma(M)$, denn für $B_j$, $B_j\cap B_k = \emptyset$ für $k\neq j$ gilt $m^{-1}(B_j)\cap m^{-1}(B_k) =\emptyset$ und 
	 $$\sum \nu(B_j) = \sum \mu(m^{-1}(B_j)) = \mu(m^{-1}(B)) = \nu(B), \bigcup B_j = B.$$
	 Es gilt nun: 
	 \begin{align*}
	 	\|f(M_m)\|_{B(L^2)} = \|f\circ m\|_{L^\infty(S,\mu)} = \|f\|_{L^\infty(\sigma(M), \nu)}
	 \end{align*}
	Also für $f\in B_b(\sigma(M_m))\rightarrow f(M_m)\in B(L^2(S,\mu))$ ist $\Phi_M(f) = f(M_m)$ ein Algebrahomomorphismus uns 
	$$\|f(M_m)\| = \|f\|_{L^\infty(\sigma(M_m),\nu)}.$$
	
	\subsection*{Funktionalkalkül für selbstadjungierte Operatoren}
	
	Sei $A = UM_m U^{-1}$ Spektraldarstellung mit $U,\mu,n,\nu$ wie oben.
	$$\xymatrix{
		H\ar[r]^A & H \\
		L^2(\mu)\ar[u]^U \ar[r]_{M_m}& L^2(U)\ar[u]_U
		}$$
	
	\begin{Definition}
		$f(A) = Uf(M_m)U^{-1}$ für $f\in B_b(\sigma(A))$.
	\end{Definition}
	
	Beachte $\sigma(A) = \sigma(M) = R_\mu(m)$. Für den Funktionalkalkül
	$$f\in B_b(\sigma(A))\rightarrow f(A) \in B(H)$$
	gilt
	\begin{enumerate}[(i)]
		\item $f\rightarrow f(A)$ ist ein Algebrahomomorphismus.
		\item $\|f(A)\| = \|f\|_{L^\infty(\sigma(A),\nu)}$.
		\item $|f_n(t)|\leq C$, $f_n(t)\rightarrow f(t)$ $\mu$-fast überall $\Rightarrow$ $f_n(A)x\rightarrow f(A)x$ für alle $x\in H$.
		\item $f(A)$ ist stets normal.\\
		$f(A)$ ist selbstadjungiert $\Leftrightarrow$ $f$ ist reell.\\
		$f(A)$ ist unitär $\Leftrightarrow$ $|f(t)| = 1$ $\mu$-fast überall.
		\item Für $B\in B(H)$ gelte, dass $BA = AB$. Dann gilt auch $f(A)B = Bf(A)$ für alle $f\in B_b(\sigma(A))$.
	\end{enumerate}
	 
	 \Bew (i) $f(A)g(A) = Uf(M_m)U^{-1} U g(M_m)U^{-1}$ liefert Behauptung.\\
	 (ii) $\|f(A)\| = \|f(M_M)\|$, da $U$ Isometrie liefert Behauptung.\\
	 (iii) Zu zeigen: $f_n(M_m) h\rightarrow f(M_m)h$ in $L^2(S,\mu)$. Folgt aus dem Satz von Lebesgue.\\
	 (v) Übung
	 \qed

	\section{Spektralprojektion}
	
	\begin{Beispiel}[Multiplikationsoperator]
		Sei $(S,\mu)$ ein Maßraum, $m:S\rightarrow \R$ messbar. Dann gibt es zu $\epsilon>0$ messbare Mengen $B_1,...,B_k\subset S$ und $m_1,...,m_k\in \R$, sodass 
		\begin{align*}
			\left|m(s)-\sum_{i = 1}^{k} m_i\chi_{B_i}(s)\right|\leq \epsilon \text{ für }s\in S
		\end{align*}
		Zur Konstruktion: Wähle $V_j = (\epsilon\m j, e\m ({j+1}))$, $j\in J$, sodass $R_\mu(m)\subset \bigcup_{j\in J}V_j$. Setze $B_j = m^{-1}(V_j)$ messbar, $m_j = \epsilon\m j$ Dann gilt 
		\begin{align*}
			\left|m(s)-\sum m_j\chi_{B_j}(s) \right|<\epsilon.
		\end{align*}
		Setze $Mh(s) = m(s) h(s)$ und $\tilde{M}h(s) = \tilde m(s) h(s)$, wobei $\tilde{m} = \sum_{j\in J} m_j \chi_{B_j}(s)$. Dann gilt
		\begin{align*}
			\| M-\tilde{M}\|_{B(L^2)} = \|m-\tilde{m}\|_{L^\infty}\leq \epsilon.
		\end{align*}
		und $\tilde{M}h(s) = m_j h(s)$ für $s\in B_j$, $j\in J$. $B_j = m^{-1}(V_j)$, d.h. $\chi_{V_j}\circ m = \chi_{B_j}$. Setze 
		$$M_j(h) = \chi_{B_j}h = (\chi_{V_j}\circ m)h = \chi_{V_j}(M) = \Phi_m(\chi_{V_j}).$$
		Damit folgt 
		$$\tilde{M} = \sum_{j\in J} m_j \chi_{V_j}(M),$$
		Approximation von $M$ mit Hilfe des Funktionalkalküls.
	\end{Beispiel}
	
	Im folgenden wird stets angenommen: $A\in B(H)$ auf einem Hilbertraum $H$, $A$ selbstadjungiert, 
	\begin{align*}
		(S,\mu),\ U:L^2(S,\mu)\rightarrow H,\ m:S\rightarrow\sigma(A)
	\end{align*}
	seien eine Spektraldarstellung für $A$, d.h. 
	$$A = UM_mU^{-1},$$
	$\Phi_A(f) = f(A)$ der Funktionalkalkül von $A$.
	
	\begin{Definition}
		Für $V$ setze $\Phi_A(\chi_V) = \chi_V(A) =: P_V$. $P_V$ ist eine \textbf{Spektralprojektion} von $A$ zur messbaren Teilmenge $V\subset S$.
	\end{Definition}
	
	
	\begin{Prop}
		Eigenschaften der Spektralprojektion $P_A$.
		\begin{enumerate}[(i)]
			\item $P_V$ ist immer eine Orthogonalprojektion auf $H$, d.h. $\|P_V\| = 1$, $P_V^2 = P_V$.
			\item $P_{\emptyset} = 0$, $P_\R = \id_H$.
			\item $P_{V_1}\m P_{V_2} = P_{V_1\cap V_2}$.
			\item Falls $V = \bigcup V_i$ mit $V_i\cap V_j = \emptyset$ für $i\neq j$, dann gilt $P_Vx = \lim\limits_{n\rightarrow\infty} \sum_{i = 1}^{n} P_{V_i}x$ in $H$.
			\item Sei $V_j$ wie in (iv). $P_{V_i}(H)\perp P_{V_j}(H)$ für $i\neq j$ und  $\|P_{V}x\|^2 = \sum_{i=1}^\infty\|P_{V_i}x\|^2$ (falls $V = S$, $x=\sum_{i = 1}^{\infty}P_{V_j}x$, $\|x\|^2 = \sum \|P_{V_i}x\|^2$), Spektralzerlegung von $H$ bezüglich $A$.
			\item $AP_V = P_VA$, d.h. $A(P_V(H))\subset P_V(H)$ und $A_V = A|_{P_V(H)}$ hat $\sigma(Av) = \sigma(A)\cap \bar V$.
		\end{enumerate}
	\end{Prop}
	
	\Bew (i) $P_V^2 = [\chi_B(A)]^2 = \chi_B^2(A) = \chi_B(A) = P_V$. $\|P_V\| = \|\chi_V\|_{L^\infty (,\mu)} = \left\{\begin{array}{l}
	0,~ \mu(B) = 0,\\
	1,\text{ sonst}
	\end{array} \right.$.\\
	(ii) $P_{V_1} \cap P_{V_2} = \chi_{V_1}(A) \m \chi_{V_2}(A) = \chi_{V_1\cap V_2}(A) = P_{V_1\cap V_2}(A)$.\\
	(iii) $\chi_V(s) = \lim\limits_{n\rightarrow \infty} \sum_{i = 1}^{n}\chi_{V_i}(s)$ $\overset{\text{Konvergenzeigenschaft}}{\Longrightarrow}$ $\chi_V(A) x = \lim\limits_{n\rightarrow\infty} \sum_{i = 1}^{m} \chi_{V_i}(A)x$ in $H$ für alle $x\in H$.\\
	(iv) $i\neq j$: 
	\begin{align*}
		\langle P_{V_i}(x),P_{V_j}(y)\rangle &= \langle x,P_{V_i}\m P_{V_j}y\rangle \\
		&= \langle x, P_{V_i\cap V_j}y\rangle = 0,
	\end{align*}
	wenn $V_i\cap V_j = \emptyset$.
	\begin{align*}
		\|P_V x\|^2 &= \left\langle\sum P_{V_i}x,\sum P_{V_i}x\right\rangle \\
		&= \sum\langle P_{V_i}x,P_{V_i}x\rangle = \sum \| P_{V_i}x\|^2
	\end{align*}
	(v) und (vi) Übung.
	\qed
	
	\begin{Korollar}
		Zu $\epsilon>0$ gibt es eine messbare Partition $V_1,...,V_k$ von $\sigma(A)$ und $m_1,...,m_k$, sodass
		\begin{align*}
			\left\| A-\sum_{j = 1}^{n}m_j P_{V_j}\right\|_{B(L^2)}\leq \epsilon,
		\end{align*}
		d.h. $x = \sum_{j = 1}^{\infty} P_{V_j}x$ $\Rightarrow$ $\tilde{A}x = \sum_{j = 1}^{\infty}m_j P_{V_j}x$ erfüllt
		$$\|A-\tilde{A}\|\leq \epsilon.$$
	\end{Korollar}
	
	
	\Bew Für $M_m$, $\tilde{M}$ siehe Beispiel 4.1. Dann gilt
	\begin{align*}
		\tilde{A} &= U^{-1}\tilde M U \\
		&= U^{-1}\left(\sum m_i\chi_{V_i}(M)\right) U \\
		&= \sum m_i U[\chi_{V_i}(M)] U^{-1}\\
		&= \sum m_i \chi_{V_i}(A)\\
		\|A-\tilde A\| &= \|M-\tilde{M}\| \leq \epsilon,
	\end{align*}
	da $U$ Isometrie.
	\qed
	
	\begin{Definition}
		$V\in B(\R)\rightarrow P_V\in \{\text{Orthogonalprojektionen auf }H\}$ heißt das Spektralmaß von $A$ und $P_t = P_{(-\infty, t]}$, $t\in\R$, heißt \textbf{Spektralschar} von $A$.
	\end{Definition}
	
	\begin{Bemerkung}
		Idee einer Darstellung selbstadjungierter Operatoren durch projektionswertige Spektralmaße.
		\begin{enumerate}[a)]
			\item $f(t) = \sum_{i} \alpha_i\chi_{V_i}$, $V_i\cap V_j =\emptyset$, $V_j$ messbar. Dann gilt $f(A) = \sum_i \alpha_i P_{V_i}$.
			\item $(f_n)$ seien Elementarfunktionen, $|f_n|\leq C$, $f_n\rightarrow f$ für $n\rightarrow \infty$. Dann gilt
			$$f_n(A)x\rightarrow f(A)x.$$
			\item Insbesondere: Für $f:\R\rightarrow \R$ stetig, beschränkt gilt
			\begin{align*}
				f_n(t) &=\sum f\left(\frac{j}{n}\right) \chi_{(j/n,(j+1)/n]}(t)\\
				f(t) &= \lim f_n(t),~ \lim|f_n(t)| \leq |f(t)|\leq C,\\
				f(A)x &= \lim f_n(A)x = \lim\left[f\left(\frac{j}{n} \right)P_{(j/n,(j+1)/n]} \right]\\
				\Rightarrow f(A)x &= \lim \sum f\left(\frac{j}{n} \right)\left(P_{(j+1)/n}-P_{j/n} \right).
			\end{align*}
			Vergleich mit Riemann-Integral, Stieltjes-Integral, legt die \glqq Schreibweise\grqq\ nahe		
			\begin{align*}
				\boxed{f(A) = \int_{\R} f(t) dP_t.}
			\end{align*}
			$P_t$ ist $\sigma$-additiv, $V=\bigcup V_j$, $V_i\cap V_j\neq \emptyset$ $P_V x=\lim \sum_{i = 1}^n P_{V_i}x$. Siehe Werner - Funktionalanalysis, Weidmann - Lineare Operatoren auf Hilberträumen, Schmüdgen - Unbeschränkte selbstadjungierte Operatoren.
			\item Für $x\in H$, $\|x\| = 1$, wurde Spektralmaß $\mu_x$ zu $x\in X$, definiert in Kapitel 2.2. Betrachte zu $(a,b]$ Funktionen $g_n\in C_b(\R)$ mit $g_n(t) \rightarrow \chi_{(a,b]}(t)$ für alle $t\in \R$, z.B. 
			\begin{align*}
				\mu_x((a,b]) &= \int \chi_{(a,b]}(t) d\mu_x(t) \\
				&= \lim\limits_{n\rightarrow\infty}\int_{R} g_n(t)d\mu_x(t)\\
				&=\lim\limits_{n\rightarrow\infty} \langle g_n(A)x,x\rangle \\
				&= \langle \chi_{(a,b]}(A)x, x\rangle \\
				&= \langle (P_b-P_a)x,x\rangle\\
				\Rightarrow \mu_x(a,b] &= \langle (P_a-P_b)x, x\rangle.
			\end{align*}
		\end{enumerate}
	\end{Bemerkung}
	
	
	
	\begin{Satz}[Spektralprojektion und Spektrum]
		~
		\begin{enumerate}[a)]
			\item $\lambda \in \rho(A)$ $\Leftrightarrow$ Es gibt eine offene Umgebung $V$ von $X$ mit $P_V = 0$.
			\item $\lambda\in \R$ ist Eigenwert von $A$ $\Leftrightarrow$ $P_{\{\lambda\}} \neq 0$. In diesem Fall projiziert $P_{\{\lambda\}}$ auf den Eigenraum von $A$ bezüglich dem Eigenwert $\lambda$.
			\item Jeder isolierte Punkt $\lambda\in \sigma(A)$ ist ein Eigenwert.
		\end{enumerate}
	\end{Satz}
	 
	 \Bew a) \glqq$\Rightarrow$\grqq\ Sei $\lambda\in \rho(A)$. $P_{\rho(A)} = P_{\sigma(A)\cap \rho(A)} = 0$, da $\rho(A)\cap \sigma(A) =\emptyset$.\\
	 \grqq$\Leftarrow$\grqq\ $P_U = 0$ für $\lambda\in U$ offen. Setze 
	 $$f(t) = \left\{\begin{array}{l}
	 \frac{1}{t-\lambda},\text{ für }t\notin U,\\
	 0,\text{ sonst}
	 \end{array} \right. .$$
	 Dann ist $f\in B_b(\R)$. Setze weiter $g(t) = t-\lambda$.
	 \begin{align*}
	 	f(A)(\lambda-A) &= f(A)g(A)\\
	 	&= (f\m g)(A) = \chi_{U^C}(A) + \chi_U(A)\\
	 	&= \chi_\R(A) = \id_H.
	 \end{align*}
	 Also $(\lambda-A)^{-1} = f(A)$, somit ist $\lambda\in \rho(A)$.\\
	 b) Zeige: $\Bild P_{\{\lambda\}} = \ker(\lambda \id-A)$. \glqq$\subseteq$\grqq\ $0\neq x\in \Bild P_{\{\lambda\}}$ $\Rightarrow$ $P_{\{\lambda\}} x= x$
	 \begin{align*}
	 	\langle (\lambda-A)x,y\rangle &= \langle (\lambda-A) P_{\{\lambda\}}x,y \\
	 	&= \int (\lambda-t)\chi_{\{\lambda\}}(t) d\langle P_t x,y\rangle = 0
	 \end{align*}
	 für alle $y\in Y$. Also $(\lambda-A)x = 0$.\\
	 \glqq$\supseteq$\grqq Sei $x\in \ker(\lambda\id-A)$. $Ax = \lambda x$ $\Rightarrow$ $\forall f\in B_b(\R)$ $f(A)x = f(\lambda)x$.
	 $$\chi_{\{\lambda\}}(t) = \lim\limits_{h\rightarrow 0} g_n(t)$$
	 $$P_{\{\lambda\}} x = \lim g_n(A)x = \lim g_n(\lambda)x = \lambda x.$$
	c) $\lambda$ sei isolierter Punkt von $\sigma(A)$, d.h. es gibt ein $x\in U$ offen, $\sigma(A)\cap U = \{x\}$. Da $U\backslash\{\lambda\} \subset \rho(A)$ $\Rightarrow$ $P_{U\backslash\{\lambda\}} = 0$. Wäre $P_{\{\lambda\}} = 0$, dann $P_U = P_{U\backslash\{\lambda\}} + P_{\{\lambda\}} = 0$ dann nach a) $\lambda\in \rho(A)$ Widerspruch. Also $P_{\{\lambda\}} \neq 0$ und $\lambda$ ist ein Eigenwert nach b). 
	
	\begin{Satz}[Stone's Formel]
		Sei $A\in B(H)$ selbstadjungiert. Für alle $a<b$ gilt 
		$$\left(\frac{1}{2}P_{[a,b]} - \frac{1}{2}P_{(a,b)}\right)x = \lim\limits_{\epsilon\rightarrow 0} \frac{1}{2\pi i}\int_{a}^{b}[(A-\lambda-i\epsilon)^{-1}x- (A-\lambda+i\epsilon)x]d\lambda = \lim\limits_{\epsilon\rightarrow 0 }\int_{\Gamma_\epsilon\cup \Gamma_{-\epsilon}} R(\mu,A)d\mu$$
		$\sigma(A) \subset\R$
	\end{Satz}
	
	\Bew Sei $f_\epsilon(t) = \frac{1}{2\pi i}\int_{a}^{b}\left[\frac{1}{t-\lambda-i\epsilon} -\frac{1}{t-\lambda+i\epsilon}\right] dt$. Behauptung:
	$$\lim\limits_{\epsilon\rightarrow 0}f_\epsilon(t) = \left\{\begin{array}{l}
	0,~ t\notin [a,b]\\
	1/2,~ t=a \text{ oder } t= b\\
	1,~ t\in (a,b)
	\end{array}
	\right.$$
	Aus der Behauptung folgt mit der Konvergenzeigenschaft:
	\begin{align*}
		\frac{1}{2\pi i} \int_{a}^{n}[(A-\lambda-i\epsilon)^{-1}-(a-\lambda+i\epsilon)^{-1}]x d\lambda &= f_\epsilon(A)x \\
		&\rightarrow \frac{1}{2}[\chi_{[a,b]}(A) + \chi_{(a,b)} (A)]x\\
		&= \frac{1}{2}\left(P_{[a,b]}+P_{(a,b)}\right)
	\end{align*}
	\begin{align*}
		f_\epsilon(t) &= \frac{1}{2\pi i}\int_a^b \frac{2i\epsilon}{(t-\lambda-i\epsilon)(t-\lambda+i\epsilon)}dt\\
		&= \frac{1}{\pi}\int_a^b \frac{\epsilon}{(t-\lambda)^2+\epsilon^2}d\lambda\\
		&= \frac{1}{\pi}\int_a^b \frac{1}{(t/\epsilon-\lambda/\epsilon)^2 + 1}\frac{d\lambda}{\epsilon}\\
		&= \frac{1}{\pi}\int_{a/\epsilon}^{b/\epsilon}\frac{1}{\lambda^2 + 1}d\lambda
		&\rightarrow \frac{1}{\pi}\int_{-\infty}^{\infty}\frac{1}{1+\lambda^2}d\lambda = 1
	\end{align*}
	Also $t\in (a,b)$: $\lim\limits_{\epsilon\rightarrow 0}f_\epsilon(t) = 1$, $t\notin [a,b]$: $\lim\limits_{\epsilon\rightarrow 0}f_\epsilon(t) = 0$, $t = 1$ oder $t = b$ $\lim\limits_{\epsilon\rightarrow 0}f_\epsilon(t) = 1/2$. $|f_\epsilon(t)| \leq 1$.
	\qed
	
	\section{Beispiele für Spektraldarstellungen} 
	
	\begin{Beispiel}
		Sei $k\in L^1(\R^n)$. Setze $Tf(t) = \int_{\R^n}k(t-s) f(s)ds$. Dann gilt: $T\in B(L^2(\R^d)$ nach dem Satz von Young.
		\begin{itemize}
			\item $T^*f(t) = \int_{\R} k^*(t-s) f(s) ds$, $k^*(t) = \overline{k(-t)}$.
			\item $T^*T = TT^*$, d.h. $T$ ist normal (Faltungsoperator vertauschen).
			\item $T$ selbstadjungiert $\Leftrightarrow$ $k(t) = k^*(t) = \overline{k(-t)}$.
		\end{itemize}
		Sei $T$ selbstadjungiert.
		\begin{align*}
			\F[Tf](t)&= \hat k(t) \hat f(t)\\
			Tf &=\F^{-1}[\hat k (t)\f]
		\end{align*}
		Ablesen der Spektraldarstellung. $U=\F^{-1}$, $L^2(S,\mu) = L^2(\R^n, \text{ Lebesguemaß})$. $(Mg)(t)= \hat k(t) g(t)$, $U$ Isometrie. $A = UMU^{-1}$. $\overline{k(-t)} = k(t)$ $\Rightarrow$ $\hat k$ reelwertig, stetig auf $\R^n$ nach Lemma von Riemann-Lebesgue. Spektralprojektion zu $I = [a,b]$. 
		$$P_I = \chi_{[a,b]}(A) = \F^{-1}[\chi_{[a,b]}(\hat k(\m))\f(\m)].$$
		Sei $J = k^{-1}([a,b])$ endlich, $\chi_{[a,b]}(\hat k(\m)) = \chi_J(\m)$.
		\begin{align*}
			P_I &= \F^{-1}[\chi_J\m \f] = \F^{-1}(\chi_J)* f\\
			\F^{-1}(\chi_J)(t) &= \int_c^d e^{2\pi i t s}\chi_J(s) ds \\
			&=  \frac{e^{2\pi i t d} - e^{2\pi i t c}}{2\pi i t},
		\end{align*}
		falls $J = (c,d)$.
	\end{Beispiel}
	
	\begin{Beispiel}[diskreter Laplace Operator]
		~
		\begin{enumerate}[a)]
			\item Motivation:
			\begin{itemize}
				\item $i u_n'(t) + u_{n+1}(t) + u_{n-1}(t) - 2 u_n(t) = 0.$
				System von abzählbar vielen gekoppelten Oszillatoren, wobei aber der $n$-te Oszillator nur seine beiden Nachbarn beeinflusst.
				\item $i\partial_t u(x,t) +\partial_{xx}u(x,t) = 0$, $t\in \R$, $x\in \R$. Approcimation durch Differenzenquotienten. 
				$$\partial_{xx}u(x_0) = \lim\limits_{h\rightarrow 0} \frac{u(x_0+h)+u(x_0-h)-2u(x_0)}{h^2}.$$
				Für $h>0$ betrachte das Gitter $h\Z$ auf $\R$.
				$$\laplace u(nh) = \frac{u(nh+h)+u(nh-h)-2u(nh)}{h^2}.$$
				$h = 1$ $iu_n(t) + \laplace_{\text{diskret}}u_n(t) = 0$.
			\end{itemize}
				\item $H = l^2(\Z^d)$, $j\in \{1,...,d\}$.
				\begin{align*}
					\partial_j f(n)&= f(n+e_j)-f(n), ~n\in \Z^d, ~e_j\text{ Einheitsvektor}.\\
					\partial_j^*f(n) &= f(n - e_j)-f(n) ,~ n\in \Z^d\\
				\end{align*}
				Definiere $\laplace_{\text{diskret}}f = -\sum_{j = 1}^{n}\partial_j^* \partial_j$. $\laplace_{\text{diskret}}u(n) = \sum_{|n-m| = 1} u(m) - 2 du(n)$ auf $\Z^d$. $A = \laplace_{\text{diskret}}$ ist ein selbstadjungierter, beschränkter Operator auf $H = l^2(\Z)$. Behauptung: 
				\begin{itemize}
					\item $\laplace_d\in B(H)$ selbstadjungiert.
					\item $\rho(\laplace_{\text{diskret}})\supset (-\infty, 0]$.
				\end{itemize}
				\Bew 
				\begin{align*}
					\|(\lambda +\laplace) u\|^2 &= \sum_n (\lambda-2d)u_n^2 +\sum_n\sum_{|m-n|= 1}|u(m)|^2 = \lambda\|(u_n)\|^2
				\end{align*}
			\item Spektraldarstellung $u: L^2([0,1]^d,\text{ Lebesgue})\rightarrow H$.
			$$U^{-1} = \F: l^2(\Z^d)\rightarrow L^2([0,1]^d), (u_n)\rightarrow\sum_{n\in \Z^d}u_n e^{2\pi i (n\m k)}.$$
			$U,\F$ unitär, $E_n(k) = e^{2\pi i(n\m k)}$ bilden ONB von $L^2([0,1]^d)$.
			
			Definiere $m(k) = -4\sum_{j = 1}^{d}\sin(\pi k_j)$, $k = (k_1,...,k_d)\in[0,1]^d$. $Mg(\m) = m(\m)g(\m)$ Multiplikationsoperator auf $L^2[0,1]^d$. Behauptung
			$$\laplace_{\text{diskret}} = UMU^{-1}, ~ L^2(S,\mu) = L^2([0,1]^d).$$
			\Bew $u = (u_n)\in l^2(\Z^d)$, $f(k) = \sum_{n\in \Z^d} u_n E_n(k)$. Dann $$u_n = \langle f,E_n\rangle_{L^2}.$$
			Dann
			\begin{align*}
				\laplace u(n) &= \sum_{j = 1}^{d}[u(n+e_j)+u(n-e_j)-2u(n)]\\
				&=\sum_{j = 1}^{d} \langle f, E_{n+e_j}\rangle +\langle f, E_{n-e_j}\rangle - 2 \langle f, E_n\rangle \\
				&= \int_{[0,1]^d} \int_{j = 1}^{d}(e^{2\pi i k_j} + e^{-2\pi i k_j} - 2)(e^{-2\pi i n\m k}) f(k)\\
				&= \int_{[0,1]^d} m(k) f(k) e^{-2\pi in k}dk = \F^{-1}[m\F f].
			\end{align*}
		\end{enumerate}
	\end{Beispiel}
	
	\begin{Beispiel}[Spektraldarstellung stationärer Prozesse\footnote{nicht Prüfungsrelevant}]
		\begin{enumerate}[a)]
		\item $(\Omega,\Sigma, P)$ sei Wahrscheinlichkeitsraum, $X_n\in L^2(\Omega,\Sigma, P)$ Zufallsveränderliche. Annahme: 
		$$E(X_n) - E(X_0) = \int_{\Omega} X_0(\om)dP(\om).$$
		Definition: $X_n$ stationär $\Leftrightarrow$ $$E(X_{n+l}\overline{X_{m+l}}) = E(X_n\m \overline X_m) = \int_{\Omega} X_n(\om) \overline{X_m(\om)}dP(\om) = \cov(X_n,X_m)$$
		mit $\cov(X_{n+l},X_{m+l}) = \cov(X_n,X_m)$ für alle $n,m,l$.\\
		Setze $M = \spann(X_n)\subset L^2(\Omega, P)$, $M$ Hilbertraum. Behauptung: Es gibt eine unitäre Transformation $U:M\rightarrow M$ mit $U(X_n) = X_{n+1}$.
		$$X = \sum_{j = 1}^{m}a_jX_j\in M$$
		Setze $U(X) = \sum_{j = 1}^{m}a_j X_{j+1}$, $U$ linear.
		\begin{align*}
			\|U(X)\|_{L^2(P)}^2 &= \left\langle \sum_{i=1}^{m}a_j X_{j+1}, \sum_{k = 1}^{m} a_kX_{k+1}\right\rangle \\
			&= \sum_{j,k = 1}^{m} a_j\overline{a}_k\langle X_{j+1},X_{k+1}\rangle \\
			&= \sum_{j,k = 1}^{m} a_j \overline{a}_k E(X_{j+1}\m \overline{X}_{k+1})\\
			&= \sum_{j,k = 1}^{m}a_j\overline{a}_k E(X_j\overline{X}_k) = \|X\|_{L(P)}^2
		\end{align*}
		Also ist $U$ Isometrie auf $X$.
		\item Zu einer Unitären Transformation $U:M\rightarrow M$ gibt es eine selbstadjungierte Abbildung $T\in B(M)$ mit $\sigma(A)\subset[0,2\pi]$, sodass
		$$U = e^{iA} = \cos(A)+i\sin(A).$$
		Sei $t\in[0,2\pi]\rightarrow P_t\in\{\text{Orthogonalprojektion auf }M\}$ die Spektralschar von $A$, d.h. 
		\begin{align*}
			f(U)X = f(e^{iA})X = \int_{0}^{2\pi} f(e^{is})dP_s(X)
		\end{align*}
		für $f\in C(S^1)$ (Einheitskreis). Sei $X = X_0$, $f(\lambda) = \lambda^n$. Dann
		$$X_n = U^n(X_0) = \int (e^{is})^n dP_sX_0.$$
		Setze $Y(t) = P_s X_0 \in L^2(\Omega,P)$ folgt
		$$X_n = \int_{0}^{2\pi} e^{ins} dY(s),$$
		d.h. der Prozess $(X_n)$ ist die Fouriertransformation des Prozesses $Y(s)$, also die Fouriertransformation des Spektralmaßes von $U$.
		\item Kovarianzfunktion von $(X_n)$. 
		$$\cov(X_n,X_0) = E(X_n\overline{X}_0)\overset{\text{b)}}{=} \int_{0}^{2\pi} e^{ins}d\langle P_s X_0,X_0\rangle = \int_{0}^{2\pi}e^{ins}d\mu_{X_0},$$
		wobei $\mu_{X_0}$ das Spektralmaß zu $A$ ist, d.h. $X_0\in M$ \glqq Kovarianzfunktion des stationären Prozesses $(X_n)$\grqq\ ist die Fouriertransformation des Spektralmaßes $\mu_{X_0}$.
		\item Stochastische Integrale und das Spektralmaß. Für $g= \sum \alpha_n e^{ins}\in L^2[0,2\pi]$ setze $Jg = \sum \alpha_n  X_n \in M\subset L^2(P)$.
		\begin{align*}
			\|Jg\|_{L^2(\Omega,P)}^2 &= \left\langle\sum \alpha_n X_n, \sum\alpha_k X_k\right\rangle\\
			&= \sum \alpha_n\overline\alpha_k E[X_n\overline{X}_k]\\
			&= \sum \alpha_n\overline{\alpha}_k\int e^{i(n-k)s}d\mu_{X_0}\\
			&= \int \left(\sum \alpha_n e^{ins} \right)\overline{\left(\sum \alpha_k e^{iks} \right)}d\mu_{X_0}(s) \\
			&= \left\| \sum \alpha_n e^{ins}\right\|_{L((0,2\pi), \mu_{X_0})}^2,
		\end{align*}
		d.h. $J: L^2[0,2\pi]\rightarrow M$ ist eine Isometrie mit $J(e^{ins}) = X_n$, $n\in \Z$.
		\begin{align*}
			Jg &= \int_{0}^{2\pi}g(s) dY(s)
			= \lim\limits_{n\rightarrow\infty} \sum_{k = 1}^{n}g(s_k^n)[J(s_k^n) - J(s_{k-1}^n)]
		\end{align*}
		mit $s_k^n = \frac{2\pi}{n}k$, $k = 0,...,n$.
		\end{enumerate}
	\end{Beispiel}
	
	\begin{Beispiel}
		Sei $T\in B(H)$ selbstadjungiert und kompakt.
		\begin{enumerate}[a)]
			\item 1. Fall: Alle Eigenwerte sind einfach, $\Kern A = \{0\}$, d.h. $\sigma(A) = \{0,\lambda_n \}$, $\lambda_1>\lambda_2,...,\lambda_n>...>0$, $\lambda_\rightarrow0$. Seien $(h_n)$ die zu $\lambda_n$ gehörigen Eigenvektoren. Aus Funktionalanalysis:
			$$Tf = \sum \lambda_n \langle f,h_n\rangle h_n,$$
			d.h. $U:l^2\rightarrow H$, $U(\alpha_n) = \sum \alpha_n h_n$ ist Isometrie auf $H$, $M:l^2\rightarrow l^2$, $M(\alpha_n) =(\lambda_n\alpha_n)$. Dann ist $A = UMU^{-1}$. Wie hängt dieses frühere Ergebnis mit der allgemeinen Konstruktion der Spektraldarstellung zusammen?
			
			Angabe eines zyklischen Vektors: Fixiere $\gamma_n>0$ mit $\sum_{n = 1}^{\infty}\gamma_n^2 = 1$. Setze $x_0 = \sum_{n = 1}^{\infty}\gamma_n h_n$. Behauptung: $x_0$ ist zyklischer Vektor, d.h. $\overline{\spann\{x_0, A^nx_0\}} = H$. 
			
			\Bew Sei $y= \sum_{j = 1}^{m}\beta_j h_j\in H$. Zu zeigen: Es gibt $\alpha_0,...,\alpha_{m-1}$ mit $y = \sum_{j = 0}^{m-1}\alpha_j A^j x_0$. 
			\begin{align*}
				\sum_{j = 0}^{m}\beta_j h_j = y &= \sum_{j = 0}^{m-1}\alpha_j A^j x_0\\
				&=\sum_{j = 1}^{m-1}\alpha_j \left(\sum_{n=1}^{\infty} \gamma_n A^jh_n \right)\\
				&= \sum_{n = 1}^{\infty}\left(\sum_{j = 0}^{m-1} \gamma_n \lambda_n^j \alpha_j \right)h_n
			\end{align*}
			$\Rightarrow$ $\beta_n = \sum_{j = 0}^{m-1}(\gamma_n \lambda_n^j)\alpha_j$, $n = 1,...,m$. Zu zeigen: Zu $\beta_1,...,\beta_m$ gibt es $\alpha_0,...,\alpha_{m-1}$ mit $0\neq \det(\gamma_n\lambda_n^j)_{n,j} = \left(\prod_{n = 1}^{m} \gamma_n^m \right)\det(\lambda_n^j) = \gamma \neq 0$.
			Spektralmaß von $A$:
			\begin{align*}
				\int f(t)d\mu_{x_0}(t) &= \langle f(A)x_0,x_0\rangle,~ x_0 = \sum\gamma_n h_n\\
				&= \left(\sum \gamma_n^2 \delta_{\lambda_n} \right)f
			\end{align*}
			Also 
			$$\boxed{\mu_{x_0} = \sum_{n = 1}^{\infty}\gamma_n^2 \delta_{\lambda_n}}, \|\mu_{x_0}\|^2 = \sum \gamma_n^2 = 1.$$
			d.h. formal sieht die Spektraldarstellung so aus:
			$$L^2(\R,\mu_x) = L^2(\R,\sum \gamma_n \delta_{\lambda_n} = l^2(\N,\gamma_n^2),$$
			d.h. $\|(\alpha_n)\|_{l^2(\N,\gamma_n^2)} = \left(\sum_{n = 1}^{\infty}|\alpha_n|^2\gamma_n^2 \right)^{1/2}$. $U:L^2(\R,\mu_x)\rightarrow H$. Polynom $p\rightarrow p(A)x_0$, $M g(t) = tg(t)$, $t\in\{\lambda_n:\nLeftarrow\in \N \}$, $Mg(\lambda_n) = \lambda_n g(\lambda_n)$.
			\item $A$ hat mehrfache Eigenwerte, $0<\lambda_j$ sei ein $m_j$-facher Eigenwert. Sei $\Kern A = \{0\}$. Abzählung der Eigenwerte:
			\begin{align*}
				\lambda_{1,1} = \cdots = \lambda_{1,m_1}>\lambda_{2,1} = \cdots = \lambda_{2,m_2}>\cdots
			\end{align*}
			Die Zerlegung von $H$ in Unterräume mit zyklischen Vektoren von $A$: Notation $h_{n,j}$ sei ein Eigenvektor zu $\lambda_{n,j}$, sodass alle $h_{n,j}$ für $j = 1,...,m_n$, $n \in \N$ eine ONB von $H$ ist. Setze ferner $h_{n,j} = 0$ für $j>m_n$
			
			Setze $H_1=\overline{\spann}\{h_{n,1}:n\in \N \}$. $H_2 = \overline{\spann}\{h_{n,2}: n\in \N \}$ und analog $H_3$... Dann $H = \sum_{k = 1}^{\infty} \bigoplus H_k$ und $H_j$ hat den zyklischen Vektor 
			$$x_j = \sum_{j\in \N} \gamma_n h_{n,j}$$
			und $A|_{H_j}$ hat das Spektralmaß $\mu_j = \sum \gamma_n\delta_{\lambda_n, j}$.
		\end{enumerate}
	\end{Beispiel}


\section{Unbeschränkte selbstadjungierte Operatoren}

\begin{Wiederholung}
	~
	\begin{enumerate}[a)]
		\item $X$ Banachraum, $X\supset D(A) \overset{A}{\rightarrow} X$ linear, $D(A)$ linearer Teilraum von $X$. $A$ heißt abgeschlossen, genau dann wenn eine der äquivalenten Eigenschaften erfüllt ist.
		\begin{enumerate}[(i)]
			\item $(D(A),\|\m\|_A)$, $\|x\|_A = \|x\|_X + \|Ax\|_X$ für $x\in D(A)$ ist vollständig.
			\item $\Graph(A) = \{(x,Ax): x\in D(A) \}$ abgeschlossen in $X\times X$.
			\item Falls $x_n\in D(A)$, $x_n\rightarrow x$ in $X$ und $Ax_n\rightarrow y$ in $X$, dann $x\in D(A)$ und $Ax = y$.
		\end{enumerate}
		\item $X\supset D(A)\overset{A}{\rightarrow} X$ linear heißt \textbf{abschließbar}, falls es eine abgeschlossene Erweiterung $A_1$ von $A$ gibt, d.h. $X\supset D(A_1)\overset{A_1}{\rightarrow}X$ ist abgeschlossen und für $x\in D(A)\subset D(A_1)$ gilt $Ax = A_1x$.
		\item $D\subset D(A)$ heißt \textbf{wesentlicher Definitionsbereich} (core), falls $D$ in $D(A)$ dicht bezüglich der Graphennurm $\|\m\|_A$ ist.
	\end{enumerate}
\end{Wiederholung}

\paragraph{Bemerkung.} $A$ ist abschließbar, falls aus $D(A)\ni x_n\rightarrow 0$ in $X$ und $Ax_n\rightarrow y$ in $X$, dass $y = 0$ (Übung).\\

Im folgenden sei $X$ stets ein komplexer Hilbertraum mit $\langle \m,\m\rangle$.

\begin{Definition}
	Sei $A\colon X\supset D(A)\rightarrow X$ ein linearer Operator mit dichtem Definitionsbereich $D(A)$ in $(X,\|\m\|)$.
	\begin{align*}
		D(A^*) = \{x\in X\colon \exists y_x\in X \text{ mit } \langle Ay, x\rangle = \langle y,y_x\rangle\text{ für } y\in D(A) \}
	\end{align*}
	Für $x\in D(A^*)$ setze $A^*x = y_x$.
\end{Definition}

\begin{Bemerkung}
	~
	\begin{enumerate}[a)]
		\item $A^*x$ ist eindeutig bestimmt, denn $\langle y_x,y\rangle = \langle y_x', y\rangle$ für alle $y\in D(A)$ folgt $y_x = y_x'$, da $D(A)$ dicht in $X$. 
		\item $x\in D(A^*)$ bedeutet, dass sich die lineare Abbildung $y\in D(A) \rightarrow \langle Ay,x\rangle$ zu einem beschränkten linearen Funktional auf $X$ fortsetzen lässt, d.h. $\exists C_x$ mit $|\langle Ay, x\rangle|\leq C_x\|y\|$, $y\in D(A)$ $(*)$.
		\item (vergleiche mit Funktionalanalysis)
		$$\xymatrix{
			& X\supset D(A) \ar[r]^A & X \\
			X \ar[r]^J & X'\\
			\supset D(A^*)\ar[r]_J \ar[d]& D(A')\subset\ar[d]\\
			X\ar[r]_J & X'
			}$$
			mit 
	\end{enumerate}
\end{Bemerkung}

\Bew a) Unter $(*)$ gibt es nach dem Rieszschen Darstellungssatz ein $y_x\in X$ mit $\langle Ay, x\rangle = \langle x, y_x\rangle$.  

	
\begin{Prop}[Eigenschaften der Adjungierten]
	Seien $D(A)$, $D(A_1)$ und $D(A_2)$ dicht.
	\begin{enumerate}[a)]
		\item Sei $A_1$ eine Erweiterung von $A_2$, d.h. $D(A_2)\subset D(A_1)$, $A_1 x = A_2x$ für $x\in D(A_2)$ $\Rightarrow$ $A_2^*$ ist eine Erweiterung von $A_1^*$, d.h. $D(A_1^*)\subset D(A_2^*)$, $A_1^*x = A_2^*x$ für $x\in D(A_1^*)$.
		\item $\Graph A^* = [J(\Graph A)]^\perp$, wobei $J\colon X\times X\rightarrow X\times X$, $J(x,y) = (-y,x)$.
		\item $A^*$ ist abgeschlossen.
		\item $A$ ist abschließbar, genau dann, wenn $D(A^*)$ dicht ist in $X$ und dann gilt $\overline{A} = A^{**}$.
		\item $\Bild(A)^\perp = \Kern(A^*)$.
		\item $\sigma(A^*) = \{\bar{\lambda}\colon \lambda\in A \}$ und für $\lambda \in \rho(A)$
		$$(\bar{\lambda} \id - A^*)^{-1} = [(\lambda\id - A)^{-1}]^*.$$
	\end{enumerate}
\end{Prop}
	
\Bew a) $x\in D(A_1^*)$ $\Leftrightarrow$ $\exists C$, sodass
\begin{align*}
	|\langle A_1y,x\rangle| \leq C\|y\| \text{ für } y\in D(A_1)
\end{align*}
gilt dann auch für $y\in D(A_2)$, d.h. $x\in D(A_2^*)$.\\
b) $x\in D(A^*)$, $z\in D(A)$
$$\langle(x, A^* x), J(z,Az)\rangle = \langle (x,A^x), (-Az, z)\rangle = \langle x,-Az\rangle + \langle A^*x, z\langle = - \langle x,Az\rangle + \langle x,Az\rangle = 0.$$
Also $\Graph A^*\subset [J(\Graph A)]^\perp$ in $X\times X$. Umgekehrt sei $(x,y)\in [J(\Graph A)]^\perp$ $\Rightarrow$ für $z\in D(A)$ gilt:
\begin{align*}
	0 = \langle (x,y), J(z, Az)\rangle = \langle (x,y), (-Az, z)\rangle = -\langle x, Az\rangle + \langle y,z\rangle
\end{align*}
$\Rightarrow$ $\forall z\in D(A)$: $\langle x, Az\rangle = \langle y,z\rangle$ $\Rightarrow$ $x\in D(A^*)$, $A^*x = y$.\\
c) $A^*$ ist immer abgeschlossen, da $\Graph A^* = [...]^\perp$ abgeschlossen ist.\\
d) Sei $D(A^*)$ dicht in $X$. $A^{**} = (A^*)^*$  existiert als abgeschlossener Operator nach c). $Ax = A^{**}x$ für $x\in D(A)$. $D(A)\subset D(A^{**})$ nach Definition.\\
e) $y\in D(A^*)$. $y\in \Bild(A)^\perp$ $\Leftrightarrow$ $\forall x\in D(A)$: $\langle y,Ax\rangle = 0$ $\Leftrightarrow$ $\forall x\in P(A)$: $\langle A^*y, x\rangle = 0$ $\Leftrightarrow$ $A^* y = 0$ $\Leftrightarrow$ $y\in \Kern A^*$.\\
f) $\lambda\in \rho(A)$ bedeutet: $\lambda \id - A\colon D(A)\rightarrow X$ ist bijektiv $\Leftrightarrow$ $(\lambda\id - A)^{-1}\colon X\rightarrow (D(A),\|\m\|_A)$ $\Leftrightarrow$ $(\lambda\id - A)^{-1}\colon X\rightarrow X$ ist beschränkt.
\begin{align*}
	S\m (\lambda\id - A) = \id &\Rightarrow [(\bar \lambda\id - A^*)] S^* = \id^* = \id.
\end{align*}
\qed
	
\begin{Definition}
	Sei $A\colon X\supset D(A)\rightarrow X$ mit $D(A)$ dicht in $X$.
	\begin{enumerate}[a)]
		\item $A$ heißt \textbf{symmetrisch}, falls für alle $x,y\in D(A)$ gilt 
		$$\langle Ax,y\rangle = \langle x, Ay\rangle.$$
		\item $A$ heißt \textbf{selbstadjungiert}, falls $A$ symmetrisch ist, und $D(A) = D(A^*)$.
	\end{enumerate}
\end{Definition}

\begin{Bemerkung}
	Falls $A$ symmetrische ist, dann ist $D(A)\subseteq D(A^*)$ und $Ax = A^*x$ für $x\in D(A)$, aber im allgemeinen gilt $D(A)\subsetneq D(A^*)$.
\end{Bemerkung}

\begin{Satz}
	Sei $A$ abgeschlossen mit dichtem Definitionsbereich in $X$. Dann sind Äquivalent.
	\begin{enumerate}[a)]
		\item $A$ ist selbstadjungiert. 
		\item $A$ ist symmetrisch und es gibt $\lambda\in \C$ mit $\lambda,\bar\lambda\in \rho(A)$.
		\item $A$ ist symmetrisch und $\sigma(A)\subset \R$.
	\end{enumerate}
\end{Satz}
	
	
\Bew a) $\Rightarrow$ c): Sei $\lambda \in \C\backslash \R$. Zu zeigen: $\lambda\in\rho(A)$.
\begin{align*}
	\|(\lambda - A)x\|^2 &= \langle (\lambda - A)x, (\lambda - A)x\rangle \\
	&= \lambda\bar{\lambda} \langle x, x, \rangle + \langle Ax, Ax\rangle - \langle Ax,\lambda x\rangle - \langle \lambda x, Ax\rangle \\
	&= |\lambda|^2\|x\|^2 - 2 Re\lambda\langle Ax, x\rangle + \|Ax\|^2\\
	&\geq ((Re\lambda)^2 + (Im\lambda)^2)\|x\|^2 - 2|Re\lambda|\|Ax\|\m\|x\| + \|Ax\|^2\\
	&= (Im\lambda)^2\|x\|^2 + (Re\lambda \|x\| - \|Ax\|)^2\\
	&\geq |Im\lambda |^2 \|x\|^2
\end{align*}
$\|(\lambda - A)x\|\geq |Im\lambda|\|x\|$, $|Im \lambda|\neq 0$ $\Rightarrow$ $(\lambda - A)$ ist injektiv, d.h. $\Bild(\lambda - A)$ ist abgeschlossen. Bleibt zu zeigen, dass $(\lambda - A)$ surjektiv. 
$$\Bild(\lambda - A)^\perp = \Kern(\lambda - A)^* = \Kern(\bar{\lambda}- A) = \{0\},$$
da $A = A^*$, $(\bar{\lambda}- A)$ injektiv (wie oben, mit $\bar{\lambda}$ anstelle von $\lambda$) $\Rightarrow$ $\Bild(\lambda - A)$ dicht, $\Bild(\lambda - A)$ abgeschlossen $\Rightarrow$ $\Bild(\lambda - A) = X$, somit ist $\lambda - A$ bijektiv.

b) $\Rightarrow$ a): Zeige $x\in D(A^*)$ $\Rightarrow$ $x\in D(A)$. Da $\Bild(\lambda-A) = X$, gibt es $y\in D(A)$ sodass
$$(\lambda - A)y = (\lambda - A^*)x.$$
Da $A^*\supseteq A$ gilt $(\lambda - A^*)y = (\lambda - A^*)x$. $(\lambda - A^*) = (\bar\lambda - A)^*$ ist injektiv, das, $\bar{\lambda}\in \rho(A)$ $\Rightarrow$ $x= y\in D(A)$.
\qed

\begin{Beispiel}
	Sei $\Omega\subseteq \C$ Borelmenge, $\mu$ Maß auf $\Omega$, $m\colon \Omega\Rightarrow \R$ Borelmaß. 
	$$D(M) = \{g\in L^2(\Omega, \mu)\colon g\m m\in L^2(\Omega, \mu) \}$$
	$g\in D(M)$: $Mg = m\m g$. Beachte:
	$$\sigma(M) = R_\mu(m) \text{ wesentlicher Definitionsbereich bezüglich }\mu.$$
	$\lambda \notin \sigma(M)$: $\|R(\lambda,\mu)\| = \left\| \frac{1}{\lambda - m(\m)}\right\|_{L^\infty(\Omega,\mu)}$ (wesentliches supremum). $M$ ist symmetrisch, da für $f,g\in D(M)$.
	$$\langle Mg, f\rangle = \int m(x)g(x) f(x) d\mu(x) = \int g(x)m(x)f(x)d\mu(x) = \langle g, Mf\rangle.$$
	$f\in D(M^*)$ $\Leftrightarrow$ $\exists C<\infty$ für alle $g\in L^2(\Omega,\mu)$
	\begin{align*}
		|\langle g, Mf\rangle| &\leq C\|g\|_{L^2}\\
		\left|\int g(x)m(x)f(x)d\mu(x)\right|&\leq C\|g\|_{L^2}
	\end{align*}
	$\Leftrightarrow f\m m\in L^2(\Omega,\mu)$ $\Leftrightarrow$ $f\in D(M)$. Also $D(M) = D(M^*)$.
\end{Beispiel}

\section{Spektralsatz für ubeschränkte selbstadjungierte Operatoren}

\begin{Beispiel}[Laplace Operator]
	Sei $H = L^2(\R^n)$, $D_0(A) = C_c^2(\R^n)$, $f\in D_(A)$: $$Af = \laplace f = \sum_{j = 1}^{n} \frac{\partial^2 f}{\partial x_j^2}.$$
	$A$ auf $D_0(A)$ ist symmetrisch:
	$$\langle \laplace f,g\rangle = -\int_{\R}\sum_{j = 1}^{n}\left(\frac{\partial}{\partial x_j}f \right)\left(\frac{\partial}{\partial x_j}g \right)dx = \langle f, \laplace g\rangle.$$
	$\laplace$ auf $D_0(A)$ selbstadjungiert?
	
	Wiederhole: $S=\R^n$, $\mu=$ Lebesgue Maß.
	$$U\colon L^2(S,\mu)\rightarrow L^2(\R^n, \mu)$$
	$U = \F^{-1}$, $m(x) = -|x|^2$. Dann 
	$$\F(\laplace f) (x) = \left(\sum_{j = 1}^{n} - x_j^2 \right)\f(x) = m(x)\f(x).$$
	Somit ist 
	$$\laplace = \F^{-1}[m(x)\f(x)] = U[M_m]\m U^{-1}$$ 
	mit $M_m g(x) = m(x)g(x) = -|x|^2g(x)$. $D(M_m) = D(M_m^*) = \{g\in L^2\colon m\m g \in L^2 \}$.
	$$D(A) = U(D(M_m)) = \F^{-1}(\{g\colon m\m g\in L^2 \}) = \{f\colon \f(x)\m|x|^2\in L^2 \} = \boxed{\dot H^2 (\R^n)} \text{ (Sobolevraum)}.$$ 
	Somit ist $\laplace$ auf $D_0(A)$ nicht selbstadjungiert, aber symmetrisch. $D_0(\laplace)$ ist ein wesentlicher Definitionsbereich.
\end{Beispiel}

\paragraph{Idee zum Spektralsatz:}
$n(t) = t(1+t^2)^{-1/2}\colon\R\rightarrow (-1,1)$ ist bijektiver, wachsender Homöomorphismus. $n^{-1}\colon (-1,1)\rightarrow \R$ sei die Inverse zu $n$. Sei $A$ unbeschränkter, selbstadjungierter Operator, $t = A$ liefert
\begin{align*}
	n(A) = A(1 + A^2)^{-1/2} =: T\in B(H)
\end{align*}
Wende Spektralsatz für beschränkte Operatoren auf $T$ an. $T\rightsquigarrow A$ (zurückrechnen mit $n^{-1}$).

\begin{Lemma}
	Sei $A$ ein selbstadjungierter Operator auf $H$. Dann ist $I+A^2$ mit Definitionsbereich $D(A^2)$ ein invertierbarer Operator und 
	$$T := (I+A^2)^{-1/2}$$
	erfüllt
	\begin{enumerate}[a)]
		\item $T$ ist beschränkt und selbstadjungiert mit $\|T\| \leq 1$.
		\item Für $x\in D(A)$ gilt $ATx = TAx$ und $T^2 = A^2(I+A^2) = \id-(I+A^2)^{-1}$.
	\end{enumerate}
\end{Lemma}

\Bew Für $x\in D(A^2)$
\begin{align*}
	\|(I+A^2)x\|^2 &= \|x\|^2 + 2\langle Ax,Ax\rangle + \|Ax\|^2 ~(\text{da } A \text{ selbstadjungiert})\\
	&\leq \|x\|^2
\end{align*}
Also ist $I+A^2$ injektiv mit abgeschlossenem Bild. $\Bild(I+A)^2$ ist dicht in $H$, denn $(I+A^2)^* = I+A^2$ auf $D(A^2)$. Da $I+A^2$ injektiv, hat $(I+A^2)$ dichtes Bild.Da $I+A^2$ injektiv ist, hat $(I+A^2)$ ein dichtes Bild 
$$(\Bild B)^\perp = \Kern B^*,~ B = I+A^2.$$
Also ist $I+A^2\colon D(A^2)\rightarrow H$ bijektiv $\Rightarrow$ $(I+A^2)^{-1}\colon H\rightarrow D(A^2)$ und $(I+A^2)^{-1}\in B(H)$ nach Graphensatz. Da $(I+A^2)^{-1}$ beschränkt ist, ist $f((I+A^2)^{-1})$ beschränkt, wobei $f(\lambda) = \lambda^{1/2}$ beschränkt auf $\sigma((I+A^2)^{-1})$, d.h. nach dem Funktionalkalkül für beschränkte Operatoren ist $(I+A^2)^{-1/2}\in B(H)$, $(I+A^2)^{-1}\geq 0$,
\begin{align*}
	 \langle (I+A^2)^{-1}x, x\rangle = \langle x,(I+A^2)x\rangle = \langle x,x\rangle +\langle Ax,Ax\rangle \geq 0, ~x\in D(A^2).
\end{align*}
Nun benutze $\overline{D(A^2)} = H$. $(I+A^2)^{-1/2}$ ist selbstadjungierter, beschränkter Operator. Setze $T:= A(I+A^2)^{-1/2}$. Z.z. $T\in B(H)$, $\|T\|\leq 1$. Für $x\in D(A^2)$:
\begin{align*}
	\|Tx\|^2 &= \langle A(I+A^2)^{-1/2}x, A(I+A^2)^{-1/2}x\rangle \\
	&= \langle (I+A^2)^{-1/2}x, A^2(I+A^2)^{-1/2}x\rangle\\
	&= \langle (I+A^2)^{-1/2} x, (I+A)^{-1/2} A^2 x\rangle \\
	&= \langle x, (I+A^2)^{-1} A^2 x\rangle \\
	&= \langle x,x\rangle - \langle x, (I+A^2)^{-1}x\rangle\\
	&\leq \|x\|^2
\end{align*}
$\Rightarrow$ $\|Tx\| \leq \|x\|$, $\|T\|\leq 1$, da $\overline{D(A^2)} = H$.

\begin{Satz}
	Sei $A$ ein selbstadjungierter Operator auf einem seperablen Hilbertraum. Dann gibt es eine Borelmenge $S\subset \N\times \R$, ein Maß $\mu$ auf $S$, eine unitäre Abbildung $U\colon L^2(S,\mu)\rightarrow H$ und eine stetige Funktion $m\colon B\rightarrow \R$ mit $D(A) = \{Ug\colon g, g\m m\in L^2(S,\mu) \}$. $Ax= U M_m U^{-1}x$, $x\in D(A)$, wobei $M_m g= g\m m$.
\end{Satz}

\Bew $n(\lambda) =  = \lambda(1+\lambda^2)^{-1/2}\colon \R\rightarrow (-1,1)$ ist bijektiv, $n^{-1}\colon (-1,1)\rightarrow \R$ stetig. Setze $T:= A(I+A^2)^{-1/2}\in B(H)$, selbstadjungiert auf $H$ nach Lemma 7.2. Nach dem Spektralsatz für beschränkte selbstadjungierte Operatoren für $T$ gibt es 
\begin{align*}
	U\colon L^2(S,\mu)\rightarrow H,~m_T\colon S\rightarrow [-1,1]\text{ stetig}
\end{align*}
mit $T = UM_{m_T}U^{-1}$, $M_{m_T} g = m_T\m g$. Ziel $T\rightsquigarrow A$. Setze $m = n^{-1}\circ m_T$.
$$D(\tilde A) = \{Ug\colon g, m\m g\in L^2(S) \}.$$
$x\in D(\tilde A)$: $\tilde{A}x = U M_m U^{-1}.$ Zu zeigen $\tilde A = A$ auf $D(A)$.
$$(I+\tilde{A}^2)x = (I+(UM_mU^{-1})^2)x = U(I+M_m^2)U^{-1}x,$$
für $x\in D(I+\tilde{A^2}) = \{Ug\colon g,g(1+m^2)\in L^2 \}$. $m = n^{-1}\circ m_T$, $n\circ m = m_T$, d.h. 
$$\tilde{A}(I+\tilde{A}^2)^{-1/2} = T= A(I+A^2)^{-1/2}.$$
\begin{align*}
	I-(I+\tilde{A}^2)^{-1} &= \tilde{A}^2(I-\tilde{A}^2)^{-1} = T^2\\
	&= A^2(I+A^2)^{-1} = I- (I+A^2)^{-1}\\
	(I+\tilde{A}^2)^{-1} &= (I+A^2)^{-1}\\
	\Rightarrow (I+\tilde{A})^{-1/2} &= (I+A^2)^{-1/2}.
\end{align*}
Für $x= (I+A^2)^{-1/2}y$, $y\in H$ ist $\Bild((I+A^2)^{-1/2})\supset D(A^2)$ ein wesentlicher Definitionsbereich, $\tilde{A} = A$, da beide selbstadjungiert.
\qed

	
	\begin{Satz}
		Sei $A$ selbstadjungierter Operator auf einem seperablen Hilbertraum. Dann gibt es ein Maß $\nu$ auf $\sigma(A)$ und einen Algebrahomomorphismus
		$$\Phi_A\colon L^\infty(\sigma(A), \nu)\rightarrow B(H), ~\Phi_A(f) = f(A)$$
		mit
		\begin{enumerate}[(i)]
			\item $\Phi\left(\frac{1}{\lambda-\m} \right)= R(\lambda, A)$ für $\lambda\notin \sigma(A)$.
			\item $\|\Phi_A(f)\|_{B(H)} = \|f\|_{L^\infty(\sigma(A), \nu)}$.
			\item $\Phi_A(f)^* = \Phi_A(\bar f)$, $\Phi_A(f)$ ist selbstadjungiert $\Leftrightarrow$ $f$ ist reelwertig.
		\end{enumerate}
	\end{Satz}
	
	\Bew Fall 1: Sei $A$ ein Multiplikationsoperator auf $L^2(s,\mu)$: $Mg(\m) = m(\m)g(\m)$, $m\colon S\rightarrow \C$. $D(M) = \{g\colon g, m\m g\in L^2(S,\mu) \}$. $\nu$-Bildmaß von $\mu$ unter $m$, d.h. für $B\subset \sigma(M)$ setze $\nu(B) = \mu(m^{-1}(B))$, $m\colon S\rightarrow \sigma(M)$. $\nu(B) = 0$ $\Leftrightarrow$ $\mu(m^{-1}(B)) = 0$. Also $\|f\|_{L^\infty(\sigma(M),\nu)} = \|f\circ m\|_{L^\infty(S,\mu)}$ für $f\colon \sigma(M)\rightarrow \C$.

	\begin{Definition}
		Sei $[\Phi_M(f)]g = (f\circ m)g$ für $g\in L^2(S,\mu)$, $f\colon \sigma(M)\rightarrow \C$. 
	\end{Definition}
	
	Nachprüfen der Eigenschaften im Falle der berschränkten, selbstadjungierten Operatoren. Fall 2: Sei $A$ beliebiger selbstadjungierter Operator. Nach 7.3 gibt es eine Spektraldarstellung für $A$, d.h. es gibt $U\colon L^2(S,\mu)\rightarrow H$, $Mg=mg$ für $m\colon S\rightarrow \C$ mit 
	$$D(A) = \{Ug\colon g\in D(M) \},~ Ax = UMU^{-1}x, ~x\in D(A).$$
	Definiere den Funktionalkalkül für $A$ wie im beschränkten Fall
	\begin{align*}
		\Phi_A(f) = U\Phi_M(f)U^{-1}
	\end{align*}
	Eigenschaften von $\Phi_A$ wie früher.
	\qed
	
	\begin{Korollar}[Konvergenzeigenschaft]
		Sei $A$ ein selbstadjungierter Operator mit $\Phi_A\colon L^2(\sigma(A), \nu) \rightarrow B(H)$. Für $f_n\in L^\infty(\sigma(A),\nu)$ mit $\|f_n\|_{L^\infty(\sigma(A),\nu)}\leq C$ $f_n(\lambda)\rightarrow f(\lambda)$ fast überall auf $\sigma(A)$ gilt
		$$f_n(A)x\overset{n\rightarrow \infty}{\longrightarrow} f(A)x ~\forall x\in H.$$
	\end{Korollar}
	
	\Bew $U\colon L^2(S,\mu)\rightarrow H$, $m\colon S\rightarrow \sigma(A)$ aus Satz 7.3. Für $g\in L^2(S,\mu)$, $x = Ug\in H$ gilt
	\begin{align*}
		\|f_n(A)x-f(A)x\|_H^2 &= \|f_n(M)g - f(M)g\|_{L^2(S,\mu)}^2\\
		&= \int |[(f_n\circ m)(s) - (f\circ m)(s)]|^2 |g(s)|^2 d\mu(s)\\
		&\rightarrow 0 \text{ für }n\rightarrow \infty
	\end{align*}
	nach Satz von Lebesgue, da $|[(f_n\circ m)(s) - (f\circ m)(s)]|\leq 2C$ und $|g(s)|\in L^\infty$.
	
	\paragraph{Bemerkung}
		Die Darstellung eines selbstadjungierten Operators als Multiplikationsoperator ist \textbf{nicht} eindeutig. Z.B. $L(\R,\mu)$ (Lebesguemaß), $m\colon \R\rightarrow \R$. 
		\begin{align*}
			\int |f\circ m(s)g(s)|^2 ds &= \|f(M)g\|_{L^2(\R, \mu)}^2\\
			&=\int |f\circ m (n(t))g(n(t))|^2 n'(t)dt, ~s = n(t)\\
			&= \int |(f\circ m_1)(t) h(t)|^2 d\mu_1(t)
		\end{align*}
		mit $d\mu_1 = n'(t)d\mu$, $m_1 = m\circ n$, $h = g\circ\nu$. Damit folgt
		\begin{align*}
			U\colon L^2(\R,\mu)&\rightarrow H;~ A= UMU^{-1}\\
			V\colon L^2(\R,\mu)&\rightarrow L^2(\R, \mu_1), \\%~ h\mapsto  Vh = h\circ n^{-1}\\
			%\int |h(t)|^2d\mu_1(t) &= \int |h(n(t))|^2 n'(t)d\mu(t)\\
			%&= \int |g(s)|^2 d\mu(s)
			\|g\|_{L^2(\mu)}^2 &= \int |g(s)|^2d\mu(s)\\
			&=\int |g(n(t))|^2n'(t)d\mu(s)\\
			&= \|g\circ n\|_{L^2(\mu_1)}^2 = \|Vg\|_{L^2(\mu_1)}^2,~ (Vg = g\circ n)\\
			V^{-1}UAU^{-1}V &= M_1 \text{ auf } L^2(S,\mu_1),~ M_1g = (m\circ n)g
		\end{align*}
		$M = VM_1V^{-1}$, $M_1 = V^{-1}M V$ auf $L^2(\R,\mu)$, $M$ auf $L^2(\R,\mu)$.
		
		
	\begin{Beispiel}[Halbgruppen für selbstadjungierte Operatoren]
		Definiere Exponentialfunktion $e^{tA}$  für $A$.
		$$f_t(\lambda) = e^{-t\lambda},~ \lambda\geq 0.$$
		Voraussetzung $\langle Ax, x\rangle \geq 0$, $\sigma(A)\subseteq \R_+$. Nach Funktionalkalkül: $T(t) = f_t(A) = \Phi_A(f_t)$ erfüllt
		\begin{itemize}
			\item $\|T(t)\|_{B(H)} = 1$, denn 
			\begin{align*}
				\|T(t)\| = \|f_t\|_{L^\infty(\R_+,\nu)} = \sup_{\lambda> 0}e^{-t\lambda} = 1.
			\end{align*}
			\item $T(t)\m T(s) = T(s+t)$, denn $f_t(\lambda)f_s(\lambda) = f_{t+s}(\lambda)$.
			\item $T(t)x\rightarrow x$ für $t\rightarrow 0$, $x\in H$, denn $f_t(\lambda)\rightarrow 1$ für $t\rightarrow 0$ aber $|f_t(\lambda)|\leq 1$ auf $\R_+$. Verwende dann 7.5.
			\item $\frac{d}{dt}T_t x = AT_tx$ für $t>0$, denn 
			$$\frac{1}{s}(T_{t+s}x - T_t x) = \Phi_A\left(\frac{f_{t+s}(\m) - f(t)}{s}\right)\rightarrow -\lambda e^{-t\lambda},~ s\rightarrow0$$
			und $q_t(s) \leq 1/t$, $|\lambda e^{-t\lambda} \leq 1/t$. Verwende dann 7.5. Damit gilt für die Lösung des Cauchyproblems $y(t) = T(t)x$:
			$$y'(t) = -Ty(t),~ y(0) = x.$$
		\end{itemize}
	\end{Beispiel}
	
	\section{Störungssatz}
	
	\paragraph{Problem:} Sei beispielsweise $A = \laplace+ \sum b_j\frac{\partial}{\partial x} + V$. Führe die Behandlung von $A$ auf den bekannten Operator $\laplace$ zurück mit Hilfe von \glqq Störungstheorie\grqq.
	
	\paragraph{Formal:} $a$ \glqq gut\grqq, $a+b$, $b$ \glqq Störung\grqq. Sei $\lambda\in \rho(a)$. Frage $\lambda\in a+b$?.
	\begin{align*}
		\mu - (a+b) &= (1-b(\mu-a))(\mu-a)\\
		(\mu-(a+b))^{-1} &= (\mu-a)^{-1}(1-b(\mu-a)^{-1})^{-1} = (\mu- a)^{-1} \sum_{k = 0}^{\infty}[b(\mu-a)^{-1}]^k
	\end{align*}
	
	\begin{Lemma}
		Sei $A$ ein abgeschlossener Operator auf einem Banachraum $X$, $B$ linear auf $X$ mit $D(A)\subset D(B)$. Falls $\lambda\in \rho(A)$ und $\|BR(\lambda, A)\|<1$. Dann ist $A+B$ mit $D(A+B) = D(A)$ abgeschlossen und $\lambda\in \rho(A+B)$ mit 
		$$R(\lambda,A+B) = R(\lambda, A)[I- BR(\lambda,A)]^{-1} = R(\lambda, A)\sum_{k = 0}^{\infty}[BR(\lambda, A)]^k$$
	\end{Lemma}
	
	\Bew Da $\|BR(\lambda, A)\| <1$ gilt nach dem Lemma über die Neumannsche Reihe
	\begin{align*}
		[\id - BR(\lambda, A)]^{-1} = \sum_{k = 0}^{\infty} BR(\lambda, A)^k
	\end{align*}
	Setze $R = R(\lambda,A)[I- BR(\lambda, A)]^{-1}$. Zu zeigen: $R$ ist Inverse von $\lambda-(A+B)$. 
	$$R(\lambda I- A- B) RX = (\lambda I - A- B)R(\lambda, A)(I-BR(\lambda, A))^{-1} = (I- BR(\lambda, A)) (I - BR(\lambda, A)^{-1})x.$$
	\begin{align*}
		R(\lambda\id - A - B) &= R(\lambda, A)\sum_{k = 0}^{\infty} [BR(\lambda, A)]^k (\lambda - A - B) x \\
		&= R(\lambda, A)(x-A-B)x + R(\lambda, A)\sum_{k = 0}^{\infty} (BR(\lambda, A))^k (\lambda - A - B)x\\
		&= x - R(\lambda, A) Bx + R(\lambda, A)\sum_{k = 1}^{\infty}(BR(\lambda, A))^k (\lambda - A)x - \sum R(\lambda, A) (BR(\lambda, A))^kBx\\
		&= x - \sum_{k = 0}^{\infty} R(\lambda, A)(BR(\lambda, A))^k Bx + \sum_{k = 1}^\infty R(\lambda, A) BR(\lambda, A)^{k-1}Bx\\
		&= x, \text{ mit Indexshift}
	\end{align*}
	
	\paragraph{Relative Kleinheitsbedingung.} $\exists$ $a,b>0$, sodass 
	$$\|Bx\| \leq a\|Ax\| + b\|x\|.$$
	
	\begin{Satz}
		Sei $H$ ein Hilbertraum, $H\supset D(A)\overset{A}{\rightarrow} H$ selbstadjungiert, $B$ ein symmetrischer Operator mit $D(B)\supset D(A)$ und es gebe $0<a<1$ und $b<\infty$, sodass für alle $x\in D(A)$
		\begin{equation}
			\|Bx\| \leq a\|Ax\| + b\|x\|.
		\end{equation}
		Dann ist $A+B$ selbstadjungiert.
	\end{Satz}
	
	\Bew Sei $\mu\in \R\backslash\{0\}$, $x\in D(A)$. 
	\begin{equation}
		(\|A+i\mu)x\|^2 = \langle (A+i\mu)x, (A+i\mu)x\rangle = \|Ax\|^2 + \mu^2\|x\|^2 + 2 Re\langle Ax,\mu x\rangle
	\end{equation}
	Für $y\in H$ und $x=(A+i\mu)^{-1}y$ folgt aus (2.3) 
	\begin{align*}
		&\|A(A+i\mu)^{-1}y\|^2 + |\mu| \|(A+i\mu)y\|^2 = \|y\|^2\\
		\Rightarrow& \|A(A+i\mu)^{-1} \|\leq 1,~ \|A+i\mu\|^{-1}\leq \frac{1}{|\mu|}
	\end{align*}
	
	Setze $x=(i\mu + A)^{-1}y$ in (2.2).
	\begin{align*}
		\|B(A + i\mu)^{-1}y\| &\leq \|A(A+i\mu)^{-1}y\| + \frac{b}{|\mu|}\|\mu (A+i\mu)^{-1}y\|\\
		&\leq a \|y\| + \frac{b}{|\mu|}\|y\|\leq (a+\frac{b}{|\mu|})\|y\|
	\end{align*}
	wobei $(a+\frac{b}{|\mu|})<1$ für $\mu$ groß genug. Also $\|B(A+i\mu)^{-1}\|<1$ für $\mu$ groß genug. Nach 8.1: $i\mu\in \rho(A+B)$ für $|\mu|$ groß genug. $A,B$ symmetrisch $\Rightarrow$ $A+B$ symmetrisch auf $D(A+B)= D(A)$. Also ist $A+B$ selbstadjungiert.
	\qed
	
	\begin{Beispiel}
		Sei $V = V_1+V_2$ mit $V_1\in L^2(\R^3)$, $V_2\in L^{\infty}(\R^3)$, $V$ reelwertig. Dann ist der \textbf{Schrödingeroperator} 
		\begin{align*}
			A = \laplace + V,~ Ax(u) = \laplace_x u(x) + V(x) u(x)
		\end{align*}
		mit Potential $V$ selbstadjungiert.
	\end{Beispiel}
	
	\begin{Lemma}
		 Sei $x\in L^2(\R^3)$ und $x\in D(A)=W^{2,2}(\R^3)$. Dann gibt es zu jedem $a>0$ ein $b<\infty$, sodass
		 \begin{align*}
		 	\|x\|_{L^\infty}\leq a\|\laplace x\|_{L^2} + b\|x\|_{L^2}.
		 \end{align*}
	\end{Lemma}
	
	\Bew Sei $x\in S(\R^3)$.
	\begin{align*}
		\|\hat x \|_{L^1(\R^3)}&\leq \left(\int |(1+|u|^2)\hat x(u)|^2du \right)^{1/2}\left(\int \left(\frac{1}{1+|u|^2} \right)^2du \right)^{1/2}\\
		&\leq C(\| |u|^2\hat x(u)\|_{L^2}+\|\hat x\|_{L^2})
	\end{align*}
	Für $r> 0$ setze $x_r(u) = x(u/r)$, $\F(x)(u) = r^3 \hat x (ru)$.
	\begin{align*}
		\|\F(x_r)\|_{L^1} &= \|\hat x\|_{L^1},\\
		\|\F(x_r)\|_{L^2} &= r^{3/2}\|\hat x\|_{L^2},\\
		\||u^2| \F(x_r)(u)\|_{L^2}&= r^{-1/2}\||u|^2\hat x(u)\|_{L^2},\\
		\|\hat x\|_{L^1} &= \|\F(x_r)\|_{L^1} \text{ mit 1. Ungleichung}\\
		&\leq C(\| |u|^2 \F(x_r)(u) \|_{L^2} + \|\F x_r\|_{L^2})\\
		&\leq C(r^{-1/2}\| |u|^2 \hat x (u)\|_{L^2} + r^3 \|\hat x\|_{L^2}).	
	\end{align*}
	Mit dem Riemannschem Lemma und der Plancherelidentität folgt
	\begin{align*}
		\|x\|_{L^\infty} \leq \|\hat x\|_{L^1}\leq C(r^{1/2}\|\laplace x\|_{L^2} + r^3\|x\|_{L^2}),
	\end{align*}
	da $\F^{-1}(\laplace x)(u) = -|u|^2x(u)$. Für $r$ genügend groß gilt dann $a = C r^{-1/2}$, $b = Cr^3$ und die Behauptung folgt.
	\qed
	
	\emph{Beweis des Beispiels:} Wähle $Bx(n) = V(h)x(n)$ in 8.3. Für $x\in D(\laplace)$
	\begin{align*}
		\|V x\|_{L^2}\leq \|V_1\|_{L^2}\|x\|_{\infty} + \|V_2\|_{L^\infty}\|x\|_{L^2}.
	\end{align*}
	Nach 8.4 gilt
	\begin{align*}
		\|Vx\|_{L^2} &\leq \|V_1\|_{L^2}(a\|\laplace x\|_{L^2} + b\|x\|_{L^2}) + \|V_2\|_{L^\infty} \|x\|_{L^\infty}\\
		&= a\|V_1\|_{L^2} \|\laplace x\|_{L^2} + (b\|V_1\|_{L^2} + \|V_2\|_{L^\infty}) \|x\|_{L^2}.
	\end{align*}
	Für $a\|V_1\|_{L^2}<1$ ist die Voraussetzung von 8.4 erfüllt und $\laplace+V$ ist selbstadjungiert.
	
	\subsection*{Grenzen der selbstadjungierten Theorie}
	
	\begin{itemize}
		\item Sei $H$ ein Hilbertraum, $A=\laplace$, $B = \sum_{j = 1}^{d}b_j \partial_{x_j}$. $A+B$ ist selbstadjungiert, wenn $b_j=b_j(u)$, $u\in \R^d$, denn $A+B$ ist nicht normal.
		\item $H = L^2(\R^d)$, $Ax(u) = a(u)\laplace x(u)$. Dieses $A$ ist nicht normal, denn 
		$$\langle Ax, y\rangle = \sum_{j = 1}^{d}\int_{\R^d} a(u)\partial_j^2 xa(u) \overline{y(u)}du = \sum_{j =1}^{d} \int x(u)\left(\frac{\partial^2}{\partial x_j} a(u) \overline{y(u)}\right)du$$
		\item $X= L^p(\R^d)$, $A=\laplace$ Keine Theorie selbstadjungierter Operatoren. $p \in [1,2)$, $p\neq p'$: $\F\colon L^p(\R^d)\rightarrow L^p(\R^d)$.
	\end{itemize}
	
	\begin{Definition}
		Sei $X\supset D(A)\overset{A}{\rightarrow} X$ abgeschlossener Operator, $X$ ein Banachraum. 
		\begin{itemize}
			\item $A$ sei injektiv und $\Bild(A)$ sei dicht in $X$.
			\item $\sigma(A)\subset \Sigma_\om$, $\Sigma_{\om} = \{\lambda\in \C\colon |\arg\lambda|<\om \}$.
			\item $\exists C$: $\|R(\lambda, A)\|\leq C/|\lambda|$, falls $\lambda\notin \overline{\Sigma_\om}$ $(+)$.
		\end{itemize}
		Dann heißt $A$ ein sektorieller Operator mit Winkel $\om$. Notation: $\om(A) = \inf \rho$, für die Bedingung $(+)$ erfüllt ist.
	\end{Definition}
	
	\begin{Beispiel}
		~
		\begin{enumerate}[a)]
			\item $A$ sei selbstadjungierter Operator, $A\geq 0$. Dann ist $\sigma(A)\subset \R_+$. $\|R(\lambda, A)\| = \frac{1}{|Im\lambda|}$, $Im\lambda=0$, also $\om(A) = 0$.
			\item Sei $iA$ selbstadjungiert. Dann ist $\sigma(A)\subset i\R$. Dann ist $\om=\pi/2$.
			\item Sei $A$ normal. $\sigma(A) \subset \Sigma_\omega$. Sann ist $A$ sektoriell mit Winkel $\om$. $\om(A) =$ der kleinste Sektor, der $\sigma(A)$ enthält.
			\item $A$ beschränkt und $\sigma(A)\subset \Sigma_\om$, $0\notin \sigma(A)$. Dann ist $A$ sektoriell und $\om(A) \leq \varphi$. Zum Beweis mit $(\lambda + A)^{-1} = \sum_{n = 0}^{\infty}\lambda^{-n-1}A^n$. 
		    \item $X = L^p(\R^d)$, $A = \laplace$ sektoriell (später).
		    \item Der nächste Störungssatz gibt viele sektorielle Operatoren an.
		\end{enumerate}
	\end{Beispiel}
	
	\paragraph{Notation.} $A$ sei sektoriell vom Winkel $\varphi$. 
	\begin{align*}
		M_\varphi(A) = \sup \{\|\lambda R(\lambda, A)\|, \|A R(\lambda, A)\|: \lambda \notin \Sigma_\om\} <\infty\\
		AR(\lambda, A) = \lambda R(\lambda, A)-\id.
	\end{align*}
	
	\begin{Satz}
		Sei $A$ ein sektorieller Operator mit Winkel $\varphi$ und $B$ sei Linear, $D(B)\supset D(A)$. Sei $a>0$ mit $a\m M_\varphi(A)<1$.
		\begin{enumerate}[a)]
			\item Falls $\|Bx\|\leq a\|Ax\|$ $(++)$ für $x\in D(A)$, dann ist $A+B$ auf sektoriell und $\om(A+B)\leq \varphi$.
			\item Falls $\|Bx\|\leq a\|Ax\| +b\|x\|$ für $x\in D(A)$, dann gibt es ein $\mu_0>0$, sodass $A+B+\mu_0$ sektoriell und $\om(A+B+\mu_0)\leq \varphi$.
		\end{enumerate}
	\end{Satz}
	
	\Bew a) Zeige $\|B R(\lambda, A)\|<1$.
	\begin{align*}
		\|(A+B)x\|&\leq (1+a)\|Ax\| \text{ nach } (++)\\
		\|(A+B)x\|&\geq \|Ax\|-\|Bx\| \geq (1-a)\|Ax\|, ~ a<1
	\end{align*}
	$A$ injektiv $\Rightarrow$ $(A+B)$ injektiv. $Tx = B A^{-1}x$, $x\in \Bild(A^{-1})$. Aus $(++)$ folgt für $x=A^{-1}y$
	$$\|Ty\|\leq a\|y\|.$$
	Also kann $T\in B(X)$ fortgesetzt werden, $\|T\|\leq a<1$. $I+T\in B(X)$ ist Isomorphismus. $\Bild(A+B) = (I+T)(\Bild(A))$, $\Bild(A+B)$ ist dicht. $Tx = B^{-1}x$, $x\in \Bild(A)$, $\|T\|_{B(X)}\leq a$. Zu zeigen: $\|BR(\lambda,A)\| <1$, $|\arg\lambda|>\varphi$. 
	\begin{align*}
		\|BR(\lambda,A) &\leq \|(BA^{-1})AR(\lambda, A)x\|\\
		&\leq \|T\| \m\|AR(\lambda, A)\|\\
		&\leq a M_\varphi(A) <1
	\end{align*}
	Nach Lemma 8.1
	\begin{align*}
		\lambda R(\lambda, A) &= \lambda R(\lambda, A) \left(\sum_{k = 0}^{\infty} [BA^{-1}A R(\lambda, A)]^k \right)\\
		\|\lambda R(\lambda, A+B)\|&\leq \|\lambda R(\lambda, A)\| + \|\lambda R(\lambda, A)\| \sum_{k = 1}^{\infty} \|T\|^k\|AR(\lambda,A)\|^k\\
		&\leq M_\varphi(A) + (1- a M_\varphi(A))^{-1},
	\end{align*}
	also $\om(A+B)\leq \om(A)$.
	b) $|\arg(\mu)|>\varphi$, $x\in D(A)$ 
	\begin{align*}
		\|BR(\mu,A)x\| &\leq a\|AR(\mu, A)x\| + \frac{b}{|\mu|} \|\mu R(\mu, A)x\|\\
		&\leq \left[a M_\varphi(A) + \frac{b}{|\mu|}M_\varphi(A)\right] \|x\|<\|x\|
	\end{align*}
	für $|\mu|$ groß genug. Dazu wähle $\mu_0>0$, sodass für $\mu= \lambda-\mu_0$ mit $\arg(\lambda)\geq \varphi$ gilt:
	$$\left(a+\frac{b}{|\mu|} \right)M_\varphi(A)<1.$$
	Also folgt aus Lemma 8.1 
	$$\|\lambda R(\lambda,\mu_0+A+B)\| \leq M_\varphi(A) + \frac{1}{1-c}.$$
	Somit $A+B+\mu_0$ sektoriell 
	\qed
	
	\paragraph{Erinnerung:}
	Sei $A$ linear und injektiv auf einem Banachraum $X$. $A$ heißt \textbf{sektoriell}, falls $D(A)$ und $\Bild(A)$ dicht sind in $X$. $\sigma(A)\subset \Sigma_\om$.
	$$M_\om (A) = \sup\{\|\lambda R(\lambda, A)\|\colon \lambda \notin\Sigma_\om \} < \infty.$$
	
	
	\begin{Beispiel}
		Sei $b_1,...,b_d\in L^\infty(\R^d)$, $B\colon f\rightarrow \sum_{j = 1}^{d} b_j(x)\frac{\partial}{\partial x_j} f$, $X = L^2(\R^d)$, $A = -\laplace$, $D(A) = W^{2,2}(\R^d)$, $D(B) \supset D(A)$. A ist selbstadjungiert, $B$, $A+B$ \textbf{nicht} normal, aber $A+B$ ist sektoriell nach 8.8 b), da $\|Bx\|< a\| Ax\| + b\|x\|$ (Übung).
	\end{Beispiel}
	
	\newpage
	
	\chapter{Der holomorphe Funktionalkalkül}
	
	\section{Dunfordkalkül für beschränkte Operatoren}
	
	Klar: $p(\lambda) = \sum_{j = 0}^{n}a_n \lambda^n$, $p(A) = \sum_{j = 0}^{n} a_n A^n$ für $A\in B(X)$. \\
	Nächster Schritt: $f(\lambda) = \sum_{n = 0}^{\infty} a_n \lambda^n$ mit Konvergenzradius $R$. Sei $A\in B(X)$, $X$ Banachraum, mit $r(A) = \sup_{\lambda\in \sigma(A)}|\lambda|< R$. Dann $\limsup \|a_n A^n\|^{1/n}\leq c< 1$, denn $r(A) = \lim\|A^n\|^{1/n}$, $\|a_n A^n\|<c^n$, $c<1$ $\Rightarrow$ 
	$$\boxed{f(A) = \sum_{n = 0}^{\infty}a_nA^n.}$$
	z.B.
	\begin{align*}
		e^\lambda &= \sum_{n = 0}^{\infty} \frac{1}{n!}\lambda^n,~ R = \infty,~ A\in B(X)\\
		e^A &= \sum_{n = 0}^{\infty} \frac{1}{n!} A^n
	\end{align*}
	
	\paragraph{Ziel:} 
	$$f(\lambda) = \frac{1}{2\pi i}\int_\Gamma \frac{1}{\mu-\lambda} f(\mu) d\mu.$$
	Für $\lambda= A$ erhält man Formal $\Gamma\subset \rho(A)$:
	$$f(A) = \frac{1}{2\pi i}\int_{\Gamma} R(\mu,A) f(\mu)d\mu.$$
	
	\begin{Lemma}
		Sei $A\in B(X)$ und $f\colon \{\lambda\colon |\lambda|<R \}\rightarrow \C$ analytisch und $r(A)<R'<R$.
		\begin{align*}
			\sum_{n = 0}^{\infty} a_n A^n = f(A) = \frac{1}{2\pi i}\int_{|\mu|= R'} f(\mu) R(\mu,A) d\mu.
		\end{align*}
	\end{Lemma}
	
	\Bew Für $|\mu| = R'$, $|\mu|> r(A)$ gilt 
	\begin{align*}
		R(\mu, A) &= \sum_{m = 0}^{\infty} \mu^{-m-1}A^m\\
		\frac{1}{2\pi i} \int_{|\mu| =R'} f(\mu) R(\mu, A) d\mu &= \frac{1}{2\pi i} \int_{|\mu| = R'} \left(\sum_{n = 0}^{\infty}a_n \mu^n \right) \left(\sum_{m = 0}^{\infty} \mu^{-m-1} A^m \right)d\mu\\
		&= \frac{1}{2\pi i}\sum_{n,m = 0}^{\infty} a_n A^m \left(\int_{|\mu|= R'} \mu^{n-m-1} d\mu \right)
	\end{align*}
	\qed
	
	\begin{Bemerkung}[Dunford Integrale]
		Gegeben: $\sigma(A)$. Wähle offene Mengen $U,V\in \C$ mit 
		$$U\supset \bar V\supset V\supset \sigma(A).$$
		$U\supset \partial V$, $\partial V\cap \sigma(A) =\emptyset$, $f\colon U\rightarrow \C$ analytisch. $\partial V$ besteht aus endlich vielen glatten Kurven, die positiv orientiert sind, d.h. $\sigma(A)$ liegt innerhalb von $\partial V$, $\C\backslash U$ ist außerhalb. Dann existiert das Dunford Integral $f(A) = \frac{1}{2\pi i}\int_{\partial V} f(\mu) R(\mu, A)d\mu$. Nach dem Satz von Cauchy hängt $f(A)$ nicht von der speziellen Wahl von $V$ ab, da 
		\begin{align*}
			(x',f(A)x) = \frac{1}{2\pi i}\int_{\partial V} f(\mu)(x', R(\lambda, A)x)d\mu
		\end{align*}
		für alle $x\in X$, $x'\in X'$ und damit auch für $A$.
	\end{Bemerkung}
	
	Standardvoraussetzungen: $\sigma(A)\subset V\subset \bar V\subset U$, $U,V$ offen.
	\begin{itemize}
		\item $\partial V$ besteht aus endlich vielen glatten Kurven. 
		\item $\partial V$ ist positiv orientiert.
	\end{itemize}
	
	\paragraph{Notation.} $U$ offen, $H(U) =$ holomorphen Funktionen. $S\subset \C$ abgeschlossen, $f\in H(S)$ $\Leftrightarrow$ Es gibt eine offene Umgebung $U\supset S$ und $f\colon U\rightarrow \C$ holomorph.
	
	$H^\infty(U)$ - beschränkte, analytische Funktionen auf $U$ mit $\|f\|_\infty = \sup_{\lambda\in U} |f(\lambda)|$, $X$ Banachraum
	
	\begin{Definition}
		Zu $A\in B(X)$ und $\sigma(A)\subset V\subset U$ wie in 1.2 definiere das \textbf{Dunford Integral} für $f\in H(U)$ durch
		$$\Phi_A(f)=\frac{1}{2\pi i}\int_{\partial V} f(\lambda) R(\lambda, A) d\lambda.$$
	\end{Definition}
	
	
	\begin{Satz}
		$\Phi_A\colon H(U) \rightarrow B(X)$ hat die Eigenschaften:
		\begin{enumerate}[(i)]
			\item $\Phi(1) = \id_X$, für $f(\lambda) = \lambda$ gilt $f(A) = A$.
			\item $\Phi$ ist linear und multiplikativ.
			\item $\|\Phi_A(f)\|\leq C_A \sup_{\lambda\in V} f(\lambda)$.
			\item Falls $f_n, f\in H(U)$ und $f_n\rightarrow f$ gleichmäßig auf $\bar V$, dann $\Phi(f_n) \rightarrow \Phi(f)$ in $B(X)$. $\Phi_A$ heißt der \textbf{Dunfordsche Funktionalkalkül}.
		\end{enumerate}
	\end{Satz}

	\paragraph{Notation:} $\Phi_A(f) =: f(A)$.
	
	\Bew (i) Wende 1.1 auf $f(\lambda) = 1$ an und $f(\lambda) = \lambda$.\\
	(ii) Linearität klar, $f,g\in H(U)$. 
	\begin{align*}
		f(A) \m g(A) &= \left(\frac{1}{2\pi i} \int_{\partial V} f(\mu) R(\mu, A) d\mu\right) \left(\frac{1}{2\pi i} \int_{\partial W} g(\lambda) R(\lambda, A) d\lambda \right) \\
		&= \left(\frac{1}{2\pi i}\right)^2 \int_{\partial V}\int_{\partial W} f(\mu) g(\lambda) R(\mu, A) R(\lambda, A) d\lambda d\mu\\
		&= \left(\frac{1}{2\pi i}\right)^2 \int_{\partial V} \int_{\partial W} \frac{f(\mu)g(\lambda)}{\lambda-\mu} [R(\mu, A)-R(\lambda, A)]d\lambda d\mu \text{ nach Resolventenformel}\\
		&= \frac{1}{2\pi i} \int_{\partial W} g(\lambda) \left(\frac{1}{2\pi i} \int_{\partial V} \frac{f(\mu)}{\mu-\lambda}d\mu \right) R(\lambda, A) d\lambda \\
		&+ \frac{1}{2\pi i} \int_{\partial V} f(\mu) \left(\frac{1}{2\pi i} \int_{\partial W} \frac{g(\lambda)}{\lambda-\mu} d\lambda \right) R(\mu, A) d\mu\\
		&= \frac{1}{2\pi i} \int_{\partial W} g(\lambda) \boxed{f(\lambda)} R(\lambda, A) d\lambda \\
		&= \frac{1}{2\pi i}\int_{\partial W} (gf)(\lambda) R(\lambda, A) d\lambda= g(A) f(A)
	\end{align*}
	Dabei sind $V,W$ offen mit $U\supset V\supset \bar W\supset W \supset \sigma(A)$.\\
	(iii) 
	\begin{align*}
		\|f(A)\| &\leq \frac{1}{2\pi} \int_{\partial V} |f(\lambda)| \|R(\lambda, A)\| d|lambda|\\
		&\leq \frac{1}{2\pi} l(\partial V) \sup_{\lambda\in \partial V} |f(\lambda)| \sup_{\lambda\in\partial V} \|R(\lambda, A)\|\\
		&\leq C\sup_{\lambda\in \bar V} |f(\lambda)|	
	\end{align*}
	(iv) $\|f_n(A) - f(A)\|\leq C\sup_{\lambda\in \partial V} |f_n(\lambda)-f(\lambda)| \rightarrow 0$ für $n\rightarrow\infty$.
	
	\begin{Korollar}
		Für $A \in B(X)$ und $f\in H(\sigma(A))$ gilt
		\begin{align*}
			\sigma(f(A)) = f(\sigma(A))
		\end{align*}
		\textbf{Spektralabbildungssatz}.
	\end{Korollar}
	
	\Bew \glqq $\subseteq$\grqq: Sei $\lambda \notin f(\sigma(A))$. Dann gilt $g(\mu) = (f(\mu)-\lambda)^{-1}\in H(\sigma(A))$ (denn es gibt ein offenes $U\supset \sigma(A)$ mit $\lambda \notin f(U)$).
	\begin{align*}
		g(A)(f(A)-\lambda) &= \Phi_A(g(\m) (f(\m)-\lambda)) = \Phi(1) = \id_X
	\end{align*}
	Ebenso $(f(A)-\lambda) g(A) = \id_X$. Also ist $g(A) = (f(A)-\lambda)^{-1}$, d.h. $\lambda \notin \sigma(f(A))$.\\
	\glqq $\supseteq$\grqq: Sei $\lambda \in \sigma(A)$. Definiere $g\in H(\sigma(A))$.
	$$g(\mu) = \left\{\begin{array}{l}
	\frac{f(\mu)-f(\lambda)}{\mu-\lambda},~ \mu\neq\lambda\\
	f(\lambda),~ \mu = \lambda
	\end{array} \right.$$
	Dann
	\begin{align*}
		f(A) - f(\lambda) =\Phi_A((f(\m)-f(\lambda)) = \Phi_A((\m-\lambda)g(\m)) = (A-\lambda)g(A) = g(A)(A-\lambda)
	\end{align*}
	$\lambda\in \sigma(A)$ $\Rightarrow$ $A-\lambda$ nicht invertierbar $\Rightarrow$ $f(A)-f(\lambda)$ nicht invertierbar, d.h. $f(\lambda)\in \sigma(f(A))$.
	\qed
	
	\begin{Satz}[Spektralprojektion]
		Sei $\sigma(A)= \sigma_1\overset{\m}{\cup} \sigma_2$ mit $\sigma_1,\sigma_2$ kompakt. Dann gibt es Projektionen $P_1,P_2\in  B(X)$ mit 
		\begin{enumerate}
			\item $\id= P_1+P_2$, $P_1\m P_2 = P_2\m P_1 = 0$.
			\item $X = X_1\oplus X_2$ mit $X_j = \Bild P_j$.
			\item $A(X_j) \subset X_j$, $j = 1,2$.
			\item Für $A_j = A|_{X_j}$ gilt $\sigma(A_j) = \sigma_j$, $j =1,2$.
		\end{enumerate}
	\end{Satz}
	
	\paragraph{Bemerkung.} $A$ kompakt $\sigma(A) = \{0, \lambda_j, j\in J\}$, $H_j\rightarrow 0$. 
	
	\Bew 1. Wähle offene Mengen $U_1,U_2$ mit $\sigma_j\subset U_j$, $U_1\cap U_2 = \emptyset$. Setze 
	$$f_j(\lambda) =\left\{ \begin{array}{l}
	1,~ \lambda \in U_j\\0, \text{ sonst}
	\end{array} \right. .$$
	Dann ist $f_j\in H(\sigma(A))$, $f_j^2 = f_j$, $f_1 + f_2 = 1_{\chi_{U_1\cup U_2}}$. Mit Dunford Kalkül folgt für $P_j = f_j(A)$:
	$$P_j^2 = P_j, P_1+P_2 = \id_X, P_1\m P_2 = 0.$$ 
	2. $X_j = \Bild P_j$
	Dann ist $X_1\subset \Kern P_2$, $X_2\subset \Kern P_1$, $X_1+X_2 = X$ $\Rightarrow$ $X=X_1\oplus X_2$, $X_1,X_2$ abgeschlossene Unterräume.\\
	3. $x\in X_j$: $Ax = A(P_jx) = (P_j)(Ax)$ $\Rightarrow$ $Ax\in X_j$.\\
	4. $A_j = A|_{X_j}$, $A_j = g_j(A)$, $g_j(\lambda) = \lambda f_j(\lambda)$. Mit Spektralabbildungssatz folgt $\sigma(A_j) = g_j(\sigma(A)) = \sigma_j$.
	\qed
	
	\begin{Satz}
		Seien $A,B\in B(X)$, $AB = BA$. Dann $f(A)B = Bf(A)$ für alle $f\in H(\sigma(A))$.
	\end{Satz}
	
	\Bew 
	\begin{align*}
		f(A)B &= \left(\frac{1}{2\pi i} \int_{\partial V} R(\lambda, A) d\lambda \right) B\\
		&= \frac{1}{2\pi i} \int_{\partial V} f(\lambda) R(\lambda, A)B d\lambda\\
		&= \frac{1}{2\pi i} \int_{\partial V} f(\lambda) BR(\lambda, A) d\lambda \\
		&= B\left(\frac{1}{2\pi i} \int_{\partial V} f(\lambda) R(\lambda, A) d\lambda \right)
	\end{align*}
	
	\begin{Satz}
		Sei $A\in B(X)$, $f\in H(\sigma(A))$, $g\in H(\sigma(f(A)))$. Dann ist $g\circ f\in H(\sigma(A))$ und 
		$$(g\circ f) (A)  = g(f(A)).$$
	\end{Satz}
	
	\Bew Nach 1.5: $\sigma(f(A)) = f(\sigma(A))$. Wähle Umgebung $V\supset \sigma(f(A))$ und $U\supset \sigma(A)$ mit $U,V$ offen, $\overline{f(U)}\subset V$. Zunächst sei $\lambda\in \partial V$. 
	$$R(\lambda, f(A)) = \frac{1}{2\pi i} \int_{\partial U} \frac{1}{\lambda - f(\mu)} R(\mu, A) d\mu.$$
	\begin{align*}
		g(f(A)) &= \frac{1}{2\pi i} \int_{\partial V} g(\lambda) R(\lambda, f(A)) d\lambda\\
		&= \frac{1}{2\pi i} \int_{\partial V} g(\lambda)\left( \frac{1}{2\pi i} \int_{\partial U} \frac{1}{\lambda - f(\mu)} R(\mu,A) d\mu \right)d\lambda\\
		&= \frac{1}{2\pi i} \int_{\partial U} \left(\frac{1}{2\pi i} \int_{\partial V} \frac{g(\lambda)}{\lambda - f(\mu)} d\lambda \right) R(\mu, A) d\mu\\
		&=(g\circ f)(A)
	\end{align*}
	
	
	\begin{Prop}[Dualität]
		Sei $A\in B(X)$, $U\supset \sigma(A)$ offen. Für $f\in H(U)$ analytisch gilt:
		\begin{enumerate}[a)]
			\item $f(A)' = f(A')$.
			\item Falls $X$ Hilbertraum mit $\langle \m,\m\rangle$, dann ist $f(A)^* = \tilde f(A^*)$ mit $\tilde f(\lambda) = \overline{f(\bar \lambda)}$ (z.B. $f(\lambda) = \sum a_n\lambda^n$ $\rightsquigarrow$ $\tilde f(\lambda) = \sum \bar a_n \lambda^n$), $\tilde f$ wieder analytisch, $\tilde f\in H(U)$.
		\end{enumerate}
	\end{Prop}
	
	\Bew b) 
	\begin{align*}
		\langle f(A)x, y\rangle &= \langle \frac{1}{1\pi i}\int_{\partial U} f(\lambda) R(\lambda, A) x d\lambda, y\rangle \\
		&= \frac{1}{2\pi i} \int_{\partial U} f(\lambda) \langle R(\lambda, A)x, y\rangle d\lambda\\
		&= \frac{1}{2\pi i}\int_{\partial U} f(\lambda)\langle x, R(\lambda,A)^*y\rangle d\lambda\\
		&= \overline{\langle x, \left(\frac{1}{2\pi i} \right)} \int_{\partial U} \overline{f(\lambda)} R(\bar \lambda, A^*) yd\lambda, y\rangle\\
		&= \langle x, \frac{1}{2\pi i} \int_{\partial U} \overline{f(\bar{\lambda})} R(\lambda, A^*) yd\lambda\rangle \\
		&= \langle x, \tilde f(A)y\rangle 
	\end{align*}
	\qed
	
	\begin{Definition}[Fréchet Ableitung]
		$t\in (a,b)\rightarrow f(t)\in X$, $X$ Banachraum. 
		\begin{align*}
			(+)~~f'(t_0) = \lim\limits_{t\rightarrow t_0} \frac{f(t) - f(t_0)}{t-t_0},~ t_0\in (a,b)
		\end{align*}
		$f$ ist \textbf{Fréchet differenzierbar} in $t_0$, falls $(+)$ existiert.
	\end{Definition}
	
	\begin{Prop}
		Sei $A\in B(X), U\supset \sigma(A)$ offen. Gegeben sei $t\in (a,b)\rightarrow f_t\in H^\infty(U)$.
		\begin{enumerate}[a)]
			\item Falls $t\in (a,b)\rightarrow f_t\in H(U)$ Fréchet differenzierbar ist, dann ist auch 
			$$t\in (a,b)\rightarrow f_t(A)\in B(X)$$
			Fréchet differenzierbar.
			\item Sei $t\in (a,b) \rightarrow f_t\in H^\infty(U)$ integrierbar und 
			$$g= \int_{a}^{b}f_t dt\in H^\infty(U),$$
			dann ist auch 
			$$t\in (a,b)\rightarrow f_t(A)\in B(X)$$
			integrierbar und 
			$$\int_{a}^{b} f_t(A) dt = g(A)$$
			(integrierbar: $f_t$ ist stetiges Riemannintegral oder Bochner Integral).
		\end{enumerate}
	\end{Prop}
	
	\Bew Übung.
	\qed
	
	\section{Halbgruppen mit beschränkten Erzeugern}
	
	Sei $A\in B(X)$. Wie versteht man \glqq $e^{tA}$\grqq?
	\begin{enumerate}
		\item Halbgruppen: Setze $e_z(\mu) = e^{z\mu}$. Definiere $e^{zA} = e_z(A)$, $e_z\in H(\C)$ für $z\in \C$. Nach Lemma 1.1: 
		$$e^{zA} = \sum_{n = 0}^{\infty} \frac{1}{n!} z^n A^n).$$
		Eigenschaften: $e_{z_1}(\mu) e_{z_2}(\mu) = e_{z_1+z_2}(\mu)$ $\Rightarrow$ $e^{z_1} e^{z_2 A} = e^{(z_1+z_2) A}$ (Halbgruppeneigenschaft).
		$e_z(\mu)\rightarrow 1$ für $z\rightarrow 0$ gleichmäßig auf kompaktem $\sigma(A)\subset \C$ $\Rightarrow$ $e^{zA}\rightarrow \id_X$ in $B(X)$ für $z\rightarrow 0$ ($C_0$-Stetigkeit der Halbgruppe).
		\item Halbgruppe und Resolvente: $r_\lambda(\mu):=\frac{1}{\lambda-\mu}$, $\lambda \notin \sigma(A)$, $r_\lambda\in H(\sigma(A))$. 
		$$R(\lambda,A) = r_\lambda(A) = (\lambda - A)^{-1},~ \lambda\notin\sigma(A).$$
		Resolventengleichung:
		\begin{align*}
			(\lambda_1-\mu)^{-1} - (\lambda_2-\mu)^{-1} &= (\lambda_2-\lambda_1) (\lambda_1-\mu)^{-1} (\lambda_2-\mu)^{-1}\\
			\Rightarrow (\lambda_1-A)^{-1} - (\lambda_2-A)^{-1} &= (\lambda_2-\lambda_1) (\lambda_1-A)^{-1} (\lambda_2-A)^{-1}
		\end{align*}
		\begin{align*}
			(\lambda-\mu)^{-1} &= \int_{0}^{\infty} e^{-\lambda t} e^{t\mu} dt\\
			\Rightarrow R(\lambda, A) &= \int_{0}^{\infty} e^{-\lambda t} e^{tA} dt
		\end{align*}
		(die Resolvente ist die Laplacetransformierte der Halbgruppe). Umgekehrt:
		\begin{align*}
			e^{tA} = \frac{1}{2\pi i}\int_{\partial U}e^{t\lambda} R(\lambda, A)d\lambda
		\end{align*}
		Widder-Umkehrformel.
		\begin{align*}
			\left(\frac{n}{t} \right)^n \left(\frac{n}{t}-\mu \right)^{-n} &= \left(1-\frac{t\mu}{n} \right)^{-n} \overset{n\rightarrow \infty }{\longrightarrow} e^{t\mu}\\
			\Rightarrow \left(\frac{n}{t} \right)^n R\left(\frac{n}{t}, A\right) &\overset{n\rightarrow\infty}{\longrightarrow} e^{tA}\text{ in } B(X),~ t>0
		\end{align*}
		\item Das Cauchy Problem: Gegeben sei $A\in B(X)$, $y_0\in X$. Gesucht ist $y\colon \R_+\rightarrow X$ mit 
		$$(+) ~y'(t) = Ay(t) \text{ für } t\geq 0,~ y(0) = y_0.$$
		Beispiel $x = \C^n$, $y_k'(t) = \sum_{k = 1}^{n} a_{kj}y_k(t)$, $y_k(0) = y_k$, $k = 1,...,n$. Damit ist $A = (a_{kj}) \in B(\C^n)$, $y_0=\left(\begin{matrix}
		y_0\\
		y_1
		\end{matrix} \right)$. Lösung von $(+)$: $\frac{d}{dt}e^{t\mu} = \mu e^{t\mu}$. Nach 1.11 gilt 
		$$T(t) = e^{tA}, ~ \frac{d}{dt}T(t) = AT(t).$$
		Setze $y(t) = T(t)y_0$. Dann ist $y'(t) = Ay(t)$. $y(0) = T(0) y_0 = y_0$ und $y$ löst $(+)$.\\
		\item Stabilität: Im Falle des obigen Beispiel ist bekannt: Ist
		$$s(A) = \sup \{Re(\lambda)\colon \lambda \in \sigma(A) \}<0$$
		dann gilt für die Lösung $y(t)$ von $(+)$ $\|y(t)\|\rightarrow 0$ für $t\rightarrow \infty$ (exponentiell).
	\end{enumerate}
	
	\begin{Prop}
		$A\in B(X)$, $s(A)<0$. Dann gibt es ein $\delta>0$, sodass $\|e^{tA}\|\leq Ce{-\delta t}$ für $t\geq 0$, $C<\infty$.
	\end{Prop}
	
	\Bew Mit Spektralabbildungssatz gilt $\sigma(e^{TA}) = \{e^{T\lambda}\colon \lambda \in \sigma(A) \}$.
	\begin{align*}
		r(e^{TA}) = \sup \{|e^{T(Re\lambda + iIm \lambda)}|\colon \lambda \in \sigma(A) \} = e^{Ts(A)}
	\end{align*}
	\begin{align*}
		\lim\limits_{n\rightarrow \infty} \|e{nTA}\|^{1/n} &= \lim\limits_{n\rightarrow\infty} \|(e^{TA})^n\|^{1/n} = r(e^{TA})
		= e^{Ts(A)}
	\end{align*}
	Wähle $\delta>$, sodass $s(A)<-\delta<0$ und $n$ mit $\| e^{nTA}\|\leq e^{-Tn\delta}$. Für beliebiges $t>0$ schreibe $t = nT + u$ mit $u\in [0,T]$ $\Rightarrow$ $\|e^{tA}\| \leq \|e^{ntA}\|\m \|e^{uA}\|\leq \sup\{\|e^{uA}\|\colon  \}$.
	
	\paragraph{5. Gebrochene Potenzen von Operatoren}
	
	Sei $A\in B(X)$, $(-\infty,0] \subset \rho(A)$. Sei $\mu_\alpha = \mu^\alpha$ analytisch auf $\C\backslash\R_-$, $\alpha\in (0,1)$. Definiere $A^\alpha:= \mu_\alpha(A)$.
	\begin{align*}
		\mu^{-\alpha} = \frac{1}{2\pi i}\int_{\partial \Sigma_\sigma} z^{-\alpha} (z-\mu)^{-1}dz, ~ Re\mu> 0
	\end{align*}
	Für $\sigma\rightarrow \pi$:
	\begin{align*}
		\mu^{-\alpha} = \frac{\sin(\pi a)}{\pi} \int_{0}^{\infty} t^{-\alpha} (t+\mu)^{-1}dt
	\end{align*}
	Für $(1-\alpha)$ statt $\alpha$, $0<1-\alpha < 1$. 
	$$(2)~ \mu^{\alpha} = \frac{\sin(\pi\alpha)}{\pi} \int_{0}^{\infty} t^{\alpha-1} \mu(t+\mu)^{-1}dt$$
	Für $\mu = A$.
	\begin{align*}
		(1)~A^{-\alpha} &= \frac{\sin(\pi\alpha)}{\pi}\int_{0}^{\infty} t^{-\alpha} R(-t,A)dt\\
		(2)~A^\alpha &= \frac{\sin(\pi\alpha)}{\pi} \int_{0}^{\infty} A R(t,-A)dt\\
		(3)~ \Gamma(\alpha) &= \int_{0}^{\infty} s^{\alpha-1} e^{-s} ds, ~ 0<\alpha<1.
	\end{align*}
	Setze $s = \mu t$: 
	\begin{align*}
		\mu^{-\alpha} &= \frac{1}{\Gamma(\alpha)}\int_{0}^{\infty} t^{\alpha-1} e^{-\mu t}dt\\
		A^{-\alpha} &= \frac{1}{\Gamma(\alpha)} \int_{0}^{\infty} t^{-\alpha-1} e^{-tA}dt.
	\end{align*}
	$\alpha = \pm1/2$:
	\begin{align*}
		A^{1/2} &= \frac{\sin(\pi/2)}{\pi} \int_{0}^{\infty} t^{-1/2} AR(t,-A)dt\\
		A^{-1/2} &= \frac{1}{\Gamma(1/2)} \int_{0}^{\infty} t^{-1/2-1} AR(t,-A)dt
	\end{align*}
	
	\paragraph{5. Cauchy Problem 2. Ordnung}
	Skalar: $a>0$. 
	$$y''(t) = -ay(t), ~ y(0) = x_0,~ y'(0) = y_0$$
	hat die Lösung:
	$$y(t) = \cos(a^{1/2}t)x_0 + \sin(a^{1/2}t)(a^{-1/2}y_0).$$
	Nun sei $A\in B(X)$, $(-\infty, 0]\subset \rho(A)$. 
	$$y(t) = \cos(A^{1/2}t) x_0 + \sin(A^{1/2}t)(A^{-1/2}y_0)$$
	ist die Lösung für das Operatorproblem.
	
	
	\section{Funktionalkalkül für sektorielle Operatoren}
	
	\paragraph{Problem:} $A$ unbeschränkt auf $X$, z.B. $A = (-\laplace)$ auf $L^p(\Omega)$.
	
	\begin{Definition}
		$X$ Banachraum, $X\supset D(A)\overset{A}{\rightarrow} X$ abgeschlossen. $A$ sei injektiv, $D(A)$ und $R(A) = \Bild A$ dicht in $X$. Dann ist $A$ \textbf{sektoriell} für ein $\om\in (0,\pi)$, falls 
		\begin{itemize}
			\item $\sigma(A)\subset \Sigma_\om = \{\lambda\in \C\colon |\arg\lambda|<\om \}$.
			\item Es gibt eine Konstante $C_\om$ mit $\|\lambda R(\lambda, A)\| \leq C_\om$ für alle $\lambda\notin \Sigma_\om$.
		\end{itemize}
	\end{Definition}
	
	\paragraph{Notation:} $\om(A) = \inf\{\om \text{ wie oben}\}$.
	
	\begin{Beispiel}
		~
		\begin{enumerate}[a)]
			\item Sei $A_0$ selbstadjungiert auf einem Hilbertraum $X$, $\theta\in (-\pi/2,\pi/2)$. Setze $A= e^{i\theta} A_0$. Also $\sigma(A)\subset e^{i\theta}\R$. Für $\theta \neq 0$ ist $A$ nicht selbstadjungiert, aber sektoriell. 
			$$\|R(\lambda, A)\| = \sup_{s>0}\frac{1}{|\lambda-e^{i\theta}s|} = \frac{1}{d}.$$
			\item $X= L^p(S,\mu)$, $Ax = mx$, $m\colon S\rightarrow \Sigma_\om\backslash\{0\}$. $A$ ist sektoriell auf $X = L^p(S,\mu)$, $\om(A)= \om$, denn 
			$$\|R(\lambda, A)\|= \sup_{s\in S} \frac{1}{|\lambda - m(s)|} = \frac{1}{\sin(\theta-\om)} \frac{1}{|\lambda|}.$$
			\item $A\in B(X)$, $\sigma(A)\subset \Sigma_\om \backslash\{0\}$, denn für $|\lambda >2\|A\|$ gilt
			\begin{align*}
				R(\lambda, A) =\left\| \sum_{n = 0}^{\infty} \frac{1}{\lambda^{n+1}}A^n\right\| \leq \frac{1}{|\lambda|} \sum_{n = 0}^\infty \left(\frac{\|A\|}{|\lambda|} \right)^n
			\end{align*}
			\item $A$ sei \glqq Erzeuger\grqq\ einer Operatorenhalbgruppe $T(t)$ mit $\|T(t)\|\leq C$ für $t>0$, d.h. 
			$$R(\lambda, A) = \int_{0}^{\infty} e^{-\lambda t}T(t) dt \text{ (Laplacetransformation)}.$$
			Ein solcher Erzeuger ist immer sektoriell, denn 
			\begin{align*}
				\|\lambda R(\lambda, A)\|\leq \int_0^\infty |\lambda e^{-\lambda t}| \|T(t)\| dt\leq C.
			\end{align*}
		\end{enumerate}
	\end{Beispiel}
	
	\begin{Definition}
		~
		\begin{enumerate}[a)]
			\item $H^{\infty}(\Sigma_\sigma) = \{f\colon \Sigma_\sigma\rightarrow \C \text{ analytisch und beschränkt auf }\Sigma_\sigma \}$, $\|f\|_{H^\infty} = \sup_{\lambda\in \Sigma_\sigma} |f(\lambda)| <\infty$.
			\item Sei $\rho(\lambda) = \lambda/(1+\lambda)^2$. 
			$$H_0^\infty(\Sigma_\sigma) = \{f\in H^\infty(\Sigma_\sigma)\colon \forall \lambda \in \Sigma_\sigma \exists \epsilon >0\colon |f(\lambda)|\leq C|\rho(\lambda)|^\epsilon \}$$
			$f\in H_0^\infty(\Sigma_\sigma)$ bedeutet insbesondere:
			$|f(\lambda)|\leq C|\lambda|^\epsilon$ für $|\lambda|$ klein und $|f(\lambda)|\leq C/|\lambda|^\epsilon$ für $|\lambda|$ groß.
		\end{enumerate}
	\end{Definition}
	
	\begin{Definition}
		Sei $A$ $\om$-sektoriell und $f\in H_0^\infty(\Sigma_\sigma)$. Sei weiter $\om < \sigma$ und wähle $\om<\nu<\sigma$.
		$$\Phi_A(f) = f(A) = \frac{1}{2\pi i}\int_{\partial \Sigma_\nu} f(\lambda) R(\lambda,A) d\lambda. ~ (+)$$
	\end{Definition}
	
	\paragraph{Bemerkung.} $f(A)\in B(X)$ definiert durch das Bochner Integral $(+)$, denn $\lambda\in \partial \Sigma_\nu$: 
	\begin{align*}
		|f(\lambda)| \|R(\lambda, A)\|\leq \|f\|_{H^{\infty}} \frac{1}{|\lambda|}.
	\end{align*}
	und das Integral $(+)$ existiert als uneigentliches Integral.
	
	\begin{Satz}
		Sei $A$ $\om$-sektoriell und $\om<\nu<\sigma$. Die Abbildung 
		$$f\in H_0^{\infty}(\Sigma_\sigma)\rightarrow f(A)\in B(X)$$ ist linea, multiplikativ, stetig und es gibt die Konvergenzeigenschaft: Falls für $\lambda\in \Sigma_\sigma$, $f_n,f\in H_0^{\infty}(\Sigma_\sigma)$ mit $|f_n(\lambda)|\leq C$, $f_n(\lambda)\rightarrow f(\lambda)$, dann $f_n(A)\rightarrow f(A)$ in $B(X)$.
	\end{Satz}
	
	\paragraph{Bemerkung.} Diese Aussagen \glqq erweitern\grqq\ den Dunfordkalkül. Sei $A\in B(X)$, $0\in \rho(A)$, $\sigma(A)\subset \Sigma_\om$. Hier:
	\begin{align*}
		f(A) &=\frac{1}{2\pi i} \int_{\Gamma} f(\lambda) R(\lambda, A) d\lambda\\
		&= \frac{1}{2\pi i} \int_{\partial \Sigma_\nu} f(\lambda) R(\lambda, A) d\lambda
	\end{align*}
	Für $A\in B(X)$: Dunfort = neuer Kalkül.
	
	\paragraph{Notation.} $\Sigma(\sigma_1,\sigma_2) = \{\lambda\in \C \colon \sigma_1<\arg\lambda <\sigma_2 \}\backslash\{0\}$, $\partial\Sigma(\sigma_1,\sigma_2)$ erhält eine positive Orientierung.
	
	\begin{Lemma}
		Sei $G\colon \Sigma(\sigma_1,\sigma_2)\rightarrow Y$, analytisch, $Y$ Banachraum (äquivalent: $x'\circ G\colon \Sigma(\sigma_1,\sigma_2)\rightarrow\C$ analytisch, für alle $x'\in X'$).
		\begin{enumerate}[a)]
			\item (Satz von Cauchy): Sei $\|G(\lambda)\| \leq C \frac{1}{1+|\lambda|^{1+\epsilon}}$ für $\lambda \in \overline{\Sigma(\sigma_1,\sigma_2)}$, $\epsilon>0$. Dann ist 
			$$\int_{\partial\Sigma(\sigma_1,\sigma_2)} \sigma(\lambda) d\lambda = 0.$$
			\item (Cauchy Formel): Sei $\|G(\lambda)\| \leq C\frac{1}{1+|\lambda|^{1+\epsilon}}$ für $\lambda \in \overline{\Sigma(\sigma_1,\sigma_2)}$. Dann ist 
			$$G(\mu) = \frac{1}{2\pi i}\int \frac{G(\lambda)}{\lambda- \mu}d\lambda.$$
		\end{enumerate}
	\end{Lemma}
	
	\Bew $\Gamma_n = \{\lambda \in \partial \Sigma(\sigma_1,\sigma_2)\colon 1/n\leq|\lambda|\leq n \}$, $\gamma_n =\{\lambda \in \Sigma(\sigma_1,\sigma_2)\colon |\lambda|=1/n \text{ oder } |\lambda| = n \}$.\\
	a) 
	\begin{align*}
		\int_{\partial \Sigma(\sigma_1,\sigma_2)} G(\lambda) d\lambda = \lim_{n\rightarrow \infty} \int_{\Gamma_n} G(\lambda) d\lambda + \lim\limits_{n\rightarrow\infty} \int_{\gamma_n} G(\lambda) d\lambda
	\end{align*}
	denn 
	\begin{align*}
		\left\| \int_{\gamma_n} G\lambda d\lambda\right\| &\leq \int_{\gamma_{n,1}} \|G(\lambda)\| d|\lambda| + \int_{\gamma_{n,2}} \|G(\lambda)\|d\lambda \\
		&\leq l(\gamma_{n,1}) \sup\|G(\lambda)\| + l(\gamma_{n,2}) \sup\|G(\lambda)\|\\
		&\leq 2\pi (\sigma_1+\sigma_2) \frac{1}{n} C + 2\pi(\sigma_1+\sigma_2) n \frac{1}{1 + |n|^{1+\epsilon}} \rightarrow 0.
	\end{align*}
	b) analog.
	\qed
	
	\subsection*{Ausblick zum $H^\infty$-Kalkül}
	
	\begin{Definition}
		A hat einen beschränkten $H^\infty$-Kalkül, falls $A$ sektoriell ist, $\sigma(A)\subset \Sigma_\sigma$ und es eine Konstante $C$ gibt, mit 
		$$\|f(A)\|_{B(X)} \leq C\|f\|_{H^\infty(\Sigma_\sigma)}$$ 
		für alle $f\in H_0^\infty(\Sigma_\sigma)$.
	\end{Definition}
	
	\paragraph{Bemerkung.} Falls $(+)$ gilt, dann lässt sich der zunächst auf einer dichten Teilmenge $D(A)\cap R(A)$ definierte Operator
	$$f(A) = \frac{1}{2\pi i} \int_{\partial \Sigma_\sigma} f(\lambda) R(\lambda, A) xd\lambda,$$
	$x\in D(A)\cap R(A)$ zu einem beschränkten Operator $f(A) \in B(X)$ fortsetzen, das heißt $f\in H^\infty(\Sigma_\sigma)\rightarrow B(X)$ ($H^\infty$-Kalkül).\\
	
	Für einen sektoriellen Operator $A$ auf einem Hilbertraum $X$ sind äquivalent:
	\begin{enumerate}[(i)]
		\item $A$ hat einen beschränkten $H^\infty$-Kalkül mit $\sigma<\pi/2$.
		\item Es gibt auf $X$ ein zu $\langle\m,\m\rangle$ äquivalentes Skalarprodukt $[\m,\m]$, sodass 
		\begin{align*}
			[Ax,x]\in \Sigma_\sigma ~ \forall x\in D(A).
		\end{align*}
		\item $A$ erzeugt eine analytische Halbgruppe $T(z)$, $z\in \Sigma_{\pi/2 - \sigma}$, für die in einer äquivalenten Hilbertraumnorm $\||\m|\|$ gilt: $\|| T(z)|\| \leq 1$.
		\item $A$ hat eine \textbf{Dilation} zu einem Multiplikationsoperator, d.h. es gibt eine isometrische Einbettung $J\colon X\rightarrow L^2(\R, X)$, eine Orthogonalprojektion $P = JJ^*$ von $L^2(\R, X)$ auf $J(X)$ und einen Multiplikationsoperator $Mf(t) = (it)^\alpha f(t)$, ($\alpha >2\sigma/\pi$), sodass
		$$\xymatrix{
			L^2(\R,X)\ar[r]^M & L^2(\R, X)\\
			J_X\ar[r] \ar[u] & X\ar[u]_J
			}$$
	\end{enumerate}
	
	\begin{Satz}
		Sei $A$ $\om$-sektoriell, $\om<\nu<\sigma$. Dann ist die Abbildung $\Phi_A\colon f\in H_0^\infty(\Sigma_\sigma) \rightarrow f(A) \in B(X)$ linear, multiplikativ, beschränkt und es gilt (Konvergenzeigenschaft): $f, f\in H_0^\infty(\Sigma_\sigma)$ mit $|f_n(\lambda)|\leq C$ und $f_n(\lambda)\rightarrow f(\lambda)$, $\lambda\in \Sigma_\sigma$. Außerdem gilt für alle $g\in H_0^\infty(\Sigma_\sigma)$:
		$$\lim\limits_{n\rightarrow\infty} \Phi_A(gf_n) = \Phi(gf)\text{ in } B(X).$$
	\end{Satz}
	
	\paragraph{Beachte:}$(+)$ $\|f(A)\|_{B(X)} \leq C_\nu \int_{\partial \Sigma_\sigma} |f(\lambda)|\frac{d|\lambda|}{|lambda|}$
	
	\Bew 
	$$\|f(A)\|\leq \frac{1}{2\pi} \int_{\partial\Sigma_\sigma} \frac{|f(\lambda)|}{|\lambda|} \|\lambda R(\lambda, A)\| d|\lambda| \leq \left(\frac{C_\nu'}{2\pi} \right) \int_{\partial\Sigma_\sigma} \frac{|f(\lambda)|}{|\lambda|} d\lambda.$$
	 Also gilt $(+)$, falls $f\in H_0^\infty(\Sigma_\sigma)$.
	 
	 $\Phi_A$ ist offensichtlich linear.
	 
	 $\Phi_A$ ist multiplikativ: $\Phi_A(g)\Phi_A(f) = \Phi_A(gf)$. Wähle $\nu_1,\nu_2$ mit $\om <\nu_1 <\nu_2 <\sigma$. 
	 \begin{align*}
	 	\Phi_A(f)\m \Phi_A(g) &= \left(\frac{1}{2\pi i} \int_{\partial\Sigma_{\nu_1}} f(\mu) R(\mu,A) d\mu \right) \left(\frac{1}{2\pi i}\int_{\partial \Sigma_{\nu_2}} g(\lambda) R(\lambda, A) d\lambda \right)\\
	 	&= \frac{1}{(2\pi i)^2} \int_{\partial \Sigma_{\nu_1}} \left(
	 	\int_{\partial\Sigma_{\nu_2}}f(\mu) g(\lambda) R(\mu,A) R(\lambda,A) d\lambda \right)d\mu \\
	 	&=\frac{1}{2\pi i}\int_{\partial\Sigma_{\nu_2}} g(x) \left[\frac{1}{2\pi i} \left(\int_{\partial\Sigma_{\nu_1}} \frac{f(\mu)}{\mu-\lambda} d\mu \right) \right]R(\lambda, A) d\lambda\\
	 	&+\frac{1}{2\pi i} \int_{\partial\Sigma_{\nu_1}} f(\mu) \left[\frac{1}{2\pi i} \int_{\partial\Sigma_{\nu_2}} \frac{g(\lambda)}{\lambda-\mu} d\lambda \right]R(\mu,A) d\mu\\
	 	&= \frac{1}{2\pi i} \int_{\partial\Sigma_{\nu_1}} f(\mu)g(\mu) R(\mu, A) d\mu\\
	 	&= \Phi_A(f\m g).
	 \end{align*}
	 Konvergenzeigenschaft:
	 \begin{align*}
	 	H_n(\lambda) &= \left[f_n(\lambda) - f(\lambda) \right] g(\lambda) R(\lambda, A) \in B(X),~ \lambda\in \partial\Sigma_\nu\\
	 	\|H_n(X)\| &\leq |f_n(\lambda) - f(\lambda)| \frac{g(\lambda)}{|\lambda|} \|\lambda R(\lambda, A)\|,\\
	 	\|H_n(\lambda)\|&\rightarrow 0, ~ n\rightarrow \infty.\\
	 	\|\Phi_A(f_n g)- \Phi_A(fg) \| &\leq  \frac{1}{2\pi} \| \int_{\partial\Sigma_\nu} (f_n(\lambda) - f(\lambda)) g(\lambda) R(\lambda, A) d\lambda\| \\
	 	&\leq \frac{1}{2\pi} \int_{\partial \Sigma_\nu} \|H_n(\lambda)\| d|\lambda|\rightarrow 0
	 \end{align*}
	 \qed
	 
	 \paragraph{Ziel:} $e^{-\lambda}$, $\frac{1}{\mu-\lambda}$, rationale Funktionen. 
	 
	 \begin{Lemma}
	 	Sei $f\in H^\infty(\Sigma_\sigma)$ und es erfülle
	 	\begin{itemize}
	 		\item $f$ ist analytisch in $B(0,2\delta)$, $\delta >0$.
	 		\item $|f(\lambda)| \leq \frac{C}{(1+|\lambda|)^\epsilon}$, $\epsilon> 0$. 
	 	\end{itemize}
	 	Für einen $\om$-sektoriellen Operator $A$ mit $\om<\delta$und $x\in \Bild(A)$ gilt:
	 	\begin{enumerate}[(i)]
	 		\item $I(f)x = \int_{\partial\Sigma\nu} f(\lambda) R(\lambda, A) d\lambda = \int_{\partial \Gamma_\delta} f(\lambda) R(\lambda, A)d\lambda$.
	 		\item $\|I(f)x\|\leq M\left(\inf\int_{\Gamma_\epsilon} |f(x)| \frac{d|\lambda|}{|\lambda|} \right) \|x\|$ mit 
	 		$$M = \sup_{\lambda\in \partial\Sigma_\nu} \|\lambda(R(\lambda, A)), ~ \Gamma_\delta= \partial[\Sigma_\nu\cup B(0,\delta)].$$
	 	\end{enumerate}
	 \end{Lemma}
	 
	 \Bew (i) Da $x\in \Bild A$, wähle $y\in D(A)$ mit $x = Ay$. $\Rightarrow$ $R(\lambda, A)x = R(\lambda, A) Ay = \lambda R(\lambda, A)y - y$. $\|R(\lambda, A)x\|$ ist beschränkt in $B(0,2\delta)$.
	 \begin{align*}
	 	\int_{\partial\Sigma_\nu} f(\lambda) R(\lambda, A)x d\lambda &= \lim\limits_{\epsilon\rightarrow 0} \int_{\partial\Gamma_\epsilon} f(\lambda) R(\lambda, A) xd\lambda \text{ (nach Lebesgue)}\\
	 	&= \int_{\Gamma_\delta} f(\lambda) R(\lambda, A) d\lambda \text{ (nach Chauchy)}\\
	 	&\Rightarrow \text{ (i)}
	 \end{align*}
	 
	 (ii) $$\|I(f)x \| \leq \int_{\partial \Gamma_\delta} \frac{|f(\lambda)|}{|\lambda|} \|\lambda R(\lambda, A)x\| d|\lambda| \leq \tilde{C}' \|x\|.$$
	 \qed
	 
	 \begin{Beispiel}
	 	Sei $\rho(\lambda) = \frac{\lambda}{1+\lambda^2}$. Dann gilt
	 	\begin{align*}
	 		\rho(A) = A(I+A)^{-2}\text{ und}\\
	 		\Bild(\rho(A)) = D(A)\cap \Bild(A).
	 	\end{align*}
	 \end{Beispiel}
	 
	 \Bew Wende Lemma 3.9 an auf $f(\lambda) = \frac{1}{1+\lambda}$.
	 $$R(-1,A)x = -\frac{1}{2\pi i} \int_{\partial \Sigma_\nu} \frac{1}{\lambda + 1} R(\lambda, A) x d\lambda$$
	 wobei $\nu\in (\om, \pi)$. 
	 \begin{align*}
	 	A R(-1,A)^2 x &= \frac{1}{2\pi i}\int_{\partial\Sigma_\nu} \frac{-1}{\lambda + 1} A R(-1, A) R(\lambda, A) xd\lambda\\
	 	&=  \frac{1}{2\pi i} \int_{\partial\Sigma_\nu} \frac{1}{\lambda + 1} \frac{AR(\lambda, A) x - A R(-1,A)x}{\lambda + 1} d\lambda\\
	 	&= \frac{1}{2\pi i} \int_{\partial\Sigma_\nu} \frac{1}{(1+\lambda)^2} R(\lambda, A) d\lambda + \frac{1}{2\pi i} \int_{\partial\Sigma_\nu} \frac{1}{1+\lambda^2}d\lambda (-x-AR(-1,A))x\\
	 	&= \rho(A) + 0
	 \end{align*}
	 $AR(-1,A) x = A(I+A)^{-2}x = \rho(A)x$ für $x\in \Bild(A)$, dicht in $X$.
	 
	 Bleibt zu zeigen: $R(A(1+A)^{-2}) = D(A)\cap \Bild(A)$.\\
	 \glqq $\subseteq$\grqq: Sei $y\in \Bild(A(I+A)^{-2})$. Wähle $x\in X$ mit $y= A(I+A)^{-2}x$ $\Rightarrow$ $y = (I+A)^{-1} z\in D(A)$, da $z= A(I+A)^{-1}\in X$. $y= A\tilde{z} \in \Bild(A)$, da $\tilde{z} = (I+A)^{-1}(I+A)^{-1}x\in D(A)$.\\
	 \glqq $\supseteq$\grqq $y:= (1+A) A^{-1}(I+A)x = (1+A)A^{-1}x + (1+A)x$. $A(I+A)^{-2}y = A(I+A)^{-1}(I+A)^{-1}(I+A)A^{-1} (I+A)x = x$.
	 \qed
	 
	 \begin{Definition}
	 	Sei $A$ $\om$-sektoriell auf $X$ $f\in H^\infty(\Sigma_\sigma)$, $\sigma> \om$, $x\in D(A)\cap \Bild(A)$. Setze $f(A)x = (f\m\rho)(A)y$ mit $y=\rho(A)^{-1}x$. Also ist $f(A)$ auf der dichten Teilmenge $D(A)\cap \Bild(A)$ definiert. 
	 \end{Definition}
	 
	 \begin{Bemerkung}
	 	\begin{enumerate}
	 		\item $f\rho\in H_0^\infty(\Sigma_\sigma)$. Also $(f\rho)(A)\in B(X)$ nach 3.7, 3.8. 
	 		\item Falls $f\in H_0^{\infty}(\Sigma_\sigma)$, dann stimmen die beiden Definitionen 
	 		$$f(A)x = (f\rho)(\rho^{-1}(x))$$
	 		überein, da $\Phi_A$ auf $H_0^\infty$ multiplikativ ist. 
	 		\item $f(A)$, $D(f(A)) = D(A)\cap \Bild(A)$ ist abschließbar (Übung).
	 	\end{enumerate}
	 \end{Bemerkung}
	 
	 \begin{Definition}
	 	Sei $A$ $\om$-sektoriell auf $X$, $\sigma > \om$. 
	 	$$H_A^\infty(\Sigma_\sigma) = \{f\in H^\infty(\Sigma_\sigma)\colon f(A) \in B(X) \}.$$
	 	Falls $H_A^\infty(\Sigma_\sigma) = H^\infty(\Sigma_\sigma)$, dann hat $A$ einen \textbf{beschränkten} $\mathbf{H^\infty}$\textbf{-Funktionalkalkül}.
	 \end{Definition}
	
	\begin{Bemerkung}
		$H_0^\infty(\Sigma_\sigma)\subset H_A^\infty(\Sigma_\sigma)$. $f$ wie in 3.9 ist auch in $H_A^\infty(\Sigma_\sigma)$. Es gibt auch Operatoren, die keinen beschränkten $H^\infty$-Kalkül haben. Diese sind jedoch exotisch.
	\end{Bemerkung}
	
	\begin{Beispiel}
		~
		\begin{enumerate}[a)]
			\item $H_0^\infty(\Sigma_\sigma) \subseteq H_A^\infty(\Sigma_\sigma)$.
			\item $r_\mu(\lambda) = \frac{1}{\mu-\lambda}$, $\mu\notin \overline{\Sigma_\sigma}\colon r_\mu(A) = R(\mu, A)$.\\
			\item $f(\lambda)\equiv 1$: $f(A) = \id_X$.
			\item $f(\lambda) = \prod_{j = 1}^n \frac{a_j-\lambda}{\mu_j-\lambda}$  mit $a_j \in \C$, $\mu_j\notin \overline{\Sigma_\sigma}$: $f(A) = \prod_{j = 1}^{ n }(a_j- A) R(\mu_j, A)$ [Übung mit Lemma 3.9].
		\end{enumerate}
	\end{Beispiel}
	
	\begin{Satz}
		$\Phi_A\colon H_A^\infty(\Sigma_\sigma) \rightarrow B(X)$ ist linear, multiplikativ und es gilt: 
		\begin{itemize}
			\item[$(+)$] Sei $|f_n(\lambda)|\leq C_1$, $f_n(\lambda)\rightarrow f(\lambda)$ für $\lambda\in \Sigma_\sigma$ mit $f_n\in H_A^\infty(\Sigma_\sigma)$ und $f\in H^\infty(\Sigma_\sigma)$ und $\|f_n(A)\|_{B(X)}\leq C$.
		\end{itemize}
		Dann ist $f\in H_A^\infty(\Sigma_\sigma)$ und $\|f(A)\|_{B(X)}\leq C$.
	\end{Satz}
	
	\paragraph{Folgerung.} Die lineare, unbeschränkte Abbildung $H^\infty(\Sigma_\sigma)\supseteq H_A^\infty(\Sigma_\sigma)\overset{\Phi_A}{\rightarrow} B(X)$ ist abgeschlossen.
	
	\Bew Linear und multiplikativ wie bisher. Zu $(+)$: Es gilt für $\rho(\lambda) = \lambda(1+\lambda^2)^{-1}\in H_0^\infty(\Sigma_\sigma)$
	\begin{align*}
		\|(f_n\rho)(A) - (f\rho)(A)\|_{B(X)} & \rightarrow 0\\
		\Rightarrow \|f_n(A)x-f(A)x\|_X& \Rightarrow 0
	\end{align*}
	für $x= \rho(A)y\in D(A)\cap\Bild(A)$ für $y\in X$.
	\begin{align*}
		&\Rightarrow \lim \|f_n(A)x\| = \|f(A)x\| \text{ für } x\in D(A)\cap \Bild(A)\\
		&\Rightarrow \|f(A)x\|\leq C\|X\| \text{ für } x\text{ in dichter Teilmenge von } X\\
		&\Rightarrow f(A)\in B(X), ~ \|f(A)\|\leq C.
	\end{align*}
	
	\begin{Definition}
		$A$ hat einen beschränkten $H^\infty(\Sigma_\sigma)$-Kalkül, falls $H_A^\infty(\Sigma_\sigma) = H^\infty(\Sigma_\sigma)$.
	\end{Definition}
	
	\begin{Beispiel}
		~
		\begin{enumerate}[a)]
			\item Fourierreihen auf $L^p(0,1)$:
			\begin{align*}
				e_n(t) = e^{2\pi int},~ t\in [0,1], \text{ ONB von } L^2[0,1]
			\end{align*}
			Fourierkoeffizienten für $x\in L^p(0,1)$: 
			$$\hat x(n) = \int_{0}^{1}e^{-2\pi int} x(t)dt,~ n \in \Z.$$
			Fourierdarstellung: 
			$$x= \sum_{n\in \Z} \hat x(n) e_n \text{ in } L^p(0,1), ~n<\infty.$$
			\item Diagonaloperatoren bezüglich der Fourierreihen. Sei $\alpha_n > 0$, $n\in \Z$, gegeben. Definiere einen unbeschränkten Operator $A$ auf $L^p(0,1)$:  
			$$D(A)=\{x\in L^p(0,1)\colon \sum_{n\in \Z}\hat x(n) \alpha_n e_n \text{ konvergiert in } L^p \}.$$
			$x\in D(A)$: $Ax= \sum_{n\in \Z} \hat x(n) \alpha_n e_n$. $H^\infty$-Kalkül für $A$:
			\begin{enumerate}[(i)]
				\item Sei $\alpha_n = n^2$. Dann ist 
				$A = -\laplace = -\frac{d^2}{dt^2}$ ein beschränkter $H^\infty$-Kalkül für $p\in (1,\infty)$ aber für $p = 1$, $p= \infty$ \textbf{nicht}.
				\item Sei $\alpha_n= 2^n$. Dann hat $A$ einen beschränkten $H^\infty$-Kalkül für $p = 2$, aber für \textbf{kein} $p\neq 2$.
				\item $\alpha_n = p_n\in \mathcal{P}$. $A$ hat keinen $H^\infty$-Kalkül.
				\item $\alpha_n\sim n^q$, $q\in \R^+$. Dann hat $A$ einen $H^\infty$-Kalkül.
			\end{enumerate}
		\end{enumerate}
	\end{Beispiel}
	
	\begin{Satz}
		Sei $A$ ein $\om$-sektorieller Operator auf einem Banachraum $X$ mit $\om<\pi/2$. Wähle $\delta\in (0, \pi/2 - \om)$, $f_z(\lambda) = e^{-z\lambda}$, $z\in \Sigma_\delta$. Dann gilt für $T(z) = f_z(A)$:
		\begin{enumerate}[(i)]
			\item $z\in \Sigma_\delta\rightarrow T(z)\in B(X)$ beschränkt und analytisch und $\frac{d}{dz}T(z) = -AT(z)$, ($T(z) = e^{-zA}$.
			\item $T(z_1)T(z_2) = T(z_1+z_2)$, $z_1,z_2\in \Sigma_\delta$.
			\item $T(z)x\rightarrow x$ für $z\rightarrow 0$ und $x\in X$.
			\item Außerdem gilt für $Re\lambda<0$
			$$R(\lambda, A) = \int_0^\infty e^{-\lambda t}T(t) dt.$$
		\end{enumerate}
	\end{Satz}
	
	\begin{Definition}
		$T(z)$ mit den Eigenschaften (i)-(iii) heißt \textbf{analytische Halbgruppe} mit Erzeuger $A$.
	\end{Definition}
	
	\Bew Zeige $T(z)$, $z\in \Sigma_\sigma$, ist beschränkt in $B(X)$. Nach Lemma 3.9 gilt wegen $T(z) = f_z(A)$
	\begin{align*}
		\|T(z)\| &= \|f_z(A)\| \leq \inf \int_{\Gamma_\epsilon} \frac{|f_z(\lambda)|}{|\lambda|} d|\lambda|\\
		\int_{\Gamma_\epsilon} |e^{-z\lambda}|\frac{d|\lambda}{|\lambda|} &\leq \int_{\Gamma_\epsilon\cap B_0(\epsilon)} |e^{-z\lambda}|\frac{d|\lambda|}{|\lambda|} +\int_{\Gamma_\epsilon \cap B_0(i)} |e^{-z\lambda}|\frac{d|\lambda|}{|\lambda|} = (*)
	\end{align*}
	Parametrisierung der Komplexen Wegintegrale $\theta\rightarrow \epsilon e^{i\theta}$ (erstes Integral), $t\rightarrow \lambda = te^{i\pm\nu}$ (zweites Integral).
	\begin{align*}
		(*)&\leq \int_0^{2\pi} e^{-\epsilon r\cos(\om+\theta)}\frac{\epsilon d\theta}{\epsilon} + \int_\epsilon^\infty e^{-tr\cos(\pm\nu + \om)}\frac{dt}{t}\\
		&\leq 2\pi e^{+\epsilon r} + \sum_{+,-} e \int_{\epsilon a\pm} \frac{dt}{t}\leq C(\om) 
	\end{align*}
	
	\paragraph{Bemerkung.} Sektorielle Operatoren mit $\om<\pi/2$ $\Leftrightarrow$ $A$ ist Erzeuger einer analytischen Halbgruppe.
	
	
	
	
	
	
	
	
	
	
	
	
	
	
	
	
	
	
	
	
	
	
	
	
	
	
	
	
	
	
	
	
	
\end{document}