\documentclass[12pt,a4paper,titlepage]{scrartcl}

\usepackage[utf8]{inputenc} 
\usepackage[T1]{fontenc} 
\usepackage[ngerman]{babel} 

\title{Spektraltheorie}
\author{Prof. Dr. Lutz Weis}
\date{Sommersemester 2016}
\publishers{Karlsruher Institut für Technologie}


\usepackage{amsmath}
\usepackage{amsfonts}
\usepackage{amssymb}
\usepackage{graphicx}
\usepackage{geometry}
\usepackage{caption}
\usepackage{amssymb}
\usepackage{amsmath}
\usepackage{mathtools}
\usepackage{mathrsfs}
\usepackage{subfigure}
\usepackage{stmaryrd}
\usepackage{enumerate}
\usepackage[ngerman]{babel}
\usepackage{mathrsfs}
%\usepackage{subcaption}
\usepackage[arrow, matrix, curve]{xy}
\geometry{a4paper,left=20mm,right=20mm,top=25mm,bottom=20mm}

% 1.1, 1.2 etc durchnummeriert, ebenso die Gleichungen mit (1.1), (1.2) ..
\newtheorem{Satz}{Satz}[subsection]
\newtheorem{Definition}[Satz]{Definition} 
\newtheorem{Klass}[Satz]{Klassifizierung} 
\newtheorem{Lemma}[Satz]{Lemma}	
\newtheorem{Bemerkung}[Satz]{Bemerkung}	
\newtheorem{Folgerung}[Satz]{Folgerung}  
\newtheorem{Beispiel}[Satz]{Beispiel} 
\newtheorem{DefBem}[Satz]{Definition + Bemerkung} 
\newtheorem{Prop}[Satz]{Proposition}
\newtheorem{PropDef}[Satz]{Proposition + Definition}
\newtheorem{Erinnerung}[Satz]{Erinnerung}
\newtheorem{Korollar}[Satz]{Korollar}
\newtheorem{Motivation}[Satz]{Motivation}
\newtheorem{Example}{Example}

\DeclareMathOperator{\trdeg}{trdeg}
\DeclareMathOperator{\Spec}{Spec}
\DeclareMathOperator{\Ann}{Ann}
\DeclareMathOperator{\Kern}{Kern}
\DeclareMathOperator{\Bild}{Bild}
\DeclareMathOperator{\Sh}{Sh}
\DeclareMathOperator{\Quot}{Quot}
\DeclareMathOperator{\Hom}{Hom}
\DeclareMathOperator{\Mor}{Mor}
\DeclareMathOperator{\id}{id}
\DeclareMathOperator{\ord}{ord}
\DeclareMathOperator{\Pic}{Pic}
\DeclareMathOperator{\GL}{GL}
\DeclareMathOperator{\Ob}{Ob}
\DeclareMathOperator{\Supp}{Supp}
\DeclareMathOperator{\Proj}{Proj}
\DeclareMathOperator{\Rang}{Rang}
\DeclareMathOperator{\res}{res}
\DeclareMathOperator{\Char}{char}
\DeclareMathOperator{\Res}{Res}
\DeclareMathOperator{\tr}{tr}
\DeclareMathOperator{\pic}{Pic}
\DeclareMathOperator{\ob}{ob}
\DeclareMathOperator{\SetOp}{Set}
\DeclareMathOperator{\KVecOp}{K-Vect}
\DeclareMathOperator{\RModOp}{R-Mod}
\DeclareMathOperator{\GroupOp}{Group}
\DeclareMathOperator{\AbelianOp}{Ab}
\DeclareMathOperator{\TopOp}{Top}
\DeclareMathOperator{\HoTopOp}{HoTop}
\DeclareMathOperator{\Graph}{Graph}
\DeclareMathOperator{\spann}{span}
\DeclareMathOperator{\supp}{supp}

\numberwithin{equation}{section} 

% einige Abkuerzungen
\newcommand{\C}{\mathbb{C}} % komplexe
\newcommand{\K}{\mathbb{K}} % komplexe
\newcommand{\R}{\mathbb{R}} % reelle
\newcommand{\Q}{\mathbb{Q}} % rationale
\newcommand{\Z}{\mathbb{Z}} % ganze
\newcommand{\N}{\mathbb{N}} % natuerliche

\newcommand{\f}{\hat{f}}
\newcommand{\g}{\hat{g}}
\newcommand{\F}{\mathcal{F}}
\newcommand{\G}{\mathcal{G}}
\newcommand{\om}{\omega}
\newcommand{\intR}{\int_{-\infty}^{\infty}}
\newcommand{\m}{\cdot}
\newcommand{\eF}{e^{-2i\pi \om t}}
\newcommand{\eIF}{e^{-2i\pi \om t}}
\newcommand{\Aff}{\mathbf{Aff}}
\newcommand{\A}{\mathbb{A}}
\newcommand{\PR}{\mathbb{P}}
\newcommand{\Bew}{\emph{Beweis: }}
\newcommand{\pw}{\wp}
\newcommand{\qscr}{\mathscr{Q}}
\newcommand{\Ocal}{\mathcal{O}}
\newcommand{\Hc}{\check{H}}
\newcommand{\Ccal}{\mathcal{C}}
\newcommand{\laplace}{\Delta}

\newcommand{\Set}{\underline{\SetOp}}
\newcommand{\KVect}{\underline{\KVecOp}}
\newcommand{\RMod}{\underline{\RModOp}}
\newcommand{\Grp}{\underline{\GroupOp}}
\newcommand{\Ab}{\underline{\AbelianOp}}
\newcommand{\Top}{\underline{\TopOp}}
\newcommand{\HoTop}{\underline{\HoTopOp}}


%\newcommand{\L}{\mathcal{L}}
\newcommand{\qed}{\begin{flushright}
		$\square$
	\end{flushright}}



\begin{document}
	\maketitle
	

	\section*{Vorwort}
	Dieses Skript wurde im Sommersemester 2016 von David Bückel geschrieben und von Martin Belica korrekturgelesen und ärgenzt. Es ist ein inoffizielles Skript und beinhaltet die Mitschriften aus der Vorlesung von Prof.~Dr.~Weis am Karlsruhe Institut für Technologie sowie die Mitschriften einiger Übungen.

	\thispagestyle{empty}

	\subsection*{Einleitung}

	Die Spektraltheorie verallgemeinert die Theorie von Eigenwerten und Normalformen von Matrizen für unendlichdimensionale Operatoren auf Funktionenräumen, wie Differential- und Integraloperatoren. Sie vermittelt eine wesentliche Methodik für viele Anwendungsgebiete, wie partielle Differentialgleichungen, mathematische Physik und numerische Analysis. \\

	Zu den Themen gehören:

  	\begin{itemize}
    	\item Spektrum und Resolvente linearer (unbeschränkter) Operatoren
    	\item Fouriertransformation und der Funktionalkalkül des Laplace-Operators
    	\item Der Funktionalkalkül selbstadjungierter Operatoren
    	\item Der holomorphe Funktionalkalkül sektorieller Operatoren
    	\item Cauchy-Problem für sektorielle Operatoren
 	\end{itemize}
  
	Diese Vorlesung bereitet auf zukünftige Vorlesungen und Seminare im Bereich der deterministischen und stochastischen Evolutionsgleichungen vor.

	\subsection*{Erforderliche Vorkenntnisse}
	Wir setzen ein grundlegendes Verständnis funktionalanalytischer Methoden voraus, wie sie z.B. in den Vorlesungen "Differentialgleichungen und Hilberträume" oder "Funktionalanalysis" vermittelt werden.
	
	\newpage	
	\tableofcontents
	\newpage	
	
	\newpage
	\section{Einführung}
	
	\subsection{Wiederholung aus der Funktionalanalysis}
	
	\subsubsection{Abgeschlossene Operatoren}
	
	Sei $X$ ein Banachraum, $X\supset D(A)\overset{A}{\rightarrow} X$ linear auf einem linearen Teilraum $D(A)$ von $X$. $D(A)$ heißt \textbf{Definitionsbereich}. 
	
	\begin{Definition}[Abgeschlossener Operator]
		$A$ heißt abgeschlossen, falls für alle $x_n\in D(A)$, $x_n\rightarrow x$ in $||.||_X$, $Ax_n\rightarrow y$ in $||.||_y$ $\Rightarrow$ $x\in D(A)$, $Ax= y$.
	\end{Definition}
	
	Notation: $||x||_A = ||x||+||Ax||$, $x\in D(A)$ heißt Graphennorm von $A$. Also: 
	$$A:(D(A),||.||_A)\rightarrow (X,||.||_X)$$
	stetig.
	
	\begin{Satz}
		$A:X\supset D(A)\rightarrow X$ ist abgeschlossen $\Leftrightarrow$ $\Graph(A) = \{(x,Ax)\in X\times X,~ x\in D(A) \}$ ist abgeschlossen in $X\times X$ $\Leftrightarrow$ $(D(A),||.||_A)$ vollständig normierter Raum.
	\end{Satz}
	
	\begin{Korollar}
		$X = D(A)$, $A$ abgeschlossen $\Leftrightarrow$ $A:X\rightarrow X$ stetig $\Leftrightarrow$ $A(U_X)$ beschränkt in $X$, $U_X = $ offene Einheitskugel in $X$.
	\end{Korollar}
	
	Notation: $B(X)$ ist der Banachraum aller beschränkten/stetigen linearen Operatoren $A:X\rightarrow X$ mit der Norm $||A|| = \sup_{x\in U_X}||Ax|| <\infty$.
	
	\begin{Bemerkung}
		Sei $D\subseteq X$ ein dichter, linearer Teilraum von $X$, d.h. $\bar{D} = X$. Sei $A:D\rightarrow X$ ein linearer Operator mit $||Ax||\leq C||x||$ für alle $x\in D$. Dann gibt es \textbf{genau eine} stetige Fortsetzung $\tilde{A}: X\rightarrow X$, d.h. $\tilde{A}\in B(X)$, $\tilde{A}|D = A$, $||\tilde{A}||\leq C$ (Sei $x\in X$ mit $x_n=\lim x_n$, $x_n\in D$. Dann $\tilde{A}x=\lim_{n\rightarrow\infty} Ax_n$).
	\end{Bemerkung}
	
	\subsubsection{Spektrum und Resolvente}
	
	Sei $X$ ein Banachraum, $A:X\supset D(A)\rightarrow X$ linearer, abgeschlossener Operator.
	
	Sei gegeben: $\lambda x-Ax=y$,$\lambda\in \C$, $y\in X$, $x\in D(A)$ ist gesucht. Formal $x=(\lambda-A)^{-1}y$ ist Lösung, falls $(\lambda-A)^{-1}$ existiert.
	
	\begin{Definition}
		$\lambda\in \rho(A)$ falls $\lambda -A: D(A)\rightarrow X$ bijektiv oder äquivalent: 
		$$\lambda-A:(D(A),||.||_A)\rightarrow (X,||.||_X)$$
		ist ein Isomorphismus. $\rho(A)$ heißt die \textbf{Resolventenmenge} von $A$. $\sigma(A)=\C\backslash\rho(A)$ heißt das \textbf{Spektrum} von $A$.
		
		$R(\lambda, A):= (\lambda-A)^{-1}: X\rightarrow D(A)\subset X$, für ein $\lambda\in \rho(A)$. $R(\lambda,A)\in B(X)$, aber $\Bild(R(\lambda,A))\subset D(A)$.
	\end{Definition}
	
	\begin{Bemerkung}
		Fall $A\in B(X)$, $D(A) = X$, dann ist $R(\lambda, A): X\rightarrow X$ ein 
		Isomorphismus.
	\end{Bemerkung}
	
	\begin{Satz}~
		\begin{enumerate}
			\item[a)] $\rho(A)$ ist offen, $\sigma(A)$ abgeschlossen.
			\item[b)] Falls $A\in B(X)$, dann: $|\lambda|\leq ||A||$ für alle $\lambda\in \sigma(A)$. Insbesondere: $\sigma(A)$ Kompakt. $\sigma(A)\neq\emptyset$.
		\end{enumerate}
	\end{Satz}
	
	\begin{Satz}[Resolventendarstellung]~
		\begin{enumerate}
			\item[a)] Sei $\lambda_0\in \rho(A)$, $|\lambda-\lambda_0|\leq\frac{1}{||R(\lambda_0,A)||}$. Dann ist
			$$R(\lambda,A) =\sum_{n = 0}^{\infty} (\lambda_0-\lambda)^n R(\lambda_0, A)^{n+1}$$
			analytisch.
			\item[b)] Sei $A\in B(X)$ und $|\lambda|>||A||$, dann ist 
			$$R(\lambda, A) = \sum_{n = 0}^{\infty}\lambda^{-(n+1)}A^n.$$ 
		\end{enumerate}
	\end{Satz}
	
	
	\begin{Satz}[Resolventenregel]
		Für $\lambda,\mu\in \rho(A)$ gilt:
		$$R(\lambda,A)-R(\mu,A) = (\mu-\lambda)R(\lambda,A)R(\mu,A).$$
		Für $\mu\rightarrow \lambda$:
		$$\frac{d}{d\lambda}R(\lambda,A) = -R(\lambda,A)^2.$$
	\end{Satz}
	
	\begin{Beispiel}~
		\begin{enumerate}
			\item[a)] Sei $X=l^p$, $(x_n)\subset\C$.
			$$D(A) = \{(x_n)\in l^p: (\sum_n |\lambda_n x_n|^p)^{1/p} \}$$
			$A(x_n) = (\lambda_n x_n)\in l^p$ für $(x_n)\in D(A)$.
			
			Diagonaloperator: $\sigma(A) = \{\lambda_n\}$, $\lambda\notin \overline{\{\lambda_n\}}$ ist 
			$$R(\lambda,A)(x_n) = (\lambda-A)^{-1}(x_n) = (\frac{1}{\lambda-\lambda_n}x_n).$$
			\item[b)] $X=l^p$, $A(x_n) = (0,x_1,x_2,...)$ (Rechts-Verschiebeoperator). $\sigma(A) = \{\lambda:|\lambda|\leq 1 \}\subset \C$. 
			\item[c)] $X = C[0,1]$, $Tf(t) = \int_0^t f(s)ds$, $t\in [0,1]$ (Volterraoperator). $\sigma(T) = \{0\}$, $||T|| \neq 0$.
		\end{enumerate}
	\end{Beispiel}
	
	\subsubsection{Spektrum und Kompaktheit}
	
	\begin{Satz}
		Sei $X$ ein Banachraum, $A\in B(A)$ kompakt, d.h. $A(U_X)$ ist relativ kompakt in $(X,||.||)$ (z.B. $A\in \{T: X\rightarrow X |\dim\Bild(T)<\infty \}$). Falls $\dim X = \infty$, dann gilt $0\in \sigma(A)$. $\sigma(A)\backslash \{0\}$ besteht aus einer Folge $(\lambda_n)\subset \C$, die aus Eigenwerten von $A$ besteht, mit endlich dimensionalem Eigenraum $\Kern(\lambda_n-A).$ $(\lambda_n)$ ist endlich oder $|\lambda_n|\rightarrow 0$.
	\end{Satz}
	
	Sei ab jetzt $X$ ein Hilbertraum mit $<.,.>$. 
	
	\begin{Definition}
		$(h_n)\subset X$ heißt Orthonormalbasis von $X$, falls $<h_n, h_m> = \delta_{n,m}$, $\overline{\spann(h_n)} = X$.
	\end{Definition}
	
	\begin{Satz}
		Jeder seperable Hilbertraum besitzt eine Orthonormalbasis $(h_n)$ und für alle $x\in X$ gilt
		\begin{eqnarray}
			x &=& \sum_n <x,h_n>h_n,\nonumber\\
			||x||^2 &=& \sum_n|<x,h_n>|^2 \nonumber
		\end{eqnarray}	
	\end{Satz}
	
	\begin{Satz}[Spektralsatz]
		Für kompakte, selbstadjungierte Operatoren auf einem seperablen Hilbertraum.
		Sei $A$ kompakt, selbstadjungiert, d.h. $<Ax, y> = <x, Ay>$. Dann gibt es eine Orthonormalbasis $(h_n)$ von $H$ und eine Folge $(\lambda_n)\subset\R$ mit $|\lambda_1|\geq |\lambda_2|\geq...\geq 0$, $\lim\limits_{n\rightarrow\infty} \lambda_n = 0$, sodass
		$$ Tx= \sum_n \lambda_n<x,h_n>h_n.$$
	\end{Satz}
	
	\begin{Bemerkung}~
		\begin{enumerate}
			\item[a)] $\lambda_n$ sind Eigenwerte von $A$, denn $Th_n = \lambda_nh_n$, $|\lambda_n|\rightarrow 0$.
			\item[b)] $||A|| = \sup_{n\in \N}|\lambda_n| = \lambda_1$.
			\item[c)] $\Kern A = \overline{\spann}\{h_n:\lambda_n = 0 \}$. $\overline{\Bild(A)} = \overline{\spann}\{h_n: \lambda_n\neq 0\}$. $X = (\Kern(A))\oplus\overline{\Bild(A)}$ (orthogonal). $A$ injektiv $\Leftrightarrow$ $\overline{\Bild(A)} = X$ $\Leftrightarrow$ $\lambda_n \neq 0$ $\forall n\in \N$.
			\item[d)] $J:l^2\rightarrow X$, $J(e_n) = h_n$ ($e_n$ Einheitsvektor). $J$ ist Isometrie, denn
			$$||\sum \alpha_n e_n||_{l^2} = (\sum |\alpha_n|^2)^{1/2} = ||\sum_n \alpha_n h_n||_X$$
			$$\xymatrix{
				l^2\ar[r]^D \ar[d]_J & l^2\\
				X\ar[r]_A & X\ar[u]_{J^{-1}}
				}$$
			$D = JJ^{-1}A$, $D(x_n) = (\lambda_n x_n)$ für $x= (x_n) \in l^2$. $A$ ist ähnlich zu einem Diagonaloperator $D$ mit $|\lambda_n|\rightarrow 0$.
		\end{enumerate}
	\end{Bemerkung}
	
	\subsection{Verpasst}
	
	\newpage
	\subsection{Fourieranalysis}
	
	\subsubsection{Die Fouriertransformation auf $L^1$, $S$ und $L^2$}
	
	\begin{Definition}
		Für $f\in L^1(\R^d)$ setze 
		$$\f(\xi) = \F(f(\xi))) = \int_{\R^d} e^{-2\pi ix\xi}f(x)dx$$
		mit $x\m \xi = \langle x,\xi\rangle$.
	\end{Definition}
	
	\paragraph{Bemerkung:} $d = 1$, $f\in L^2[0,\pi]\subset L^2(\R)$.
	$$\f(n) = \int_{0}^{2\pi} e^{-2\pi i n x}f(x) dx = \F(f(n))$$
	Klassische Fourier Koeffizienten $f\in L^[0,2\pi]$ sind die Werte von $\F f(\xi)$ für $\xi = n\in \Z$.
	$$f= \sum_{n \in \Z} \f(n) e^{2\pi i n x} \text{ Fourierreihen.}$$
	Ziel: $\sum\rightarrow \int$. Damit folgt $f(x) = \int_{\R}e^{2\pi i x \xi}\f(\xi)d\xi$.
	
	\begin{Prop}
		Für $f\in L^1(\R^d)$ ist $\F f\in L^{\infty}(\R^d)$ und $||\F f||_{L^\infty}\leq ||f||_{L^1}$. Es gilt sogar $\F f\in C_0(\R^d)$, d.h. 
		$$\lim\limits_{|x|\rightarrow 0}\F f(x) = 0.$$
	\end{Prop}
	
	\Bew $|\F f(\xi)|\leq \int |e^{-2\pi i x\xi}| |f(x)| dx$. FÜr $\F f\in C_0(\R^d)$ siehe Übung.
	\qed
	
	\begin{Prop}
		~
		\begin{enumerate}
			\item[a)] Für die Dilation $f_\delta(x) = f(\delta x)$, $\delta>0$ fest:
			$$f(\delta x) \overset{\F}{\rightarrow} \delta^{-d}\f(\delta^{-1}\xi).$$
			Beweis durch Substitution $x' = x\m \delta$ in der Definition von $\F$.
			\item[b)] $f(x+h) \overset{\F}{\rightarrow} \f(\xi)e^{2i\pi \xi h}$, $h\in \R^d$ fest. 
			
			$f(x)e^{-2i\pi x h}\overset{\F}{\rightarrow} \f(\xi+h)$. 
		\end{enumerate}
	\end{Prop}
	\Bew b) $\F[f(\m+h)](\xi) = \int e^{-2i\pi x\xi}f(x+h) dx = (*)$. Substituiere $x' = x+h$
	\begin{eqnarray}
		(*) &=& \int e^{-2i\pi (x'-h)\xi}f(x')dx'\nonumber\\
		&=& e^{2\pi i h\xi}\f(\xi).\nonumber
	\end{eqnarray}
	\qed
	
	\begin{Definition}[Faltung von $f,g\in L^1(\R^d)$]
		$$(f*g)(x):= \int_{\R^d}f(x-y)g(y)dy = \int_{\R^d}f(y) g(x-y) dy$$
		\textbf{Bemerkung:} $||f*g||_{L^1}\leq ||f||\m ||g||$.
	\end{Definition}
	
	\begin{Prop}
		$$\F(f*g)(\xi) = \f(\xi)\m \g(\xi).$$
	\end{Prop}
	
	\Bew $\F(f*g)(x) = \int e^{-2i\pi x\xi}(\int f(x-y)g(y)dy)dx$. Fubini liefert:
	\begin{eqnarray}
		\F(f*g)(x) &=& \int e^{-2i\pi y\xi}g(y) (\int e^{-2i\pi (x-y)\xi}f(x-y)dx)dy\nonumber\\
		&=&(\int e^{-2i\pi y\xi}g(y)dy)\m(\int e^{-2i\pi x\xi}f(x)d(x)) = \g(\xi)\m \f(\xi)\nonumber
	\end{eqnarray}
	\qed
	
	\paragraph{Zur Erinnerung:} $f\rightarrow f*g$ ist eine Glättung.
	
	\begin{Definition}
		$f\in C^{\infty}(\R^d)$ heißt schnell fallend, falls für alle Multiindizes $\alpha,\beta \in \N_0^d$ gilt: 
		$$||f||_{\alpha,\beta} := \sup_{\xi\in \R^d}|\xi^\beta(D^\alpha f)(\xi)| <\infty.$$
		Notation: $f\in S(\R^d)$ (Raum der schnell fallenden Funktionen/Schwarzraum). $\alpha\in \N_0^d$, $\alpha = (\alpha_1,...,\alpha_d)$, $D^\alpha = D_{x_1}^{\alpha_1}...D_{x_n}^{\alpha_n}$ und $x^\alpha$ beschreibt das komponentenweise Potenzieren mit $\alpha_i$.
	\end{Definition}
	
	\paragraph{Bemerkung:} Alle Ableitungen gehen schneller gegen Null als jedes Polynom gegen $\infty$ geht für $|x|\rightarrow \infty$, d.h. $\forall\alpha,\beta\in \N_0^d$, $\forall m\in \N$ gilt:
	$$(I+|x|)^m x^\beta D^\alpha f(x)\in L^\infty(\R^d).$$
	Insbesondere: $f\in S(\R^d)$ $\Rightarrow$ $f\in \bigcap_{p\geq 1} L^p(\R^d)$ (Himmelreich für Fubini, Differentiation unter dem Integralzeichen, Lebesgue Konvergenz...).
	
	\begin{Beispiel}
		~
		\begin{enumerate}
			\item[a)] $C_c^{\infty}(\R^d)\subset S(\R^d)$. Aber $\f(\xi) = \int_K e^{-2i\pi x\xi}f(x) dx\notin C_c^{\infty}(\R^d)$, $K$ kompakt = $\supp f$.
			\item[b)] $h(x) = e^{-\pi|x|^2}$, $h\in S(\R^d)$. Später: $\F(h) = h$.
		\end{enumerate}
	\end{Beispiel}
	
	\begin{Satz}
		Mit $f\in S(\R^d)$ sind auch $x\m f(x)$, $D^\alpha f$, $D^\alpha\F f$, $\F(D^\alpha f)$ in $S(\R^d)$ und 
		\begin{enumerate}
			\item[a)] $D^{\alpha}\F f= \F[(-2i\pi x)^\alpha f(x)]$.
			\item[b)] $(2i\pi \xi)^\alpha\F(f)(\xi) = \F(D^\alpha f)(\xi)$.
		\end{enumerate}
	\end{Satz}
	
	\Bew a) $D_{\xi_1}(\F f)(\xi) = \int D_{\xi_1}[e^{-2i\pi x\xi}]f(x) dx$
	\begin{eqnarray}
		D_{\xi_1}(\F f)(\xi) = \int (2i\pi x_1)e^{-2i\pi x\xi}f(x)dx\nonumber
	\end{eqnarray}
	Analog 
	$$D_{\xi_1}D_{\xi_2}\F f(\xi) = \int e^{-2i\pi x\xi}(-2i\pi x_2)(-2i\pi x_1)f(x) dx...$$
	b) \begin{eqnarray}
		\F f(x) &=& \lim\limits_{R\rightarrow \infty} \int_{-R}^R...\int_{-R}^{R} e^{-2i\pi x y}f(y)dy\nonumber\\
		&=& \lim\limits_{R\rightarrow\infty}\int_{-R}^R e^{-2i\pi x_1y_1}\left(\int_{-R}^{R} e^{-2i\pi x_2y_2}...e^{-2i\pi x_dy_d} f(y_1,...,y_d) dy_2...dy_d\right)dy_1 \nonumber\\
		&=& \lim\limits_{R\rightarrow\infty} \int_{-R}^{R}\frac{1}{-2i\pi x_2}e^{-2i\pi x_1 y_1}D_{y_1}I_R(y_1)dy_1 + \lim\limits_{R\rightarrow\infty}\left[\frac{1}{-2i\pi x_2} I_R(y_1)\right]_{y_1 = -R}^R \nonumber\\
		&=& \frac{1}{-2i\pi x_1}\F(Dy_1 f)(x)\nonumber
	\end{eqnarray}
	mit $I_R = \int_{-R}^{R} e^{-2i\pi x_2y_2}...e^{-2i\pi x_dy_d} f(y_1,...,y_d) dy_2...dy_d$.
	
	\begin{Beispiel}
		$\F[e^{-\pi |x|^2}](\xi) = e^{-\pi|\xi|^2}$, denn (für $d= 1$):
		
		Setze $h(\xi) = \intR e^{-\pi x^2-2i\pi x\xi}dx$, d.h. $h(\xi) = \F[e^{-\pi|x|^2}](\xi)$.
		\begin{eqnarray}
			h'(\xi) &=& \intR (-2i\pi x)e^{-\pi x^2- 2i\pi x\xi}dx \nonumber \\
			&=& i\intR \frac{d}{dx}[e^{-\pi x^2}]e^{-2i\pi x\xi}dx \nonumber\\
			&=& i \F\left[\frac{d}{dx}e^{-\pi x^2} \right](\xi)\nonumber\\
			&=& i(2i\pi\xi)\F\left[e^{-\pi x^2} \right](\xi) = -2\pi\xi h(\xi)\nonumber
		\end{eqnarray}
		Differentialgleichung: $h'(\xi) = -2\pi\xi h(\xi)$.
		\begin{eqnarray}
			\frac{h'(\xi)}{h(\xi)}=-2\pi \xi ~~\Rightarrow~~ \ln(h(\xi)) = 2\pi\int_0^x(-\xi)d\xi\nonumber\\
			\Rightarrow h(\xi) = e^{-\pi\xi^2} = \F[e^{\pi x^2}](\xi)\nonumber
		\end{eqnarray}
		Für $d>1$: 
		\begin{eqnarray}
			\F[e^{-\pi|x|^2}](\xi) &=& \int e^{-2i\pi \xi x}e^{-\pi|x|^2}dx \nonumber\\
			&=& \prod_{j = 1}^{d}\int_{\R}e^{-\pi x_j^2}e^{-2i\pi \xi_j x_j}dx_j = \prod_{j = 1}^{d} e^{-\pi\xi_j^2}\nonumber\\
			&=& e^{-\pi|\xi|^2}\nonumber
		\end{eqnarray}
		Damit folgt $\F(\exp(-\pi\epsilon^2|x|^2)) = \epsilon^{-d}\exp(-\pi|\xi|^2/\epsilon^2)$.
	\end{Beispiel}
	
	\begin{Satz}
		$\F:S(\R^d)\rightarrow S(\R^d)$ ist bijektiv und 
		$$\F^{-1}\phi(x) = \int_{R^d} e^{2i\pi x\xi}\phi(\xi)d\xi.$$
	\end{Satz}
	
	\Bew Seien $\phi,\psi\in S(\R^d)$, $\F(\phi)\in S(\R^d)$, $x\in \R^d$ fest:
	\begin{eqnarray}
		\int e^{2i\pi x\xi}(\F\phi)(\xi)d\xi &=& \int \left(\int e^{-2i\pi \xi y}\phi(y) dy\right) e^{2i\pi x\xi} \psi(\xi)d\xi\nonumber\\
		&\overset{\text{Fubini}}{=}& \int\left(\int e^{2i\pi (x-y)\xi}\psi(\xi)d\xi\right)\phi(y)dy\nonumber\\
		&=& \int\F\psi(x-y)\phi(y)dy \nonumber\\
		&=& \int \F\psi(y)\phi(x+y)dy\nonumber
	\end{eqnarray}
	Wähle $\psi(x) = e^{-\pi\epsilon^2|x|^2}$.
	\begin{eqnarray}
		\int e^{2i\pi x\xi}(\F\phi)(\xi)e^{-\pi\epsilon^2|\xi|^2}d\xi &=& \int \epsilon^{-d} e^{-pi|y|^2}\phi(y+x)dy\nonumber\\
		&\overset{y' = y/\epsilon}{=}& \int e^{2i\pi x\xi}\F\phi (\xi) e^{-\pi\epsilon^2|\xi|^2}d\xi \nonumber \\
		&=& \int e^{-\pi |y|^2}\phi (x+\epsilon y)dy\nonumber
	\end{eqnarray}
	Für $\epsilon>> 0$ folgt mit Lebesgue Konvergenz und
	\begin{eqnarray}
		\phi(x+\epsilon y)&\overset{\epsilon\rightarrow 0}{\longrightarrow}&\phi(x),~|\phi(x)|\leq C\nonumber\\
		e^{-\pi\epsilon^2|\xi|^2}&\overset{\epsilon\rightarrow 0}{\longrightarrow}& 1, ~|e^{-\pi\epsilon^2|\xi|^2}\leq 1\nonumber
	\end{eqnarray}
	Für $\epsilon\rightarrow 0$:
	\begin{eqnarray}
		\int e^{2i\pi x\xi}\F\phi(\xi)d\xi &=& \left(\int e^{-\pi|y|^2}dy \right)\phi(x)\nonumber\\
		\G f(x)&:=& \int e^{2\i\pi x\xi}f(\xi)d\xi\nonumber
	\end{eqnarray}
	Also $\G(\F) = \text{Id}$ und $\F(\G) = \text{Id}$ $\Rightarrow$ $\F^{-1} = \G$.\qed
	
	\paragraph{Folgerung:} $\F^{-1}\phi(x) = \F\phi(-x)$.
	
	\paragraph{Bemerkung} Vergleich zu Fourierreihen: 
	$$f(x)=\sum_{n\in \Z} e^{2i\pi n x}\f(n),~~ \f(n) = \int_{0}^{2\pi}e^{-2i\pi n x}f(x) dx.$$
	
	\begin{Prop}
		Seien $f, g\in S$. Dann sind $f\m g$ und $f*g$ wieder $\in S$.
	\end{Prop}	
	\Bew Zeige $f(x)g(x) \in S$, benutze Produktformel.\\
	$D^\alpha(f*g)(x) = \int (D_\alpha f)(x-y)g(y) dy$.
	\begin{eqnarray}
		(1+|x|)^n D^{\alpha}(f*g)(x) &=& \int(1+|x-y|)^n D^\alpha f(x-y) (1+|y|)^ g(y) \frac{(1+|x|)^n}{(1+|x-y|)^n (1+|y|)^n}dx \nonumber\\
		&\leq& C\int(1+|y|)^{-m}(1+|y|)^{n+m}g(y) dy\nonumber
	\end{eqnarray}
		
	\paragraph{Zusammenfassung:} 
	$$\xymatrix{
		S\ar[d]_\F \ar[r]^{D^\alpha} & S\ar[d]^\F\\
		S\ar[r]_{M_\alpha} & S
		}$$
	Mit $M_\alpha f(x) = (2i\pi x)^{\alpha} f(x)$.
	$$\xymatrix{
		S\ar[r]^{T_g}\ar[d]_\F & S\ar[d]^\F\\
		S\ar[r]_{M_g} & S
		}$$
	Mit $T_g f = f*g$, $M_g f= \g f$.
	
	\begin{Bemerkung}[Zur Lösung von Differentialgleichungen (formale Rechnung)]~
		\\
		$(*)$ $(I-D^{\alpha})f = g$, $g$ gegeben, $f$ gesucht.
		\begin{eqnarray}
			[1-(2i\pi x)^{\alpha}]\f(x) = \g(x) &\Rightarrow& \f(x) = [1 - (2i\pi x)^{\alpha}]^{-1}\g(x)\nonumber 
		\end{eqnarray}
		$m(x) = 1-(2i\pi x)^{\alpha}$. Falls $m(x)\neq 0$.
		\begin{eqnarray}
			f = [\F^{-1}[(1-2i\pi x)^{\alpha}]\g(x)]\nonumber
		\end{eqnarray}
		Angenommen $\exists k\in L^1$ mit $\hat{k}(x) = \frac{1}{m(x)}$.
		\begin{eqnarray}
		\Rightarrow f &=& [\F^{-1}[(1-2i\pi x)^{\alpha}]\g(x)]\nonumber\\
		 &=& \F^{-1}[\hat{k}\m \g]\nonumber\\
		&=& \F^{-1}\hat{k}*\F^{-1} \g = k*g\nonumber
		\end{eqnarray}
		Die Lösung von $(*)$ ist oft durch einen Faltungsoperator gegeben.
	\end{Bemerkung}
	
	\subsubsection{Fouriertransformation auf $L^2(\R^d)$}
	
	\begin{Lemma}
		Für $f, g\in S(\R^d)$ gilt mit $<f,g> = \int_{\R^{d}} f(x) \overline{g(x)}dx$.
		\begin{enumerate}
			\item[a)] $<\F f,\F g> = <f,g>$, $\F$ unitär.
			\item[b)] $||\F f||_{L^2} = ||f||_{L^2}$, $\F$ Isometrie.
		\end{enumerate}
	\end{Lemma}
	\Bew a) $\F^{-1}\F f = f$. Daraus folgt:
	\begin{eqnarray}
		<f,g> &=&\int f(x)\overline{g(x)}dx\nonumber\\
		&=& \int\left(\int e^{2i\pi x y}\F f(y)dy \right)\overline{g(x)}dx \nonumber \\
		&\overset{\text{Fubini}}{=}& \int \F(y)\left(\int e^{2i\pi xy} \overline{g(x)}dx\right)dy\nonumber\\
		&=& \int \F f(y) \int e^{-2i\pi xy}g(x)dx dy\nonumber\\
		&=& <\F f, \F g>\nonumber
	\end{eqnarray}
	b) $<f,f>$ liefert Behauptung.
	\qed
	
	\begin{Definition}[der Fouriertransformation auf $L^2(\R^d)$]
		Da $S(\R^d)$ dicht in $L^{2}(\R^d)$ liegt, gibt es zu jedem $f\in L^{2}(\R^d)$ eine Folge $(f_n)\subset S(\R^d)$ mit $||f-f_n||_{L^2}\rightarrow 0$.
		
		Setze $\F f =\lim\limits_{n\rightarrow\infty}\F(f_n)$ (in $L^2$), d.h. $\F$ ist die stetige Fortsetzung von 
		$$\F:S(\R^d)\rightarrow L^2(\R^d) \text{ auf } L^2(\R^d),$$	
		denn $||\F f||_{L^2} = ||f||^{L^2}$ für alle $f\in S(\R^d)$.
	\end{Definition}
	Achtung: Für $f\in L^2(\R^d)$ ist $\int e^{-2i\pi xy}f(y)dy$ nicht immer für fast alle $x$ als Lebesgue-Integral definiert.
	
	\begin{Korollar}
		Für $f, g\in L^2(\R^d)$ gilt:
		\begin{enumerate}
			\item[a)] $<\F f,\F g> = <f,g>$.
			\item[b)] $||\F f||_{L^2} = ||f||_{L^2}$.
		\end{enumerate}
	\end{Korollar}
	\Bew $f_n,g_n\in S^n$ mit $f_n\rightarrow f$, $g_n\rightarrow g$, $f,g\in L^2$. 3.13 und 3.14 liefern Behauptung.
	\qed
	
	\begin{Korollar}
		Sei $g\in L^{1}(\R^d)$, $T_g f= g*f$. Dann ist 
		$$||T_g||_{L^2\rightarrow L^2} = \sup_{x\in \R^d}|\g(x)|.$$
	\end{Korollar}
	\Bew $||f*g||_{L^2} = ||\F[f*g]||_{L^2} = ||\f\m\g||_{L^2}$.  Damit folgt
	$$||\f\m\g||_{L^2} = \left(\int |f(x)\bar{\g(x)}|^2 dx \right)^{1/2}\leq \sup_{x\in \R^d}|\g(x)|\m||\f||_{L^2} = \sup|\g(x)|\m ||f||_{L^2}.$$
	Somit $||T_g||\leq \sup|\g(x)|$.
	\qed 
	
	\paragraph{Bemerkung:} $||T_g||_{L^2\rightarrow L^2}\leq||g||_{L^1}$ (Youngsche Ungleichung), denn $\sup|\g(x)| \leq ||g||_{L^1}$.
	
	\subsection{Temperierte Distributionen und die Fouriertransformation}
	
	Idee $S'(\R^d)$ soll der Dualraum von $S(\R^d)$ sein. 
	\paragraph{Wiederholung:} Für $N\in \N$ setze $||f||_N = \sup\{|x^\alpha D^\beta f(x)|: x\in \R^d, \alpha,\beta\in \N_0^d, |\alpha|,|\beta|\leq N \}$.
	
	\begin{Definition}
		Für $f_n,f\in S(R^d)$ gilt $f_n\overset{S}{\rightarrow}f$ $\Leftrightarrow$ $||f-f_n||_N\rightarrow 0$ $\forall N\in \N$.
	\end{Definition}
	
	\paragraph{Bemerkung:} $f_n\overset{S}{\rightarrow} f$ ist äquivalent zu $d(f_n,f)\rightarrow 0$ für die Metrik
	$$d(f,g)=\sum_{N\in \N}2^{-N}\frac{||f-g||_N}{1+||f-g||_N}$$
	und $d(f_n,f)\rightarrow 0$ $\Leftrightarrow$ $||f-f_n|_N\rightarrow 0$ für alle $N$.
	
	\begin{Definition}
		$S'(\R^d) = \{u:S(\R^d)\rightarrow K, \text{ linear, stetig bezüglich } f_n\overset{S}{\rightarrow} f \}$. $S'(\R^d)$ ist der Dualraum von $S(\R^d)$. Mann nennt ihn auch den Raum der temperierten Distributionen.
	\end{Definition}
	
	\begin{Beispiel}
		~
		\begin{enumerate}
			\item[a)] Sei $h:\R^d\rightarrow \C$ lokal integrierbar und es gebe $n,C$, sodass $\int_{|x|\leq R}|h(x)|dx\leq C R^n$ für $R\rightarrow \infty$. Dann setze $$u_h(f) = \int f(x)h(x) dx.$$
			Zu zeigen: $u_h\in S'$. Insbesondere: $L^p(\R^d)\subset S'(\R^d)$ für alle $p$, denn 
			$$\int_{|x|\leq R}|f(x)|d<\leq R^{1/p}||f\chi_{|x|\leq R}||_{L^p}.$$
			\item[b)] Dirac Maß: $\delta_x\in S'(\R^d)$, $\delta_x(u) = u(x)$.
		\end{enumerate}
	\end{Beispiel}
	
	\begin{Satz}
		Sei $u:S(\R^d)\rightarrow \C$ linear, $u$ stetig, d.h. $u\in S'(\R^d)$ genau dann, wenn es ein $N$ gibt, sodass 
		$$|u(f)|\leq C||f||_N~~\forall f\in S(\R^d).$$
	\end{Satz}
	
	\Bew \glqq$\Leftarrow$\grqq\ klar. \glqq$\Rightarrow$\grqq: Andernfalls gibt es zu jedem $n\in \N$ ein $f_n\in S$ mit $||f_n||_n = 1$ aber $|u(f_n)|\geq n$. Setze $g_n = f_n/\sqrt{n}$. Dann gilt $||g_n||_N \leq n^{1/2}\overset{n\rightarrow\infty}{\rightarrow} 0$ für $n\geq N$.
	
	Aber $|u(g_n)|\geq \sqrt{n}\overset{n\rightarrow\infty}{\rightarrow}\infty$. Also Widerspruch zur Stetigkeit von $u$.
	
	
	\begin{Beispiel}
		~
		\begin{enumerate}
			\item[a)] \glqq$L^p(\R^d)\subset S'(\R^d)$\grqq. Sei $h:\R^d\rightarrow \C$ lokal integrierbar und
			$$(*)~~~\int_{|x|\leq R}|h(x)|dx\leq C R^n\text{ für } R> 1$$
			für ein festes $C<\infty$, $n\in \N$ fest. Dann wird durch 
			$$u_h(f)= \int_{\R^d} h(x)f(x)dx$$
			eine Distribution $u_h\in S'(\R^d)$ definiert, d.h. man erhält eine Einbettung von Funktionen $h$ mit $(*)$ nach $u_h\in S'(\R^d)$.
			\item[b)] Sei $\psi\in C^\infty(\R^d)$ und langsam wachsend, d.h. für alle $\alpha\in \N$ gibt es $N_\alpha, C_\alpha\leq \infty$ mit 
			$$|D^\alpha\psi(x)|\leq C_\alpha(1+|x|)^{N_\alpha},~~ x\in \R^d.$$
			Dann kann man für jedes $u\in S'(\R^d)$ ein Produkt $\psi\m u\in S'(\R^d)$ definieren durch $(\psi\m u)(f) = u(\psi f)$ für $f\in S(\R^d)$.
			\item[c)] Dirac Distribution: $\delta_x\in S'(\R^d)$. $\delta_x(f) = f(x)$ für $f\in S(\R^n)$.
			\item[d)] $h(x) = e^{|x|^2}$. Dann $u_h\notin S'(\R^d)$, den $f(x) = e^{-|x|^2}\in S(\R^d)$
			$$u_h(f)=\int h(x) g(x) dx = \int 1 dx = \infty.$$
		\end{enumerate}
	\end{Beispiel}
	\Bew a) Benutze 4.4. 
	\begin{eqnarray}
		|u_h(f)| &=& \int h(x)f(x)dx\nonumber\\
		&=& \int h(x)(1+|x|^2)^{-2-n}(1+|x|^2)^{2+n}f(x) dx \nonumber\\
		&\leq& \int |h(x)|(1+|x|^2)^{-2-n}dx\m \sup_{x\in \R^d}(1+|x|^2)^{2+n}|f(x)|\nonumber\\
		&\leq& C||f||_N\nonumber
	\end{eqnarray}
	Für $h\in L^p(\R^d)$ gilt mit Hölder $\int_{|x|\leq R}|h(x)|dx\leq \left(\int_{|x|\leq R}|h(x)|^p dx \right)^{1/p} R^{d/p'}$. Hier ist $(*)$ erfüllt mit $n>d/p'$.\\
	b) z.B. $\psi f\in S(\R^d)$.
	\qed
	
	\begin{Definition}[Prinzip der Dualität]
		Sei $T:S(\R^d)\rightarrow S(\R^d)$ linear und stetig. Dann definiere die \textbf{duale Abbildung} $T':S'(\R^d)\rightarrow S'(\R^d)$ durch $(U\in S'(\R^d), f\in S(\R^d))$
		$$(T'u)(f) = u(T(f)).$$
	\end{Definition}
	
	
	\paragraph{Bemerkung.} $f_n\overset{S}{\rightarrow} f$ $\Rightarrow$ $Tf_n\overset{S}{\rightarrow} Tf$. $u(Tf_n) \rightarrow u(Tf)$, $T'u(f_n)\rightarrow T'u(f)$, da $T$ und $u$ stetig und linear nach Definition.
	
	\begin{Definition}
		$\F:S(\R^d)\rightarrow S(\R^d)$ $\Rightarrow$ $\F':S'(\R^d)\rightarrow S'(\R^d)$ mit 
		$$(\F'u)(f) = u(\f),~~ f\in S,~ u\in S'.$$
	\end{Definition}
	
	\paragraph{Bemerkung.} $h\in L^1(\R^d)\rightarrow u_h\in S'(\R^d)$. Für $f\in S(\R^d)$ gilt 
	$$(\F'u_h)(f) = u_h(\f) = \int h(x) \f(x) dx \overset{\text{Sec. 3}}{=}\int \hat{h}(x)f(x) dx = u_{\hat h (x)}(f),~\forall f\in S.$$
	Also $\F'u_h = u_{\F h}$.
	
	Notation: $\F'\hat{=}\F$. Dann $\boxed{\F(u_h)=u_{\F h}$, $\hat{u}_h = u_{\hat h}.}$
	
	\begin{Prop}
		$\F: S'(\R^d)\rightarrow S'(\R^d)$ ist bijektiv.
		$$(\F^{-1}u)(f) = (\F u)(\tilde{f}), \text{ wobei } \tilde{f}(x) = f(-x).$$
	\end{Prop}
	
	\Bew $\F\m\F^{-1} = \id_S$. Dualität: 
	$$(\F^{-1})'\F' = (\F\F^{-1})' = \id_S'= \id_{S'}.$$
	$\Rightarrow$ $(\F')^{-1} = (\F^{-1})'$ und damit bijektiv. Außerdem 
	$$(F^{-1}u)(f) = u(\F^{-1}f) = u(\F f(-\bullet)) =  \F'u(f(-\bullet)) = \F'u(\tilde{f}).$$
	\qed
	
	
	\begin{Definition}
		Sei $u\in S'(\R^d)$, $\alpha\in \N_0^d$. Definiere $D^\alpha u$ durch
		$$(D^\alpha u)(f) = (-1)^{|\alpha|}u(D^\alpha f)\text{ und } D^\alpha u\in S'(\R^d).$$
	\end{Definition}
	
	\paragraph{Bemerkung:} ~
	\begin{enumerate}
		\item[a)] $D^\alpha: S(\R^d)\rightarrow S(\R^d)$. 
		$$D^\alpha u := (-1)^{|\alpha|}(D_S^\alpha)'u \text{ (im Sinne der Dualität)}$$
		denn $(D_{S'}^\alpha u)(f) = (-1)^{|\alpha|} u(D^\alpha f)$.
		\item[b)] Sei $h\in S(\R^d)$ und $u_h\in S'(\R^d)$. 
		$$D^\alpha(u_h)(f) = (-1)^{|\alpha|} u_h(D^\alpha f) = (-1)^{|\alpha|}\int h(x) D^\alpha f(x) dx \overset{\text{p.I.}}{=} \int(D^\alpha h)(x) f(x) dx.$$
		Also: $\boxed{D^\alpha (u_h) = u_{D^\alpha h}.}$
	\end{enumerate}
	
	\begin{Prop}
		Sei $h\in C^{\infty}(\R^d)$ eine langsam wachsende Funktion, $u\in S'^{\R^d}$. Dann gilt
		$$D_{x_i}(h\m u) = (D_{x_i}h) \m u + h D_{x_i}u$$
		sowie für $\alpha,\beta\in \N_0^d$
		$$D^{\alpha+\beta}u = D^\alpha(D^\beta u).$$
	\end{Prop}
	
	\Bew Übung.
	\qed
	
	\begin{Beispiel}
		~
		\begin{enumerate}
			\item[a)] Sei $d= 1$, $H(x) =\left\{\begin{array}{l}
			1, ~x\geq 0\\
			0, ~x< 0
			\end{array}
			\right.$. Behauptung: $D(u_H) = \delta_0$, denn für $\phi\in S(\R)$ gilt
			\begin{eqnarray}
				(Du_h)(\phi) = -u_h(D\phi) = -\int H(x)\phi'(x)dx = \int_0^\infty \phi'(x) dx = \phi(0)\nonumber
			\end{eqnarray}
			Außerdem: für $\delta_a$
			$$(D^\alpha\delta_a)(\phi) = (-1)^{|\alpha|}(D^\alpha\phi) = (-1)^{|\alpha|}D^\alpha\phi(0).$$
		\end{enumerate}
	\end{Beispiel}
	
	\paragraph{Vorschau:} Zu $u\in S'(\R^d)$ gibt es eine langsam wachsende, stetige Funktion $\psi:\R^d\rightarrow\C$ und $\alpha\in \N_0^d$, sodass $u= D^\alpha\psi$.
	
	\begin{Prop}
		Für $u\in S'(\R^d)$ und $\alpha\in \N_0^d$ gilt:
		\begin{enumerate}
			\item[a)] $\F(D^\alpha u) = (2i\pi \m)^\alpha \F u$.
			\item[b)] $D^\alpha (\F u) = \F((-2i\pi \m)^\alpha u)$.
		\end{enumerate}
	\end{Prop}
	\Bew a) Für $f\in S(\R^d)$ erhält man nach Kapitel 3
	\begin{eqnarray}
		[\F(D^\alpha u)](f)&\overset{\text{Def.}}{=}& (D^\alpha u)(\F f)\nonumber\\
		&\overset{\text{Def.}}{=} & (-1)^{|\alpha|} u(D^\alpha \F f)\nonumber\\
		&=& (-1)^{|\alpha|} u (\F((-2i\pi\m)^\alpha f)) \nonumber\\
		&\overset{\text{Def.}}{=}& (-1)^{|\alpha|}(\F u)((-2i\pi\m)^\alpha f)\nonumber\\
		&=& [(2i\pi)^\alpha \F u](f)\nonumber
	\end{eqnarray}
	b) analog.
	\qed
	
	\paragraph{Erinnerung:} Für $f\in S(\R^d)$, $y\in \R^d$, $a>0$ gilt
	\begin{eqnarray}
		(\tau^Y f)(x) &:=& f(x-y)\nonumber\\
		(\delta^a f)(x)&:=& f(ax)\nonumber\\
		\tilde{f}(x) &:= & f(-x)\nonumber
	\end{eqnarray}
	Dann folgt mit einfacher Substitution
	\begin{eqnarray}
		u_{\tau^y g}(f) &=& \int_{\R^d} \tau^Y(g)f dx \nonumber\\
		&=& \int_{\R^d}g(x-y) f(x) dx \nonumber\\
		&\overset{\text{Subst.}}{=}& \int_{\R^d} g(x)f(x+y)dx = u_g(\tau^{-y}f)\nonumber\\\nonumber \\
		u_{\delta^a g}(f) &=& \int_{\R^d}g(ax)f(x)dx \nonumber\\
		&\overset{\text{Subst.}}{=}& a^{-d}\int_{\R^d} g(x)f(\frac{1}{a}x)dx \nonumber\\
		&=& a^{-d}u_g(\delta^{1/a}f)\nonumber\\\nonumber\\
		u_{\tilde{g}}(f) &=& \int_{\R^d}g(x)f(x)dx \nonumber\\
		&=& \int_{\R^d}g(x)f(-x) dx\nonumber\\
		&=& u_g(\tilde{f})\nonumber
	\end{eqnarray}
	
	\begin{Prop}
		Es gelten für $u\in S'(\R^d)$, $y\in \R^d$, $a>0$ die Regeln:
		\begin{enumerate}
			\item[a)] $\F(\tau^y u) = e^{-2i\pi y x} \hat{u}$.
			\item[b)] $\tau^y \hat u =\F(e^{2i\pi x y} u)$.
			\item[c)] $\F(\delta^a u) = a^{-d}\delta^{1/a} \hat u$.
			\item[d)] $\F(\tilde{u}) = \widetilde{\F(u)}$.
		\end{enumerate}
	\end{Prop}
	
	\Bew a) Für $f\in S(\R^d)$ gilt
	\begin{eqnarray}
		[F(\tau^y u)](f) &\overset{\text{Def}}{=}& (\tau^y u)(\f)\nonumber\\
		&\overset{\text{Def}}{=}& u(\tau^{-y}\f)\nonumber\\
		&\overset{\text{Def.}}{=}& u(\F(e^{-2i\pi xy}f))\nonumber\\
		&\overset{\text{Def.}}{=}& \hat u (e^{-2i\pi xy}f)\nonumber\\
		&\overset{\text{Def.}}{=}& [e^{-2i\pi xy}\hat u](f)\nonumber
	\end{eqnarray}
	b), c), d) analog
	\qed
	
	\begin{Satz}
		Zu $u\in S'(\R^d)$ existiert eine langsam wachsende Funktion $\varphi:\R^d\rightarrow \C$ und $\gamma\in \N_0^d$, sodass
		\begin{eqnarray}
			u &=& D^\gamma\varphi (= D^\gamma u_\varphi)\nonumber\\
			( u(f) &=& (-1)^{|\gamma|}\int_{\R^d}\varphi(x) D^{\gamma} f(x)dx ) \nonumber
		\end{eqnarray}
	\end{Satz}
	
	\Bew Idee: Verwende Darstellungssatz von Riesz für $L^1$.
	\begin{enumerate}
		\item Für $N\in \N$ definiere $S_N:=(S(\R^d),||.||_N)$, wobei
		$$||f||_N = \sup_{\lambda\in \R^d}\sup_{|\alpha|,|\beta|\leq N}|x^\beta D^\alpha f(x)|$$
		und sei $S'_N$ der Dualraum von $S_N$. Definiert man 
		$$I_N:= \{\alpha\in \N_0^d: |\alpha|\leq N \}$$ 
		und $M:= \# I_N$ zu $f\in S(\R^d)$ und außerdem 
		$$\psi_f:\R^d\rightarrow \R^d,~~\psi_{f,\alpha}(x)= (1+|x|)^N D^{\alpha+1}f(x),~~\alpha \in I_N$$
		mit $1=(1,...,1)\in \N_0^d$. Dann ist $\psi_{f,\alpha}\in L^1(\R^d)$ für alle $\alpha\in I_N$, d.h. $\psi_f\in L^1(\R^d)^M$. 
		Außerdem ist $\psi_f$ durch $f$ eindeutig bestimmt.
		\item Sei nun $u\in S'(\R^d)$. Nach Satz 4.3 ist dann $u\in S'_N$ für ein $N\in \N$. Definiere außerdem $L(\psi_f):= u(f)$, $f\in S(\R^d)$. Nach 1. ist dann $L$ wohldefiniert.
		
		Weiter $|L(\psi_f)| = |u(f)|\leq C||f||_N = C\sup_x \sup_{|\alpha|,|\beta|\leq N}|x^\beta D^\alpha f(x)$
		\begin{eqnarray}
			|L(\psi_f)| &=& |u(f)|\leq C||f||_N\nonumber\\
			&=& C\sup_x \sup_{|\alpha|,|\beta|\leq N}|x^\beta D^\alpha f(x)\nonumber\\
			&\leq & C\sum_{|\alpha|\leq N} \sup_x(1+|x|)^N|D^\alpha f(x)|\nonumber\\
			&\overset{x_0 = x_{0,\alpha}}{=}& C\sum_{|\alpha|\leq N} (1+|x_{0,\alpha})^N|D^\alpha f(x_{0,\alpha}|)\nonumber\\
			&=& C \sum_{|\alpha|\leq N} (1+|x_0|)^N \left( \int_{-\infty}^{x_{0,d}}...\int_{-\infty}^{x_{0,k+1}}...\int_{x_{0,k}}^{\infty}... D^{\alpha+1}(f(y_1,...,y_d))dx  \right)\nonumber\\
			&\leq& C||\psi_f||_{L^1(\R^d)^M}\nonumber
		\end{eqnarray}
		Definiere nun $\mathcal{L}^M:=\{\psi_f:\R^d\rightarrow\R^M: f\in S(\R^d) \}\subseteq L^1(\R^d)^M$. Wegen $\alpha\psi_f+\psi_g = \psi_{\alpha f+g}$ und der Linearität von $U$ ist $L$ linear auf $\mathcal{L}^M$ und nach obigem beschränkt auf $(\mathcal{L}^M,||.||_{L^1(\R^d)^M})$, d.h. $L\in (\mathcal{L}^{M})'$. Nach Hahn-Banach existiert nun eine Erweiterung $\tilde{L}\in (L^1(\R^d)^M)'$ und nach Darstellungssatz von Riesz gilt
		$$(L^1(\R^d)^M)'\cong (L^1(\R^d)')^M\cong L^\infty(\R^d)^M.$$
		Damit gilt 
		\begin{eqnarray}
			u(f) = \tilde{L}(\psi_f) &=& \sum_{|\alpha|\leq N}\int_{\R^d}\psi_{f,\alpha}g_\alpha dx\nonumber\\
			&=&\sum_{|\alpha|\leq N}\int_{\R^d} (1+|x|)^N D^{\alpha + 1}f(x) g_\alpha(x) dx\nonumber
		\end{eqnarray}
		mit $g_\alpha\in L^\infty(\R^d)$ für $\alpha\in I_N$.
		Mit mehrfacher partieller Integration kann man dies weiter vereinfachen zu
		$$u(f) = \int_{\R^d} \tilde{\varphi}D^{\tilde{\gamma}}f dx$$
		mit $\tilde{\gamma} = (N+1,...,N+1)$ und $\tilde{\varphi}=$ Linearkombinationen aus $(1+|x|)^Ng_\alpha$ und deren Stammfunktionen. Insbesondere ist $\tilde{\varphi}$ langsam wachsende Funktion. Nochmals partielles Integrieren liefert dann ein stetiges $\varphi$ und $\gamma\in \N_0^d$ mit gewünschter Eigenschaft.
	\end{enumerate}
	\qed
	
	\subsection{Faltung und Distributionen}
	
	Wir begnügen uns damit Distributionen $u$ mit Schwarzfunktionen $f$ zu falten. Die Definition wird motiviert durch
	$$(f*g)(x)=\int_{\R^d}f(x-y)g(y) dy = u_g(f(x-\m)).$$
	
	\begin{Definition}
		Sei $u\in S'(\R^d)$ und $f\in S(\R^d)$. Dann definiere
		$$(f*u)(x):= u(f(x-\m)),~ x\in \R^d.$$
	\end{Definition}
	
	\begin{Satz}
		Sei $u\in S'(\R^d)$, $f\in S(\R^d)$. Dann ist die Funktion $x\mapsto (f*u)(x)$ eine langsam wachsende $C^\infty-$Funktion, mit 
		$$D(f*u) = (D^\alpha f)*u = f*(D^\alpha u).$$
	\end{Satz}
	
	\Bew Nach Kapitel 4 existiert zu $u\in S'(\R^d)$ ein $N\in \N$, $C<\infty$ mit 
	$$|u(f)|\leq C||f||_N.$$
	Damit folgt
	\begin{eqnarray}
		|(f*u)(x)| &=& |u(f(x-\m))|\nonumber\\
		&\leq& C\sup_y \sup_{|\alpha|,|\beta|\leq N}|y^\alpha D^\beta f(x-y)|\nonumber\\
		&=& C\sup_y\sup_{|\alpha|,|\beta|\leq N} |(x-y)^\alpha D^\beta f(y)|\nonumber\\
		&\leq& C\m \tilde{C}(1+|x|)^N, \nonumber
	\end{eqnarray}
	wobei $\tilde{C}\geq \sup_{|\alpha|,|\beta|\leq N}(1+|y|)^N|D^\beta f(u)|$.
	
	Zeige nun
	$$D_{x_i}(f*u) = (D_{x_i}f*u).$$
	Betrachte dazu den Differenzenquotienten
	\begin{eqnarray}
		\frac{1}{n}[(f*u)(x+h e_i)-(f*u)(x)] &\overset{\text{Def.}}{=}& u(\frac{1}{n}[f(x+he_i -\m)-f(x-\m)])\nonumber
	\end{eqnarray}
	Da jedes $f\in S(\R^d)$ gleichmäßig stetig (und alle Ableitungen ebenso) ist, folgt für festes $x\in \R^d$ und $N\in \N$.
	$$||[\frac{1}{n}f(x+he_i-\m)-f(x-\m)]-D_{x_i}f(x-\m)||_N\rightarrow 0\text{ für } h\rightarrow 0.$$
	Mit der Stetigkeit von $u$ folgt nun
	\begin{eqnarray}
		D_{x_i}(f*u)(x) &=& u(D_{x_i}f(x-\m))\nonumber\\
		&=& ((D_{x_i}f)*u)(x)\nonumber
	\end{eqnarray}
	Wiederholung des Arguments liefert
	$$D^\alpha (f*u) = (D^\alpha f)*u).$$
	Die letzte Behauptung folgt aus 
	\begin{eqnarray}
		(D^\alpha f*u)(x) &=& u((D^\alpha f)(x-\m))\nonumber\\
		&=& u((-1)^{|\alpha|}D^\alpha (f(x-\m))\nonumber\\
		&=& (D^\alpha u)(f(x-\m)) = (f*D^\alpha u)(x)\nonumber
	\end{eqnarray}
	\qed
	
	Für alternative Darstellungen benötigen wir das folgende
	
	\begin{Lemma}
		Seien $u\in S'(\R^d),~f,g\in S(\R^d)$. Dann gilt 
		$$u\left(\int_{\R^d}f(u-\m)g(y) dy \right) = \int_{\R^d}u(f(y-\m))g(y) dy.$$
	\end{Lemma}
	
	\Bew 
	\begin{enumerate}
		\item Für jedes $N\in \N$ sei $(Q_m)_{m= 1}^{(2N^2)^d}$ die Zerlegung von $[-N,N]^d$ in Würfel der Seitenlänge $1/N$ mit Mittelpunkt $y_m$. Zeige nun, dass die Riemannsumme $R_N(x) := \sum_{m=1}^{(2N^2)^d}f(y_m-x)g(y_m)|Q_m|$ in $S(\R^d)$ gegen $\int_{\R^d}f(y-x)g(y)dy$ konvergiert, d.h.
		$$||R_N-\int_{\R^d}f(y-\m)g(y) dy||_{x,\beta} = \sup_{\lambda\in \R^d}|\sum_{m= 1}^{(2N^2)^d}x^\alpha D^\beta(f(y_m-x))g(y_m)|Q_m| - \int_{\R^d}x^\alpha D_x^\beta (f(y-x))g(y)dy|\overset{N\rightarrow\infty}{\rightarrow} 0.$$
		Es gilt zum einen: 
		\begin{eqnarray}
			x^\alpha (-1)^{|\beta|}(D^\beta f)(y_m-x)g(y_m)|Q_m|&-&(-1)^{|\beta|}\int_{\Q_m} x^\alpha(D^\beta f)(y-x)g(y) dy\nonumber\\
			&=& (-1)^{|\beta|}\int_{Q_m}x^\alpha ((D^\beta f)(u_m-x)g(y_m)-(D^\beta f) (y-x)g(y)) dy \nonumber\\
			&=& (-1)^{|\beta|}\int_{Q_m} x^\alpha(y_m-y)[\nabla_y (D^\beta f(\m-x)g)](\xi) dy = (*)\nonumber
		\end{eqnarray}
		 für $\xi = y+\theta(y_m-y)$, $\theta \in [0,1]$. Wegen $|y|\leq |\xi|+\theta|y_m-y|\leq |\xi|+\frac{\sqrt{d}}{N}\leq |\xi|+1$ für $N>\sqrt{d}$, folgt
		 \begin{eqnarray}
		 	|(*)| &\leq& c_1\frac{|x|^{|\alpha|}}{(1+|x|)^M}\frac{\sqrt{d}}{N}\int_{Q_m} \frac{1}{(1+|y|)^M}dy\nonumber
		 \end{eqnarray}
		 Damit erhält man 
		 \begin{eqnarray}
		 	|D_N(x)|&\leq& c_1\frac{|x|^{|\alpha|}}{(1+|x|)^M}\frac{\sqrt{d}}{N} \int_{|y|\leq N}\frac{1}{(1+|y|)^M}dy+ \int_{|y|> N} |x^{|\alpha|}D^\beta f(y-x)g(y)dy \nonumber\\
		 	&\leq& c_1 \frac{|x|^{|\alpha|}}{(1+|x|)^M}\frac{\sqrt{d}}{N}\int_{|y|_\infty\leq N} \frac{1}{(1+|y|)^M}dy + c_2 \frac{|x|^{|\alpha|}}{(1+|x|)^M} \int_{|y|_\infty> N} \frac{1}{(1+|y|)^M} dy\nonumber\\
		 	&\rightarrow& 0 \text{ für } N\rightarrow 0 \text{ (unabhängig von x)}\nonumber
		 \end{eqnarray}
		 \item Mit 1. erhält man schließlich 
		 \begin{eqnarray}
		 	u\left(\int_{\R^d} f(y-\m)g(y) dy \right) &=& \lim\limits_{N\rightarrow \infty} u(R_N)\nonumber\\
		 	&=& \lim\limits_{N\rightarrow \infty}\sum_{m = 1}^{(2N^2)^d}u(f(y_m-\m))g(y_m)|Q_M|\nonumber\\
		 	&=& \int_{\R^d} u(f(y-\m))g(y)dy\nonumber
		 \end{eqnarray}
		 da $y\mapsto u(f(y-\m))g(y)\in S(\R^d)$ und damit Riemann-integrierbar
	\end{enumerate}
	\qed
	
	Damit erhält man
	\begin{Prop}
		Für $u\in S'(\R^d)$, $f\in S(\R^d)$ gilt $$(f*u)(g) = u(\tilde{f}*g) ~\forall g\in S(\R^d)$$
		($f*u$ als Distribution aufgefasst).
	\end{Prop}
	
	\Bew Nach Kapitel 3 gilt $\tilde{f}*g\in S(\R^d)$, d.h. die rechte Seite ist wohldefiniert. Außerdem gilt
	\begin{eqnarray}
		u(\tilde{f}*g) &=& u\left(\int_{\R^d}f(y-\m)g(y) dy \right)\nonumber\\
		&\overset{\text{5.3}}{=} & \int_{\R^d} u(f(y-\m))g(y)dy\nonumber\\
		&=& \int_{\R^d}(f*u)(y) g(y) dy\nonumber\\
		&=& (f*u)(g)\nonumber.
	\end{eqnarray}
	\qed
	
	\begin{Prop}
		Für $u\in S'(\R^d)$, $f,g\in S(\R^d)$ gilt 
		$$f*(g*u)= (f*g)*u.$$
	\end{Prop}
	
	\Bew 
	\begin{eqnarray}
		\left[(f*g)*u \right](x) &=& u((f*g)(x-\m)\nonumber\\
		&=& u\left(\int_{\R^d} g(x-y-\m)f(y) dy \right) \nonumber\\
		&\overset{\text{5.3}}{=}& \int_{\R^d} u(g(x-y-\m))f(y) dy\nonumber\\
		&=& \int_{\R^d} (g*u)(x-y)f(y)dy\nonumber\\
		&=& [f*(g*u)](x)\nonumber.
	\end{eqnarray}
	\qed
	
	\begin{Prop}
		Für $u\in S'(\R^d)$, $f\in S(\R^d)$ gilt
		\begin{enumerate}
			\item[a)] $\F(f*u) = \f\m \hat{u}$.
			\item[b)] $\F(f\m u) = \f *\hat{u}$.
		\end{enumerate}
	\end{Prop}
	
	\Bew a) Für $g\in S(\R^d)$ gilt
	\begin{eqnarray}
		\F(f*u)(g) &=& (f*u)(\g) \overset{\text{5.4}}{=} u(\tilde{f}*\tilde{g})\nonumber\\
		(\f\m \hat{u})(g) &=& \hat{u}(\f\m g) = u(\F(\f\m g)) = u(\F(\f)*\g) = u(\tilde{f}* g). \nonumber
	\end{eqnarray}
	b) analog.
	\qed
	
	So kommt man schließlich zu einem Dichtheitsresultat:
	
	\begin{Satz}
		Zu jedem $u\in S'(\R^d)$ existiert eine Folge $(f_k)_{k\in \N}\subseteq C_c^\infty(\R^d)$ mit $f_k\rightarrow u$ \glqq im distributiven Sinne\grqq, d.h.
		$$u_{f_k}(g)\overset{k\rightarrow \infty}{\rightarrow}u(g) ~\forall g\in S(\R^d).$$
	\end{Satz}
	
	\Bew Sei $\varphi\in C_c^\infty(\R^d)$ mit $\varphi(x) = 1$ auf $B(0,R)$ für ein $R>0$ und setze $\varphi_k(x) = \varphi(\frac{1}{k}x)$. Zeige zunächst zwei Hilfsbehauptungen. Für $f\in S(\R^d)$ gilt
	\begin{enumerate}
		\item $\lim\limits_{k\rightarrow \infty}\varphi_k f= f$ in $S(\R^d)$.
		\item $\lim\limits_{k\rightarrow\infty}\varphi_k\F(\varphi_k f) = \f$ in $S(\R^d)$.
	\end{enumerate}
	
	\Bew 1. Seien $\alpha,\beta\in \N_0^d$ beliebig. Dann gilt
	\begin{eqnarray}
		||\varphi_k f- f||_{\alpha,\beta} &=& \sup_{x\in \R^d}|x^\alpha D^\beta((\varphi_k(x)-1)f(x))| \nonumber\\
		&\overset{\text{Leibnitz}}{=}& \sup_{x\in \R^d} |x^\alpha \sum_{\gamma_1 = 0}^{\beta_1}...\sum_{\gamma_d = 0}^{\beta_d}\left(\begin{matrix}
		\beta_1\\
		\gamma_1
		\end{matrix}\right)\m...\m \left(\begin{matrix}
		\beta_d\\
		\gamma_d
		\end{matrix}\right) (D^\gamma(\varphi_k-1)) (D^{\beta\gamma}f)|\nonumber\\
		&\leq& \sup_{x\in \R^d}\left[|x^\alpha (\varphi_k (x)-1) D^\beta f(x)  + \sum_{\gamma\neq 0} \left(\begin{matrix}
		\beta\\
		\gamma
		\end{matrix}\right) \frac{1}{k^{|\gamma|}}|x^\alpha (D^\gamma\varphi)(\frac{x}{k}) D^{\beta-\gamma} f(x)|\right]\nonumber\\
		&\leq& \sup_{x\in \R^d}|x^\alpha(\varphi_k(x)- 1)D^\beta f(x)| +\frac{1}{k} c_\beta ||\varphi||_N ||f||_N
	\end{eqnarray} Nun ist $\varphi_k(x)-1 = 0$  für $|x|<kR$ und $|x^\alpha D^\beta f(x)| \leq c\frac{1}{|x|}\leq c\frac{1}{kR}$ für $|x|\leq kR$.
	Damit folgt
	$$||\varphi_k f-f||_{\alpha,\beta}\leq \frac{c}{kR}+\frac{1}{k}c_\beta||\varphi||_N ||f||_N\rightarrow 0 ~~(k\rightarrow 0).$$
	
	2. Seien $\alpha,\beta\in \N_0^d$ beliebig. Nach 1. und wegen der Stetigkeit von $\F: S\rightarrow S$ gilt
	$$\lim\limits_{k\rightarrow\infty}\F(\varphi_k f) = \f$$ 
	in $S(\R^d)$ für alle $f\in S(\R^d)$. Nach 1. und da $\F(\varphi_j f)\in S(\R^d)$ folgt dann auch 
	$$\lim\limits_{k\rightarrow\infty}\varphi_k \F(\varphi_j f) = \F(\varphi_j f)$$
	in $S(\R^d)$ für alle $j\in \N$. Wähle nun zu $\epsilon>0$ ein $j_\epsilon\in \N$ mit $||\F(\varphi_j f)-\f||_{\alpha,\beta} <\epsilon$ und $||\F(\varphi_k f)-\F(\varphi_j f)||_{\alpha,\beta}<\epsilon$ für alle $k,j\geq j_\epsilon$. Dann gilt für $j\geq j\epsilon$
	\begin{eqnarray}
		||\varphi_k\F(\varphi_k f)-\f||_{\alpha,\beta}&\leq & ||\varphi_k\F(\varphi_k f) - \varphi_k\F(\varphi_j f)||_{\alpha,\beta}\nonumber\\
		&+& ||\varphi_k\F(\varphi_j f) - \F(\varphi_j f)||_{\alpha,\beta}\nonumber\\
		&+& ||\F(\varphi_j f) - \f||_{\alpha,\beta}\nonumber\\
		&\rightarrow& 2\epsilon \text{ für } k\rightarrow\infty\nonumber
	\end{eqnarray} % TODO fehlt ein Teil
	\qed
	
	Sei nun $u\in S'(\R^d)$. Dann definiere 
	$$f_k: = \varphi_k(\hat{\varphi}* u)\in C_c^\infty(\R^d) \text{ nach Satz 5.2}.$$
	Damit folgt 
	\begin{eqnarray}
		u_{f_k}(g) &\overset{\text{5.6}}{=}& \varphi_k\m\F(\varphi_k\m\F^{-1}(u))(g) \nonumber\\
		&=& \F^{-1}(\varphi_k\F(\varphi_k g))\nonumber\\
		&\rightarrow& \F^{-1}u(\g) = u(g)~(k\rightarrow\infty)\nonumber
	\end{eqnarray}
	nach Behauptung 2.
	\qed

	\newpage
	
	\subsection{Sobolevräume}
	
	Sei $f\in L^2(\R^d)\rightarrow u_f\in S'$, $u_f(h) =\int f(x)g(x) dx$. Wann gilt $D^\alpha u_f\in S'$ $\Rightarrow$ $D^\alpha u_f \in L^2$?.
	
	\begin{Definition}
		Für $K\in \N$: 
		\begin{eqnarray}
			H^K(\R^d) =\{f\in L^2(\R^d): D^\alpha f\in L^2, \forall |\alpha|\leq K \}\nonumber\\
			\text{genauer } f\in H^\alpha\Leftrightarrow \forall|\alpha|\leq K\exists f_\alpha: U-f=D^\alpha u_f\nonumber
		\end{eqnarray}
		mit der Norm
		$$||f||_K:= \left(\sum_{|\alpha|\leq K}||D^\alpha f||_{L^2(\R^d)}^2 \right)^{1/2}.$$
	\end{Definition}
	
	\begin{Bemerkung}
		~\begin{enumerate}
			\item[a)] (schwache Ableitung) Zu $D^\alpha f\in L^2(\R^d)$ gibt es ein $f_\alpha\in L^2$, sodass für $h\in S(\R^d)$ gilt:
			\begin{eqnarray}
				\int f_\alpha(x) h(x) dx &=& D^\alpha(u_f)(h) \nonumber\\
				&=& (-1)^{|\alpha|}u_f(D^\alpha h)\nonumber\\
				&=& (-1)^{|\alpha|}\int f(x) (D^\alpha h)(x) dx.\nonumber
			\end{eqnarray}
			\item[b)] Die Räume $H^K$ sind vollständig.
			\item[c)] $H^K$ sind Hilberträume bezüglich 
			$$<f,g>_K = \sum_{|\alpha|\leq K} \int_{\R^d} (D^\alpha f)(x) \overline{D^\alpha g(x)}dx,~~ <f,f>_K = ||f||_K^2.$$
			\item[d)] Berechne $||f||_K$ mit Hilfe von $\F$.
			\begin{eqnarray}
				||D^\alpha f||_{L^2}^2 &=&||\F(D^\alpha f)||_{L^2}^2\nonumber\\
				&=& ||(2i\pi\xi)^\alpha\f(\xi)||_{L^2}^2 \nonumber\\ 
				&=& (2\pi)^{|\alpha|}\int_{\R^d}|\xi^\alpha||\f(\xi)^2 d\xi\nonumber\\
				\Rightarrow \sum_{|\alpha|\leq K}(2\pi)^{|\alpha|}\int |\xi^\alpha|^2 |\f(\xi)|^2d\xi &=& \int_{\R^d}\left(\sum_{|\alpha|\leq K} (2\pi)^{|\alpha|} |\xi^\alpha| \right)|\f(\xi)|^2d\xi\approx (1+|\xi|^2)^K\nonumber
			\end{eqnarray}
			denn:
			\begin{eqnarray}
				|x^\alpha|&\leq& 1+|x|^{2k}\nonumber\\
				(1+|x|^K)^2 &=& (1+(\sum |x_i|^2)^{K/2})^2\nonumber\\
				&\leq& (1+(\sum_{i = 1}^{d}|x_i|^K))^2\nonumber\\
				&\leq& C(\sum_{|\alpha|\leq K}|x^\alpha|^2)\nonumber
			\end{eqnarray}
			Also 
			$$||f||_K^2\cong \int_{\R^d}(1+|\xi|^2)^{K/2}|\f(\xi)|^2 d\xi,K\in \N. $$
		\end{enumerate}
	\end{Bemerkung}
	
	\begin{Definition}
		Für alle $s\in \R$ definiere den Sobolevraum
		\begin{eqnarray}
			H^s(\R^d) &=& \{f\in S'(\R^d): (1+|\xi|^2)^{s/2}\f(\xi)\in L^2(\R^d) \}\nonumber\\
			||f||_{H^s} &=& \left(\int_{\R^d}|\f(\xi)|^2(1+|\xi|^2)^sd\xi \right)^{1/2}\nonumber\\
			<f,g>_{H^s} &=& \int_{\R^d} \f(\xi)\overline{\g(\xi)}(1+|\xi|^2)^s d\xi\nonumber
		\end{eqnarray}
		für $s=K\in\N: ||f||_{H^s}\cong ||f||_{H^\alpha}$.
	\end{Definition}
	
	\begin{Prop}
		Die Räume $H^s(\R^d)$, $s\in \R$ sind Hilberträume, also insbesondere vollständig. $C_c^\infty(\R^d)$ ist dich in $H^s(\R^d)$.
	\end{Prop}
	
	\Bew $\F:S'(\R^d)\rightarrow S'(\R^d)$, $H^s\overset{\F|_{H^s}}{\leftrightarrow} L^2(\R^d, (1+|\xi|^2)^s)$, Isometrie von $H^s$ auf $L^2(\R^d, (1+|\xi|^2)^s)$, wobei $L^2(\R^d, (1+|\xi|^2)^s)$ vollständig ist. Somit ist auch $H^s$ vollständig und damit ein Hilbertraum.
	\begin{eqnarray}
		\F^{-1}: L^2(\R^d,(1+|\xi|^2)^s d\xi) &\overset{\F^{-1}}{\rightarrow}& H^s\nonumber\\
		S(\R^d)&\overset{\F^{-1}}{\rightarrow}& S(\R^d)\nonumber
	\end{eqnarray}
	Also liegt $S(\R^d)$ dicht in $H^s(\R^d)$. Elementar: $C_0^\infty(\R^d)$ in $S(\R^d)$.
	\qed
	
	\begin{Prop}
		$J:H^{-s}\rightarrow (H^s)'$, $J(u)(h) = \int \hat u(\xi) \hat h(\xi)d\xi$ für $u\in H^{-s}$, $h\in H^s$ definiert eine surjektive Isometrie von $H^{-s}$ auf $(H^s)'$ - den Banachraum dual von $H^s$ bezüglich der Dualität 
		$$(f,g) = \int \f(\xi)\g(\xi)d\xi \text{ (bilineare Abbildung)}.$$
	\end{Prop}
	
	\Bew Sei $u\in H^{-s}$, $h\in H^s$. 
	\begin{eqnarray}
		J(u)(h) &=& \int\hat u(\xi)\hat h(\xi)d\xi\nonumber\\
		&=& \int\hat u(\xi) (1+|\xi|^2)^{-s/2}\hat h(\xi)(1+|\xi|^2)^{s/2}d\xi\nonumber\\
		&\leq& \left(\int |\hat u(\xi)|^2 (1+|\xi|^2)^{-s} d\xi \right)^{1/2}\m \left(\int |\hat h(\xi)|^2 (1+|\xi|^2)^{s} d\xi \right)^{1/2}\nonumber\\
		&=& ||u||_{H^{-s}} ||h||_{H^s} \nonumber
	\end{eqnarray}
	Da $J$ stetig gilt $||J u||_{(H^s)'}\leq ||u||_{H^{-s}}$. Sei nun $u\in H^{-s}$, $||u||_{H^{-s}}= 1$ gegeben. Wähle $g = \F^{-1}[\hat u(\xi)(1+|\xi|^2)^{-s}]$. Dann gilt
	\begin{eqnarray}
		||g||_s &=& \left(\int|\hat u(1+|\xi|^2)^{-s/2}|^s d\xi \right)^{1/2}\nonumber\\
		&=& ||u||_{H^{-s}} = 1\nonumber\\
		(Ju)(g) &=& \int |\hat u(\xi)|^2(1+|\xi|^2)^{-s}d\xi\nonumber\\
		&=& ||u||_{H^{-s}} = 1\nonumber\\
		\Rightarrow ||Ju||_{(H^s)'} &=& 1\text{ für } ||u||_{H^{-s}} = 1\nonumber
	\end{eqnarray}
	Somit ist $J:H^{-s}\rightarrow (H^s)'$ eine isometrische Einbettung. 
	
	Zu zeigen bleibt, $J$ ist surjektiv. Zu $v\in(H^s)'$ gibt es ein $g\in H^s$ mit $v(h) = <h,g>_{H^s}$ nach Riesz.
	\begin{eqnarray}
		v(h) &=& \int \hat h(\xi)\overline{\g(\xi)} (1+|\xi|^2)^s d\xi\nonumber
	\end{eqnarray}
	für alle $h\in H^s$. Wähle $u=\F^{-1}[\overline{\g(\xi)}(1+|\xi|^2)^{s}]$. Dann:
	\begin{eqnarray}
		||u||_{H^{-s}} &=& ||\bar g||_{H^s} ~=~ ||g||_{H^s}\nonumber\\
		J(u)(h) &=& <h,g>_{s}~=~ v(h),~\forall h\in H\nonumber
	\end{eqnarray}
	\qed
	
	\begin{Prop}
		Ist $s<t$, so ist $H^t\subset H^s$.
	\end{Prop}
	
	\Bew Da $(1+|\xi|^2)^s\leq (1+|\xi|^2)^t$.
	\qed
	
	\begin{Satz}[Sobolevscher Einbettungssatz]
		Sei $s>d/2$. Dann gilt
		\begin{enumerate}
			\item[a)] $H^s\subset C_b(\R^d)$.
			\item[b)] $H^{s+K}\subset C_b^K(\R^d)$.
		\end{enumerate}
	\end{Satz}
	
	\Bew a) Für $u\in S(\R^d)$. 
	\begin{eqnarray}
		u(x) &=& \int_{\R^d} e^{2i\pi x\xi}(1+|\xi|^2)^{-s/2} \hat u(1+|\xi|^2)^{s/2} d\xi\nonumber\\
		&\leq& \left(\int (1+|\xi|^2)^{-s}d\xi \right)^{1/2} \left(\int |\hat u|^2 (1+|\xi|^2)^sd\xi \right)^{1/2}\nonumber 
	\end{eqnarray}
	mit $\left(\int (1+|\xi|^2)^{-s}d\xi \right)^{1/2}\leq \infty$ und $\left(\int |\hat u|^2 (1+|\xi|^2)^sd\xi \right)^{1/2} = ||u||_{H^s}$. 
	
	b) $|\alpha|\leq K$, $u\in H^{s+K}$ $\Rightarrow$ $D^\alpha\in H^s$, wende nun a) an $\Rightarrow$ Beh.
	\qed
	
	\begin{Korollar}
		Sei $s>d/2$, dann gilt: Der Raum der endlichen Maße $M(\R^d)\subset H^{-s}(\R^d)$.
	\end{Korollar}
	
	\paragraph{Bemerkung:} $\mu\in M(\R^d)$, $u_\mu\in S'(\R^d)$.
	\begin{eqnarray}
		u_\mu(h) &=& \int_{\R^d}h(x)d\mu(x)\nonumber\\
		\mu\in M(\R^d) &\Rightarrow& u_\mu \in H^{-s}(\R^d)\nonumber
	\end{eqnarray}
	
	\Bew 
	\begin{eqnarray}
		|u_\mu(h)| &=& |\int h(x)d\mu(x)|\nonumber\\
		&\leq& ||h||_{C_b(\R^d)}||\mu||_{M(\R^d)}\nonumber\\
		&\leq& ||h||_{H^s}\m ||\mu||_{M(\R^d)}\nonumber
	\end{eqnarray}
	Also: $u_\mu\in (H^s)'$, $||u_\mu||_{(H^s)'}\leq ||\mu||_{M(\R^d)}$, d.h. $u_\mu\in H^{-s}$. 
	
	z.B. $\delta_x\in H^{-s}$, $s>d/2$, $x\in \R^d$, denn $h\in H^s$, $h\in C_b(\R^d)$, $\delta_x(f) = f(x)$.
	
	\begin{Satz}
		Sei $s<t$ (d.h. $H^t\subseteq H^s$), $(u_n)_{n\in\N}\subset H^t$ und $||u_n||_{H^t}\leq 1$ und $\supp(u_n)\subset U$, $U\subset \R^d$ beschränkt. Dann hat $(u_n)$ eine in $H^s$ konvergente Teilfolge.
	\end{Satz}
	
	\paragraph{Ergänzung zur Einbettung.} $H^t\subset H^s(\R^d)$ für $t>s$.
	
	\begin{Satz}[Kompakte Einbettung auf beschränkten Mengen]
		Sei $u_n\subset H^t$, $||u_n||_{H^t}\leq C$,  und für ein $M<\infty$ $\supp u_n\subset B(0,M)$ für alle $n\in \N$. Dann hat $u_n$ eine Teilfolge, die in $H^s(\R^d)$ konvergiert für $s<t$.
	\end{Satz}
	
	\Bew 1. Schritt: $\hat u_n|_K, n\in \N$ ist relativ kompakt in $C(K)$ für alle kompakten Teilmengen $K\subset \R^d$. Denn: wähle $\psi\in C_c^\infty(\R^d)$ und $\psi\equiv 1$ auf $B(0, m)$.
	\begin{eqnarray}
		D^\alpha \hat u_n &=& D^\alpha \F(\psi u_n) \nonumber\\
		&=& D^\alpha[\hat{\psi}*\hat u_n]\nonumber\\
		&=& (D^\alpha\hat{\psi})*\hat u_n,\nonumber~~~\text{ also}\\
		|D^\alpha \hat u_n(\xi) |&\leq& \int D^\alpha \hat{\psi}(\xi-\eta)\hat u_n(\eta) d\eta\nonumber\\
		&\leq&\left(\int (1+|\eta|^2)^{-t}|D^\alpha\hat{\psi}(\xi-\eta)|^2d\eta\right)^{1/2}\left(\int |\hat u_n(\eta)|^2(1+|\eta|^2)^ d\eta \right)^{1/2}\nonumber\\
		&=:& f_\alpha(\xi)\m||u_n||_{H^t}\nonumber\\
		&\leq& Cf_\alpha(\xi)\in L^\infty(\R^d),\nonumber
	\end{eqnarray}
	denn $(1+|\eta|^2)^{-1}\in L^\infty$, $\xi\rightarrow |D^\alpha\hat{\psi}(\xi)|^2\in L^1(\R^d)$ (Youngsche Ungleichung).
	
	Da $\hat u_n|_K$ beschränkt und gleichgradig stetig (denn $D^\alpha f_n$ gleichmäßig beschränkt), folgt aus Azela-Ascoli, dass $(u_n)_{n\in \N}$ gleichmäßig konvergente Teilfolgen hat.
	
	2. Schritt: $u_n$ hat eine konvergente Teilfolge in $H^s$. Denn: Zu $\epsilon>0$ wähle ein $0<R<\infty$, sodass 
	\begin{eqnarray}
		(1+R^2)^{s-t}\leq\frac{\epsilon}{4c^2}\nonumber
	\end{eqnarray}
	Für $K=\{x\in \R^d: |x|\leq R \}$ wähle nach Schritt 1 eine Teilfolge von $(u_n)$ (wieder $u_n$ genannt), sodass $\hat u_n|_K$ gleichmäßig in $C(K)$ konvergiert. Dann
	\begin{eqnarray}
		||u_n-u_m||^2 &=& \int_{\R^d} |\hat u_n(\xi)-\hat u_m(\xi)|^2(1+|\xi|^2)^s d\xi \nonumber\\
		&=& \int_{|\xi|\leq R}...+\int_{|\xi|> R}...\nonumber
	\end{eqnarray}
	mit
	\begin{eqnarray}
		\int_{|\xi|>R}|\hat u_n(\xi)-\hat u_m(\xi)|^2(1+|\xi|^2)^sd\xi &\leq& \sup_{|\xi|>R} (1+|\xi|^2)^{s-t}||u_n-u_m||^2_{H^t}\nonumber\\
		&\leq& (1+R^2)^{s-t}(||u_n||_{H^t}+||u_m||_{H^t})\nonumber
		\leq \epsilon\nonumber
	\end{eqnarray}
	nach Wahl von $R$ und
	\begin{eqnarray}
		\int_{|\xi|\leq R} |\hat u_n(\xi)-\hat u_m(\xi)^2(1+|\xi|)^sd\xi&\leq& \int_{|\xi|\leq R}(1+|\xi|^2)^s d\xi \sup_{\xi\in K}|\hat{u_n}(\xi)-\hat u_m(\xi)|^2\nonumber\\
		&=& C_1\m\text{ Konst }\m ||\hat u_n-\hat u_m||_{C(K)}\overset{n,m\rightarrow\infty}{\longrightarrow} 0\nonumber
	\end{eqnarray}
	
	
	\begin{Prop}[Bernsteinsche Ungleichung]
		Für $f\in L^2(\R^d)$ mit $\supp\f\subseteq\{\xi: |\xi|>R \}$ gilt:
		\begin{enumerate}
			\item[a)] $f$ hat eine analytische Fortsetzung auf $\C^d$.
			\item[b)] Für alle $s<r$ gilt $f\in H^r$ und 
			$$||f||_{H^r}\leq (1+R)^{r-s}||f||_{H^s}.$$
		\end{enumerate}
	\end{Prop}
	
	\Bew a) Für $(z_1,..., z_d)\in \C^d$ setze 
	$$F(z_1,...,z_d) = \int_{|\xi|\leq R} \exp\left[2i\pi\left(\sum_{j = 1}^{d}z_j \xi_j\right)\right]\f(\xi_1,...,\xi_d)d\xi_1...\xi_d.$$
	Nach der Umkehrformel: $F(x_1,...,x_d) = f(x_1,...,x_d)$ für $x_1,...,x_d\in \R$. Also ist $F$ eine Fortsetzung von $f$ von $\R^d$ auf $\C^d$. Zu zeigen: $F$ ist partiell komplex differenzierbar nach $z_1,...,z_d$. Sei $w=w_1,...,w_d\in\C^d$ fest. Auf der Kugel $K=\{z\in \C^d: |z-w|\leq 1 \}$ ist 
	$\exp\left[2i\pi \left(\sum_{j = 1}^{d}z_j \xi_j\right) \right]$, $|\xi|\leq R$, $z\in K$ gleichmäßig beschränkt. Also ist Differentiation nach $w_1,...,w_d$ unter dem Integral erlaubt und die Behauptung folgt.
	
	b) 
	\begin{eqnarray}
		||f||_{H^r}^2 &=& \int |\f(\xi)|^2(1+|\xi|^2)^rd\xi\nonumber\\
		&\leq& \sup_{|\xi|\leq R}(1+|\xi|^2)^{r-s}\int_{|\xi|\leq R}|\f(\xi)|^2(1+|\xi|^2)d\xi\nonumber\\
		&\leq& (1+R^2)^{r-s}||f||_{H^s}^2\nonumber\\
		&\leq& (1+R)^{2(r-s)}||f||_{H^s}^2\nonumber
	\end{eqnarray}
	\qed
	
	
	\subsection{Der Funktionalkalkül des Laplace Operators}
	
	\begin{Definition}[des Laplace Operators]
		Für $f\in S(\R^d)$: $\Delta f(\m)=\sum_{j = 1}^{d}\frac{\partial^2}{\partial x_j^2} f(\m)$. 
		$$\xymatrix{
			S(\R^d)\ar[r]^{\Delta}\ar[d]_{\F} & S(\R^d)\ar[d]^{\F}\\
			S(\R^d)\ar[r]_M & S(\R^d)
			}$$
		\begin{eqnarray}
			\F(\Delta f)(\xi) &=& \sum_{j = 1}^{d}\F(\frac{\partial^2}{\partial x_j^2}f)(\xi) \nonumber\\
			&=& \sum_{j = 1}^d (2i\pi)^2\xi_j^2 (\xi)\nonumber\\
			&=& -(2\pi)^2 |\xi|^2 \f(\xi)\nonumber
		\end{eqnarray}
		$$Mg(\xi) = -(2\pi)^2|\xi|^2 g(\xi)$$
		$$\boxed{\Delta= \F^{-1}\circ M\circ \F.}$$
		$$D(A) = H^2(\R^d),~ f\in D(A):~ \F(\Delta f)(\xi) = -(2\pi)^2|\xi|^2\f(\xi).$$
	\end{Definition}
	
	\begin{Bemerkung}
		~
		\begin{enumerate}
			\item[a)] Der ungewohnte Faktor $(2\pi)^2$ kommt von der Definition der Fouriertransformation durch $e^{\underline{2\pi}i\xi x}$.
			\item[b)] $H^2 = \{f\in S': \frac{\partial^i}{\partial x_j^i}\in L^2(\R^d), i=1,2 \}$
			\begin{eqnarray}
				\Delta f = \sum_{j = 1}^d \frac{\partial^2}{\partial x_j^2} f\nonumber
			\end{eqnarray}
			ist in Ordnung, falls man $\frac{\partial^i}{\partial x_j^i}$ als distributionelle oder \textbf{schwache} Ableitungen versteht. 
			$$\int \varphi(x)\left[\frac{\partial^2}{\partial x_j^2} f(x)\right]dx = (-1)^{2}\int\left(\frac{\partial^2}{\partial x_j^2} \varphi(x) \right)dx$$
			für alle $\varphi\in C_c^\infty(\R^d)$.
			\item[c)] $\Delta: H^2\rightarrow L^2$ ist stetig und mit der Graphennorm
			$$||f||_{H^2}\cong ||f||_{L^2} + ||\Delta f||_{L^2}$$
			ist $\Delta$ insbesondere auf $L^2$ ein abgeschlossener Operator, d.h. $f_n\in D(A)$, $f_n\rightarrow f$ in $L^2$, $\Delta f_n\rightarrow g$ in $L^2$ $\Rightarrow$ $f\in D(A)$, $\Delta f = g$.
		\end{enumerate}
	\end{Bemerkung}
	
	\Bew 
	\begin{eqnarray}
		||f||_{H^2}^2 &=& \int |\f(\xi)|^2(1+|\xi|^2)^2 d\xi\nonumber\\
		&\approx& \int |\f(\xi)|^2d\xi + \int\left[|\f(\xi)||\xi|^2 \right]^2d\xi\nonumber\\
		&\approx& ||f||_{L^2}^2 + ||\Delta f||_{L^2}^2\nonumber
	\end{eqnarray}
	Also: $||\Delta u||_{L^2} \leq ||u||_\Delta\approx ||u||_{H^2}$ - Stetigkeit.
	\qed
	
	\begin{Bemerkung}[Ziele des Funktionalkalküls]
		Definition neuer Operatoren, z.B.
		\begin{itemize}
			\item $e^{-t\Delta}$ $\Rightarrow$ Lösung der Wärmeleitungsgleichung: $y'(t)=\Delta y(t)$, $y(0) = y_0$.
			\item $e^{-i t\Delta}$ $\Rightarrow$ Lösung der Schrödingergleichung $y'(t) = (i\Delta) y(t)$.
			\item $\sin(t\Delta^{1/2})$, $\cos(t\Delta^{1/2})$ $\Rightarrow$ Lösung der Wellengleichung $y''(t) = \Delta y(t)$.
		\end{itemize}
		Berechnen der Operatorennormen, z.B.
		$$||\Delta^n e^{-z\Delta}|| = ?$$
		Übertragung von Funktionalgleichungen in Operatorengleichungen, z.B. 
		$$e^{-t\lambda}e^{-s\lambda} = e^{-(s+t)\lambda} \Rightarrow e^{-t\Delta}e^{-s\Delta} = e^{-(s+t)\Delta} ?$$
	\end{Bemerkung}
	
	\begin{Definition}
		Der Funktionalkalkül des Laplaceoperators ist eine Abbildung
		$$\Phi: B_b(\R_+)\rightarrow B(L^2(\R^d))$$
		mit
		\begin{enumerate}
			\item[(i)] $\Phi(\varphi +\phi) = \Phi(\varphi)+ \Phi(\phi)$. $\Phi(\varphi\m \phi) = \Phi(\varphi)\circ\Phi(\phi)$. \textbf{Algebrahomomorphismus} von $B_b(\R_+)$ - punktweise Operationen - nach $B(L^2)$ - Operatorverknüpfungen.
			\item[(ii)] $||\Phi(\varphi)|| = ||\varphi||_{L^{\infty}(\R^d)}$. \textbf{Beschränktheit} des Kalküls.
			\item[(iii)] Sei $\varphi_n(t)|\leq 1$, $\varphi_n(t)\rightarrow \varphi(t)$, $n\rightarrow\infty$ für alle $t>0$. Dann gelte
			$$\Phi(\varphi_n)f\overset{n\rightarrow\infty}{\longrightarrow} \Phi(\varphi)f, ~\forall f\in L^2(\R^d)$$
			\textbf{Konvergenzeigenschaft}.
			\item[(iv)] Für $r_\mu(t) =\frac{1}{\mu-t}$ folgt $\Phi r_\mu = R(\mu, \Delta)$.
		\end{enumerate}
	\end{Definition}
	
	\paragraph{Notation:} Schreibe $\Phi(\varphi):= \varphi(\Delta)$.
	
	Idee zur Konstruktion von $I$:
	$$-\laplace = \F^{-1}M\F$$
	mit $Mg(\xi) = (4\pi^2 \m |\xi|^2)g(\xi)$, $m(\xi) = 4\pi^2|\xi|^2$, $$D(M) = \{g\in L^2(\R^d): ~ m\m g\in L^2, \text{ d.h. } |\xi|^2g(\xi) \in L^2 \}.$$ Konstruiere zuerst
	
	\begin{Definition}[Funktionalkalkül für $M$]
		$$M^2 g = M(Mg) = M(m\m g) = m^2 g, M^g = (m^n)g.$$
		\begin{eqnarray}
			p(\lambda)&=& \sum a_n\lambda^n,\nonumber\\
			p(M)g &=& \left(\sum a_nM^n \right)g = \left(\sum a_n m^n \right)g = \left[p(m) \right]g\nonumber
		\end{eqnarray}
		wobei $p(m)(u) = p(m(u))$ (Komposition von Funktionen).
		
		Sei $\varphi \in B_b(\R_+)$: 
		\begin{eqnarray}
			\Phi_M(\varphi)g &=& \varphi(M)g~=~\boxed{\varphi(m)}g,~ \varphi(m) =\varphi\circ m\nonumber
		\end{eqnarray}
	\end{Definition}
	
	\emph{Nachweis der Eigenschaften (i)-(iv) für $M$:} (i)
	\begin{eqnarray}
		\varphi(M)g +\psi(M)g &=& (\varphi\circ m)g+(\psi\circ m)g\nonumber\\
		&=& \left[(\varphi+\psi)\circ m \right]g \nonumber\\
		&=& (\varphi+\psi)(M)g\nonumber\\ \nonumber\\
		\left[\varphi(M)\m \psi(M) \right](g) &=& \varphi(M)\left[\psi(M)g \right]\nonumber\\
		&=& \varphi(M)\left[(\psi\circ m)g \right]\nonumber\\
		&=& (\varphi\circ m)(\psi\circ m)g\nonumber\\
		&=& \left[(\varphi\m \psi)\circ m\right] g\nonumber\\
		&=& (\varphi\m \psi)(M)g\nonumber
	\end{eqnarray}
	(ii)
	\begin{eqnarray}
		||\varphi(M)||_{B(L^2)} &=& \int |\varphi(M)g(x)|^2 dx\nonumber\\
		&=& \int|\varphi(m(x))g(x)|^2dx\nonumber\\
		&=& ||m\circ\varphi||_{L^\infty(\R^d)}\nonumber\\
		&=& \text{ess}\sup_{x\in \R^d}|\varphi(m(x))|\nonumber\\
		&=& \text{ess}\sup_{\lambda>0}|\varphi(\lambda)|\nonumber
	\end{eqnarray}
	(iii) Zu zeigen:
	\begin{eqnarray}
		\varphi_n(M)f &\rightarrow& \varphi(M)f\text{ in } L^2(\R^d).\nonumber\\
		\nonumber\\
		|\varphi(M)f-\varphi_n(M)f||_{L^2}^2 &=&\int |\varphi(m(x)) - \varphi_n(m(x))|^2 |f(x)|^2 dx\overset{n\rightarrow \infty}{\longrightarrow} 0\nonumber
	\end{eqnarray}
	nach Satz von Lebesgue, da $|\varphi(m(x)) - \varphi_n(m(x))|^2\rightarrow 0$ für $n\rightarrow\infty$ für alle $x\in \R^d$.\\
	(iv) Zu zeigen:
	\begin{eqnarray}
		r_\mu(M) &=& R(\mu, M)\nonumber\\
		\nonumber\\
		R(\mu,M)&=& (\mu-M)^{-1},\nonumber\\
		(\mu-M)g&=& (\mu-m(x))g(\m)\nonumber\\
		\Rightarrow (\mu-M)^{-1}g &=& (M-m(\m))^{-1}g(\m) = r_\mu(M)\nonumber
	\end{eqnarray}
	\qed
	
	\begin{Definition}[Konstruktion des Funktionalkalküls für $\laplace$]
		$$\varphi(-\laplace) = \F^{-1}\varphi(M)\F.$$
	\end{Definition}
	
	Nachprüfen der Eigenschaften von $\Phi$: (i)
	\begin{eqnarray}
		\varphi(-\laplace)\psi(-\laplace) &=& \left(\F^{-1}\varphi(M)\F \right)\left(\F^{-1}\varphi(M)\F \right)\nonumber\\
		&=&\F^{-1}[\varphi(M)\psi(M)]\F \nonumber\\
		&=& (\varphi\m \psi)(-\laplace)\nonumber
	\end{eqnarray}
	(ii)
	\begin{eqnarray}
		||\varphi(-\laplace)|| \overset{\text{Isometrien}}{=}||\varphi(M)|| =||\varphi||_{L^{\infty}(\R^d)}.\nonumber
	\end{eqnarray}
	(iii) 
	\begin{eqnarray}
		\varphi_n(M)f&\overset{n\rightarrow\infty}{\longrightarrow}&\varphi(M)f,~f\in L^2\nonumber\\
		\Rightarrow \varphi_n(-\laplace)g = \F^{-1}[\varphi_n(M)(\F g)]&\rightarrow &\F^{-1}[\varphi(M)(\F g)] \text{ da }\F\text{ stetig.}\nonumber
	\end{eqnarray}
	(iv)
	\begin{eqnarray}
		(\lambda-M)r_\mu(M)&=& \id\nonumber\\
		\Rightarrow \F^{-1}[(\lambda-M)r_\mu(M)]\F &=&\id\nonumber\\
		\Rightarrow R(\lambda,-\laplace) &=& r_\mu(-\laplace)\nonumber
	\end{eqnarray}
	\qed
	
	\paragraph{Beispiel.} Definition von $-(-\laplace)^{1/2}$.
	\begin{eqnarray}
		\F(-(-\laplace)^{1/2} f)(\xi) &=& 2\pi|\xi|\f(\xi):\nonumber\\
		(-(-\laplace))^2 &=& -\laplace\nonumber
	\end{eqnarray}
	
	\subsection{Das Cauchyproblem für die Schrödinger- und die Wellengleichung}
	
	\begin{Motivation}
		$$y'(t) = a(y(t))+ f(t), ~y(0) = y_0,$$
		wobei $a\in \R$, $f\in L^1(\R_+)$.
		$$y(t) = e^{at}y_0+\int_{0}^{t}e^{a(t-s)}f(s)ds.$$
	\end{Motivation}
	
	\begin{Bemerkung}[Cauchyproblem für die Schrödingergleichung]
		$$(+)~u'(t) = i\laplace u(t) +f(t),~ u(0) = y_0,$$
		wobei $y_0\in L^2(\R^d)$, $f(t)\in L^2(\R^d)$, $\int_{0}^{\infty}||f(t)||_{L^2}dt<\infty$, $A = i\laplace$.
	\end{Bemerkung}
	
	\paragraph{Notation:} $\hat{u}(t,\xi)  =\F(u(t,\m))(\xi)$ für festes $t\in \R$ heißt partielle Fouriertransformation bezüglich $x$.
	
	Anwendung der partiellen Fouriertransformation auf $(+)$.
	\begin{eqnarray}
		\partial_t \hat u(t,\xi) &=& i(-4\pi^2|\xi|^2)\hat u(t,\xi) +\f(t,\xi) 
	\end{eqnarray}
	Für festes $\xi\in \R^d$ ist das eine gewöhnliche Differentialgleichung wie in 8.1. Die Lösung von (1) für festes $\xi$ ist
	\begin{eqnarray}
		\hat u(t,\xi) &=& e^{-4\pi^2i|\xi|^2}\hat{y}_0(\xi)+\int_{0}^{\infty} e^{-4i\pi^2(t-s)|\xi|^2}\f(s,\xi)ds
	\end{eqnarray}
	Mit der inversen partiellen Fouriertransformation bezüglich $\xi$ ($t$ fest) erhält man
	\begin{eqnarray}
		u(t,\xi) &=& \F^{-1}\left[e^{-4i\pi^2|\xi|^2}\hat{y}_0(\xi) \right](x) +\int_{0}^{t}\F^{-1}\left[e^{-4i\pi^2|\xi^2(t-s)}\f(x,\xi)\right]ds
	\end{eqnarray}
	wobei im letzten Ausdruck mit Hilfe von Fubini die Integration und $\F^{-1}$ vertauscht wurden. Mit Hilfe des Funktionalkalküls kann man (3) interpretieren als
	\begin{eqnarray}
		\boxed{u(t) = e^{it\laplace}y_0 +\int_0^t e^{i(t-s)\laplace}f(s) ds,}
	\end{eqnarray}
	denn
	\begin{eqnarray}
		\F(e^{it\laplace}y_0)(\xi) &=& e^{itm(\xi)}\hat{y}_0(\xi),~~ m(\xi) = 4\pi|\xi|^2\nonumber\\
		&=& e^{-4\pi^2 it|\xi|^2}\hat{y}_0(\xi)\nonumber
	\end{eqnarray}
	
	\begin{Prop}
		Die Operatoren $T(t) = e^{it\laplace}$ erfüllen:
		\begin{enumerate}
			\item[(i)] $T(t+s) = T(s)T(t)$, $t,s\in \R$.
			\item[(ii)] $T^{-1}(t) = T(-t) = T(t)^*$, d.h. die $T(t)$ sind eine unitäre Gruppe in $B(L^{2})$. $||T(t) ||= 1$.
			\item[(iii)] $T_t f\overset{t\rightarrow 0}{\longrightarrow}f$ in $L^2$, $f\in L^2$.
			\item[(iv)] Für $f\in D(\laplace)$ gilt:
			$$\lim\limits_{t\rightarrow 0}\frac{T(t)-I}{t}f = i\laplace f, ~~ \frac{d}{dt}e^{it\laplace}|_{t=0}f = i\laplace f.$$
		\end{enumerate}
	\end{Prop}
	
	\Bew (i) 
	\begin{eqnarray}
		T(t)&=&\varphi_t(-\laplace),~ \varphi_t(x) = e^{-i\lambda t}\nonumber\\
		\varphi_t(\lambda)\varphi_s(\lambda) = \varphi_{s+t}(\lambda)&\Rightarrow& T(t) T(s) = T(t+s)\nonumber\\
		\varpi_t(\lambda)\m\varphi_{-t}(\lambda)= 1&\Rightarrow& T(t)\m T(-t) = \id\nonumber
	\end{eqnarray}
	(ii) 
	$$||T(t)|| = \sup_{\lambda>0}|e^{i\lambda t}| =1.$$
	(iii)
	$$|\rho_t(\lambda)|\leq 1, \varphi_t(\lambda)\rightarrow 1\Rightarrow T(t)f\rightarrow T(0) =\id.$$
	(iv) Da $f\in D(\laplace)$, ist $(i\laplace)f=g\in L^2$.
	\begin{eqnarray}
		\frac{1}{-i\lambda t}(e^{-i\lambda t}-1)&\rightarrow& 1\nonumber\\
		\Rightarrow \frac{1}{t}(e^{i\lambda\laplace}- I)(i\laplace)^{-1}g&\rightarrow& g\nonumber\\
		\Rightarrow \lim\limits_{t\rightarrow 0}\frac{(T(t)-I)}{t}f &=& i\laplace f.\nonumber
	\end{eqnarray}
	
%% Bitte lese das Folgende hier Korrektur, denn es began mit einer "Motivation", keine Ahnung, ob du die mit drin haben willst, und danach kam wieder Prop 8.3 und ich bin mir nicht sicher, wo die Nummerierung schief ging.
	Funktionalkalkül des Laplace: $ \phi ( - \laplace )f = F^{-1}[\phi ( 4 \pi^{2} |\xi|^{2} \hat{f}(\xi)], \phi: \R_{+} \rightarrow \C$ beschr. Borel. \\
	Cauchyproblem für die Schrödingergleichung:
		\[ \begin{cases}
			u_{t}(t, x) & = i \laplace_{x} u(t, x ) + f(t, x) \\
			u(0, x) = u_{0}(x)
		\end{cases} \]
	mit $u_{0} \in S(\R^{d}), f \colon \R_{+} \rightarrow S(\R^{d}), \int_{0}^{\infty} \| f(t) \|_{L^{1}} dt < \infty$  \\
	Lsg.: $u(t) = T(t) u_{0} + \int_{0}^{t} T(t - s) f(s) ds$
		\[ T(t) = e^{it\laplace}, \text{ d.h. } \widehat{T(t)f}(\xi) = e^{it \cdot ( 4 \pi^{2} |\xi|^{2} )} \hat{f}(\xi) \]
	$T(t)$ unitäre Gruppe auf $L^{2}(\R^{d})$.


	\setcounter{Satz}{2} % todo keine Ahnung warum noch einmal 8.3 kam, ich texe es einfach ab
	\begin{Prop}
		$T(t) f(x) = \int_{\R^{d}} k_{t}(x-y) f(y) dy$, $t > 0$ für $f \in S(\R^{d})$, \\
		wobei: $k_{t}(x) = (t \pi^{2} i t)^{-\frac{d}{2}} e^{- \frac{|x|^{2}}{4 i t}}$ 
	\end{Prop}
	
	\Bew 
		$f \in S(\R^{d}), k_{t} \in S'(\R^{d})$ $(|k_{t}(x)| = |(4 \pi^{2} t)^{-\frac{d}{2}}|)$ \\
		d.h. $k_{t} \ast f \in S'(\R^{d})$ und $(k_{t} \ast f)^{\wedge} = \hat{k_{t}} \cdot \hat{f}$, siehe $\S$ über Faltung von Distributionen. \\ \\
		Andererseits: $\widehat{T(t)f}(\xi) = e^{-it4\pi^{2}|\xi|^{2}} \hat{f}(\xi)$. Z.z.: $\hat{k_{t}}(\xi) ) e^{-it4\pi^{2}|\xi|^{2}}$ \\
		Bekannt: $\left( e^{-\pi |x|^{2}} \right)^{\wedge}(\xi) = e^{-\pi |\xi|^{2}}$. Mit Dilatation: $x \mapsto \frac{x}{\sqrt{u}}$ folgt:
		\[ \left( u^{-\frac{d}{2}} e^{-\pi\frac{|x|^{2}}{u}} \right)^{\wedge} (\xi) = e^{- \pi u |\xi|^{2}} \]
		Mit $u = 4 \pi s$: $\left( (4 \pi s)^{-\frac{d}{2}} e^{-\pi \frac{|x|^{2}}{4 \pi s}} \right)^{\wedge} = e^{-4 \pi^{2} s |\xi|^{2}}$ $(*)$
		\begin{eqnarray*}
			F_{1}(z) &=& \int_{\R^{d}} (4 \pi z)^{- \frac{d}{2}} e^{- \frac{|x|^{2}}{4z}} \hat{f}(x) dx, \quad l_{s}(x) = (4 \pi s)^{- \frac{d}{2}} e^{- \frac{|x|^{2}}{4s}} \\
			F_{2}(z) &=& \int_{\R^{d}} e^{- 4 \pi^{2} z |\xi|^{2}} f(\xi) d\xi \\
		\end{eqnarray*}
		wobei $z = s + i t$ mit $s > 0$, $t \in \R$. \\
		$F_{1}$ und $F_{2}$ sind analytisch auf $\C_{+} = \{ s + i t | s, t \in \R, s > 0 \}$ und $F_{1}(s) = F_{2}(s)$ für $s > 0$, $t = 0$, denn:
		\begin{eqnarray*}
			F_{1}(s) & = & \int_{\R^{d}} l_{s}(x) \hat{f}(x) fx \int_{\R^{d}} \hat{l_{s}}(\xi) f(\xi) d\xi \\
			& \overset{(*)}{=} F_{2}(s)
		\end{eqnarray*}
		Also $F_{1}(s + it) = F_{2}(s + it)$, $t \in \R$ fest. $F_{1}(it) = \lim_{s \rightarrow \infty} F_{1}(s + it) = \lim_{s \rightarrow 0} F_{2}(s + it) = F_{2}(it)$.
		Damit 
		 	\[ \hat{k_{t}}(f) = k_{t}(\hat{f}) = \int e^{- 4\pi^{2}it|\xi|^{2}} \hat{f}(\xi) d\xi, \quad F_{1} = F_{2} \text{ auf } i\R. \]
			 \[ \Rightarrow \hat{k_{t}}(\xi) \overset{\wedge}{=} e^{-4\pi^{2}it|\xi|^{2}} \]
	\qed


	\begin{Korollar} $T_{t} \colon L^{1}(\R^{d}) \rightarrow L^{\infty}(\R) \quad \text{und} \quad \| T_{t} \|_{L^{1}\rightarrow L^{\infty}} \leq \left( 4 \pi^{2} |t|^{2} \right)^{-\frac{d}{2}}$
	\end{Korollar}
	
	\Bew
		\[ |T_{t}f(x)| \leq \int \underbrace{|k_{t}(x - s)|}_{= (4 \pi^{2} |t|)^{-\frac{d}{2}}} |f(s)| ds = (4 \pi^{2} |t|)^{-\frac{d}{2}} \| f \|_{L^{1}} \quad \forall x \]
	\qed
	
	
	\subsection{Wellengleichung}
	\begin{Motivation}
		$u''(t) = - a u(t) + f(t)$ mit $u(0) = u_{0}, u_{t}(0) = u_{1}$, $a > 0$ $(1)$ hat die Lösung: 
		\[ u(t) = u_{0} cos(\sqrt{a} t) + \frac{u_{1}}{\sqrt{a}} sin(\sqrt{a} t) + \frac{1}{\sqrt{a}} \int_{0}^{t} sin(\sqrt{a} (t-s)) f(s) ds \]
		Wellengleichung: \\
		\[ \begin{rcases}
			\frac{\partial^{2}}{\partial t^{2}} u(t, x) = \laplace_{x} u(t, x) + f(t, x) \\
			\text{mit } u(0, x) = u_{0}(x), u_{t}(0, x) = u_{1}(x)
		\end{rcases} (2), \]
		mit $u_{0}, u_{1} \in S(\R^{d}), f(t) \in S(\R^{d}), \int_{0}^{\infty} \|f(t)\|_{L^{1}} dt < \infty$
		Für $t$ fest, partielle Fouriertransformation in $x$:
		\[ \xi \mapsto \hat{u}(t, \xi) = \int_{\R^{d}} e^{-2\pi i \xi \cdot x} u(t, x) dx \]
		Wende part. Fouriertransformation auf $(2)$ an: $t$ fest, $\xi \in \R^{d}$ \\
		\[ \begin{rcases}
			\frac{\partial^{2}}{\partial t^{2}} \hat{u}(t, \xi) = 4 \pi^{2} |\xi|^{2} \hat{u}(t, \xi) + \hat{f}(t, \xi) \\
			\hat{u}(0, \xi) = \hat{u}_{0}(\xi), \frac{\partial}{\partial t} \hat{u}(0, \xi) = \hat{u}_{1}(\xi)
		\end{rcases} (3), \]
		Sei $\xi \in \R^{d}$ fest und betrachte $(3)$für jedes $\xi$ als eine gewöhnliche Differentialgleicung für die Funktion $t \mapsto \hat{u}(t, \xi)$. Nach $(1)$ gilt dann:
		\[ \hat{u}(t, \xi) = \hat{u}_{0}(\xi) \cos(2 \pi |\xi| (t - s)) \hat{f}(s, \xi) ds \qquad (4) \]
		Das ist die Lösung von $(3)$ für festes $\xi$. Nun wenden wir für festes $t$ die inverse partielle Fouriertransformation bezüglich $\xi$ an. Dann folgt aus $(4)$:
		\begin{eqnarray*}
			u(t, x) = F^{-1} \left[ \cos( 2 \pi |\xi| t ) \hat{u}_{0}(\xi) \right](x) &+& F^{-1} \left[ \sin( 2 \pi |\xi| t ) \frac{1}{2 \pi |\xi| } \hat{u}_{1}(\xi) \right](x) \\
			&+& \int_{0}^{t} F^{-1} \left[ \sin( 2 \pi |\xi| (t-s) ) \frac{1}{2 \pi |\xi|} \hat{f}(s, \xi) \right](x) ds \qquad (5)
		\end{eqnarray*}
		Interpretiere $(5)$ mit Hilfe des Funktionalkalküls. Mit
		% Die folgende Zeile muss man Korrekturlesen, irgendetwas stimmt da nicht und ich wusste auch nicht, wie ich den rechten Teil darstellen soll
		$\varphi(-\laplace) f)^{\wedge} = 4 \pi^{2} |\xi|^{2} \hat{f}(\xi)$, $(\varphi((-\laplace)^{\frac{1}{2}})f)^{\wedge} = 2 \pi |\xi| \hat{f}(\xi)$, $\varphi(\lambda) = \cos(\lambda t), \varphi(\lambda) = \frac{sin(\lambda t)}{\lambda}$ folgt:
		\begin{eqnarray*}
			F^{-1}[\cos(2 \pi |\xi t) \hat{u}_{0}(\xi)] &=& F^{-1}[\cos(\sqrt{4 \pi^{2} |\xi|^{2}} t) F F^{-1} \hat{u}_{0}(\xi)] \\
			 &=& \cos((-\laplace)^{-\frac{1}{2}} t) u_{0}(\xi)
		\end{eqnarray*}
		\[ F^{-1} \left[ \frac{\sin(2 \pi |\xi| t)}{2 \pi |\xi]} \hat{u}_{1}(\xi) \right] = (-\laplace)^{-\frac{1}{2}} \sin((-\laplace)^{-\frac{1}{2}} t) u_{1}(x) \]
		\[ \int_{0}^{t} F^{-1} \left[ \frac{\sin(2 \pi |\xi| (t - s))}{2 \pi |\xi|} \hat{f}(s, \xi) \right] ds = \int_{0}^{t} (-\laplace)^{-\frac{1}{2}} \sin ((-\laplace)^{-\frac{1}{2}}t-s)) f(s, x) ds \]
		Damit erhält $(5)$ die Form:
		\begin{eqnarray*}
			u(t) = \cos( (-\laplace)^{\frac{1}{2}} t ) \hat{u}_{0} + (-\laplace)^{-\frac{1}{2}} \sin( (-\laplace)^{\frac{1}{2}} t ) \hat{u}_{1} + \int_{0}^{t} (-\laplace)^{-\frac{1}{2}} \sin( (-\laplace)^{\frac{1}{2}} (t-s) ) f(s) ds \qquad (6) 
		\end{eqnarray*}
		Für $d = 1$ gilt:
		\begin{eqnarray*}
			\frac{\partial^{2}}{\partial t^{2}} u(t, x) = \laplace_{x} u(t, x), \quad u(0, x) = u_{0}(x), \quad \frac{\partial u}{\partial t}(0, x) = u_{1}(x) \qquad (7)
		\end{eqnarray*}
	\end{Motivation}
	
	
	\begin{Satz}[d'Alembert]
		Die Gleichung $(7)$ mit $d = 1$ hat die Lösung:
		\[ u(t, x) = \frac{1}{2} ( u_{0}(x + t) + u_{0}(x - t) ) + \frac{1}{2} \int_{x - t}^{x + t} u_{1}(y) dy \qquad (8) \]
	\end{Satz}
	
	\Bew
		\[ \int_{x - t}^{x + t} u_{1}(y) dy = (\chi_{[-t, t]} \ast u_{1})(x) \]
		\[ (\chi_{[-t, t]})^{\wedge}(\xi) = \int_{-t}^{t} = e^{-2 \pi i \xi x} dx = \frac{1}{2 \pi i \xi} \left( e^{2 \pi i \xi t} - e^{- 2 \pi i \xi t} \right) \]
		Partielle Fouriertransformation von $(8)$:
		\begin{eqnarray*} 
			\hat{u}(t, \xi) &=& \frac{1}{2} \left( e^{-2 \pi i \xi t} + e^{2 \pi i \xi t} \right) \hat{u}_{0}(\xi) + \frac{1}{4 \pi i \xi} \left( e^{-2 \pi i \xi t} - e^{- 2 \pi i \xi t} \right) \\
				&=& \cos( 2 \pi t |\xi| ) \hat{u}_{0}(\xi) + \frac{1}{2 \pi |\xi|} \sin(2 \pi t |\xi|) \hat{u}_{1}(\xi) 
		\end{eqnarray*}
		Das stimmt mit $(5)$ überein, d.h. $u(t, x)$ ist die Lösung.
	\qed
	
	
	\begin{Folgerung}
		\[ u(t, x) = \underbrace{\frac{1}{2}u_{0}(x + t) + \frac{1}{2} \int_{0}^{x+t} u_{1}(y)dy}_{= F(x+t)} + \underbrace{\frac{1}{2} u_{0}(x-t) + \frac{1}{2}\int_{x-t}^{0} u_{1}(y)dy}_{= G(x-t)} \]
	\end{Folgerung}
	
	
	\begin{Definition}
		Seien $d = 3$, $f \in S(\R^{d})$
		\begin{eqnarray*}
			M_{t}(f)(x) &=& \frac{1}{4 \pi} \int_{S^{2}} f(x - t\gamma) d\sigma(\gamma) \\
			&=& \int_{|x-y|=t} f(y) d\tilde{\sigma}_{t}(y)
		\end{eqnarray*}
		$\tilde{\sigma}_{t}(y)$ normierte Oberflächenmaß oder Sphäre um $x$ mit Radius $t$.
	\end{Definition}
	
	
	\begin{Satz}
		$u_{0}, u_{1} \in S(\R^{3}), f \equiv 0$. Dann hat $(2)$ die Lösung
		\[ u(t, x) = \frac{\partial}{\partial t} \left( t u_{t}(u_{0})(x) \right) + t u_{t}(u_{1})(x) \]	
	\end{Satz}
	
	\Bew
		Stein, Shakarchi: Fourieranalysis, Chap. 6.
	\qed	
	
	\newpage
	\section{Spekraltheorie selbstadjungierter Operatoren}
	
	\subsection{Beschränkte normale Operatoren}
	
	\begin{Definition}
		Seien $H_j$ Hilberträume mit $<.,.>_j$, $j = 1,2$. Zu jedem $T\in B(H_1,H_2)$ gibt es einen Hilbertraum adjungierten Operator $T^*\in B(H_2,H_1)$ mit 
		$$<Tx,y> = <x,T^*y>.$$
	\end{Definition}
	
	\begin{Bemerkung}
		~
		\begin{enumerate}
			\item[a)] $A^*$ ist eindeutig, $A^* \in B(H)$ mit $||A^*|| = ||A||$.
			\item[b)] Beziehung zwischen der Hilbertraum-Adjungierten $T^*$ und der Banachraum-Adjungierten $T'$: 
			\begin{eqnarray}
				\Phi_1:H_1&\rightarrow& H_1', \Phi_1(x)(y) = <y,x>_1\nonumber\\
				\Phi_2: H_2&\rightarrow& H_2', \Phi_2(x)(y) = <y,x>_2,~x,y\in Hz.\nonumber
			\end{eqnarray}
			(Riesz-Homomorphismen).
		\end{enumerate}
	\end{Bemerkung}
	
	Behauptung:
	$$\xymatrix{
		H_1\ar[r]^{\Phi_1} & H_1'\\
		H_2\ar[u]^{T^*}\ar[r]_{\Phi_2} & H_2' \ar[u]_{T'}
		}, ~~ T^* = \Phi_1^{-1}(T')\Phi_2.$$
		
	\begin{eqnarray}
		\Phi_1(T^*x)(y) &=& <y,T^*x>_1\nonumber\\
		&=& <Ty,x>_2\nonumber\\
		&=& <\Phi_2(x),Ty>_1
	\end{eqnarray}
	
	\begin{Prop}
		Seien $S,T\in B(H_1,H_2)$, $R\in B(H_2,H_3)$, $\lambda\in \C$. Dann
		\begin{enumerate}
			\item[a)] $(S+T)^* = S^*+T^*$.
			\item[b)] $(\lambda S)^* = (\overline{\lambda})S^*$, $(\lambda S)^* = \lambda S'$.
			\item[c)] $(RS)^* = S^* R^*$.
			\item[d)] $S^{**} = S$.
			\item[e] $||S S^*|| = ||S^* S|| = ||S||^2$.
			\item[f)] $\Kern(S) = \Bild(S^*)^{\perp}$, $\Kern(S^*) = \Bild(S)^{\perp}$.
		\end{enumerate}
	\end{Prop}
	
	\Bew e) 
	\begin{eqnarray}
		||Sx||^2 = <Sx,Sx> = <x,S^*Sx> \leq ||x||^2||S^*S|| = ||x||^2||S^*||\m||S^*|| = ||x||^2||S||^2\nonumber
	\end{eqnarray}
	Supremum über $x$ mit $||x|| = 1$:
	$$||S||^2\leq ||S^* S||\leq ||S||^2.$$
	
	
	\begin{Definition}
		~
		\begin{enumerate}
			\item[a)] $T: H_1\rightarrow H_2$ ist \textbf{unitär}, falls $T$ invertierbar ist und $T^{-1} = T^*$.
			\item[b)] $T\in B(H)$ ist \textbf{selbstadjungiert}, falls $T^* = T$.
			\item[c)] $T\in B(H)$ ist \textbf{normal} fallst $TT^* = T^* T$.
		\end{enumerate}
	\end{Definition}
	
	\begin{Bemerkung}
		~
		\begin{enumerate}
			\item[a)] Selbtstadjungierte und unitäre Operatoren sind normal.
			\item[b)] $T$ unitär $<Tx, Ty> = <x,T^*Ty> = <x,T T^*y> = <x, TT^{-1}y> ? <x,y>$ $\Rightarrow$ eine unitäre Abbildung erhält das Skalarprodukt und $||Tx|| = ||x||$.
		\end{enumerate}
	\end{Bemerkung}
	
	\begin{Beispiel}
		$H = L^2(\Omega)$, $\Omega\subset \R^d$, $m\in L^\infty(\Omega)$. 
		$$T:H\rightarrow H,~~ Tf(\om) = m(\om)f(\om).$$
		Dann gilt $||T|| = ||m||_{L^\infty}$.
		\begin{eqnarray}
			<Tf,g> &=& \int_{\Omega} m(x)f(x) \overline{g(x)} dx \nonumber\\
			&=& \int f(x)\overline{\overline{m(x)}g(x)}dx\nonumber\\
			&=& <f,T^* g>\nonumber
		\end{eqnarray}
		$\Rightarrow$ $T^*g(x) = \overline{m(x)}g(x)$. Also 
		\begin{itemize}
			\item $T$ selbstadjungiert $\Leftrightarrow$ $m(x)\in \R$.
			\item $T$ unitär $\Leftrightarrow$ $|m(x)| = 1$.
		\end{itemize}
		$$T^{-1}f(x) = m(x)^{-1} f(x),~m(x)^{-1} = \overline{m(x)}$$.
	\end{Beispiel}
	
	\begin{Satz}
		Sei $T\in B(H)$. Dann
		\begin{enumerate}
			\item[a)] $T$ normal: $r(T) = ||T||$.
			\item[b)] $T$ selbstadjungiert $\Rightarrow$ $\sigma(T)\subset \R$.
			\item[c)] $T$ unitär $\Rightarrow$ $\sigma(T)\subset \{\lambda:|\lambda| = 1 \}$.
		\end{enumerate}
	\end{Satz}
	
	\Bew a) $$||T^2||^2 = ||(T^2)(T^2)^*|| = ...= ||T T^*||^2 = ||T||^4.$$
	$\Rightarrow$ $||T^2|| = ||T||^2$ $\Rightarrow$ $ ||T^{2^n}|| = ||T||^{2n}$.

	b),c) Übung.
	\qed
	
	\begin{Satz}
		$T\in B(H)$ ist selbstadjungiert genau dann, wenn $<Tx,x>\in \R$ $\forall x\in H$ und $$||T|| = \sup_{||x||= 1}<Tx,x>.$$
	\end{Satz}
	
	\Bew Übung.
	
	\begin{Satz}
		$P\in B(H)\backslash\{0\}$ sei eine Projektion. Dann sind äquivalent:
		\begin{enumerate}
			\item[a)] $P$ ist eine Orthogonalprojektion, d.h. $\Kern P\perp \Bild P$.
			\item[b)] $||P|| = 1$.
			\item[c)] $P$ selbstadjungiert.
			\item[d)] $P$ normal.
			\item[e)] $<Px,x>\geq 0$.
		\end{enumerate}
	\end{Satz}
	
	\Bew Funkana Übung, Buch Werner.
	
	\begin{Satz}[Lax-Milgram]
		Sei $H$ ein komplexer Hilbertraum mit $<.,.>$ und $b:H\times H\rightarrow\C$ sei eine sesquilineare Form ($b(x,\lambda y)= \overline{\lambda}b(x,y)$).
		\begin{enumerate}
			\item[a)] Falls $|b(x,y)|\leq C||x||\m||y||$ $(+)$, dann existiert ein eindeutig bestimmter Operator $T\in B(H)$ mit 
			$$b(x,y) = <x,Tx>\text{ und } ||T||\leq C.$$
			\item[b)] Falls zusätzlich $b(x,x)\geq \delta||x||^2$, $\delta\geq 0$, dann ist $T$ invertierbar und $||T^{-1} \leq \delta^{-1}$.
		\end{enumerate}
	\end{Satz}
	
	\Bew a) Sei $y\in H$ fest. Dann ist $x\rightarrow b_y(x) := b(x,y)$ ein lineares Funktional auf $H$ mit $||b_y||\leq C||y||$ (nach $(+)$). Da $b_y\in H'$, gibt es nach dem Satz von Riesz ein $z\in H$ mit $<x,z> = b_y(x) = b(x,y)$ und $||z|| = ||b_y||_{H'}\leq C||y||$.
	
	Definiere $Ty:= z$. Dann ist $||Ty||\leq ||z||\leq C||y||$.
	$$<x,Ty> = <x,z>= b(x,y)$$
	$||T||\leq C$, $T$ linear.
	
	b) Es gilt $b(x,x) \geq \delta ||x||^2$. Für $T$ aus a) und $y\in H$ gilt 
	$$<y, Ty> = b(y,y)\geq \delta ||y||^2.$$
	\begin{itemize}
		\item Aus $Ty = 0$ folgt $||y|| = 0$, d.h. $T$ ist injektiv.
		\item $\Bild T$ ist abgeschlossen, denn für $y_n\in H$ mit $T(y_n)\rightarrow z\in H$ folgt 
		$$||y_n-y_m||^2\leq \delta^{-1} b(y_n-y_m,y_n-y_m) = \delta^{-1}<y_n-y_m,T(y_n-y_m)> \leq \delta^{-1}2\sup ||y_n||\m||T(y_n)-T(y_m)||$$
		Da $\sup ||y_n||<0$, $T(y_n)\rightarrow z$, gilt $||y_n-y_m||\rightarrow 0$ für $n,m\rightarrow \infty$. Also $(y_n)$ sind Cauchy Folge und $y=\lim y_n$ existiert. Dann $Ty=\lim Ty_n = z$ und $z\in \Bild T$, d.h. $\Bild T$ ist abgeschlossen.
		\item $\Bild T = H$. Wähle $z\in (\Bild T)^{\perp}$. Dann 
		$$||z||^2\leq \delta^{-1}<z,Tz> = 0\Rightarrow z= 0$$
		Somit gilt $\Bild T = H$.
	\end{itemize}
	Also $T\in B(H)$ ist surjektiv und $T^{-1}\in B(X)$ (open-map).
	\begin{eqnarray}
		||T^{-1}x||^2\leq \delta^{-1}<T^{-1}, T^{-1}Tx> \leq \delta^{-1}||T^{-1}x||\m ||x||\nonumber
	\end{eqnarray}
	Also $||T^{-1}x||\leq \delta^{-1}||x||$.
	
	\subsection{Funktionalkalkül für beschränkte selbstadjungierte Operatoren}
	
	\paragraph{Notation} $\mathcal{P}$: Menge der Polynome auf $\R$ mit komplexen Koeffizienten. 
	
	\begin{Definition}[Funktionalkalkül für Polynome]
		Sei $A\in B(X)$, $X$ Hilbertraum, $<.,.>$, $A$ selbstadjungiert. Für 
		$$p(\lambda) = \sum_{j = 0}^{n} a_n\lambda^n, ~ a_n\in \mathbf{K}$$
		setze 
		$$p(A) = \sum_{j = 0}^{^n}a_n A^n\in B(X).$$
	\end{Definition}
	
	\begin{Definition}
		Die Abbildung $p\in\mathcal{P}\rightarrow p(A) \in B(X)$ hat die Eigenschaften ($f,g\in \mathcal{P}, \lambda\in \C$)
		\begin{enumerate}[(i)]
			\item $(\alpha f +g)(A) = \alpha f(A)+g(A)$, linear. \\
			$(f\m g)(A) = f(A)\m g(A)$, multiplikativ.
			\item $f_0(\lambda)\equiv 1$, $f_1(\lambda) =\lambda$ $\Rightarrow$ $f_0(A) = \id_X$, $f_1(A) = A$.
			\item $f(A)^* =\overline{f}(A)$.
			\item $\boxed{\|p(A)\|\leq \sum_{j = 0}^{n}|a_j|\m\|A\|^j}$ für $p(\lambda) = \sum_{j = 0}^{n} a_j\lambda^j$
		\end{enumerate}
	\end{Definition}
	
	\Bew (i) Übung, (ii) folgt aus Definition, (iii) 
	\begin{align*}
		p(A)^* = (\sum_{j = 0}^{n}a_j A^n)* = \sum_{j = 0}^{n} (a_n A^n)^* = \sum_{j = 0}^{n} \overline{a}_j (A^*)^n = \overline{p}(A^*) = \overline{p}(A)
	\end{align*}
	(iv) $\|p(A)\| \leq\sum_{j = 0}^{n}|a_j|\m\|A^j\|\leq \sum_{j = 0}^{n}|a_j|\m\|A\|^j$.
	\qed
	
	\begin{Satz}[Spektralabbildungssatz]
		Für einen selbstadjungierten Operator $A\in B(X)$ und $p\in \mathcal{P}$ gilt 
		\begin{align*}
			\sigma(p(A)) = p(\sigma(A)) = \{p(\lambda):\lambda\in \sigma(A) \}.
		\end{align*}
	\end{Satz}
	
	\Bew \glqq$\supseteq$\grqq\ Sei $\mu\in \sigma(A)$. Zu zeigen: $p(\mu)\in \sigma(p(A))$. Dazu wähle ein Polynom $q$, sodass 
	\begin{align*}
		&~ p(\mu) - p(\lambda) = (\mu-\lambda)q(\lambda), ~\lambda>0\\
		\Rightarrow & ~p(\mu)-p(A) = (\mu-A)q(A) = q(A)(\mu-A)
	\end{align*}
	Da $\mu-A$ keine Inverse hat ($\mu\in \sigma(A)$), hat auch $p(\mu)-p(A)$ keine Inverse, d.h. $p(\mu)\in \sigma(p(A))$.
	
	\glqq$\subseteq$\grqq\ Sei $\mu\in \sigma(p(A))$. Zeige: $\mu\in p(\sigma(A))$. Seien $\lambda_1,...,\lambda_n$ die Wurzeln von $\lambda\rightarrow \mu-p(\lambda)$, d.h. $\mu-p(\lambda) = a(\lambda-\lambda_1)\m...\m (\lambda-\lambda_n)$, $(+)$. Damit folgt $\mu-p(A) = a(A-\lambda_1)\m...\m (A-\lambda_n)$. Wäre $\lambda_1,...,\lambda_n\notin \sigma(A)$, dann folgt 
	\begin{align*}
		(\mu-p(A))^{-1} = a^{-1}(A-\lambda_n)^{-1}\m...\m (A-\lambda_1)^{-1}
	\end{align*}
	also wäre $\mu\notin \sigma(p(A))$. Da $\mu \in \sigma(p(A))$, ist mindestens eines der $\lambda_i$ in $\sigma(A)$. Dann folgt aus $(+)$ $\mu=p(\lambda_j)$, $\mu \in p(\sigma(A))$.
	\qed
	
	
	\begin{Lemma}
		Für $A\in B(X)$ selbstadjungiert und $p\in \mathcal{P}$ gilt 
		$$\|p(A)\|_{B(X)} = \sup_{\lambda\in \sigma(A)}|p(x)|.$$	
	\end{Lemma}
	
	\Bew 
	\begin{align*}
		\|p(A)\|^2 = \|p(A)p(A)^*\| = \|p(A)\overline{p}(A)\| = \| (p\m \overline{p})(A)\| = \||p|^2(A)\| = r(|p|^2(A))
	\end{align*}
	da $|p(\m)|^2$ reell und $|p|^2(A)$ selbstadjungiert ist. Weiter ist
	\begin{align*}
		r(|p|^2(A)) = \sup\{|\lambda| : \lambda\in \sigma(|p|^2(A)) \} \overset{\text{2.2.3}}{=} \sup\{|p|^2(\lambda): \lambda\in \sigma(A) \}
	\end{align*}
	Durch Wurzelziehen erhält man 
	$$\|p(A)\| = \sup|\{p(\lambda)\in \sigma(A) \}|.$$
	\qed
	
	Idee: $f(A) = \lim p_n(A)$ für $p_n\in \mathcal{P}$ mit $\|p_n-f\|_{C(\sigma(A))}\rightarrow 0$.
	
	
	\begin{Satz}[Funktionalkalkül für stetige Funktionen]
		Es gibt eine lineare, multiplikative Abbildung 
		\begin{align*}
			\Phi: C(\sigma(A)) \rightarrow B(X) ~(\text{schreibe} \Phi(f) = f(A))
		\end{align*}
		mit $\Phi(p) = p(A)$ für $p\in \mathcal{P}$ und
		\begin{enumerate}[(i)]
			\item $\|f(A)\| = \sup\{|f(\lambda)|: \lambda\in \sigma(A) \}$.
			\item $f(A)^* = \bar{f}(A)$, $f(A)$ normal, $f(A)$ selbstadjungiert $\Leftrightarrow$ $f$ reellwertig. $f(A)\geq 0$ $\Leftrightarrow$ $f(\lambda)\geq 0$ für $\lambda\in \sigma(A)$.
			\item $Ax = \lambda x$ $\Rightarrow$ $f(A)x = f(\lambda)x$.
			\item $\sigma(f(A)) = f(\sigma(A))$.
		\end{enumerate}
	\end{Satz}
	
	\Bew Nach dem Satz von Weierstraß sind die Polynome dicht in $(C(\sigma A),\|\m\|_\infty)$, da $\sigma(A)\subset \R$. $\Phi$ ist die stetige Fortsetzung der Abbildung $p\in \mathcal{P}\rightarrow p(A)\in B(X)$, d.h. für $p\in \mathcal{P}$ mit $\|p_n-f\|_\infty\rightarrow 0$ setze $\Phi(f) = \lim\lim\limits_{n\rightarrow \infty} p_n(A)$ in $B(X)$. Das ist möglich, da $p\in \mathcal{P}\rightarrow p(A)\in B(X)$ eine Isometrie ist, wegen 2.2.3 
	$$\|p(A)\| = \sup\{|p(\lambda)|:\lambda\in \sigma(A) \}.$$
	Da $$\|f(A)\| = \lim\limits_{n\rightarrow\infty} \|p_n(A)\| = \lim\lim\limits_{n\rightarrow\infty}sup_{\lambda\in \sigma(A)}|p_n(x)| = \sup_{\lambda\in \sigma(A)}|f(x)$$
	gilt also 
	$$\|\Phi(f)\| = \sup_{\lambda\in\sigma(A)}|f(\lambda)| \text{ nach (i)},$$
	d.h. $\Phi$ ist linear und multiplikativ.
	$$f(A)^* = \lim p_n(A)^* = \lim \overline{p}_n(A) = \overline{f}(A)$$
	Zeige nun $f(A)$ ist normal: 
	$$f(A)\m f(A)^* = f(A) \m \overline{f}(A) = (f\m \overline{f})(A) = (\overline{f}\m f)(A) = \overline{f}(A)\m f(A) = f(A)^*\m f(A).$$
	$f(A)$ selbstadjungiert $\Leftrightarrow$ $f$ ist reelwertig:
	$$f(A) = f(A)^* = \overline{f}(A) \Leftrightarrow f = \overline{f}	\Leftrightarrow f\text{ ist reelwertig.}$$
	$f\geq 0$ $\Rightarrow$ $f(A)\geq 0$, d.h. $<f(A)x,x> \geq 0$ für alle $x$. Wähle $g\geq 0$ mit $f = g^2$. Dann 
	$$<f(A)x,x> = <g^2(A)x,x> = <g(A)g(A)x,x> = <g(A)x,g(A)^*x> = <g(A)x,g(A)x> \geq 0.$$
	(iii) $p\in \mathcal{P}$, $Ax = \lambda x$
	\begin{align*}
		\Rightarrow p(A) = \left(\sum_{j = 0}^{n}a_j A^j \right)x = \left(\sum_{j = 0}^{n} a_j \lambda^j \right)x
	\end{align*}
	$\Rightarrow$ $f(A)x = f(\lambda)x$ mit Hilfe von Approximationen von $f$ durch Polynome $p_n$.
	(iv) \glqq$\subseteq$\grqq\ Sei $\mu\in f(\sigma(A))$. Setze $g(\lambda) = (f(\lambda)-\mu)^{-1}\in C(\sigma(A))$. Dann gilt
	\begin{align*}
		g(f-\mu) = (f-\mu) g\equiv 1\\
		\Rightarrow g(A)(f(A)-\mu) = (f(A)-\mu)g(A) = \id_X\Rightarrow \mu\notin\sigma(f(A)).
	\end{align*}
	\glqq$\supseteq$\grqq\ Sei $\lambda\in\sigma(A)$. Zeige: $f(\mu)\in \sigma(f(A))$. Wähle $p_n\in \mathcal{P}$ mit $\|f-p_n\|_{C(\sigma(A))}\rightarrow 0$. 
	\begin{align*}
		\|f(\mu)-f(A)-(p_n(\mu)-p_n(A))\|_{B(X)} \leq \sup_{\lambda\in \sigma(A)}|f(\mu)-f(\lambda)-p_n(\mu)+p_n(\lambda)| \overset{n\rightarrow\infty}{\rightarrow}0\\
		f(A)^* = \lim p_n(A)^* = \lim \overline{p}_n(A) = \overline{f}(A)
	\end{align*}
	Nach 2.2.3 gilt $p_n(\mu)\in \sigma(p_n(A))$. Wäre $f(\mu)-f(A)$ invertierbar, so wäre wegen $p_n(\mu)-p_n(A)\rightarrow f(\mu)-f(A)$ in $B(X)$ auch $p_n(\mu)-p_n(A)$ für große $n$ invertierbar, da die Menge der invertierbaren Operatoren in $B(X)$ offen ist. Also $f(\mu)-f(A)$ nicht invertierbar, d.h. $f(\mu)\in\sigma(f(A))$.
	
	\paragraph{Ziele.} 
	\begin{enumerate}[1)]
		\item $\Phi: B_b(\sigma(A)) \rightarrow B(X)$, $B_b$ beschränkte Borelfunktionen.
		\item Finde ein Maß $\mu$ auf $\C$ und eine Isometrie $U: L^2(\C,\mu)\rightarrow X$, sodass 
		$$A=UMU^{-1},\text{ wobei } Mg(\lambda = m(\lambda)g(\lambda).$$
	\end{enumerate}
	
	Nach Riesz gilt $C(\sigma(A))' = M(\sigma(A))$. Zu $|l\in C(\sigma(A))$ gibt es ein Maß $\mu$:
	$$l(f)\int_{\sigma(A)}f(\lambda d\mu(\lambda)).$$

	
	
	
	
	
	
	
	
	
	
	
	
	
	
	
	
	
	
	
	
	
	
	
	
	
	
	
	
	
	
	
	
	
	
	
	
	
	
	
	
	
	
	
	
	
	
\end{document}