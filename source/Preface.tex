%!TEX root = Funktionalanalysis - Vorlesung.tex

\chapter*{Vorwort}
Dieses Skript wurde im Sommersemester 2016
von Martin Belica geschrieben. Es ist ein inoffizielles Skript und beinhaltet die Mitschriften aus der Vorlesung von Prof.~Dr.~Weis am Karlsruhe Institut für Technologie sowie die Mitschriften einiger Übungen.

\thispagestyle{empty}

\section*{Einleitung}

Die Spektraltheorie verallgemeinert die Theorie von Eigenwerten und Normalformen von Matrizen für unendlichdimensionale Operatoren auf Funktionenräumen, wie Differential- und Integraloperatoren. Sie vermittelt eine wesentliche Methodik für viele Anwendungsgebiete, wie partielle Differentialgleichungen, mathematische Physik und numerische Analysis.

Zu den Themen gehören:

  \begin{itemize}
     \item Spektrum und Resolvente linearer (unbeschränkter) Operatoren
     \item Fouriertransformation und der Funktionalkalkül des Laplace-Operators
     \item Der Funktionalkalkül selbstadjungierter Operatoren
     \item Der holomorphe Funktionalkalkül sektorieller Operatoren
     \item Cauchy-Problem für sektorielle Operatoren
  \end{itemize}
  
Diese Vorlesung bereitet auf zukünftige Vorlesungen und Seminare im Bereich der deterministischen und stochastischen Evolutionsgleichungen vor.

\section*{Erforderliche Vorkenntnisse}
Wir setzen ein grundlegendes Verständnis funktionalanalytischer Methoden voraus, wie sie z.B. in den Vorlesungen "Differentialgleichungen und Hilberträume" oder "Funktionalanalysis" vermittelt werden.


\begin{center}	

\end{center}